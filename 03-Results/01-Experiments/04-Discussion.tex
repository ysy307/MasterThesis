\subsection{本解析手法の妥当性と適用範囲に関する総括}
\label{Sec:general_discussion}

以上の結果に基づき,本解析手法の適用範囲と工学的有用性について総括する.
実験初期に見られた過冷却現象に伴う温度の不一致は,核形成確率などの相変化の微視的メカニズムを捨象したマクロな連続体モデルの限界を示すものである.しかしながら,人工凍結工法の設計や施工管理において主たる関心事となるのは,過冷却解消後の定常的な凍結進展挙動や,最終的に形成される凍結土の造成範囲である.本数値モデルは,凍結進行に伴う潜熱放出と地下水流による熱供給のバランスを正確に再現できており,長期的な温度場形成を予測する上で十分な精度を有していると判断できる.

また,逆解析によって同定された境界条件が実測値と乖離した事実は,実験計測値には不可避的に環境からの外乱が含まれることを示唆している.逆に言えば,本研究で用いた逆解析手法は,そうした外乱の影響をフィルタリングし,数値計算上で物理的整合性を保つための実効的な境界条件を抽出する有効な手段であるといえる.不確実性の高い実験境界条件を直接入力するのではなく,逆解析を通じてシステム全体の熱収支と整合する値を同定するアプローチは,複雑な熱的境界を有する地盤凍結問題を扱う上で極めて合理的である.

したがって,構築した数値解析モデルおよび逆解析によるパラメータ同定手法は,凍結現象の主要なプロセスを捉える妥当なアプローチであり,次章以降の検討における信頼性は担保されたものと結論付ける.本検証により,数値モデルは実験室規模の現象を再現可能であることが確認されたため,次章では本ソルバーを用いて,より広範な条件下でのパラメータスタディや,実規模を想定した解析への適用を進める.

\FloatBarrier