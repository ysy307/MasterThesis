\subsection{結果と考察}
\label{Sec:results_discussion}

\subsubsection{逆解析を用いた境界条件推定}
\label{Sec:bc_estimation}
\cref{Sec:inv_analysis}で述べた逆解析手法を用いて,境界条件の温度について推定を行った.
実験期間中の外部環境温度の変化を\cref{Fig:exp_temp_change}に示す.
また,実験期間中の外部環境温度に関する統計量を\cref{Table:temp_stats}に示す.

まず,\cref{Fig:exp_temp_change},\cref{Table:temp_stats}より,凍結管表面の熱電対の温度は,往き側・還り側はそれぞれ平均約\qty{-1.44}{\degreeCelsius},\qty{-1.18}{\degreeCelsius}であり,実験期間中においては比較的安定していたことが分かる.
また,室温は平均約\qty{5.05}{\degreeCelsius}であり,初期温度場として\qty{5.0}{\degreeCelsius}を設定したことも妥当であり,実験装置周囲の放射熱交換も大きく変動しないと考えられるため,断熱境界条件の妥当性も支持される.
つまり,実験期間中においては,外部環境温度は比較的一定であったため,逆解析における温度境界条件推定において,境界での温度を時間発展させなくても安定した結果が得られることが期待される.

\begin{figure}[tbp]
  \centering
  \includegraphics[width=\linewidth,pagebox=cropbox,clip]{3-1/Exp_Case3_8.pdf}
  \caption{実験時の外部環境温度変化}
  \label{Fig:exp_temp_change}
\end{figure}

\begin{table}[tbp]
  \centering
  \caption{実験期間中の外部環境温度に関する統計量}
  \label{Table:temp_stats}
  \begin{tabular}{lccccc}
    \toprule
    Variable      & Min        & Median     & Max        & Mean        & Std. Dev.  \\
    \midrule
    Inlet         & \num{-1.6} & \num{-1.4} & \num{1.2}  & \num{-1.44} & \num{0.09} \\
    Outlet        & \num{-1.3} & \num{-1.2} & \num{2.0}  & \num{-1.18} & \num{0.13} \\
    Room Temp.    & \num{3.9}  & \num{5.0}  & \num{5.7}  & \num{5.05}  & \num{0.28} \\
    Water Temp.   & \num{3.0}  & \num{3.1}  & \num{4.0}  & \num{3.26}  & \num{0.27} \\
    Chiller Temp. & \num{-3.1} & \num{-2.8} & \num{-2.7} & \num{-2.83} & \num{0.06} \\
    \bottomrule
  \end{tabular}
\end{table}

逆解析により推定された境界温度は,下端が\qty{4.219}{\degreeCelsius},往き側が約\qty{-2.500}{\degreeCelsius},還り側が約\qty{-1.970}{\degreeCelsius}となり,いずれも実験期間中の実測平均値とは異なる値を示した.
まず,往き側および還り側の推定温度が実測値の平均(それぞれ約\qty{-1.44}{\degreeCelsius},\qty{-1.18}{\degreeCelsius})よりも低くなった点について考察する.この乖離は,実験装置の断熱が完全ではないことに起因すると考えられる.すなわち,周囲環境からの放射および自然対流による受熱が,配管表面や熱電対そのものに影響を与え,実測値が内部を流れる実際のブライン温度よりも高く観測されていた可能性が示唆される.推定された境界温度はチラーの設定温度(約\qty{-2.8}{\degreeCelsius})に近い値であり,逆解析によって外乱の影響を除外した実効的な境界温度が得られたと解釈できる.
また,往き側の推定温度が還り側より低くなった結果は,凍結管が往き側から還り側へと直列に接続された流路構造と整合する.これは,往き側で土壌冷却に寄与したブラインが昇温して還り側へ流入する過程を正しく反映しているといえる.
一方,下端境界温度がタンク内の循環水温度(約\qty{3.26}{\degreeCelsius})よりも高く推定された点については,以下の二つの要因が挙げられる.第一に,実験装置が室内に設置されているため,循環水がタンクから試験体下端に到達するまでの経路において周囲環境からの吸熱が生じたこと,第二に,循環ポンプの機械的損失に伴う発熱が水温を上昇させたことである.これらを鑑みれば,推定された下端境界温度がタンク内水温よりも高くなることは物理的に妥当である.

\FloatBarrier