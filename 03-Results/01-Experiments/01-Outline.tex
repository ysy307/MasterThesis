\subsection{実験概要}
\label{Sec:ExperimentalSetup}
実験には,\cref{Fig:experimental_setup}に示す円筒形の凍結実験装置を用いた.装置は内径\qty{0.2}{\meter},外径\qty{0.23}{\meter},高さ\qty{0.6}{\meter}のアクリル円筒と,円筒下端から\qty{0.2}{\meter}の位置を貫通する外径\qty{0.006}{\meter}のU字型凍結管で構成される.
供試体は鳥取砂丘砂を水中充填法により作製し,底部給水・上部排水により飽和状態を維持した.凍結開始前にDarcy流速\qty{1.0}{\meter.\day^{-1}}で36時間以上給水した後,凍結管内にブライン(エチレングリコール水溶液,比重\qty{1100}{\kilogram.\meter^{-3}})を循環させて内部から冷却した.

\subsection{数値解析条件}
\begin{figure}[btp]
  \centering
  \begin{subfigure}[b]{0.45\textwidth}
    \centering
    \begin{tikzpicture}
      %==== パラメータ設定 ====
      \def\r{2cm}
      \def\h{7cm}
      \def\t{0.6cm}
      \def\R{2.6cm}
      \def\legW{0.2cm}
      \def\legL{1.5cm}
      \def\legPosRad{\r}
      \def\aspect{0.25}

      %==== 色設定 ====
      \colorlet{soilFill}{gray!5}
      \colorlet{soilDots}{black!80}
      \colorlet{plateColor}{gray!60}
      \colorlet{pipeColor}{gray!20}
      \colorlet{metalColor}{gray!50}
      \colorlet{fzPipeColor}{gray!40}
      \colorlet{fzPipeBorder}{black!70}

      %==== 計算 ====
      \pgfmathsetlengthmacro{\eCyl}{\r * \aspect}
      \pgfmathsetlengthmacro{\ePlate}{\R * \aspect}
      \pgfmathsetlengthmacro{\eLeg}{\legPosRad * \aspect}

      %=================================================
      % 1. 最背面:足
      %=================================================
      \begin{scope}[shift={({110}:{\legPosRad} and {\eLeg})}]
        \draw[fill=metalColor, draw=black!70]
        (-\legW/2,-\t) rectangle (\legW/2,-\t-\legL);
      \end{scope}

      \begin{scope}[shift={({-150}:{\legPosRad} and {\eLeg})}]
        \draw[fill=metalColor, draw=black!70]
        (-\legW/2,-\t) rectangle (\legW/2,-\t-\legL);
      \end{scope}

      % 下の外部パイプ(供給)
      \draw[double=pipeColor, double distance=10pt, draw=black!70]
      (0,-\t) -- ++(0,-0.6)
      to[out=-90, in=180] ++(1.0,-0.6)
      -- ++(2.0,0);

      %%%% ADD: 下パイプラベル(供給)
      \node[
        anchor=west,
        fill=white,
        inner sep=1.5pt,
        text width=2.2cm,
        align=left
      ] at (3.1,-1.8) {\large Water  supply};

      % 手前足
      \begin{scope}[shift={({-30}:{\legPosRad} and {\eLeg})}]
        \draw[fill=metalColor, draw=black!70]
        (-\legW/2,-\t) rectangle (\legW/2,-\t-\legL);
      \end{scope}

      %=================================================
      % 2. 下円盤
      %=================================================
      \fill[plateColor]
      (-\R,0) -- (-\R,-\t)
      arc (180:360:{\R} and {\ePlate})
      -- (\R,0)
      arc (360:180:{\R} and {\ePlate}) -- cycle;

      \fill[plateColor] (0,0) ellipse ({\R} and {\ePlate});
      \draw[thick] (0,0) ellipse ({\R} and {\ePlate});
      \draw[thick] (-\R,0)--(-\R,-\t);
      \draw[thick] ( \R,0)--( \R,-\t);
      \draw[thick] (-\R,-\t) arc (180:360:{\R} and {\ePlate});

      %=================================================
      % 3. 土壌カラム
      %=================================================
      \fill[
        pattern=dots,
        pattern color=soilDots,
        preaction={fill=soilFill}
      ]
      (-\r,0) -- (-\r,\h)
      arc (180:360:{\r} and {\eCyl})
      -- (\r,0)
      arc (360:180:{\r} and {\eCyl}) -- cycle;

      %=================================================
      % 4. 内部凍結管
      %=================================================
      \def\fzLev{\h/3}
      \def\fzSep{0.5cm}
      \def\fzRun{0.8cm}
      \def\fzRise{0.5cm}
      \def\fzRad{1.5pt}

      \tikzset{
        hPipe/.style={
            double=fzPipeColor,
            double distance=2*\fzRad,
            draw=fzPipeBorder,
            line cap=round,
            thick
          }
      }

      \coordinate (OutFront) at (-\fzSep,\fzLev);
      \coordinate (InFront)  at ( \fzSep,\fzLev);
      \coordinate (OutBack)  at ($(OutFront)-(\fzRun,\fzRise)$);
      \coordinate (InBack)   at ($(InFront)-(\fzRun,\fzRise)$);

      \draw[hPipe] (OutFront)--(OutBack);
      \draw[hPipe] (InFront)--(InBack);

      \node[anchor=north,fill=white,inner sep=1pt,
        xshift=1mm,yshift=-1.2mm] at (InBack) {Inlet};
      \node[anchor=north,fill=white,inner sep=1pt,
        xshift=-1mm,yshift=-1.2mm] at (OutBack) {Outlet};

      %=================================================
      % 5. 外枠
      %=================================================
      \draw[thick] (-\r,0)--(-\r,\h);
      \draw[thick] ( \r,0)--( \r,\h);
      \draw[thick] (-\r,0) arc (180:360:{\r} and {\eCyl});
      \draw[thick] (-\r,\h) arc (180:360:{\r} and {\eCyl});
      \draw[dashed,thick,gray]
      (\r,\h) arc (0:180:{\r} and {\eCyl});

      %=================================================
      % 6. 上円盤・排出パイプ
      %=================================================
      \fill[plateColor]
      (-\R,\h)--(-\R,\h+\t)
      arc (180:360:{\R} and {\ePlate})
      --(\R,\h)
      arc (360:180:{\R} and {\ePlate})--cycle;

      \fill[plateColor] (0,\h+\t) ellipse ({\R} and {\ePlate});

      \draw[thick] (-\R,\h) arc (180:360:{\R} and {\ePlate});
      \draw[thick] (-\R,\h)--(-\R,\h+\t);
      \draw[thick] ( \R,\h)--( \R,\h+\t);
      \draw[thick] (-\R,\h+\t) arc (180:360:{\R} and {\ePlate});
      \draw[thick] (\R,\h+\t) arc (0:180:{\R} and {\ePlate});

      \begin{scope}[shift={(0,\h+\t)}]
        \fill[black!60] (0,0) ellipse (0.6cm and 0.2cm);
        \draw[double=pipeColor, double distance=10pt, draw=black!70]
        (0,0)--++(0,0.8)
        to[out=90,in=180] ++(1.0,0.5)
        --++(2.0,0);

        %%%% ADD: 上パイプラベル(排出)
        \node[
          anchor=west,
          fill=white,
          inner sep=1.5pt,
          text width=2.2cm,
          align=left
        ] at (3.1,1.3) {\large Water Discharge};

      \end{scope}
    \end{tikzpicture}
    % \caption{実験装置の模式図}
    \caption{}
    \label{Fig:experimental_setup}
  \end{subfigure}
  \hfill
  \begin{subfigure}[b]{0.45\textwidth}
    \centering
    \includegraphics[width=\linewidth,pagebox=cropbox,clip]{3-1/Experimental_Domain_FluxBC.pdf}
    \caption{}
    \label{Fig:numerical_domain_setup}
  \end{subfigure}
  \caption{実験装置の模式図および対応する計算領域.(a) 実験装置の模式図,(b) 数値解析に用いた計算領域.(b) 中のアスタリスクでの熱移動境界条件として,逆解析により推定したパラメータ(底面 (*1) および左右 (*2,*3) の凍結管表面における境界温度)を用いる.}
  \label{Fig:experiment_and_domain}
\end{figure}

実験結果との比較を行うため,\cref{Fig:experimental_setup}の実験装置をモデル化した計算領域を\cref{Fig:numerical_domain_setup}に示す.
解析は凍結管に垂直な断面をとり,二次元水平断面として実施した.

% 境界条件の設定において,装置底面および凍結管表面の熱伝達特性は,実験条件ごとの接触熱抵抗や流体挙動の影響を受けるため,一意に決定することが困難である.
% そこで本解析では,別途実施した逆解析(第〇節参照)の結果に基づき,以下の有効境界温度を設定した.

% \begin{itemize}
%   \item \textbf{底面境界(*1)}: 給水による熱供給を考慮し,有効温度 $T_{\mathrm{bot}} = \dots ^\circ\mathrm{C}$ を固定境界条件として与えた.
%   \item \textbf{凍結管表面(*2,*3)}: ブライン循環による冷却効果を表すため,有効表面温度 $T_{\mathrm{pipe}} = \dots ^\circ\mathrm{C}$ をDirichlet境界として与えた.
% \end{itemize}

その他の側面境界については断熱条件とした.
% 初期条件および土質パラメータについては,実験開始時の実測値および第〇節で示した物性値を用いた

\subsection{結果と考察}
% (ここにグラフの図を配置して,温度変化や凍結深さの一致度について記述)