\subsection{実験概要}

\begin{tikzpicture}
  %==== パラメータ設定 ====
  \def\r{2cm}
  \def\h{6cm}
  \def\t{0.6cm}
  \def\R{2.6cm}
  \def\legW{0.2cm}
  \def\legL{1.5cm}
  \def\legPosRad{\r}
  \def\aspect{0.25}

  %==== 色設定 ====
  \colorlet{soilFill}{gray!5}
  \colorlet{soilDots}{black!80}
  \colorlet{plateColor}{gray!60}
  \colorlet{pipeColor}{gray!20}
  \colorlet{metalColor}{gray!50}
  \colorlet{fzPipeColor}{gray!40}
  \colorlet{fzPipeBorder}{black!70}

  %==== 計算 ====
  \pgfmathsetlengthmacro{\eCyl}{\r * \aspect}
  \pgfmathsetlengthmacro{\ePlate}{\R * \aspect}
  \pgfmathsetlengthmacro{\eLeg}{\legPosRad * \aspect}

  %=================================================
  % 1. 最背面:足
  %=================================================
  \begin{scope}[shift={({110}:{\legPosRad} and {\eLeg})}]
    \draw[fill=metalColor, draw=black!70]
    (-\legW/2,-\t) rectangle (\legW/2,-\t-\legL);
  \end{scope}

  \begin{scope}[shift={({-150}:{\legPosRad} and {\eLeg})}]
    \draw[fill=metalColor, draw=black!70]
    (-\legW/2,-\t) rectangle (\legW/2,-\t-\legL);
  \end{scope}

  % 下の外部パイプ(供給)
  \draw[double=pipeColor, double distance=10pt, draw=black!70]
  (0,-\t) -- ++(0,-0.6)
  to[out=-90, in=180] ++(1.0,-0.6)
  -- ++(2.0,0);

  %%%% ADD: 下パイプラベル(供給)
  \node[
    anchor=west,
    fill=white,
    inner sep=1.5pt,
    text width=2.4cm,
    align=left
  ] at (3.1,-1.8) {\large Water supply};

  % 手前足
  \begin{scope}[shift={({-30}:{\legPosRad} and {\eLeg})}]
    \draw[fill=metalColor, draw=black!70]
    (-\legW/2,-\t) rectangle (\legW/2,-\t-\legL);
  \end{scope}

  %=================================================
  % 2. 下円盤
  %=================================================
  \fill[plateColor]
  (-\R,0) -- (-\R,-\t)
  arc (180:360:{\R} and {\ePlate})
  -- (\R,0)
  arc (360:180:{\R} and {\ePlate}) -- cycle;

  \fill[plateColor] (0,0) ellipse ({\R} and {\ePlate});
  \draw[thick] (0,0) ellipse ({\R} and {\ePlate});
  \draw[thick] (-\R,0)--(-\R,-\t);
  \draw[thick] ( \R,0)--( \R,-\t);
  \draw[thick] (-\R,-\t) arc (180:360:{\R} and {\ePlate});

  %=================================================
  % 3. 土壌カラム
  %=================================================
  \fill[
    pattern=dots,
    pattern color=soilDots,
    preaction={fill=soilFill}
  ]
  (-\r,0) -- (-\r,\h)
  arc (180:360:{\r} and {\eCyl})
  -- (\r,0)
  arc (360:180:{\r} and {\eCyl}) -- cycle;

  %=================================================
  % 4. 内部凍結管
  %=================================================
  \def\fzLev{\h/3}
  \def\fzSep{0.5cm}
  \def\fzRun{0.8cm}
  \def\fzRise{0.5cm}
  \def\fzRad{1.5pt}

  \tikzset{
    hPipe/.style={
        double=fzPipeColor,
        double distance=2*\fzRad,
        draw=fzPipeBorder,
        line cap=round,
        thick
      }
  }

  \coordinate (OutFront) at (-\fzSep,\fzLev);
  \coordinate (InFront)  at ( \fzSep,\fzLev);
  \coordinate (OutBack)  at ($(OutFront)-(\fzRun,\fzRise)$);
  \coordinate (InBack)   at ($(InFront)-(\fzRun,\fzRise)$);

  \draw[hPipe] (OutFront)--(OutBack);
  \draw[hPipe] (InFront)--(InBack);

  \node[anchor=north,fill=white,inner sep=1pt,
    xshift=1mm,yshift=-1.2mm] at (InBack) {Inlet};
  \node[anchor=north,fill=white,inner sep=1pt,
    xshift=-1mm,yshift=-1.2mm] at (OutBack) {Outlet};

  %=================================================
  % 5. 外枠
  %=================================================
  \draw[thick] (-\r,0)--(-\r,\h);
  \draw[thick] ( \r,0)--( \r,\h);
  \draw[thick] (-\r,0) arc (180:360:{\r} and {\eCyl});
  \draw[thick] (-\r,\h) arc (180:360:{\r} and {\eCyl});
  \draw[dashed,thick,gray]
  (\r,\h) arc (0:180:{\r} and {\eCyl});

  %=================================================
  % 6. 上円盤・排出パイプ
  %=================================================
  \fill[plateColor]
  (-\R,\h)--(-\R,\h+\t)
  arc (180:360:{\R} and {\ePlate})
  --(\R,\h)
  arc (360:180:{\R} and {\ePlate})--cycle;

  \fill[plateColor] (0,\h+\t) ellipse ({\R} and {\ePlate});

  \draw[thick] (-\R,\h) arc (180:360:{\R} and {\ePlate});
  \draw[thick] (-\R,\h)--(-\R,\h+\t);
  \draw[thick] ( \R,\h)--( \R,\h+\t);
  \draw[thick] (-\R,\h+\t) arc (180:360:{\R} and {\ePlate});
  \draw[thick] (\R,\h+\t) arc (0:180:{\R} and {\ePlate});

  \begin{scope}[shift={(0,\h+\t)}]
    \fill[black!60] (0,0) ellipse (0.6cm and 0.2cm);
    \draw[double=pipeColor, double distance=10pt, draw=black!70]
    (0,0)--++(0,0.8)
    to[out=90,in=180] ++(1.0,0.5)
    --++(2.0,0);

    %%%% ADD: 上パイプラベル(排出)
    \node[
      anchor=west,
      fill=white,
      inner sep=1.5pt,
      text width=2.4cm,
      align=left
    ] at (3.1,1.3) {\large Water Discharge};

  \end{scope}

\end{tikzpicture}