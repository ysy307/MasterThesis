\subsection{提案モデルの適用範囲と限界}
\label{Sec:Param_Discussion}

\subsubsection{土質条件に関する適用性}
提案した予測モデルの適用性は,地盤の水理学的特性,特に透水係数によって支配される.\cref{Eq:predictive_model_l} で示された回帰係数は,\cref{Table:sim_properties} で用いた砂質土の特性に基づいて導出されたものであり,これは凍結工法の実務的な制約と整合している.工学的観点からは,地下水流が凍結を阻害するリスクは,砂や礫などの高透水性地盤において特に重要となる.粘性土(粘土やシルトなど)では,通常,透水係数が十分に低く,閉塞を妨げるほどの高いダルシー流速が生じることは稀である.したがって,透水性の高い砂質土に焦点を当てた本研究の範囲は,地下水流条件下での凍結工法設計において最もクリティカルな領域をカバーしている.具体的な回帰係数は土質の熱的特性によって変化しうるが,凍土壁の総延長と配管密度の相互作用という物理的な依存関係は,熱伝導とポテンシャル流れの基礎物理に支配されている.そのため,\cref{Eq:predictive_model_l} の定性的な傾向と数式構造は他の透水性地盤にも適用可能であり,数値シミュレーションと双曲線フィッティングを組み合わせた本提案フローは,サイト固有のパラメータを導出するための汎用的なツールとなる.

\subsubsection{理論的および幾何学的制約}
本研究で提案した $l$ の予測モデルは,高志の解析的枠組みに組み込むことを前提として設計されているため,その導出の基礎となる理論的仮定を継承している.Takashi の解析モデルにおける $l$ の概念は,非圧縮性完全流体が平板周りを流れるポテンシャル流れ理論に基づいている点に留意が必要である.彼の理論定式化において,$l$ は流れに対して直交して置かれた平板の幅に対応し,これは凍土壁の数学的な抽象化である.したがって,限界流速は直線状の凍土壁によって生じる堰上げ(dam-up)圧力差によって決定される.ゆえに,本研究の予測モデルは,原則として直線状の凍土壁配置にのみ適用される.

円形立坑などの異なる幾何形状では,ポテンシャル場が根本的に異なる.例えば,高志はは形立坑に対して,直線状凍土壁とは異なる個別のポテンシャル関数を導出している.このような場合,直線状凍土壁として定義された $l$ を単純に外挿することはできない.しかし,大規模な地下構造物(長大な直線擁壁や大口径立坑など)において,凍結管間隔に対して曲率半径が十分に大きい場合,局所的な流れ場は直線条件に近似できる.そのようなシナリオでは,提案モデルは工学的に妥当な近似値を与える.さらに,直線と円形のポテンシャル場の相違は,本研究で提案した「数値的キャリブレーション」アプローチの必要性を浮き彫りにしており,将来的には非線形形状へと拡張することで,形状固有の代表長さを導出することが可能である.

\subsubsection{温度境界条件の単純化}
本研究では,凍結管表面を一定温度境界(\qty{-30}{\degreeCelsius})として単純化した.現場では動的な熱変動が生じるものの,標準的な凍結工法の運用実態を考慮すれば,この単純化は正当化される.通常,ブラインは高い流量で循環され,管内を乱流状態に保つことで熱交換効率が最大化される.この運転条件により,供給温度と還り温度の差は数度以内に保たれることが多く,管軸方向の温度変化は,凍結を駆動する大きな半径方向の温度勾配に比べて無視できるほど小さい.したがって,仮定した一定温度は,平均的な有効冷却温度の妥当な近似となる.配管ラインでの熱損失や還り管でのわずかな温度上昇は詳細設計における要素であるが,これらを変数として組み込むことは,配管形状が限界流速に与える本質的な影響を定量化するという本研究の主目的を不明瞭にする恐れがある.実務的な設計においては,一定温度の入力値を予想される還り温度(ループ内で最も高い温度)に置き換えることで,熱効率に対する安全率を暗黙的に組み込む保守的なアプローチが可能である.

\FloatBarrier