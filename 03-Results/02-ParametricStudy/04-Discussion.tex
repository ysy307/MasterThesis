\subsection{考察}
\label{Sec:Param_Discussion}

\subsubsection{代表長さの逆算と評価}
前節の結果に基づき,限界流速式で算出される限界流速を数値的な漸近値に置き換えることで,式中の経験的パラメータである「代表長さ」を数値計算によって逆算することを試みた.
従来,この代表長さとしては凍土壁全長(凍結管列の全幅)が慣例的に用いられてきたが,管間隔や密度効果は考慮されていなかった.
このようにして数値的に求めた代表長さと,慣例的に代表長さとして用いられてきた凍土壁全長との比較を行ったところ,凍結管配置によっては両者に有意な乖離が生じることが確認された.

\subsubsection{新たな代表長さ算定式の構築}
上記の知見を踏まえ,より合理的な設計パラメータを導出するため,凍結管間隔,凍結管密度およびそれらの相互作用項を説明変数とする重回帰分析を行い,新たな代表長さの算定式を構築した.
分析の結果,提案式を用いることで,凍結管配置を考慮した代表長さを良好に整理できることが確認され,本手法の有効性が示された.

本研究で構築した代表長さの算定式は,凍結管間隔や凍結管密度など,あらかじめ決定可能な幾何学的条件のみを説明変数としている.
このため,従来設計者の経験的判断に委ねられていた代表長さを,客観的かつ定量的に決定することが可能となる.
本研究の成果は,解析的設計手法の信頼性および実用性を向上させ,地下水流動を伴う人工地盤凍結工事の設計において,過度に保守的な設計や不要な地盤改良の解消に寄与することが期待される.

\FloatBarrier