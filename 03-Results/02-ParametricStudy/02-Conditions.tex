\subsection{解析条件および計算ケース}
\label{Sec:Param_Conditions}

\subsubsection{計算領域および初期・境界条件}
\label{Sec:Param_Conditions_Domain}
解析対象として,\cref{Fig:ComputationDomain}に示すように,凍結管が配置されている空間内のある一定の深さにおける\qtyproduct{30 x 30}{\meter}の水平断面領域を設定した.
\begin{figure}[tbp]
  \centering
  \includegraphics[width=\linewidth,pagebox=cropbox,clip]{3-2/mesh_base.pdf}
  \caption{計算領域と凍結管配置(凍結管12本,間隔\qty{1.0}{\meter}).円は鉛直凍結管,破線は解析対象断面,網掛け部は凍結イベントの探索範囲を示す.}
  \label{Fig:ComputationDomain}
\end{figure}
凍結管は表面温度一定(\qty{-30}{\degreeCelsius})とし,解析領域の左端から\qty{10}{\meter}の地点に,領域中心軸に対して上下対称となるように配置した.
計算ケースとして,凍結管の本数を2,4,8,12本の4種類とし,それぞれの配置において凍結管間隔を\qtylist{0.6;0.8;1.0}{\meter}の一定間隔とした組み合わせについて検討を行った.
また,地下水流の影響を詳細に評価するため,初期地下水流速については,\qtyrange{0.0}{1.2}{\meter.\day^{-1}}の範囲で計19通りの流速条件を設定した.
これにより,流速の増大に伴う凍結閉塞時間の変化および閉塞限界近傍の非線形な挙動を捉えることを可能とした.
解析に用いた砂質地盤の熱・水理物性値を\cref{Table:sim_properties}に示す.
\begin{table}[t]
  \centering
  \caption{計算に用いた砂質地盤の熱・水理物性値}
  \label{Table:sim_properties}
  \begin{tabular}{llcl@{\hspace{2em}}lcl}
    \toprule
    \multicolumn{4}{c}{Thermal Properties}                  & \multicolumn{3}{c}{Hydraulic Properties}                                                                                                                                \\
    \cmidrule(r){1-4} \cmidrule(l){5-7}
    Parameter                                               & Component                                & Symbol                 & \multicolumn{1}{c}{Value} & Symbol              & Unit                  & \multicolumn{1}{c}{Value} \\
    \midrule
    Porosity                                                & --                                       & $\phi$                 & \num{0.3}                 & $\theta_\mathrm{s}$ & \unit{-}              & \num{0.3}                 \\
    Density                                                 & Soil                                     & $\rho_\mathrm{s}$      & \num{2800}                & $\theta_\mathrm{r}$ & \unit{-}              & \num{0.0}                 \\
    \multicolumn{1}{r}{\unit{\kg.\m^{-3}}}                  & Water                                    & $\rho_\mathrm{w}$      & \num{1000}                & $\alpha_1$          & \unit{\meter^{-1}}    & \num{0.2}                 \\
                                                            & Ice                                      & $\rho_\mathrm{ice}$    & \num{917}                 & $n_1$               & \unit{-}              & \num{1.8}                 \\
    Specific heat                                           & Soil                                     & $c_\mathrm{s}$         & \num{921}                 & $K_\mathrm{s}$      & \unit{\meter.\s^{-1}} & \num{1.97e-7}             \\
    \multicolumn{1}{r}{\unit{\joule.\kg^{-1}.\kelvin^{-1}}} & Water                                    & $c_\mathrm{w}$         & \num{4180}                &                     &                       &                           \\
                                                            & Ice                                      & $c_\mathrm{ice}$       & \num{2100}                &                     &                       &                           \\
    Thermal conductivity                                    & Soil                                     & $\lambda_\mathrm{s}$   & \num{3.78}                &                     &                       &                           \\
    \multicolumn{1}{r}{\unit{\watt.\m^{-1}.\kelvin^{-1}}}   & Water                                    & $\lambda_\mathrm{w}$   & \num{0.6}                 &                     &                       &                           \\
                                                            & Ice                                      & $\lambda_\mathrm{ice}$ & \num{2.2}                 &                     &                       &                           \\
    \bottomrule
  \end{tabular}
\end{table}
また,境界条件は左端が地下水流の流入境界で温度は初期温度に固定し,右端が流出境界となるように設定した.上下端は断熱・不透水境界とし,凍結管表面は\qty{-30}{\degreeCelsius}に固定し,不透水境界とした.
初期条件としては,地盤温度を一様に\qty{18}{\degreeCelsius},間隙水圧を設定した初期流速を満たすラプラス方程式に基づく分布とした.

\subsubsection{凍結の完了について}
\label{Sec:Param_Conditions_FreezeEnd}

凍結閉塞時間の評価にあたっては,解析領域内で形成される凍結領域が連結し,遮水壁として機能する状態を凍結の完了と定義した.具体的には,\qty{0}{\degreeCelsius} 等温線に代表される凍結開始温度のみに基づく単純な温度判定ではなく,透水係数の低下に伴う水理学的な遮水機能に着目した基準を設けた.

一般に,土壌の透水係数が \qty{1.0e-9}{\meter.\second^{-1}} 程度を下回ると水理学的に不透水層(難透水層)とみなされる\parencite{Freeze-1979}.\cref{Fig:T_Si_K_comparison} は,本研究で対象とした砂質土の不凍水特性および透水係数関数\eqref{Eq:Hydraulic_Conductivity_Ice}に基づき,温度低下に伴う凍結率および透水係数の推移を算出した結果である.計算の結果,不透水基準 \qty{1.0e-9}{\meter.\second^{-1}} を満たす理論的な凍結率は $0.765$ であり,これは温度約 \qty{-0.24}{\degreeCelsius} に相当することが示された.

\begin{figure}[tbp]
  \centering
  \includegraphics[width=\linewidth,pagebox=cropbox,clip]{3-2/K-T-Si_validation.pdf}
  \caption{砂質地盤における温度・凍結率・透水係数の関係.透水係数が \qty{1.0e-9}{\meter.\second^{-1}} を下回る理論的な凍結率は $0.765$であり,対応する温度は約 \qty{-0.24}{\degreeCelsius}である.}
  \label{Fig:T_Si_K_comparison}
\end{figure}

本研究では,数値解析上の設計基準として十分な安全率を確保するため,上述の理論値よりも保守的な閾値として凍結率 $\Sice=0.8$を採用した.凍結率が $0.8$,凍結率に対応する温度が約 \qty{-0.3}{\degreeCelsius}に達した際の透水係数は約 \qty{7.8e-10}{\meter.\second^{-1}} であり,不透水基準を十分に下回る遮水性能が担保されていることが確認された.

計算データからの閉塞時間の抽出には,以下の数値的手順を用いた.まず,地下水流の影響による閉塞点の流下(下流側へのシフト)を考慮し,凍結管配置の対称軸である $Y = \qty{15}{\meter}$ を基準として,凍結管ピッチ $L$ の $\pm \qty{10}{\percent}$ に位置する計 3 箇所の $Y$ 軸横断面を解析対象とした.
さらに,計算メッシュの非対称性や要素拡大率の不均一性に起因する数値誤差の影響を低減するため,凍結管が 4 本以上配置されるケースにおいては,中央の凍結管間だけでなく,その上下に隣接する凍結管間についても同様の検証を行った.
各断面の離散的な時間ステップデータに対し,\cref{Fig:BinarySearch} に示す二分探索法を適用することで,解析対象範囲の全域において凍結率が $0.8$ を上回る時刻を特定した.最終的に,すべての解析断面の中で最も遅い時刻を,全体の代表凍結完了時間 $t_\mathrm{freeze}$ と定義した.
\begin{figure}
  \begin{tikzpicture}[
    box/.style={draw, minimum width=0.9cm, minimum height=0.9cm, align=center},
    pointer/.style={-{LaTeX[length=4mm]}, line width=1mm},
    every node/.style={font=\normalsize},
    scale=1.2
    ]

    % 左4つ
    \node (EventLabel) at (-1.2, 0) {\large{\textbf{Event}}};
    \node (EventLabel) at (-1.2, -0.9) {\large{\textbf{Frame}}};
    \foreach \i/\val in {0/\(\times\), 1/\(\times\), 2/\(\times\), 3/\(\times\)} {
        \node[box] (boxL\i) at (\i*1.1, 0) {\Large\val};
      }
    \foreach \i in {0, 1, 2, 3} {
        \node (indexL\i) at (\i*1.1, -0.9) {\normalsize\i};
      }

    \node (dotsL) at (4.4, 0) {\(\cdots\)};
    \node[box] (midbox) at (5.5, 0) {\(\bigcirc\)};
    \node (indexLmid) at (5.5, -0.9) {4320};

    % 右の点々
    \node (dotsR) at (6.6, 0) {\(\cdots\)};

    % 右4つ
    \foreach \i/\val in {0/\(\bigcirc\), 1/\(\bigcirc\), 2/\(\bigcirc\), 3/\(\bigcirc\)} {
        \node[box] (boxR\i) at ({7.7 + \i*1.1}, 0) {\Large\val};
      }
    \foreach \i/\val in {0/8637, 1/8638, 2/8639, 3/8640} {
        \node (indexL\i) at ({7.7 + \i*1.1}, -0.9) {\normalsize\val};
      }

    % ポインタ
    \draw[pointer] (boxL0.north) ++(0,0.2) -- ++(0,0.5) node[above] {Left};
    \draw[pointer] (boxR3.north) ++(0,0.2) -- ++(0,0.5) node[above] {Right};
    \draw[pointer] (midbox.north) ++(0,0.2) -- ++(0,0.5) node[above] {\textbf{Mid}};

    % 補足説明(オプション)
    \node[align=left, anchor=west] at (-1.3, -2.8) {
      \textbf{Binary Search on Event Occurrence:} \\
      Each frame represents whether an event has occurred (\(\bigcirc\)) or not (\(\times\)). \\
      The goal is to find the first position where the event occurs. \\
      Use binary search to minimize the number of checks.
    };
  \end{tikzpicture}
  \caption{二分探索法による凍結完了時間の抽出手順.各フレームはイベント発生(\(\bigcirc\))または非発生(\(\times\))を示す.}
  \label{Fig:BinarySearch}
\end{figure}

\subsubsection{限界流速の同定および挙動分類の手法}
本研究では,シミュレーション結果から限界流速を定量的に抽出するために,数値実験的なアプローチとしてカーブフィッティングを採用した.フィッティング関数の選定は,以下の物理的および数学的な考察に基づいて決定された.

物理的な観点からは,地下水流速がある臨界閾値(限界流速)に達すると,地下水流による対流熱供給と凍結管による熱移動が平衡状態となる.この状態では凍土壁の拡大が停止し,未凍結部を閉塞することなく定常状態に至る.すなわち,完全閉塞に要する時間は,流速が限界値に近づくにつれて無限大に発散する.
数学的な観点からは,この挙動は限界流速において垂直漸近線を持つ関数で記述される必要がある.指数関数,ロジスティック関数,あるいは多項式などの標準的な成長関数は,通常有限の値に収束する水平漸近線を持つか,あるいは特異点を持たないため,時間が無限大へ発散するこの臨界遷移を表現するには不適切である.したがって,観測される単調増加性を維持しつつ,所定の漸近条件を一意に満たす数学形式として,双曲線関数(hyperbolic function)が最適であると結論付けられる.
以上の理由により,本研究では流速 $V$ と凍結完了時間 $t_\mathrm{freeze}$ の関係を以下の双曲線関数で近似した.
\begin{equation}
  t_\mathrm{freeze}\pab{V\,} = \mathcal{F}\pab{V\,} = a + \dfrac{b}{V - V_\mathrm{crit}^\mathrm{n}}
  \label{Eq:hyperbolic_fitting}
\end{equation}
ここで,$V$ は初期流速 \unit{\meter.\day^{-1}},$a, b, V_\mathrm{crit}^\mathrm{n}$ はフィッティングパラメータ \unit{\meter.\day^{-1}} である.\cref{Eq:hyperbolic_fitting} において,垂直漸近線は $V = V_\mathrm{crit}^\mathrm{n}$ に位置する.したがって,このパラメータは数値的に導出された限界流速 $V_\mathrm{crit}^\mathrm{n}$ と定義される.
% 実際,基本的な2本配管のケースにおいて,このパラメータ $c$ の値が数値シミュレーションから直接得られる限界流速と良好に一致することを確認している.
$V_\mathrm{crit}^\mathrm{n}$ が物理的に意義のある限界流速として解釈できるため,本研究ではこの値を,高志の式の枠組みにおける代表長さを評価するための基礎データとして用いる.なお,パラメータ $a$ および $b$ は,それぞれ曲線のオフセットおよび曲率を表す.

さらに,凍結挙動の特徴的なプラトー(Plateau)およびジャンプ(Jump)現象を定量的に解析するため,フィッティング関数の導関数 $\odv{t}/{V}$ に基づいて2つの領域を定義した.本研究では,経験的な閾値を 10 と設定し,以下のように分類を行った.
\begin{description}
  \item[プラトー領域] ($\odv{t}/{V} < 10$): 導関数の値が小さく,流速の変化に対する凍結時間の感度が低い領域.
  \item[ジャンプ領域] ($\odv{t}/{V} \ge 10$): 導関数の値が大きく,わずかな流速の増加が凍結時間の劇的な延長を招く高感度な領域.
\end{description}
この手法は,複雑な数値シミュレーション結果から主要な設計パラメータをロバストに抽出し,簡易な解析理論との橋渡しを行うための有効な手段となる.

\FloatBarrier