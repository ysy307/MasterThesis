\subsection{解析条件および計算ケース}
\label{Sec:Param_Conditions}

解析対象として,凍結管が配置されている空間内のある一定の深さにおける\qtyproduct{30 x 30}{\meter}の水平断面領域を設定した.
凍結管は表面温度一定(\qty{-30}{\degreeCelsius})とし,解析領域の左端(流入境界)から\qty{10}{\meter}の地点に,領域中心軸に対して上下対称となるように配置した.

計算ケースとして,凍結管の本数を2,4,8,12本の4水準とし,それぞれの配置において凍結管間隔を\qtylist{0.6;0.8;1.0}{\meter}の一定間隔とした組み合わせについて検討を行った.
また,地下水流の影響を詳細に評価するため,初期地下水流速については,\qty{0.0}{\meter.\day^{-1}}から\qty{1.2}{\meter.\day^{-1}}の範囲で計19通りの流速条件を設定した.
これにより,流速の増大に伴う凍結閉塞時間の変化および閉塞限界近傍の非線形な挙動を捉えることを可能とした.
その他の解析条件(境界条件の設定等)および物性値については,前章の実験検証と同様とした.

\FloatBarrier