\subsection{結果}
\label{Sec:Param_Results}

\subsubsection{凍土壁閉塞時間の算出}
各計算ケースにおいて,得られた温度分布の経時変化に基づき,凍土壁が閉塞するまでの所要時間を算出した.
ここで「閉塞」とは,隣接する凍結管から成長した凍結領域が連結し,凍結管列の間隙が全て\qty{0}{\degreeCelsius}以下の領域で満たされた状態と定義した.

\subsubsection{双曲線フィッティングによる限界流速の同定}
それぞれの凍結管間隔・本数に対し,初期流速と凍土が閉塞するまでの時間について双曲線フィッティングを行った.
凍結管本数が多い条件では,初期流速が閉塞成立の限界(限界流速)に近づくにつれて閉塞時間が急増する傾向が見られたが,双曲線フィッティングはこうした全体的な挙動を概ね良好に捉えることができた.

特に凍結管が2本の場合,フィッティングされた双曲線の漸近値(閉塞時間が無限大となる流速)は,既存の理論式から算出される理論的な限界流速とよく一致した.
これより,本数値解析結果のフィッティングより得られる漸近値は,理論上の閉塞限界と等価であるとみなせる.
したがって,以下の考察においては,この数値的な漸近値をその計算ケースにおける「実効的な限界流速」として取り扱う.

\FloatBarrier