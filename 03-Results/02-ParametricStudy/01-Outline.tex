\subsection{概要}
\label{Sec:Param_Outline}

前章までは,室内模型実験およびその再現解析を通じて,構築した数値解析モデルの妥当性を検証した.
本章では,検証済みの数値モデルを実規模を想定した広域な解析領域に適用し,地下水流が存在する地盤における凍結挙動についてのパラメータスタディを行う.

地下水流は,上流部より温度の高い液状水を運ぶとともに,凍結管周辺の冷却された間隙水を下流側に移流させ,凍土壁の形成を阻害する主要な要因となる.
特に,複数の凍結管によって形成された凍土が結合し,遮水壁として機能するまでの所要時間は,地下水流速や凍結管の本数,配置間隔といった設計因子に強く依存する.
そこで本章では,解析領域内の初期地下水流速および凍結管の配置条件をパラメータとして変化させ,これらが凍結壁の閉塞時間に及ぼす影響を定量的かつ網羅的に検討する.
また,得られた数値解析結果に基づき,既存の設計理論式に含まれる幾何学的パラメータである代表長さの再評価を試みる.

\FloatBarrier