\section{研究背景}

\subsection{人工地盤凍結工法}

建設技術の高度化と都市空間の深部利用が進む現代において,土壌物理学や地盤工学が直面する課題は複雑化の一途を辿っている.
特に,大深度における高水圧下の帯水層や軟弱地盤における掘削工事,あるいは汚染物質の封じ込めといった極限的な状況において,地盤の強度増加と完全な遮水性を同時に,かつ可逆的に実現する技術への社会的な要求は高まるばかりである.
こうした要請に応えうる唯一無二の技術として,人工地盤凍結工法(Artificial Ground Freezing Method, AGF)が存在する.

人工地盤凍結工法とは、地盤中に一定間隔で埋設した凍結管に、冷凍機で冷却されたブライン(塩化カルシウム水溶液などの不凍液)や液体窒素(\ce{LN2})等の冷媒を循環させ、管周囲の地盤熱を奪うことで土中の間隙水を凍結させる技術である.この工法が他の地盤改良技術(薬液注入工法や高圧噴射攪拌工法など)と決定的に異なる点は、土粒子そのものを化学的に固結させるのではなく、間隙水を固体である氷へと相変化させる物理現象を利用する点にある.この物理的相変化を用いることで、以下の工学的利点が存在する.
\begin{itemize}
  \item 均質かつ高強度な改良体の造成: \\
        土の種類や粒度分布に依存せず、含水さえしていれば凍結が可能である.凍土の強度は温度に依存して一義的に定まり、-10℃程度でコンクリートの約3分の1、-40℃~-162℃(LNGタンク周辺)といった極低温下ではコンクリートと同等以上の圧縮強度を発現する.
  \item 完全な遮水性: \\
        間隙が氷で充填されるため、透水係数は実質的にゼロとなり、完全な不透水層(遮水壁)を形成する.
  \item 可逆性と無公害性: \\
        工事終了後に冷却を停止すれば地盤は融解し、元の状態に戻る.薬液による地下水汚染のリスクがないため、環境保全性が求められる現代の建設プロジェクトにおいて再評価されている.
\end{itemize}
一方で、水が氷になる際の体積膨張(約9\%)や、未凍結部からの水分吸引に伴う凍上(Frost Heave)、および融解時の融解沈下(Thaw Settlement)といった負の側面も併せ持つ.これらの挙動をいかに精緻に予測し制御するかが、本工法の設計・施工における核心的課題となる.