\section{数値解析モデルの妥当性と解析アプローチの合理性}
\label{Sec:Param_Discussion_Validity}
\subsection{現象の再現性}
\label{Sec:Param_Discussion_PhenomenaReproduction}
本研究で構築した凍結融解を含む水・熱移動数値解析ソルバーは,ゼロカーテン効果や定常的な温度場形成を正確に再現しており,潜熱放出,移流・伝導のエネルギーバランスを計算する上で十分な精度を有している.
また,FEMをベースにした本数値解析モデルは,地下水流が存在する条件下での凍結管周囲の温度分布を実験的に観測されたデータと高い精度で一致させており,地下水流の影響を考慮した熱伝達挙動を適切に捉えていることが確認された.
\subsection{逆解析の有効性}
\label{Sec:Param_Discussion_InverseAnalysis}
実験値に含まれる環境外乱をフィルタリングし,物理的整合性の取れた境界条件を同定する手法として,逆解析アプローチは極めて合理的であった.
逆解析によって,不確実性を含むデータを用いても,安定かつ一貫性のある境界条件を特定できた.
さらに,逆解析により得られた境界条件を用いて再度順方向解析を実行した結果,観測データと高い一致度が得られたことから,逆解析手法の有効性が実証された.

\section{代表長さ $l$ を支配する物理的メカニズムの解明}
\label{Sec:Param_Discussion_PhysicalMeaning}
\subsection{水理的堰上げ(dam-up)効果}
代表長さ$l$ は単なる幾何形状の指標ではなく,地下水流に対する水理的抵抗の大きさを定量化するパラメータである.凍結管の本数が増えるほど,あるいは管間隔が広くなるほど,上流側の圧力が上昇し,未凍結部を通過する局所流速が加速されるため,限界流速 $V_\mathrm{crit}$ は低下する.この現象は,流れが障害物によって部分的に遮られることに起因する堰上げ(dam-up)効果として理解できる.

\subsection{幾何学的相互作用}
回帰分析の結果,代表長さ $l$ は凍土壁の総延長 $l_\mathrm{total}$ だけでなく,配管密度($2a/L$)およびそれらの相互作用項に強く依存することが明らかとなった.特に,配管密度の増加は,単に凍土壁の延長を増やす以上に,水理的堰上げ効果を強化することが示された.

\subsection{提案モデルの工学的有用性と実務への適用}
従来,設計者の経験や仮定に委ねられていた $l$ を,具体的な配管レイアウトから算出可能な予測変数へと変換した .これにより,高志の式を用いた簡便な限界流速判定の信頼性が大幅に向上した.また, 配管密度を高めることによる水理的堅牢性の向上は,特に長大な凍土壁において顕著である.この相乗効果を定量化したことは,大規模プロジェクトにおけるコストとリスクの最適化に直結しうる.

\subsection{適用範囲と今後の課題}
本モデルは砂質土などの高透水性地盤を対象としており,地下水流が凍結を阻害する最もクリティカルな条件下での設計を支援する.また,現時点では直線状の配置を前提としているが,大規模な円形立坑などでは直線条件への近似が可能である.今後は,曲率を持つ非線形な形状への拡張が期待される.さらに実務においては,還り温度などの保守的な値を入力することで,熱効率に対する安全率を考慮した設計運用が可能である.