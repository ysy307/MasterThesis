\section{数値解析モデルの妥当性と解析アプローチの合理性}
\label{Sec:Param_Discussion_Validity}
現象の再現性:
構築した熱・水理カップリングソルバーは,ゼロカーテン効果や定常的な温度場形成を正確に再現しており,潜熱放出と対流・伝導のエネルギーバランスを計算する上で十分な精度を有しています .

逆解析の有効性:
実験値に含まれる環境外乱をフィルタリングし,物理的整合性の取れた境界条件を同定する手法として,逆解析アプローチは極めて合理的です.これにより,次章以降の広範なパラメータスタディの信頼性が担保されています.

\section{代表長さ $l$ を支配する物理的メカニズムの解明}
\label{Sec:Param_Discussion_PhysicalMeaning}
水理的堰上げ(dam-up)効果: $l$ は単なる幾何形状の指標ではなく,地下水流に対する水理的抵抗の大きさを定量化するパラメータです .凍結管の本数が増えるほど,あるいは管間隔が広くなるほど,上流側の圧力が上昇し,未凍結部を通過する局所流速が加速されるため,限界流速 $V_{crit}$ は低下します .

幾何学的相互作用: 回帰分析の結果,代表長さ $l$ は凍土壁の総延長 $l_{total}$ だけでなく,配管密度($2a/L$)およびそれらの相互作用項に強く依存することが明らかになりました .

\section{提案モデルの工学的有用性と実務への適用}
曖昧さの排除: 従来,設計者の経験や仮定に委ねられていた $l$ を,具体的な配管レイアウトから算出可能な予測変数へと変換しました .これにより,高志の式を用いた簡便な限界流速判定の信頼性が大幅に向上します .

設計の最適化: 配管密度を高めることによる水理的堅牢性の向上は,特に長大な凍土壁において顕著です .この相乗効果を定量化したことは,大規模プロジェクトにおけるコストとリスクの最適化に直結します .

\section{適用範囲と今後の課題}
土質条件: 本モデルは砂質土などの高透水性地盤を対象としており,地下水流が凍結を阻害する最もクリティカルな条件下での設計を支援します .

幾何学的制約: 現時点では直線状の配置を前提としていますが,大規模な円形立坑などでは直線条件への近似が可能です .今後は,曲率を持つ非線形な形状への拡張が期待されます .

安全側の設計: 実務においては,還り温度などの保守的な値を入力することで,熱効率に対する安全率を考慮した設計運用が可能です .