\section{数値解析手法の妥当性}
\label{Sec:Validity_Analysis}
本研究で構築した水・熱移動数値解析ソルバーは,ゼロカーテン効果や定常的な温度場形成を再現しており,潜熱放出および移流・伝導のエネルギーバランスを計算する上で十分な精度を有している.FEMをベースとした本モデルは,地下水流が存在する条件下での凍結管周囲の温度分布について実験値と整合しており,地下水流の影響を考慮した熱伝達挙動を適切に捉えている.
また,実験値に含まれる環境外乱を排除し,物理的整合性のある境界条件を同定する手法として,逆解析アプローチは有効であった.逆解析により同定された境界条件を用いて再度順方向解析を実行した結果,観測データとの整合性が確認され,解析手法および境界条件設定の合理性が実証された.

\section{代表長さ $l$ の物理的解釈と工学的有用性}
\label{Sec:Physical_Interpretation_Utility}
代表長さ $l$ は単なる幾何形状の指標ではなく,地下水流に対する水理的抵抗の大きさを定量化するパラメータである.凍結管群の存在は流れを阻害する堰上げ(dam-up)効果を生じさせ,上流側の圧力上昇と未凍結部における局所流速の加速を引き起こすことで,限界流速 $V_\mathrm{crit}$ を低下させる.回帰分析の結果,$l$ は凍土壁の総延長 $l_\mathrm{total}$ および配管密度 $2a/L$ ,さらにそれらの相互作用項に依存することが示された.特に配管密度の増加は,単なる延長の増加以上に水理的堰上げ効果を強化する要因となる.

本モデルは,従来設計者の経験や仮定に依存していた $l$ を,具体的な配管レイアウトから算出可能な予測変数へと変換するものである.これにより,高志の式を用いた限界流速判定の信頼性が向上した.本手法は,地下水流の影響が顕著な高透水性地盤を対象としており,大規模な円形立坑などにおいては直線近似を行うことで適用が可能である.実務においては,還り温度などの保守的な値を入力することで,安全率を考慮した設計運用に寄与する.

\section{今後の課題}
\label{Sec:Future_Issues}

本研究では特に,飽和地盤に限った数値解析および議論を進めてきた.しかしながら,実際の現場では不飽和地盤が多く存在し,その場合には毛管現象が水・熱移動に与える影響を考慮する必要がある.今後の課題として,不飽和地盤における水・熱移動モデルの拡張と,凍結管周囲の不飽和領域での挙動解析が挙げられる.
さらに,代表長さ $l$ の物理的解釈を深化させるために,異なる地盤条件や配管レイアウトに対する感度解析を実施することが重要である.これにより,$l$ の一般化と,より広範な設計条件への適用可能性が期待される.