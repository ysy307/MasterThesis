\clearpage
\begin{center}
  {\LARGE\headingfont\bfseries 要旨}
\end{center}

\vspace{1.5\baselineskip}

\normalsize
\setlength{\baselineskip}{22.35pt}

近年,建設技術の高度化および都市空間の深部利用の進展に伴い,大深度高水圧下の帯水層や軟弱地盤における掘削工事,さらには汚染物質の封じ込めといった難条件への対応が必要となっている.
そのため,地盤の強度増加と遮水性を同時に,かつ可逆的に確保可能な地盤改良技術が強く求められている.これらの要求に応え得る有効な工法の一つとして,人工地盤凍結工法が挙げられる.
本工法は,地盤中に一定間隔で設置した凍結管内に,冷媒を循環させることで地盤から熱を奪い,間隙水を凍結させる地盤改良技術である.
土粒子を化学的に固結させる従来工法とは異なり,間隙水の相変化という物理現象を利用する点が特徴である.
これにより,均質で高強度な凍結地盤を形成できること,実質的な完全遮水性,ならびに地盤を原状回復できる可逆性といった利点を有する.
一方で,地下水流れが速い地盤では流れによって凍土の造成が阻害され,その適用可能性には限りがある.
既往の解析的な検討において,閉塞可能な上限流速である限界流速は,凍土壁の規模を示す代表長さによって支配される.
しかし,凍土壁閉塞に影響するのは凍土周辺の局所的な範囲に限られるため,慣例的に凍土壁全長を代表長さとして用いた場合,実際の限界流速は過小評価される.
そのため,実際には造成可能な地下水流速でも凍土が閉塞しないと判定され,凍土壁全長を代表長さとみなして限界流速式を適用し,工法の可否を判定する従来手法には課題が残る.
そこで本研究では開発した凍結・融解を含む飽和二次元熱・水移動数値解析ソルバーを用いた数値解析によって,凍結管配置に基づく代表長さの算出式を構築した.

本研究で開発した数値解析ソルバーは,エネルギー保存則に基づく熱移動支配方程式と間隙水質量保存則に基づく水分支配方程式を連成して構築した.
液状水と氷の相平衡については一般化クラジウス・クラペイロン式を用い,算出された液状水の圧力を水分保持関数に代入することで,ある温度条件下での不凍水分量を計算した.
なお,本解析では解析領域を凍結範囲に対して十分に大きく設定し,境界条件の影響を排除するとともに,相変化に伴う体積変化は無視した.

ここで,解析対象として,凍結管が配置されている空間内のある一定の深さにおける\qtyproduct{30 x 30}{\meter}の断面領域をとり,初期地下水流速を \qtyrange{0.0}{1.2}{\meter.\day^{-1}}の範囲で19通り設定した.
表面温度が\qty{-30}{\degreeCelsius}である凍結管を解析領域の左端から\qty{10}{\meter}の地点に上下対称に2,4,8,12本の管をそれぞれ\qtylist{0.6;0.8;1.0}{m}の一定間隔で設置し,得られた温度分布の経時変化に基づき,凍土壁が閉塞するまでの所要時間を算出した.
まず,それぞれの凍結管間隔・本数に対し,初期流速と凍土が閉塞するまでの時間について双曲線フィッティングを行った.
凍結管本数が多い条件では,初期流速が閉塞成立の限界に近づくにつれて閉塞時間が急増する傾向があったが,双曲線フィッティングは全体的な挙動を概ね良好にとらえた.
特に凍結管が2本の場合,フィッティングされた双曲線の漸近値は理論的な限界流速とよく一致した.
これより,数値解析結果のフィッティングより得られる漸近値は,理論上の閉塞限界と等価であるとみなせる.
したがって,限界流速式で算出される限界流速をこの数値的な漸近値に置き換えることで,経験的パラメータである代表長さを数値計算によって逆算することが可能となる.
このようにして数値的に求めた代表長さと,慣例的に代表長さとして用いられてきた凍土壁全長との比較を行った.
その結果を踏まえ,凍結管間隔,凍結管密度およびそれらの相互作用項を説明変数とする重回帰分析を行い,新たな代表長さの算定式を構築した.これにより,凍結管配置を考慮した代表長さを良好に整理でき,本手法の有効性を示した.
本研究で構築した代表長さの算定式は,凍結管間隔や凍結管密度など,あらかじめ決定可能な幾何学的条件のみを説明変数としている.
このため,従来設計者の判断に委ねられていた代表長さを,客観的かつ定量的に決定することが可能となる.本研究の成果は,解析的設計手法の信頼性および実用性を向上させ,地下水流動を伴う人工地盤凍結工事の設計において,過度に保守的な設計や不要な地盤改良の解消に寄与することが期待される.

