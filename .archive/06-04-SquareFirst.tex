\subsection{四角形一次要素}
\begin{figure}[H]
  \centering
  \begin{tikzpicture}
    % \draw[help lines] (0,0) grid (20,10);
    \node at (0,0) [below left]{$O$};
    \draw[-{latex}, line width=0.4mm] (0,0) -- (7,0) node[below]{$x$};
    \draw[-{latex}, line width=0.4mm] (0,0) -- (0,7) node[left]{$y$};
    \node at (13,4) [below left]{$O$};
    \draw[-{latex}, line width=0.4mm] (9.5,4) --  (16.5,4) node[below]{$\xi$};
    \draw[-{latex}, line width=0.4mm] (13,0.5) --  (13,7.5) node[right]{$\eta$};
    \coordinate[label=below left:\ctext{$1$}]  (p1) at (1,1);
    \coordinate[label=below right:\ctext{$2$}] (p2) at (7,1.5);
    \coordinate[label=above right:\ctext{$3$}] (p3) at (6,6);
    \coordinate[label=above left:\ctext{$4$}]  (p4) at (1.5,5);

    \coordinate[label=below left:\ctext{$1$}]  (p5) at (11,2);
    \coordinate[label=below right:\ctext{$2$}] (p6) at (15,2);
    \coordinate[label=above right:\ctext{$3$}] (p7) at (15,6);
    \coordinate[label=above left:\ctext{$4$}]  (p8) at (11,6);

    \draw (p1)--(p2)--(p3)--(p4)--cycle;
    \draw (p5)--node[midway, below right]{$-1$}(p6)--node[midway, below right]{$+1$}(p7)--node[midway, above right]{$+1$}(p8)--cycle node[midway, below left]{$-1$};

    \node at (4, 8.2) {Cartesian coordinate system};
    \node at (13, 8.2) {Normalized coordinate system};
    \draw[{latex}-{latex}, line width=0.6mm] (7.5,8.2) -- (9,8.2);

  \end{tikzpicture}
  \caption{四角形一次要素における直交座標系と正規化座標系との変換}\label{fig:CtoN-2D4}
\end{figure}
直交座標系での座標での温度を次のように表すとする.
\begin{equation}
  T\pab{x,y}=a + bx + cy + dxy
\end{equation}
最高次数の係数が$x$と$y$の積であるので,この要素は4 noded bilinear iso-parametric elementと呼ばれる.
正規化座標系上で基本ルールに適合した基底関数は次のようになる.
\begin{subequations}
  \begin{align}
    \psi_1\pab{\xi, \eta} & = \dfrac{1}{4}\pab{1-\xi}\pab{1-\eta} \\
    \psi_2\pab{\xi, \eta} & = \dfrac{1}{4}\pab{1+\xi}\pab{1-\eta} \\
    \psi_3\pab{\xi, \eta} & = \dfrac{1}{4}\pab{1+\xi}\pab{1+\eta} \\
    \psi_4\pab{\xi, \eta} & = \dfrac{1}{4}\pab{1-\xi}\pab{1+\eta}
  \end{align}
\end{subequations}
先に.基底関数の和が1であることを確認しておこう.また,基底関数の$\xi$,$\eta$についての偏微分は次のようになる.
\begin{subequations}
  \label{Eq:psi_derivative}
  \begin{align}
    \label{Eq:psi1_xi}
    \pdv{\psi_1}{\xi}  & = -\dfrac{1}{4}\pab{1-\eta} \\
    \label{Eq:psi2_xi}
    \pdv{\psi_2}{\xi}  & = +\dfrac{1}{4}\pab{1-\eta} \\
    \label{Eq:psi3_xi}
    \pdv{\psi_3}{\xi}  & = +\dfrac{1}{4}\pab{1+\eta} \\
    \label{Eq:psi4_xi}
    \pdv{\psi_4}{\xi}  & = -\dfrac{1}{4}\pab{1+\eta} \\
    \label{Eq:psi1_eta}
    \pdv{\psi_1}{\eta} & = -\dfrac{1}{4}\pab{1-\xi}  \\
    \label{Eq:psi2_eta}
    \pdv{\psi_2}{\eta} & = -\dfrac{1}{4}\pab{1+\xi}  \\
    \label{Eq:psi3_eta}
    \pdv{\psi_3}{\eta} & = +\dfrac{1}{4}\pab{1+\xi}  \\
    \label{Eq:psi4_eta}
    \pdv{\psi_4}{\eta} & = +\dfrac{1}{4}\pab{1-\xi}
  \end{align}
\end{subequations}
よって,基底関数を用いて温度を補間すれば,
\begin{equation}
  T\pab{\xi, \eta} = \sum_{i=1}^{4}\psi_i\pab{\xi, \eta}T_i
\end{equation}
同様に,それぞれ$x$,$y$について基底関数を用いて表せば
\begin{align}
  \label{Eq:x-4noded-inter}
  x\pab{\xi, \eta} & = \sum_{i=1}^{4}\psi_i\pab{\xi, \eta}x_i  \\
  \label{Eq:y-4noded-inter}
  y\pab{\xi, \eta} & = \sum_{i=1}^{4}\psi_i\pab{\xi, \eta}y_i
\end{align}
ここで,$\odif{x}\odif{y}$と$\odif{\xi}\odif{\eta}$の関係を考える.まず,$x$,$y$の全微分をとると,
\begin{subequations}
  \label{Eq:xy-diff-perfect}
  \begin{align}
    \odif{x} & = \pdv{x}{\xi}\odif{\xi} + \pdv{x}{\eta}\odif{\eta} \\
    \odif{y} & = \pdv{y}{\xi}\odif{\xi} + \pdv{y}{\eta}\odif{\eta}
  \end{align}
\end{subequations}
\eqref{Eq:xy-diff-perfect}式を行列表示すれば,
\begin{equation}
  \begin{bmatrix}
    \odif{x} \\
    \odif{y}
  \end{bmatrix}
  =
  \begin{bmatrix}
    \displaystyle\pdv{x}{\xi} & \displaystyle\pdv{x}{\eta} \\[3mm]
    \displaystyle\pdv{y}{\xi} & \displaystyle\pdv{y}{\eta}
  \end{bmatrix}
  \begin{bmatrix}
    \odif{\xi} \\
    \odif{\eta}
  \end{bmatrix}
\end{equation}
ここで,Jacobian行列$\mat{J}=\partial x_i/\partial \xi_j$を導入すれば,
\begin{equation}
  \begin{bmatrix}
    \odif{x} \\
    \odif{y}
  \end{bmatrix}
  =\mat{J}^\top
  \begin{bmatrix}
    \odif{\xi} \\
    \odif{\eta}
  \end{bmatrix}
\end{equation}
具体的なJacobian行列は\eqref{Eq:x-4noded-inter},\eqref{Eq:y-4noded-inter}式をそれぞれ,$\xi$,$\eta$について微分すればよいので,
\begin{equation}
  \mat{J}^\top =
  \begin{bmatrix}
    \displaystyle\pdv{x}{\xi} & \displaystyle\pdv{x}{\eta} \\[3mm]
    \displaystyle\pdv{y}{\xi} & \displaystyle\pdv{y}{\eta}
  \end{bmatrix}
  = \begin{bmatrix}
    \displaystyle\sum_{i=1}^{4}\pdv{\psi_i}{\xi}x_i & \displaystyle\sum_{i=1}^{4}\pdv{\psi_i}{\eta}x_i  \\[5mm]
    \displaystyle\sum_{i=1}^{4}\pdv{\psi_i}{\xi}y_i & \displaystyle\sum_{i=1}^{4}\pdv{\psi_i}{\eta}y_i
  \end{bmatrix}
\end{equation}
またJacobian行列の要素について,\eqref{Eq:psi_derivative}式を用いることで次のように書き下せる.
\begin{subequations}
  \label{Eq:Jacobian-4noded-compoments}
  \begin{align}
    \pdv{x}{\xi} = \sum_{i=1}^{4}\pdv{\psi_i}{\xi}x_i & = \pdv{\psi_1}{\xi}x_1 + \pdv{\psi_2}{\xi}x_2 + \pdv{\psi_3}{\xi}x_3 + \pdv{\psi_4}{\xi}x_4\notag                        \\
                                                      & = -\dfrac{1}{4}\pab{1-\eta}x_1 + \dfrac{1}{4}\pab{1-\eta}x_2 + \dfrac{1}{4}\pab{1+\eta}x_3 -\dfrac{1}{4}\pab{1+\eta}x_4
  \end{align}
  \begin{align}
    \pdv{x}{\eta} = \sum_{i=1}^{4}\pdv{\psi_i}{\eta}x_i & = \pdv{\psi_1}{\eta}x_1 + \pdv{\psi_2}{\eta}x_2 + \pdv{\psi_3}{\eta}x_3 + \pdv{\psi_4}{\eta}x_4\notag                \\
                                                        & = -\dfrac{1}{4}\pab{1-\xi}x_1 -\dfrac{1}{4}\pab{1+\xi}x_2 + \dfrac{1}{4}\pab{1+\xi}x_3 + \dfrac{1}{4}\pab{1-\xi}x_4
  \end{align}
  \begin{align}
    \pdv{y}{\xi} = \sum_{i=1}^{4}\pdv{\psi_i}{\xi}y_i & = \pdv{\psi_1}{\xi}y_1 + \pdv{\psi_2}{\xi}y_2 + \pdv{\psi_3}{\xi}y_3 + \pdv{\psi_4}{\xi}y_4\notag                        \\
                                                      & = -\dfrac{1}{4}\pab{1-\eta}y_1 + \dfrac{1}{4}\pab{1-\eta}y_2 + \dfrac{1}{4}\pab{1+\eta}y_3 -\dfrac{1}{4}\pab{1+\eta}y_4
  \end{align}
  \begin{align}
    \pdv{y}{\eta} = \sum_{i=1}^{4}\pdv{\psi_i}{\eta}y_i & = \pdv{\psi_1}{\eta}y_1 + \pdv{\psi_2}{\eta}y_2 + \pdv{\psi_3}{\eta}y_3 + \pdv{\psi_4}{\eta}y_4\notag                \\
                                                        & = -\dfrac{1}{4}\pab{1-\xi}y_1 -\dfrac{1}{4}\pab{1+\xi}y_2 + \dfrac{1}{4}\pab{1+\xi}y_3 + \dfrac{1}{4}\pab{1-\xi}y_4
  \end{align}
\end{subequations}
次に,閉じた曲線内の面積$A$は次のように表すことができる.
\begin{equation}
  \label{Eq:area-4noded-integral}
  A = \iint \odif{x}\odif{y} = \dfrac{1}{2}\oint x\odif{y} - y\odif{x}
\end{equation}
ここで,$\odif{x}$,$\odif{y}$は\eqref{Eq:xy-diff-perfect}式を用いれば,
\begin{align}
  A & = \dfrac{1}{2}\oint x\pab{\pdv{y}{\xi}\odif{\xi} + \pdv{y}{\eta}\odif{\eta}} - y\pab{\pdv{x}{\xi}\odif{\xi} + \pdv{x}{\eta}\odif{\eta}}\notag \\
  \label{Eq:area-4noded}
    & = \dfrac{1}{2}\oint \pab{x\pdv{y}{\eta} - y\pdv{x}{\eta}}\odif{\eta} - \pab{y\pdv{x}{\xi}-x\pdv{y}{\xi}}\odif{\xi}
\end{align}
ここで,正規化座標系でのDivergence定理は
\begin{equation}
  \label{Eq:divergence-theorem-normalized}
  \iint \pab{\pdv{q_{\xi}}{\xi} + \pdv{q_{\eta}}{\eta}}\odif{\xi}\odif{\eta} = \oint q_{\xi}\odif{\eta} - q_{\eta}\odif{\xi}
\end{equation}
ここで,$q_{\xi}$,$q_{\eta}$を\eqref{Eq:area-4noded}を注視して次のように定義すれば,
\begin{subequations}
  \label{Eq:q_xi_q_eta}
  \begin{align}
    q_{\xi}  & = x\pdv{y}{\eta} - y\pdv{x}{\eta} \\
    q_{\eta} & = y\pdv{x}{\xi} - x\pdv{y}{\xi}
  \end{align}
\end{subequations}
よって,\eqref{Eq:area-4noded-integral}式は\eqref{Eq:divergence-theorem-normalized}式と\eqref{Eq:q_xi_q_eta}式を用いれば,
\begin{align}
  A & = \iint \odif{x}\odif{y}\notag                                                                                                                                                                                                           \\
    & = \dfrac{1}{2}\oint x\odif{y} - y\odif{x}\notag                                                                                                                                                                                          \\
    & = \dfrac{1}{2}\oint \pab{x\pdv{y}{\eta} - y\pdv{x}{\eta}}\odif{\eta} - \pab{y\pdv{x}{\xi}-x\pdv{y}{\xi}}\odif{\xi}\notag                                                                                                                 \\
    & = \dfrac{1}{2}\oint q_{\xi}\odif{\eta} - q_{\eta}\odif{\xi}\notag                                                                                                                                                                        \\
    & = \dfrac{1}{2}\iint \pab{\pdv{q_{\xi}}{\xi} + \pdv{q_{\eta}}{\eta}}\odif{\xi}\odif{\eta}\notag                                                                                                                                           \\
    & = \dfrac{1}{2}\iint \bab{\displaystyle\pdv{}{\xi} \pab{\displaystyle x\pdv{y}{\eta} - y\pdv{x}{\eta}} + \displaystyle\pdv{}{\eta} \pab{\displaystyle y\pdv{x}{\xi} - x\pdv{y}{\xi}}} \odif{\xi}\odif{\eta}\notag                         \\
    & = \dfrac{1}{2}\iint \pab{\pdv{x}{\xi}\pdv{y}{\eta}+x\pdv{y}{\xi,\eta}-\pdv{y}{\xi}\pdv{x}{\eta}-y\pdv{x}{\xi,\eta}+\pdv{y}{\eta}\pdv{x}{\xi}+y\pdv{x}{\xi,\eta}-\pdv{x}{\eta}\pdv{y}{\xi}-x\pdv{y}{\xi,\eta}}\odif{\xi}\odif{\eta}\notag \\
    & = \iint \pab{\pdv{x}{\xi}\pdv{y}{\eta}-\pdv{y}{\xi}\pdv{x}{\eta}}\odif{\xi}\odif{\eta}\notag                                                                                                                                             \\
    & =\iint \begin{vmatrix}
               \displaystyle \pdv{x}{\xi}  & \displaystyle\pdv{y}{\xi}  \\[3mm]
               \displaystyle \pdv{x}{\eta} & \displaystyle\pdv{y}{\eta}
             \end{vmatrix}\odif{\xi}\odif{\eta}\notag                                                                                                                                                                 \\
  \label{Eq:area-4noded-final}
    & =\iint \det \mat{J} \odif{\xi}\odif{\eta}
\end{align}
\eqref{Eq:area-4noded-final}式より,直交座標系での$f\pab{x,y}$の積分は,\eqref{Eq:xy-to-xieta-4noded}式に示すように正規化座標系での$f\pab{\xi, \eta}$の積分に変換できる.
\begin{equation}
  \label{Eq:xy-to-xieta-4noded}
  \iint f\pab{x,y}\odif{x}\odif{y} = \iint f\pab{\xi, \eta}\det \mat{J} \odif{\xi}\odif{\eta}
\end{equation}
Gauss-Legendre積分を用いれば,
\begin{align}
  \iint f\pab{x,y}\odif{x}\odif{y} & = \int_{-1}^{+1}\int_{-1}^{+1} f\pab{\xi, \eta}\det \mat{J} \odif{\xi}\odif{\eta}                            \\
                                   & \approx \sum_{i=1}^{N_\mathrm{sample}}\sum_{j=1}^{N_\mathrm{sample}}w_i w_j f\pab{\xi_i, \eta_j}\det \mat{J}
\end{align}

ここで,$\vect{\psi}^\top\vect{\psi}$の重積分を考える.
\begin{align}
  \iint_{\Omega^e}\vect{\psi}^\top\vect{\psi}\odif{\Omega^e}
   & =\iint_{\Omega^e}
  \begin{bmatrix}
    \psi_1^2     & \psi_1\psi_2 & \psi_1\psi_3 & \psi_1\psi_4 \\
    \psi_2\psi_1 & \psi_2^2     & \psi_2\psi_3 & \psi_2\psi_4 \\
    \psi_3\psi_1 & \psi_3\psi_2 & \psi_3^2     & \psi_3\psi_4 \\
    \psi_4\psi_1 & \psi_4\psi_2 & \psi_4\psi_3 & \psi_4^2
  \end{bmatrix}\odif{x}\odif{y}\notag                     \\
   & = \int_{-1}^{+1}\int_{-1}^{+1}
  \begin{bmatrix}
    \psi_1^2     & \psi_1\psi_2 & \psi_1\psi_3 & \psi_1\psi_4 \\
    \psi_2\psi_1 & \psi_2^2     & \psi_2\psi_3 & \psi_2\psi_4 \\
    \psi_3\psi_1 & \psi_3\psi_2 & \psi_3^2     & \psi_3\psi_4 \\
    \psi_4\psi_1 & \psi_4\psi_2 & \psi_4\psi_3 & \psi_4^2
  \end{bmatrix}\det \mat{J} \odif{\xi}\odif{\eta}                     \\
   & \approx \sum_{i=1}^{N_\mathrm{sample}}\sum_{j=1}^{N_\mathrm{sample}}w_i w_j 
  \begin{bmatrix}
    \psi_1^2     & \psi_1\psi_2 & \psi_1\psi_3 & \psi_1\psi_4 \\
    \psi_2\psi_1 & \psi_2^2     & \psi_2\psi_3 & \psi_2\psi_4 \\
    \psi_3\psi_1 & \psi_3\psi_2 & \psi_3^2     & \psi_3\psi_4 \\
    \psi_4\psi_1 & \psi_4\psi_2 & \psi_4\psi_3 & \psi_4^2
  \end{bmatrix}\det \mat{J}
\end{align}
次に基底関数の微分について考える.2次元の直交座標系の基底関数は$x$と$y$の関数であるので,基底関数の全微分は次のようになる.
\begin{equation}
  \label{Eq:psi_i_diff_perfect}
  \odif{\psi_i} = \pdv{\psi_i}{x}\odif{x} + \pdv{\psi_i}{y}\odif{y}
\end{equation}
ここで,\eqref{Eq:psi_i_diff_perfect}式をさらに$\xi$,$\eta$について微分すれば,
\begin{subequations}
  \label{Eq:psi_i_diff_xi_eta}
  \begin{align}
    \label{Eq:psi_i_diff_xi}
    \pdv{\psi_i}{\xi}  & = \pdv{\psi_i}{x}\pdv{x}{\xi} + \pdv{\psi_i}{y}\pdv{y}{\xi}   \\
    \label{Eq:psi_i_diff_eta}
    \pdv{\psi_i}{\eta} & = \pdv{\psi_i}{x}\pdv{x}{\eta} + \pdv{\psi_i}{y}\pdv{y}{\eta}
  \end{align}
\end{subequations}
\eqref{Eq:psi_i_diff_xi_eta}式を行列表示すれば.
\begin{equation}
  \begin{bmatrix}
    \displaystyle\pdv{\psi_i}{\xi} \\[3mm]
    \displaystyle\pdv{\psi_i}{\eta}
  \end{bmatrix}
  =
  \begin{bmatrix}
    \displaystyle\pdv{x}{\xi}  & \displaystyle\pdv{y}{\xi}  \\[3mm]
    \displaystyle\pdv{x}{\eta} & \displaystyle\pdv{y}{\eta}
  \end{bmatrix}
  \begin{bmatrix}
    \displaystyle\pdv{\psi_i}{x} \\[3mm]
    \displaystyle\pdv{\psi_i}{y}
  \end{bmatrix}
  =\mat{J}
  \begin{bmatrix}
    \displaystyle\pdv{\psi_i}{x} \\[3mm]
    \displaystyle\pdv{\psi_i}{y}
  \end{bmatrix}
\end{equation}
よって基底関数の微分は次のようになる.
\begin{equation}
  \label{Eq:psi_i_diff}
  \begin{bmatrix}
    \displaystyle\pdv{\psi_i}{x} \\[3mm]
    \displaystyle\pdv{\psi_i}{y}
  \end{bmatrix}
  =\mat{J}^{-1}
  \begin{bmatrix}
    \displaystyle\pdv{\psi_i}{\xi} \\[3mm]
    \displaystyle\pdv{\psi_i}{\eta}
  \end{bmatrix}
  =\dfrac{1}{\det\mat{J}}
  \begin{bmatrix}
    \displaystyle\pdv{y}{\eta}  & \displaystyle-\pdv{y}{\xi} \\[3mm]
    \displaystyle-\pdv{x}{\eta} & \displaystyle\pdv{x}{\xi}
  \end{bmatrix}
  \begin{bmatrix}
    \displaystyle\pdv{\psi_i}{\xi} \\[3mm]
    \displaystyle\pdv{\psi_i}{\eta}
  \end{bmatrix}
\end{equation}
よって四角形一次要素において,$\mat{B}$は次のようになる.
\begin{equation}
  \mat{B} = \begin{bmatrix}
    \displaystyle\pdv{\psi_1}{x} & \displaystyle\pdv{\psi_2}{x} & \displaystyle\pdv{\psi_3}{x} & \displaystyle\pdv{\psi_4}{x} \\[3mm]
    \displaystyle\pdv{\psi_1}{y} & \displaystyle\pdv{\psi_2}{y} & \displaystyle\pdv{\psi_3}{y} & \displaystyle\pdv{\psi_4}{y}
  \end{bmatrix}
\end{equation}
$\mat{R}$マトリックスがIsotropicであれば,
\begin{equation}
  \label{Eq:R-iso-4noded}
  \begin{split}
     & \iint_{\Omega^e}\mat{B}^\top\mat{R}\mat{B}\odif{\Omega^e} \\
     & =\lambda\iint_{\Omega^e}
    \small
    \begin{bmatrix}
      \displaystyle\pdv{\psi_1}{x}\pdv{\psi_1}{x} + \pdv{\psi_1}{y}\pdv{\psi_1}{y} & \displaystyle\pdv{\psi_1}{x}\pdv{\psi_2}{x} + \pdv{\psi_1}{y}\pdv{\psi_2}{y} & \displaystyle\pdv{\psi_1}{x}\pdv{\psi_3}{x} + \pdv{\psi_1}{y}\pdv{\psi_3}{y} & \displaystyle\pdv{\psi_1}{x}\pdv{\psi_4}{x} + \pdv{\psi_1}{y}\pdv{\psi_4}{y} \\[3mm]
      \displaystyle\pdv{\psi_2}{x}\pdv{\psi_1}{x} + \pdv{\psi_2}{y}\pdv{\psi_1}{y} & \displaystyle\pdv{\psi_2}{x}\pdv{\psi_2}{x} + \pdv{\psi_2}{y}\pdv{\psi_2}{y} & \displaystyle\pdv{\psi_2}{x}\pdv{\psi_3}{x} + \pdv{\psi_2}{y}\pdv{\psi_3}{y} & \displaystyle\pdv{\psi_2}{x}\pdv{\psi_4}{x} + \pdv{\psi_2}{y}\pdv{\psi_4}{y} \\[3mm]
      \displaystyle\pdv{\psi_3}{x}\pdv{\psi_1}{x} + \pdv{\psi_3}{y}\pdv{\psi_1}{y} & \displaystyle\pdv{\psi_3}{x}\pdv{\psi_2}{x} + \pdv{\psi_3}{y}\pdv{\psi_2}{y} & \displaystyle\pdv{\psi_3}{x}\pdv{\psi_3}{x} + \pdv{\psi_3}{y}\pdv{\psi_3}{y} & \displaystyle\pdv{\psi_3}{x}\pdv{\psi_4}{x} + \pdv{\psi_3}{y}\pdv{\psi_4}{y} \\[3mm]
      \displaystyle\pdv{\psi_4}{x}\pdv{\psi_1}{x} + \pdv{\psi_4}{y}\pdv{\psi_1}{y} & \displaystyle\pdv{\psi_4}{x}\pdv{\psi_2}{x} + \pdv{\psi_4}{y}\pdv{\psi_2}{y} & \displaystyle\pdv{\psi_4}{x}\pdv{\psi_3}{x} + \pdv{\psi_4}{y}\pdv{\psi_3}{y} & \displaystyle\pdv{\psi_4}{x}\pdv{\psi_4}{x} + \pdv{\psi_4}{y}\pdv{\psi_4}{y}
    \end{bmatrix}\odif{\Omega^e}
  \end{split}
\end{equation}
ここで,\eqref{Eq:R-iso-4noded}式の行列(1,1)成分を例にして面積分を考える.具体的な要素は展開すれば,
\begin{equation}
  k_{11} = \pdv{\psi_1}{x}\pdv{\psi_1}{x} + \pdv{\psi_1}{y}\pdv{\psi_1}{y}
\end{equation}
\eqref{Eq:psi_i_diff}式を用いれば,
\begin{subequations}
  \label{Eq:k11-4noded}
  \begin{align}
    \pdv{\psi_1}{x} & = \dfrac{1}{\det\mat{J}}\pab{\pdv{y}{\eta}\pdv{\psi_1}{\xi} - \pdv{y}{\xi}\pdv{\psi_1}{\eta}}  \\
    \pdv{\psi_1}{y} & = \dfrac{1}{\det\mat{J}}\pab{-\pdv{x}{\eta}\pdv{\psi_1}{\xi} + \pdv{x}{\xi}\pdv{\psi_1}{\eta}}
  \end{align}
\end{subequations}
ここで,関数$\mathcal{F}_{11}$を次のように定義する.
\begin{equation}
  \label{Eq:func-11-4noded}
  \mathcal{F}_{11}\pab{\xi, \eta} = \pab{\pdv{\psi_1}{x}\pdv{\psi_1}{x} + \pdv{\psi_1}{y}\pdv{\psi_1}{y}}\det\mat{J}
\end{equation}
よって,\eqref{Eq:k11-4noded}式を\eqref{Eq:func-11-4noded}式に代入し,\eqref{Eq:xy-to-xieta-4noded}式の関係を用いれば,
\begin{equation}
  \iint_{\Omega^e}k_{11}\odif{\Omega^e} = \int_{-1}^{+1}\int_{-1}^{+1}\mathcal{F}_{11}\pab{\xi, \eta}\odif{\xi}\odif{\eta}
\end{equation}

一方$\mat{R}$マトリックスがAnisotropicであれば,
\begin{equation}
  \begin{split}
     & \iint_{\Omega^e}\mat{B}^\top\mat{R}\mat{B}\odif{\Omega^e}=\iint_{\Omega^e}\lambda_{ij}\pdv{\psi_k}{x_i}\pdv{\psi_m}{x_j}\odif{\Omega^e} \\
     & =\iint_{\Omega^e}
    \begin{cases}
      \lambda_{xx}\pdv{\psi_1}{x}\pdv{\psi_1}{x}+ \lambda_{xy}\pab{\pdv{\psi_1}{x}\pdv{\psi_1}{y} + \pdv{\psi_1}{y}\pdv{\psi_1}{x}} + \lambda_{yy}\pdv{\psi_1}{y}\pdv{\psi_1}{y} \quad D^e(1,1) \\
      \lambda_{xx}\pdv{\psi_1}{x}\pdv{\psi_2}{x}+ \lambda_{xy}\pab{\pdv{\psi_1}{x}\pdv{\psi_2}{y} + \pdv{\psi_1}{y}\pdv{\psi_2}{x}} + \lambda_{yy}\pdv{\psi_1}{y}\pdv{\psi_2}{y} \quad D^e(1,2) \\
      \lambda_{xx}\pdv{\psi_1}{x}\pdv{\psi_3}{x}+ \lambda_{xy}\pab{\pdv{\psi_1}{x}\pdv{\psi_3}{y} + \pdv{\psi_1}{y}\pdv{\psi_3}{x}} + \lambda_{yy}\pdv{\psi_1}{y}\pdv{\psi_3}{y} \quad D^e(1,3) \\
      \lambda_{xx}\pdv{\psi_1}{x}\pdv{\psi_4}{x}+ \lambda_{xy}\pab{\pdv{\psi_1}{x}\pdv{\psi_4}{y} + \pdv{\psi_1}{y}\pdv{\psi_4}{x}} + \lambda_{yy}\pdv{\psi_1}{y}\pdv{\psi_4}{y} \quad D^e(1,4) \\
      \lambda_{xx}\pdv{\psi_2}{x}\pdv{\psi_1}{x}+ \lambda_{xy}\pab{\pdv{\psi_2}{x}\pdv{\psi_1}{y} + \pdv{\psi_2}{y}\pdv{\psi_1}{x}} + \lambda_{yy}\pdv{\psi_2}{y}\pdv{\psi_1}{y} \quad D^e(2,1) \\
      \lambda_{xx}\pdv{\psi_2}{x}\pdv{\psi_2}{x}+ \lambda_{xy}\pab{\pdv{\psi_2}{x}\pdv{\psi_2}{y} + \pdv{\psi_2}{y}\pdv{\psi_2}{x}} + \lambda_{yy}\pdv{\psi_2}{y}\pdv{\psi_2}{y} \quad D^e(2,2) \\
      \lambda_{xx}\pdv{\psi_2}{x}\pdv{\psi_3}{x}+ \lambda_{xy}\pab{\pdv{\psi_2}{x}\pdv{\psi_3}{y} + \pdv{\psi_2}{y}\pdv{\psi_3}{x}} + \lambda_{yy}\pdv{\psi_2}{y}\pdv{\psi_3}{y} \quad D^e(2,3) \\
      \lambda_{xx}\pdv{\psi_2}{x}\pdv{\psi_4}{x}+ \lambda_{xy}\pab{\pdv{\psi_2}{x}\pdv{\psi_4}{y} + \pdv{\psi_2}{y}\pdv{\psi_4}{x}} + \lambda_{yy}\pdv{\psi_2}{y}\pdv{\psi_4}{y} \quad D^e(2,4) \\
      \lambda_{xx}\pdv{\psi_3}{x}\pdv{\psi_1}{x}+ \lambda_{xy}\pab{\pdv{\psi_3}{x}\pdv{\psi_1}{y} + \pdv{\psi_3}{y}\pdv{\psi_1}{x}} + \lambda_{yy}\pdv{\psi_3}{y}\pdv{\psi_1}{y} \quad D^e(3,1) \\
      \lambda_{xx}\pdv{\psi_3}{x}\pdv{\psi_2}{x}+ \lambda_{xy}\pab{\pdv{\psi_3}{x}\pdv{\psi_2}{y} + \pdv{\psi_3}{y}\pdv{\psi_2}{x}} + \lambda_{yy}\pdv{\psi_3}{y}\pdv{\psi_2}{y} \quad D^e(3,2) \\
      \lambda_{xx}\pdv{\psi_3}{x}\pdv{\psi_3}{x}+ \lambda_{xy}\pab{\pdv{\psi_3}{x}\pdv{\psi_3}{y} + \pdv{\psi_3}{y}\pdv{\psi_3}{x}} + \lambda_{yy}\pdv{\psi_3}{y}\pdv{\psi_3}{y} \quad D^e(3,3) \\
      \lambda_{xx}\pdv{\psi_3}{x}\pdv{\psi_4}{x}+ \lambda_{xy}\pab{\pdv{\psi_3}{x}\pdv{\psi_4}{y} + \pdv{\psi_3}{y}\pdv{\psi_4}{x}} + \lambda_{yy}\pdv{\psi_3}{y}\pdv{\psi_4}{y} \quad D^e(3,4) \\
      \lambda_{xx}\pdv{\psi_4}{x}\pdv{\psi_1}{x}+ \lambda_{xy}\pab{\pdv{\psi_4}{x}\pdv{\psi_1}{y} + \pdv{\psi_4}{y}\pdv{\psi_1}{x}} + \lambda_{yy}\pdv{\psi_4}{y}\pdv{\psi_1}{y} \quad D^e(4,1) \\
      \lambda_{xx}\pdv{\psi_4}{x}\pdv{\psi_2}{x}+ \lambda_{xy}\pab{\pdv{\psi_4}{x}\pdv{\psi_2}{y} + \pdv{\psi_4}{y}\pdv{\psi_2}{x}} + \lambda_{yy}\pdv{\psi_4}{y}\pdv{\psi_2}{y} \quad D^e(4,2) \\
      \lambda_{xx}\pdv{\psi_4}{x}\pdv{\psi_3}{x}+ \lambda_{xy}\pab{\pdv{\psi_4}{x}\pdv{\psi_3}{y} + \pdv{\psi_4}{y}\pdv{\psi_3}{x}} + \lambda_{yy}\pdv{\psi_4}{y}\pdv{\psi_3}{y} \quad D^e(4,3) \\
      \lambda_{xx}\pdv{\psi_4}{x}\pdv{\psi_4}{x}+ \lambda_{xy}\pab{\pdv{\psi_4}{x}\pdv{\psi_4}{y} + \pdv{\psi_4}{y}\pdv{\psi_4}{x}} + \lambda_{yy}\pdv{\psi_4}{y}\pdv{\psi_4}{y} \quad D^e(4,4)
    \end{cases}\odif{\Omega^e}
  \end{split}
\end{equation}
