\section{水分移動支配方程式の離散化}
\eqref{Eq:Continuity_void_strain_chain_8}式を書き換えれば,\eqref{Eq:Hydraulic_GPDE_Simplified}式が得られる.
\begin{Strongbox}{液状水移動のみを考慮した非圧縮性水分移動支配方程式}{Hydraulic_GPDE_Simplified}
  \begin{equation}
    \label{Eq:Hydraulic_GPDE_Simplified}
    \pdv{\Qw}{t}+\frac{\rho_\mathrm{ice}}{\rho_\mathrm{w}}\pdv{\Qice}{t}=\pdv*{\bab{\frac{k_0 k_\mathrm{r}}{\mu_\mathrm{w}}\pab{\pdv{P}{x_j}-\rho_\mathrm{w}\vect{g}}}}{x_i} - S_\mathrm{H}
  \end{equation}
  \small
  $\Qw$:体積含水率$\bab{\si{m^3.m^{-3}}}$,$\Qice$:体積含氷率$\bab{\si{m^3.m^{-3}}}$,$\rho_\mathrm{w}$:間隙水の密度$\bab{\si{kg.m^{-3}}}$,$\rho_\mathrm{ice}$:間隙氷の密度$\bab{\si{kg.m^{-3}}}$,$k_0$:固有透水係数$\bab{\si{m^2}}$,$k_\mathrm{r}$:相対透水係数$\bab{\si{-}}$,$\mu_\mathrm{w}$:水の粘性係数$\bab{\si{Pa.s}}$,$P$:間隙水の圧力$\bab{\si{m}}$,$\vect{g}$:重力ベクトル$\bab{\si{-}}$,$S_\mathrm{H}$:液状水吐き出し項$\bab{\si{m^3.m^{-3}.s^{-1}}}$
\end{Strongbox}
圧力境界条件については,\eqref{Eq:water-BC}式のように表される.
\begin{NoteBox}{圧力境界条件}{Hydraulic_BC}
  \begin{subequations}
    \label{Eq:water-BC}
      1. 圧力既定境界
      \begin{equation}
        \label{Eq:water-BC-Dirichlet}
        P\pab{x_k, t} = {P}_\mathrm{D}\pab{x_k}
        \qquad\text{for} \quad x_k\in\Gamma_\mathrm{HD}    
      \end{equation}
      2. 圧力勾配既定境界
      \begin{equation}
        \label{Eq:water-BC-Neumann}
        \pab{\pdv{P}{x_j}-\rho_\mathrm{w}\vect{g}}n_i = N_\mathrm{H}\pab{x_k}
        \qquad\text{for} \quad x_i\in\Gamma_\mathrm{HN}
      \end{equation}
      3. 水分フラックス境界
      \begin{equation}
        \label{Eq:water-BC-Flux}
        -\frac{k_0 k_\mathrm{r}}{\mu_\mathrm{w}}\pab{\pdv{P}{x_j}-\rho_\mathrm{w}\vect{g}}n_i = Q_\mathrm{HF}\pab{x_k, t}
        \qquad\text{for} \quad x_k\in\Gamma_\mathrm{HF}
      \end{equation}
    \end{subequations}
  \end{NoteBox}
