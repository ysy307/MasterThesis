\subsection{一次要素の境界上の積分}
境界上の積分を考える.まず最初にイメージをつかむため,Fig.~\ref{Fig:tri-boundary}に示す三角形要素における境界上の積分を考える.四角形要素においても同様の手法で計算できる.
\begin{figure}
  \centering
  \begin{tikzpicture}
    % \draw[help lines] (0,0) grid (15,12);
    % \node at (0,0) [below left]{$O$};
    \coordinate (p1) at (3,2);
    \coordinate (p2) at (8,4);
    \coordinate (p3) at (5,7);
    \coordinate (G) at ($(p1)!1/3!(p2)!1/3!(p3)$);
    % 縮小率(0.6倍)
    \def\scale{0.6}
    % 内部の三角形の頂点
    \coordinate (p1s) at ($(G)!\scale!(p1)$);
    \coordinate (p2s) at ($(G)!\scale!(p2)$);
    \coordinate (p3s) at ($(G)!\scale!(p3)$);
    \def\lscale{1.18}
    \coordinate (p1o) at ($(G)!\lscale!(p1)$);
    \coordinate (p2o) at ($(G)!\lscale!(p2)$);
    \coordinate (p3o) at ($(G)!\lscale!(p3)$);

    \coordinate[label=below:$q^{(1)}_1$] (p121) at (3.5, 0.75);
    \coordinate[label=below:$q^{(1)}_2$] (p122) at (8.2, 3.5);
    \coordinate[label=above right:$q^{(2)}_1$] (p231) at (8.2, 4.2);
    \coordinate[label=above :$q^{(2)}_2$] (p232) at (5.5, 7.5);
    \coordinate[label=above left:$q^{(3)}_1$] (p311) at (4.5, 7.2);
    \coordinate[label=left:$q^{(3)}_2$] (p312) at (2.0, 2.4);

    \coordinate (p121o) at (3.75, 0);
    \coordinate (p122o) at (8.8,2.0);
    \coordinate (p231o) at (9.0,5.0);
    \coordinate (p232o) at (6.0,8.0);
    \coordinate (p311o) at (3.0,7.75);
    \coordinate (p312o) at (0.86207,2.7552);
    
  
    \draw[fill, color=red!20!white] (p1)--(p2)--(p122)--(p121) -- cycle;
    \draw[fill, color=blue!20!white] (p2)--(p3)--(p232)--(p231) -- cycle;
    \draw[fill, color=green!20!white] (p3)--(p1)--(p312)--(p311) -- cycle;
    \draw[line width=0.4mm] (p1)--(p2)--(p3)--cycle;
      % 内部の縮小三角形の描画
    \draw[-{Latex}] (p1s)--node[midway, above] {$S$}(p2s);
    \draw[-{Latex}] (p2s)--node[midway, right] {$S$}(p3s);
    \draw[-{Latex}] (p3s)--node[midway, right] {$S$}(p1s);
    \draw (p121)--(p122);
    \draw (p231)--(p232);
    \draw (p311)--(p312);
    \foreach \i in {0,1,2,3,4,5,6,7,8,9} {
      \draw[thick, -{Latex}] ($(p1)!\i/9!(p2)$) -- ($(p121)!\i/9!(p122)$);
      \draw[thick, -{Latex}] ($(p2)!\i/9!(p3)$) -- ($(p231)!\i/9!(p232)$);
      \draw[thick, -{Latex}] ($(p3)!\i/9!(p1)$) -- ($(p311)!\i/9!(p312)$);
    }
    \node at (p1o) {$P_1$};
    \node at (p2o) {$P_2$};
    \node at (p3o) {$P_3$};
    \draw[{Latex}-{Latex}] (p121o)--node[midway, below] {$L^{(1)}$}(p122o);
    \draw[{Latex}-{Latex}] (p231o)--node[midway, above right] {$L^{(2)}$}(p232o);
    \draw[{Latex}-{Latex}] (p311o)--node[midway, left] {$L^{(3)}$}(p312o);
  \end{tikzpicture}
  \caption{三角形要素の境界条件}
  \label{Fig:tri-boundary}
\end{figure}\\
境界上の基底関数$\vect{\psi}_\mathrm{b}$は一次元要素と同様に次のように定義される.
\begin{equation}
  \vect{\psi}_\mathrm{b} = \begin{bmatrix}
    1-\dfrac{S}{L}&\dfrac{S}{L}
  \end{bmatrix}
\end{equation}
ここで,$L$は要素境界の長さ,$S$は境界上の局所座標である.よって,$\vect{\psi}_\mathrm{b}$の積の積分は局所座標系に落とし込むことで計算できる.
\begin{equation}
  \int_{\Gamma} \vect{\psi}_\mathrm{b}^\top \vect{\psi}_\mathrm{b} \odif{\Gamma}=\int_{0}^{L}
  \begin{bmatrix}
    \pab{1-\dfrac{S}{L}}\pab{1-\dfrac{S}{L}} & \pab{1-\dfrac{S}{L}}\dfrac{S}{L}\\[2mm]
    \dfrac{S}{L}\pab{1-\dfrac{S}{L}}   & \dfrac{S}{L}\dfrac{S}{L}
  \end{bmatrix}\odif{S}
  =\dfrac{L}{6}
  \begin{bmatrix}
    2 & 1\\
    1 & 2
  \end{bmatrix}
\end{equation}
また,$\vect{\psi}_\mathrm{b}^\top$の積分は次のようになる.
\begin{equation}
  \int_{\Gamma} \vect{\psi}_\mathrm{b}^\top \odif{\Gamma}=\int_{0}^{L}
  \begin{bmatrix}
    1-\dfrac{S}{L}\\[2mm]
    \dfrac{S}{L}
  \end{bmatrix}\odif{S}=\dfrac{L}{2}
  \begin{bmatrix}
    1\\1
  \end{bmatrix}
\end{equation}
ロビン境界を考える場合には少々工夫が必要となる.
\begin{figure}
  \centering
  \begin{tikzpicture}
    \coordinate (p1) at (2,1);
    \coordinate (p2) at (4,6);
    \coordinate[label=below left:$O$] (O) at (0,-1);

    \draw [-{Latex[length=3mm]}] (O) -- (7,-1) node[right] {$x$};
    \draw [-{Latex[length=3mm]}] (O) -- (0,7) node[above] {$y$};
    \draw [very thick] (p1)--(p2);


    \draw [dashed, -{Latex[length=3mm]}] (p1)--++(1.0,0) node[right] {$\vect{q}_x^1$};
    \draw [dashed, -{Latex[length=3mm]}] (p2)--++(0.7,0) node[right] {$\vect{q}_x^2$};
    \draw [dashed, -{Latex[length=3mm]}] (p1)--++(0,-0.8) node[below] {$\vect{q}_y^1$};
    \draw [dashed, -{Latex[length=3mm]}] (p2)--++(0,-1.0) node[below] {$\vect{q}_y^2$};

    \draw [red, -{Latex[length=3mm]}] (p1)--++(1.137931034482759, -0.4551724137931035) node[below right] {$\vect{q}_1\cdot\vect{n}$};
    \draw [red, -{Latex[length=3mm]}] (p2)--++(0.9482758620689657, -0.37931034482758624) node[below right] {$\vect{q}_2\cdot\vect{n}$};

    \fill [black] (p1) circle (2.2pt) node[below left] {$P_1$};
    \fill [black] (p2) circle (2.2pt) node[above right] {$P_2$};
  \end{tikzpicture}
\end{figure}\\
境界上の節点には,直交座標系における熱フラックス$\vect{q}_x$,$\vect{q}_y$が与えられている.これを局所座標系に変換するために,$\vect{q}_x$,$\vect{q}_y$を$\vect{n}$に射影する.
\begin{subequations}
  \begin{align}
    \vect{Q}_\mathrm{Cn,1}&=\vect{q}_1\cdot\vect{n}=\vect{q}_x^1n_x + \vect{q}_y^1n_y\\
    \vect{Q}_\mathrm{Cn,2}&=\vect{q}_2\cdot\vect{n}=\vect{q}_x^2n_x + \vect{q}_y^2n_y
  \end{align}
\end{subequations}
この射影したベクトル$\vect{Q}_\mathrm{Cn}$を用いて,ロビン境界の節点における積分を計算する.