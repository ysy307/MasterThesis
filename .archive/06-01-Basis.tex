\subsection{基底関数について}
まずは基底関数$\psi$について基本ルールについておさらいしておこう.ここでは,2次元での基底関数について考えるが,1次元や3次元についても引数の変更によって同様の議論ができる.
\begin{NoteBox}{基底関数の基本ルール}{Rule_of_basis_function}
  \begin{equation}
    \label{Eq:shape-func-rule-1}
    \psi_i\pab{x_j, y_j} = \delta_{ij}
  \end{equation}
  \begin{equation}
    \label{Eq:shape-func-rule-2}
    \sum_{i=1}^{N_\mathrm{basis}}\psi_i\pab{x, t} = 1
  \end{equation}
  \small
  $\psi$:基底関数,$N_\mathrm{basis}$:基底関数の数,$\delta_{ij}$:クロネッカーのデルタ
\end{NoteBox}
