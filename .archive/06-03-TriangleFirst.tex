\subsection{三角形一次要素}
基底関数の存在を認めれば,三角形要素内の温度分布は次のようになる.
\begin{equation}
  T\pab{x,y}=\psi_1\pab{x,y}T_1+\psi_2\pab{x,y}T_2+\psi_3\pab{x,y}T_3
\end{equation}
温度の代わりに座標を代入すると,
\begin{subequations}
  \label{Eq:tri-basis}
  \begin{align}
    x &= \psi_1 x_1 + \psi_2 x_2 + \psi_3 x_3\\
    y &= \psi_1 y_1 + \psi_2 y_2 + \psi_3 y_3
  \end{align}
  さらに基底関数の基本ルールより,基底関数の和は常に1であるので,
  \begin{equation}
    \psi_1 + \psi_2 + \psi_3 = 1
  \end{equation}
\end{subequations}
\eqref{Eq:tri-basis}式の関係を用いれば次の連立方程式を得る.
\begin{equation}
  \begin{bmatrix}
    1 & 1 & 1\\
    x_1 & x_2 & x_3\\
    y_1 & y_2 & y_3
  \end{bmatrix}
  \begin{bmatrix}
    \psi_1\pab{x,y}\\ \psi_2\pab{x,y}\\ \psi_3\pab{x,y}
  \end{bmatrix}
  =
  \begin{bmatrix}
    1\\ x\\ y
  \end{bmatrix}
\end{equation}
よって,各要素の面積を$A_m$とすれば,
\begin{align}
  \psi\pab{x,y}=\dfrac{1}{2A_m}\pab{a_{m}+b_{m}x+c_{m}y}
\end{align}
となり,基底関数の各係数はTable~\ref{tab:tri-basis-coef}のようになる.
\begin{table}[]
  \centering
  \caption{基底関数の係数}
  \label{tab:tri-basis-coef}
  \begin{tabular}{@{}cccc@{}}
  \toprule
           & $a$             & $b$       & $c$       \\ \midrule
  $\psi_1$ & $x_2y_3-x_3y_2$ & $y_2-y_3$ & $x_3-x_2$ \\
  $\psi_2$ & $x_3y_1-x_1y_3$ & $y_3-y_1$ & $x_1-x_3$ \\
  $\psi_3$ & $x_1y_2-x_2y_1$ & $y_1-y_2$ & $x_2-x_1$ \\ \bottomrule
  \end{tabular}
\end{table}\\
ここで,$\mat{B}$マトリックスについて考える.$\mat{B}$マトリックスは基底関数の導関数を並べた行列である.
\begin{equation}
  \mat{B}=
  \begin{bmatrix}
    \displaystyle\pdv{\psi_1}{x} & \displaystyle\pdv{\psi_2}{x} & \displaystyle\pdv{\psi_3}{x}\\
    \displaystyle\pdv{\psi_1}{y} & \displaystyle\pdv{\psi_2}{y} & \displaystyle\pdv{\psi_3}{y}
  \end{bmatrix}
  = \dfrac{1}{2A_m}
  \begin{bmatrix}
    b_1 & b_2 & b_3\\
    c_1 & c_2 & c_3
  \end{bmatrix}
\end{equation}
$\mat{R}$マトリックスがIsotropicであれば,要素中の熱伝導率の平均$\bar{\lambda}$を用いると
\begin{equation}
  \iint_{\Omega^e}\mat{B}^\top\mat{R}\mat{B}\odif{\Omega^e}=\dfrac{\bar{\lambda}}{4A_m}
  \begin{bmatrix}
    b_1 b_1 + c_1 c_1 & b_1 b_2 + c_1 c_2 & b_1 b_3 + c_1 c_3\\
    b_2 b_1 + c_2 c_1 & b_2 b_2 + c_2 c_2 & b_2 b_3 + c_2 c_3\\
    b_3 b_1 + c_3 c_1 & b_3 b_2 + c_3 c_2 & b_3 b_3 + c_3 c_3
  \end{bmatrix}
\end{equation}
一方,$\mat{R}$マトリックスがAnisotropicであれば,
\begin{equation}
  \iint_{\Omega^e}\mat{B}^\top\mat{R}\mat{B}\odif{\Omega^e}=\dfrac{1}{4A_m}
  \begin{cases}
    \lambda_{xx}b_1b_1 + \lambda_{xy}(b_1c_1 + c_1b_1) + \lambda_{yy}c_1c_1 \qquad \text{Matrix}(1,1) \\
    \lambda_{xx}b_1b_2 + \lambda_{xy}(b_1c_2 + c_1b_2) + \lambda_{yy}c_1c_2 \qquad \text{Matrix}(1,2) \\
    \lambda_{xx}b_1b_3 + \lambda_{xy}(b_1c_3 + c_1b_3) + \lambda_{yy}c_1c_3 \qquad \text{Matrix}(1,3) \\
    \lambda_{xx}b_2b_1 + \lambda_{xy}(b_2c_1 + c_2b_1) + \lambda_{yy}c_2c_1 \qquad \text{Matrix}(2,1) \\
    \lambda_{xx}b_2b_2 + \lambda_{xy}(b_2c_2 + c_2b_2) + \lambda_{yy}c_2c_2 \qquad \text{Matrix}(2,2) \\
    \lambda_{xx}b_2b_3 + \lambda_{xy}(b_2c_3 + c_2b_3) + \lambda_{yy}c_2c_3 \qquad \text{Matrix}(2,3) \\
    \lambda_{xx}b_3b_1 + \lambda_{xy}(b_3c_1 + c_3b_1) + \lambda_{yy}c_3c_1 \qquad \text{Matrix}(3,1) \\
    \lambda_{xx}b_3b_2 + \lambda_{xy}(b_3c_2 + c_3b_2) + \lambda_{yy}c_3c_2 \qquad \text{Matrix}(3,2) \\
    \lambda_{xx}b_3b_3 + \lambda_{xy}(b_3c_3 + c_3b_3) + \lambda_{yy}c_3c_3 \qquad \text{Matrix}(3,3)
  \end{cases}
\end{equation}
ここで領域$\Omega^e$での基底関数の面積分を考えるためFig.~\ref{Fig:AreaCood}に示す面積座標を導入する.
\begin{figure}[H]
  \centering
  \begin{tikzpicture}[scale=1.5]
    \coordinate[label=below left:$O$](O)at(0,0);
    \coordinate(XS)at(-0.5,0);
    \coordinate(XL)at(8,0);
    \coordinate(YS)at(0,-0.5);
    \coordinate(YL)at(0,5);
    \draw[semithick,-{Latex}](XS)--(XL)node[below]{$x$};
    \draw[semithick,-{Latex}](YS)--(YL)node[left]{$y$};
    \coordinate[label=below:$P_1\pab{x_1,y_1}$](P1)at(0.85,0.75);
    \coordinate[label=right:$P_2\pab{x_2,y_2}$](P2)at(6,1.5);
    \coordinate[label=above:$P_3\pab{x_3,y_3}$](P3)at(3,4);
    \coordinate(P231)at(4,3.166666);
    \coordinate(P232)at(4.8,2.5);
    \coordinate(P)at(3.5,2.5);

    \draw(P1)--(P2)--(P3)--cycle;
    \draw(P)--(P1);
    \draw(P)--(P2);
    \draw(P)--(P3);

    \coordinate[label=below:$A_{12}^e$](G1)at($(P)!1/3!(P1)!1/3!(P2)$);
    \coordinate[label=above:$A_{23}^e$](G2)at($(P)!1/3!(P2)!1/3!(P3)$);
    \coordinate[label=above:$A_{31}^e$](G3)at($(P)!1/3!(P3)!1/3!(P1)$);

    \node[right, fill=white, inner sep=2pt, xshift=6pt, yshift=0pt] at (P) {$P{(\xi_1,\xi_2,\xi_3)}$};
    \draw (P1) node[circle, inner sep=2pt, fill=black]{};
    \draw (P2) node[circle, inner sep=2pt, fill=black]{};
    \draw (P3) node[circle, inner sep=2pt, fill=black]{};
    \draw (P) node[circle, inner sep=2pt, fill=black]{};
  \end{tikzpicture}
  \caption{面積座標の概念図}
  \label{Fig:AreaCood}
\end{figure}
$\triangle P_1 P_2 P_3 $は3つの三角形$\triangle P P_1 P_2 $,$\triangle P P_2 P_3 $,$\triangle P P_3 P_1 $に分割できる.これらを全体の三角形の面積$A^e$で割ったものを順に面積座標$\xi_1$,$\xi_2$,$\xi_3$と定義する.ただしそれぞれの三角形の面積は$A_{12}^{e}$,$A_{23}^{e}$,$A_{31}^{e}$とする.
\begin{subequations}
  \label{Eq:DefACoord}
  \begin{align}
    \xi_1 &= \frac{A_{12}^{e}}{A^e}\\
    \xi_2 &= \frac{A_{23}^{e}}{A^e}\\
    \xi_3 &= \frac{A_{31}^{e}}{A^e}
  \end{align}
\end{subequations}
また,$\xi$は次の関係があるのは自明である.
\begin{equation}
  \label{Eq:Relxi}
  \xi_1+\xi_2+\xi_3=1
\end{equation}
つまり,$\xi_1$と$\xi_2$が決定すると\eqref{Eq:Relxi}式より,自動的に$\xi_3=1-\xi_1-\xi_2$と決定でき$\xi_1$と$\xi_2$のみが独立変数となる.よって$0\leq\xi_2\leq 1-\xi_1$の条件下の領域$\Omega^e$内では$\forall P(x,y)\rightarrow P(\xi_1,\xi_2,1-\xi_1-\xi_2)$と表せる.ここで分割されたそれぞれの三角形の面積について考えることで,面積座標について考える.
\begin{subequations}
  \label{Eq:AreaSp}
  \begin{align}
    \xi_1 &= \frac{A_{12}^{e}}{A^e}\notag\\
    &=\frac{1}{2A^e}\bab{x\pab{y_2-y_3}+x_2\pab{y_3-y}+x_3\pab{y-y_2}}\notag\\
    &=\frac{1}{2A^e}\bab{\pab{x_2 y_3-x_3 y_2}+\pab{y_2-y_3}x+\pab{x_3-x_2}y}\notag\\
    &=\frac{1}{2A^e}\pab{a_1+b_1 x+c_1 y}\notag\\
    \label{Eq:AreaSp1j}
    &=\psi_1
  \end{align}
  同様に
  \begin{align}
    \xi_2 &= \frac{A_{23}^{e}}{A^e}\notag\\
    &=\frac{1}{2A^e}\bab{x\pab{y_3-y_1}+x_3\pab{y_1-y}+x_1\pab{y-y_3}}\notag\\
    &=\frac{1}{2A^e}\bab{\pab{x_3 y_1-x_1 y_3}+\pab{y_3-y_1}x+\pab{x_1-x_3}y}\notag\\
    &=\frac{1}{2A^e}\pab{a_2+b_2 x+c_2 y}\notag\\
    \label{Eq:AreaSp2j}
    &=\psi_2
  \end{align}
  \begin{align}
    \xi_3 &= \frac{A_{31}^{e}}{A^e}\notag\\
    &=\frac{1}{2A^e}\bab{x\pab{y_1-y_2}+x_1\pab{y_2-y}+x_2\pab{y-y_1}}\notag\\
    &=\frac{1}{2A^e}\bab{\pab{x_1 y_2-x_2 y_1}+\pab{y_1-y_2}x+\pab{x_2-x_1}y}\notag\\
    &=\frac{1}{2A^e}\pab{a_3+b_3 x+c_3 y}\notag\\
    \label{Eq:AreaSp3j}
    &=\psi_3
  \end{align}
\end{subequations}
よって三角形一次要素の基底関数は面積座標と一致する.そのため,面積座標上での面積分をすることによって,基底関数の積分を求めることができる.ここで\eqref{Eq:tri-basis-integral}式で示すように階乗を用いることで,面積座標での積分を解析的に求めることができる.
\begin{equation}
  \label{Eq:tri-basis-integral}
  \iint_{\Omega^e}\xi_1^p~\xi_2^q~\xi_3^r\odif{\Omega^e}=\dfrac{p!q!r!}{\pab{p+q+r+2}!}2A_m
\end{equation}
よって,\eqref{Eq:tri-basis-integral}式を用いれば,基底関数の積の積分は次のようになる.
\begin{equation}
  \iint_{\Omega^e}\vect{\psi}^\top\vect{\psi}\odif{\Omega^e}=\iint_{\Omega^e}
  \begin{bmatrix}
    \psi_1^2 & \psi_1\psi_2 & \psi_1\psi_3\\
    \psi_2\psi_1 & \psi_2^2 & \psi_2\psi_3\\
    \psi_3\psi_1 & \psi_3\psi_2 & \psi_3^2
  \end{bmatrix}\odif{\Omega^e}
  =\dfrac{A_m}{12}
  \begin{bmatrix}
    2 & 1 & 1\\
    1 & 2 & 1\\
    1 & 1 & 2
  \end{bmatrix}
\end{equation}
したがって,
\begin{equation}
  \iint_{\Omega^e}C_\mathrm{w}\vect{Q}_\mathrm{w}\vect{\psi}^\top\vect{\psi}\odif{\Omega^e}=C_\mathrm{w}\vect{Q}_\mathrm{w}\dfrac{A_m}{12}
  \begin{bmatrix}
    2 & 1 & 1\\
    1 & 2 & 1\\
    1 & 1 & 2
  \end{bmatrix}
\end{equation}
また,
\begin{align}
  &\iint_{\Omega^e} C_\mathrm{w} \vect{\psi}^\top \vect{q}^\top \mat{B} \odif{\Omega^e}\notag
  \\&=\dfrac{C_\mathrm{w}}{6}
  \begin{bmatrix}
    b_1q_x + c_1q_y & b_2q_x + c_2q_y & b_3q_x + c_3q_y\\
    b_1q_x + c_1q_y & b_2q_x + c_2q_y & b_3q_x + c_3q_y\\
    b_1q_x + c_1q_y & b_2q_x + c_2q_y & b_3q_x + c_3q_y
  \end{bmatrix}
\end{align}
また,
\begin{equation}
  \iint_{\Omega^e}C_\mathrm{a}\vect{\psi}^\top\vect{\psi}\odif{\Omega^e}=C_\mathrm{a}\dfrac{A_m}{12}
  \begin{bmatrix}
    2 & 1 & 1\\
    1 & 2 & 1\\
    1 & 1 & 2
  \end{bmatrix}
\end{equation}
このマトリックスにlumpingを適用すれば,
\begin{equation}
  \iint_{\Omega^e}C_\mathrm{a}\vect{\psi}^\top\vect{\psi}\odif{\Omega^e}=C_\mathrm{a}\dfrac{A_m}{3}
  \begin{bmatrix}
    1 & 0 & 0\\
    0 & 1 & 0\\
    0 & 0 & 1
  \end{bmatrix}
\end{equation}