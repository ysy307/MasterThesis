\subsection{GCC式の微分形}
定式化に伴い飽和度$\Sw$の微分が必要になるため,GCC式$\mathcal{F}_\mathrm{GCC}$と水分保持関数$\mathcal{F}_\mathrm{WRF}$を微分形に変形する.
\begin{align}
  \Sw &= \mathcal{F}_\mathrm{WRF}\pab{P_\mathrm{ice}-\mathcal{F}_\mathrm{GCC}\pab{T^{\ast}, P, P_\mathrm{ice}}}\notag\\
  \label{Eq:Sw_form}
  &= \mathcal{F}_\mathrm{WRF}\pab{P_\mathrm{iw}}
\end{align}
GCC式の全微分は,温度微分と圧力微分の和として表される.
\begin{equation}
  \label{Eq:GCC_differential}
  \odif{\Sw} = \pdv{\Sw}{P} \odif{P} +\pdv{\Sw}{P_\mathrm{ice}} \odif{P_\mathrm{ice}} + \pdv{\Sw}{T^{\ast}} \odif{T^{\ast}} 
\end{equation}
\eqref{Eq:Sw_form}式を考慮すれば,\eqref{Eq:GCC_differential}式は次のように表される.
\begin{align}
  \odif{\Sw} &= \pdv{\mathcal{F}_\mathrm{WRF}}{P_\mathrm{iw}}\pdv{P_\mathrm{iw}}{P} \odif{P} + \pdv{\mathcal{F}_\mathrm{WRF}}{P_\mathrm{iw}}\pdv{P_\mathrm{iw}}{P_\mathrm{ice}} \odif{P_\mathrm{ice}} + \pdv{\mathcal{F}_\mathrm{WRF}}{P_\mathrm{iw}}\pdv{P_\mathrm{iw}}{T^{\ast}} \odif{T^{\ast}}\notag\\
  \label{Eq:GCC_differential_form}
  &= \pdv{\mathcal{F}_\mathrm{WRF}}{P_\mathrm{iw}}\pab{\frac{\rho_\mathrm{ice}}{\rho_\mathrm{w}}-1} \odif{P} + \pdv{\mathcal{F}_\mathrm{WRF}}{P_\mathrm{iw}}\odif{P_\mathrm{ice}} - \pdv{\mathcal{F}_\mathrm{WRF}}{P_\mathrm{iw}}\frac{L_\mathrm{f} \rho_\mathrm{ice}}{T^{{\ast}}} \odif{T^{\ast}}
\end{align}