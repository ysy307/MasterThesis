\section{熱移動支配方程式の離散化}
\eqref{Eq:First_Law_Thermo_Closed_Time_8}を書き換えれば,\eqref{Eq:Thermal_GPDE_Simplified}が得られる.
\begin{Strongbox}{液状水移動のみを考慮した非圧縮性熱移動支配方程式}{Thermal_NonVapor_NonCompressible}
  \begin{equation}
    \label{Eq:Thermal_GPDE_Simplified}
    \pab{C_\mathrm{p} - L_\mathrm{f}\rho_\mathrm{ice}\pdv{\Qice}{T}}\pdv{T}{t} = \pdv*{\pab{\lambda_{ij}\pdv{T}{x_j}}}{x_i} - C_\mathrm{w}\pdv{q_i T}{x_i} + S_\mathrm{T}
  \end{equation}\\[-4mm]
  \small
  $C_\mathrm{p}$:土壌全体の体積熱容量$\bab[big]{\si{J.m^{-3}.K^{-1}}}$,$L_\mathrm{f}$:水の液固相間の潜熱$\bab[big]{\si{J.kg^{-1}}}$,$\rho_\mathrm{ice}$:氷の密度$\bab[big]{\si{kg.m^{-3}}}$,$\Qice$:氷の体積分率$\bab[big]{\si{m^3.m^{-3}}}$,$\lambda_{ij}$:土壌全体の熱伝導率$\bab[big]{\si{W.m^{-1}.K^{-1}}}$,$C_\mathrm{w}$:水の体積熱容量$\bab[big]{\si{J.m^{-3}.K^{-1}}}$,$q_i$:液状水フラックス$\bab[big]{\si{m.s^{-1}}}$,$S_\mathrm{T}$:熱の吸い込み・湧き出し$\bab[big]{\si{J.m^{-3}.s^{-1}}}$
\end{Strongbox}\noindent
境界条件は次のように設定するものとする.
\begin{NoteBox}{温度境界条件}{Thermal_BC}
  \begin{subequations}
    \label{Eq:heat-BC}
    1. 温度既定境界
    \begin{equation}
      \label{Eq:heat-BC-Dirichlet}
      T\pab{x_k, t} = T_\mathrm{D}\pab{x_k} \qquad\text{for} \quad x_k\in\Gamma_\mathrm{HD}
    \end{equation}
    2. 温度勾配既定境界
    \begin{equation}
      \label{Eq:heat-BC-Neumann}
      \pdv{T}{x_j}\pab{x_k, t}n_i  = N_\mathrm{T}\pab{x_k} \qquad\text{for} \quad x_i\in\Gamma_\mathrm{HN}
    \end{equation}
    3. 熱フラックス境界
    \begin{equation}
      \label{Eq:heat-BC-Flux}
      Q_\mathrm{HF}\pab{x_k, t} = -\lambda_{ij}\pdv{T\pab{x_k}}{x_j}n_i \qquad\text{for} \quad x_k\in\Gamma_\mathrm{HF}
    \end{equation}
    4. ロビン境界
    \begin{equation}
      \label{Eq:heat-BC-Robin}
      Q_\mathrm{HC}\pab{x_k, t} = -\lambda_{ij}\pdv{T\pab{x_i}}{x_j}n_i + T\pab{x_k}C_\mathrm{w}q_i n_i \qquad\text{for} \quad x_k\in\Gamma_\mathrm{HC}
    \end{equation}
    5. 熱伝達境界
    \begin{equation}
      \label{Eq:heat-BC-Convection}
      Q_\mathrm{HH}\pab{x_k, t} = h\bab{T\pab{x_k, t}-T_\mathrm{env}\pab{x_k, t}} \qquad\text{for} \quad x_k\in\Gamma_\mathrm{HH}
    \end{equation}
    6. 熱放射境界
    \begin{align}
      \label{Eq:heat-BC-Radiation}
      Q_\mathrm{HR}\pab{x_k, t} & = \varepsilon\sigma F\bab{T\pab{x_k, t}^4-T_\mathrm{r}\pab{x_k, t}^4}\notag                                     \\
                                & =\alpha_\mathrm{HR}\bab{T\pab{x_k, t}-T_\mathrm{r}\pab{x_k, t}} \qquad\text{for} \quad x_k\in\Gamma_\mathrm{HR} \\
      \alpha_\mathrm{HR}        & = \varepsilon\sigma F\pab[big]{T+T_\mathrm{r}}\pab{T^2+T_\mathrm{r}^2}\notag
    \end{align}
  \end{subequations}
\end{NoteBox}
ここで,見かけの熱容量$C_\mathrm{a}\bab{\si{J.m^{-3}.K^{-1}}}$を次のように表す.
\begin{equation}
  C_\mathrm{a} = C_\mathrm{p}+L_\mathrm{f}\rho_\mathrm{ice}\pdv{\Qice}{T}
\end{equation}
支配方程式\ref{Gov:Thermal_NonVapor_NonCompressible}を有限要素法で離散化することを考える.重み関数を$\chi$とすれば弱形式は
\begin{equation}
  \iint_{\Omega} \bab{C_\mathrm{a}\pdv{T}{t}-\pdv*{\pab{\lambda_{ij}\pdv{T}{x_j}}}{x_i}+C_\mathrm{w}\pdv{q_i T}{x_i}-S_\mathrm{T}}\chi\odif{\Omega}=0
\end{equation}
ここで,重み関数は任意な関数であるが,以下の条件を満たしている必要がある.
\begin{NoteBox}{重み関数$\chi$の条件}{Weight_Function_Condition}
  \begin{itemize}
    \item 重み関数$\chi$は,有限要素の節点で連続である.
    \item 重み関数$\chi$は,Dirichlet境界条件において,値がゼロである.
    \item 重み関数$\chi$は,Neumann境界条件において,値が1で微分値がゼロである.
    \item 重み関数$\chi$は無次元である.
    \item 重み関数$\chi$は,互いに独立関数でなくてはならない.
  \end{itemize}
\end{NoteBox}
ガウスの発散定理より
\begin{equation}
  \label{Eq:heat-weak}
  \begin{split}
    \iint_{\Omega} C_\mathrm{a} \pdv{T}{t} \chi \odif{\Omega}+\iint_{\Omega}\lambda_{ij}\pdv{T}{x_j}\pdv{\chi}{x_i} \odif{\Omega} - \oint_{\Gamma}\lambda_{ij}\pdv{T}{x_j}n_i\chi\odif{\Gamma} \\
    +\iint_{\Omega}C_\mathrm{w}\pdv{q_i}{x_i} T \chi\odif{\Omega} + \iint_{\Omega}C_\mathrm{w}q_i \pdv{T}{x_i} \chi\odif{\Omega} - \iint_{\Omega}S_\mathrm{T}\chi\odif{\Omega} = 0
  \end{split}
\end{equation}
\eqref{Eq:heat-weak}の第3項について境界条件を差し込むと,
\begin{equation}
  \begin{split}
    \oint_{\Gamma}-\lambda_{ij}\pdv{T}{x_j}n_i\chi\odif{\Gamma}
     & = 
    -\int_{\Gamma_\mathrm{HN}}\lambda_{ij}N_\mathrm{T}\chi\odif{\Gamma_\mathrm{HN}}
    + \int_{\Gamma_\mathrm{HF}}Q_\mathrm{HF}\chi\odif{\Gamma_\mathrm{HF}}
    + \int_{\Gamma_\mathrm{HC}}\pab{Q_\mathrm{HC} - C_\mathrm{w}q_i n_i T}\chi\odif{\Gamma_\mathrm{HC}} \\                                                      
     & + \int_{\Gamma_\mathrm{HH}}Q_\mathrm{HH}\chi\odif{\Gamma_\mathrm{HH}}
    + \int_{\Gamma_\mathrm{HR}}Q_\mathrm{HR}\chi\odif{\Gamma_\mathrm{HR}}
  \end{split}
\end{equation}
よって,\eqref{Eq:heat-weak}は
\begin{equation}
  \label{Eq:heat-weak2}
  \begin{split}
     & \iint_{\Omega} C_\mathrm{a} \pdv{T}{t} \chi \odif{\Omega}
    + \iint_{\Omega} \lambda_{ij}\pdv{T}{x_j}\pdv{\chi}{x_i} \odif{\Omega}
    + \iint_{\Omega} C_\mathrm{w}\pdv{q_i}{x_i} T \chi\odif{\Omega}                                        \\                                                    
     & + \iint_{\Omega} C_\mathrm{w}q_i \pdv{T}{x_i} \chi\odif{\Omega}
    - \iint_{\Omega} S_\mathrm{T}\chi\odif{\Omega}
    - \int_{\Gamma_\mathrm{HN}} \lambda_{ij}N_\mathrm{T}\chi\odif{\Gamma_\mathrm{HN}}
    + \int_{\Gamma_\mathrm{HF}} Q_\mathrm{HF}\chi\odif{\Gamma_\mathrm{HF}}                                 \\                                                       
     & + \int_{\Gamma_\mathrm{HC}} \pab{Q_\mathrm{HC} -C_\mathrm{w}q_i n_i T}\chi\odif{\Gamma_\mathrm{HC}}
    + \int_{\Gamma_\mathrm{HH}} Q_\mathrm{HH}\chi\odif{\Gamma_\mathrm{HH}}
    + \int_{\Gamma_\mathrm{HR}} Q_\mathrm{HR}\chi\odif{\Gamma_\mathrm{HR}}= 0
  \end{split}
\end{equation}
今,考えている領域を有限要素に分割し,それらの節点における温度$T_m$に基底関数$\psi_m$を乗じることによって,有限要素内の任意点でのそれらの値を表すことができるとすれば,
\begin{equation}
  \hat{T} = \sum_{m=1}^{N_n}\psi_m T_m = \vect{\psi}_\mathrm{T} \vect{T}
\end{equation}
ただし,$N_n$は総節点数,$\vect{\psi}_\mathrm{T}$は基底関数ベクトル,$\vect{T}$は温度ベクトルである.また,重み関数が基底関数で表現できるとすれば,
\begin{equation}
  \chi =
  \begin{cases}
    \vect{\psi}_\mathrm{T}\vect{\chi}\qquad              & \text{On elements}   \\
    \vect{\psi}_\mathrm{Tb}\vect{\chi}_\mathrm{Tb}\qquad & \text{On boundaries}
  \end{cases}
\end{equation}
ここで,$\psi_\mathrm{Tb}$は境界節点での基底関数である.
基底関数を偏微分した行列を$\mat{B}$マトリックス,熱伝導率行列を$\mat{R}$マトリックスとすれば,Fourier則より熱フラックス$\vect{q}$は次のようになる.
\begin{equation}
  \vect{q} = - \mat{R}\mat{B}\vect{T}
\end{equation}
境界では,次のように境界上の基底関数を用いて熱フラックスを表す.
\begin{equation}
  \vect{Q} = \vect{\psi}_\mathrm{Tb}\vect{Q}_\mathrm{Tb}
\end{equation}
ここで,$\vect{Q}_\mathrm{Tb}$は各境界条件で境界節点に与えられる熱フラックスである.
よって,
\begin{equation}
  \label{Eq:heat-weak3}
  \begin{split}
     & 
    \sum_{e=1}^{N_e}\iint_{\Omega^e} C_\mathrm{a} \vect{\chi}^\top \vect{\psi}_\mathrm{T}^\top \vect{\psi}_\mathrm{T}\pdv{\vect{T}}{t}\odif{\Omega^e}
    + \sum_{e=1}^{N_e}\iint_{\Omega^e}\vect{\chi}^\top \mat{B}^\top \mat{R} \mat{B} \vect{T}\odif{\Omega^e}
    + \sum_{e=1}^{N_e}\iint_{\Omega^e} C_\mathrm{w} \vect{Q}_\mathrm{w} \vect{\chi}^\top \vect{\psi}_\mathrm{T}^\top \vect{\psi}_\mathrm{T}\vect{T}\odif{\Omega^e}                                             \\                                                        
     & + \sum_{e=1}^{N_e}\iint_{\Omega^e} C_\mathrm{w} \vect{\chi}^\top \vect{\psi}_\mathrm{T}^\top \vect{q}^\top \mat{B} \vect{T} \odif{\Omega^e}
    - \sum_{e=1}^{N_e}\iint_{\Omega^e} \vect{\chi}^\top \vect{\psi}_\mathrm{T}^\top\vect{\psi}_\mathrm{T} \vect{S}_\mathrm{T}\odif{\Omega^e}
    - \sum_{s=1}^{N_{\Gamma_\mathrm{TN}}}\int_{\Gamma_\mathrm{TN}} \vect{\chi}^\top \vect{\psi}_\mathrm{Tb}^\top \lambda\vect{\psi}_\mathrm{Tb} \vect{N}_\mathrm{T} \odif{\Gamma_\mathrm{TN}}                  \\                                                        
     & + \sum_{s=1}^{N_{\Gamma_\mathrm{TF}}}\int_{\Gamma_\mathrm{TF}} \vect{\chi}^\top \vect{\psi}_\mathrm{Tb}^\top \vect{\psi}_\mathrm{Tb} \vect{Q}_\mathrm{TF} \odif{\Gamma_\mathrm{TF}}
    + \sum_{s=1}^{N_{\Gamma_\mathrm{TC}}}\int_{\Gamma_\mathrm{TC}} \vect{\chi}^\top \vect{\psi}_\mathrm{Tb}^\top \vect{\psi}_\mathrm{Tb} \vect{Q}_\mathrm{TC} \odif{\Gamma_\mathrm{TC}}
    - \sum_{s=1}^{N_{\Gamma_\mathrm{TC}}}\int_{\Gamma_\mathrm{TC}} \vect{\chi}^\top \vect{\psi}_\mathrm{Tb}^\top C_\mathrm{w} \vect{q}^\top \vect{n} \vect{\psi}_\mathrm{Tb}\vect{T} \odif{\Gamma_\mathrm{TC}} \\                                                        
     & + \sum_{s=1}^{N_{\Gamma_\mathrm{TH}}}\int_{\Gamma_\mathrm{TH}} \vect{\chi}^\top h\vect{\psi}_\mathrm{Tb}^\top \vect{\psi}_\mathrm{Tb}\pab{\vect{T}-\vect{T}_\mathrm{env}} \odif{\Gamma_\mathrm{TH}}
    + \sum_{s=1}^{N_{\Gamma_\mathrm{TR}}}\int_{\Gamma_\mathrm{TR}} \vect{\chi}^\top \alpha_\mathrm{TR} \vect{\psi}_\mathrm{Tb}^\top \vect{\psi}_\mathrm{Tb}\pab{\vect{T}-\vect{T}_\mathrm{r}} \odif{\Gamma_\mathrm{TR}} = 0
  \end{split}
\end{equation}
ただし,$N_e$は総要素数,$N_{\Gamma_\mathrm{TN}}$は温度勾配境界の数,$N_{\Gamma_\mathrm{TF}}$は熱フラックス境界の数,$N_{\Gamma_\mathrm{TC}}$はロビン境界の数,$N_{\Gamma_\mathrm{TH}}$は熱伝達境界の数,$N_{\Gamma_\mathrm{TR}}$は熱放射境界の数である.また,$\vect{Q}_\mathrm{w}$は水分フラックス勾配ベクトルである.
\begin{equation}
  \label{Eq:heat_fluxgrad}
  \vect{Q}_\mathrm{w}=\sum_{i=1}^{N_\mathrm{Dim}}\pdv{q_i}{x_i}
\end{equation}
ここで,$\vect{\chi}$は任意の関数を取れるので,$\vect{\chi}$についてまとめると,
\begin{equation}
  \label{Eq:heat-weak4}
  \begin{split}
     & \sum_{e=1}^{N_e}\iint_{\Omega^e} C_\mathrm{a} \vect{\psi}_\mathrm{T}^\top \vect{\psi}_\mathrm{T}\pdv{\vect{T}}{t}\odif{\Omega^e}
    +\sum_{e=1}^{N_e}\iint_{\Omega^e} \mat{B}^\top \mat{R} \mat{B} \vect{T}\odif{\Omega^e}
    +\sum_{e=1}^{N_e}\iint_{\Omega^e} C_\mathrm{w} \vect{Q}_\mathrm{w} \vect{\psi}_\mathrm{T}^\top \vect{\psi}_\mathrm{T}\vect{T}\odif{\Omega^e}                                             \\                                                          
     & +\sum_{e=1}^{N_e}\iint_{\Omega^e} C_\mathrm{w} \vect{\psi}_\mathrm{T}^\top \vect{q}^\top \mat{B} \vect{T} \odif{\Omega^e}
    -\sum_{e=1}^{N_e}\iint_{\Omega^e} \vect{\psi}_\mathrm{T}^\top \vect{\psi}_\mathrm{T} \vect{S}_\mathrm{T}\odif{\Omega^e}
    - \sum_{s=1}^{N_{\Gamma_\mathrm{TN}}}\int_{\Gamma_\mathrm{TN}} \vect{\psi}_\mathrm{Tb}^\top \vect{\psi}_\mathrm{Tb} \lambda\vect{N}_\mathrm{T} \odif{\Gamma_\mathrm{TN}}                 \\                                                          
     & +\sum_{s=1}^{N_{\Gamma_\mathrm{TF}}}\int_{\Gamma_\mathrm{TF}} \vect{\psi}_\mathrm{Tb}^\top \vect{\psi}_\mathrm{Tb} \vect{Q}_\mathrm{F} \odif{\Gamma_\mathrm{TF}}
    +\sum_{s=1}^{N_{\Gamma_\mathrm{TC}}}\int_{\Gamma_\mathrm{TC}} \vect{\psi}_\mathrm{Tb}^\top \vect{\psi}_\mathrm{Tb} \vect{Q}_\mathrm{C} \odif{\Gamma_\mathrm{TC}}
    -\sum_{s=1}^{N_{\Gamma_\mathrm{TC}}}\int_{\Gamma_\mathrm{TC}} \vect{\psi}_\mathrm{Tb}^\top C_\mathrm{w} \vect{q}^\top \vect{n} \vect{\psi}_\mathrm{Tb}\vect{T} \odif{\Gamma_\mathrm{TC}} \\                                                          
     & +\sum_{s=1}^{N_{\Gamma_\mathrm{TH}}}\int_{\Gamma_\mathrm{TH}} h\vect{\psi}_\mathrm{Tb}^\top \vect{\psi}_\mathrm{Tb}\pab{\vect{T}-\vect{T}_\mathrm{env}} \odif{\Gamma_\mathrm{TH}}
    +\sum_{s=1}^{N_{\Gamma_\mathrm{TR}}}\int_{\Gamma_\mathrm{TR}} \alpha_\mathrm{TR}\vect{\psi}_\mathrm{Tb}^\top \vect{\psi}_\mathrm{Tb}\pab{\vect{T}-\vect{T}_\mathrm{r}}  \odif{\Gamma_\mathrm{TR}} = 0
  \end{split}
\end{equation}
よって,\eqref{Eq:heat-weak4}を行列表示すれば,
\begin{equation}
  \label{Eq:heat-weak-Matrix}
  \mat{K}_\mathrm{T}\vect{T} + {\mat{C}_\mathrm{T}}\pdv{\vect{T}}{t} = {\vect{F}_\mathrm{T}}
\end{equation}
ここで,
\begin{subequations}
  \begin{align}
    \label{Eq:Matrix_KT}
    \begin{split}
      \mat{K}_\mathrm{T}
       & = \sum_{e=1}^{N_e}\iint_{\Omega^e}\mat{B}^\top \mat{R} \mat{B} \odif{\Omega^e}
      + \sum_{e=1}^{N_e}\iint_{\Omega^e} C_\mathrm{w} \vect{Q}_\mathrm{w} \vect{\psi}_\mathrm{T}^\top \vect{\psi}_\mathrm{T}\odif{\Omega^e}
      + \sum_{e=1}^{N_e}\iint_{\Omega^e} C_\mathrm{w} \vect{\psi}_\mathrm{T}^\top \vect{q}^\top \mat{B} \odif{\Omega^e}                                                                   \\                                                               
       & + \sum_{s=1}^{N_{\Gamma_\mathrm{TC}}}\int_{\Gamma_\mathrm{TC}}  C_\mathrm{w} \vect{q}^\top \vect{n} \vect{\psi}_\mathrm{Tb}^\top\vect{\psi}_\mathrm{Tb}\odif{\Gamma_\mathrm{TC}}
      + \sum_{s=1}^{N_{\Gamma_\mathrm{TH}}}\int_{\Gamma_\mathrm{TH}} h\vect{\psi}_\mathrm{Tb}^\top \vect{\psi}_\mathrm{Tb} \odif{\Gamma_\mathrm{TH}}
      + \sum_{s=1}^{N_{\Gamma_\mathrm{TR}}}\int_{\Gamma_\mathrm{TR}} \alpha_\mathrm{TR}\vect{\psi}_\mathrm{Tb}^\top \vect{\psi}_\mathrm{Tb} \odif{\Gamma_\mathrm{TR}}
    \end{split} \\
    \label{Eq:Matrix_CT}
    \mat{C}_\mathrm{T} & = \sum_{e=1}^{N_e}\iint_{\Omega^e} C_\mathrm{a} \vect{\psi}_\mathrm{T}^\top \vect{\psi}_\mathrm{T}\odif{\Omega^e}                                                                                                                 \\
    \label{Eq:Matrix_FT}
    \begin{split}
      \vect{F}_\mathrm{T}
       & = \sum_{e=1}^{N_e}\iint_{\Omega^e} \vect{\psi}_\mathrm{T}^\top \vect{\psi}_\mathrm{T}\vect{S}_\mathrm{T}\odif{\Omega}
      + \sum_{s=1}^{N_{\Gamma_\mathrm{TN}}}\int_{\Gamma_\mathrm{TN}} \vect{\psi}_\mathrm{Tb}^\top \vect{\psi}_\mathrm{Tb} \lambda\vect{N}_\mathrm{T} \odif{\Gamma_\mathrm{TN}}
      - \sum_{s=1}^{N_{\Gamma_\mathrm{TF}}}\int_{\Gamma_\mathrm{TF}} \vect{\psi}_\mathrm{Tb}^\top \vect{\psi}_\mathrm{Tb} \vect{Q}_\mathrm{TF} \odif{\Gamma_\mathrm{TF}}    \\                                                                
       & - \sum_{s=1}^{N_{\Gamma_\mathrm{TC}}}\int_{\Gamma_\mathrm{TC}} \vect{\psi}_\mathrm{Tb}^\top \vect{\psi}_\mathrm{Tb} \vect{Q}_\mathrm{TC} \odif{\Gamma_\mathrm{TC}}
      +\sum_{s=1}^{N_{\Gamma_\mathrm{TH}}}\int_{\Gamma_\mathrm{TH}} h\vect{\psi}_\mathrm{Tb}^\top \vect{\psi}_\mathrm{Tb} \vect{T}_\mathrm{env} \odif{\Gamma_\mathrm{TH}}
      + \sum_{s=1}^{N_{\Gamma_\mathrm{R}}}\int_{\Gamma_\mathrm{TR}} \alpha_\mathrm{R}\vect{\psi}_\mathrm{Tb}^\top \vect{\psi}_\mathrm{Tb}\vect{T}_\mathrm{r} \odif{\Gamma_\mathrm{TR}}
    \end{split}
  \end{align}
\end{subequations}
\eqref{Eq:heat-weak-Matrix}を時間積分する.時間積分については詳細は\cref{Sec:BDF}を参照されたい.ここで,特に後退差分法 (BDF-1)について定式化する.
\begin{equation}
  \label{Eq:heat-weak-Matrix2}
  \adif{t}_{n}\eval{\mat{K}_\mathrm{T}}{n+1} \vect{T}_{n+1} + \eval{\mat{C}_\mathrm{T}}{n+1}\vect{T}_{n+1} - \eval{\mat{C}_\mathrm{T}}{n+1}\vect{T}_{n} = \adif{t}_{n}\eval{\vect{F}_\mathrm{T}}{n+1}
\end{equation}
ここで下付き添字の$n$は時間ステップを表し,$n+1$は次のステップ,$n$は現在のステップを表す.$\adif{t}_{n}$は現在の計算時間ステップとの時間差分$t_{n+1}-t_{n}$を表す.
熱支配方程式の残差$\vect{\Phi}_\mathrm{T}$は 
\begin{equation}
  \vect{\Phi}_\mathrm{T} = \adif{t}_{n}\eval{\mat{K}_\mathrm{T}}{n+1} \vect{T}_{n+1} + \eval{\mat{C}_\mathrm{T}}{n+1}\vect{T}_{n+1} - \eval{\mat{C}_\mathrm{T}}{n+1}\vect{T}_{n} - \adif{t}_{n}\eval{\vect{F}_\mathrm{T}}{n+1}
\end{equation}
したがって,${\vect{\Phi}_\mathrm{T}}$の増分は
\begin{subequations}
  \begin{align}
    \odif{{\vect{\Phi}_\mathrm{T}}}      & = \eval{\mat{K}_\mathrm{T}^\star}{n+1} \odif{\vect{T}_{n+1}}                                                                                                                 \\
    \eval{\mat{K}_\mathrm{T}^\star}{n+1} & = \adif{t}_{n}\pab{\eval{\mat{K}_\mathrm{T}}{n+1} + \eval{\pdv{\mat{K}_\mathrm{T}}{\vect{T}}}{n+1} \vect{T}_{n+1} -  \eval{\pdv{\vect{F}_\mathrm{T}}{\vect{T}}}{n+1}} \notag \\
                                         & \quad + \pab{\eval{\mat{C}_\mathrm{T}}{n+1} + \eval{\pdv{\mat{C}_\mathrm{T}}{\vect{T}}}{n+1} \vect{T}_{n+1} - \eval{\pdv{\mat{C}_\mathrm{T}}{\vect{T}}}{n+1}\vect{T}_{n}}
  \end{align}
  ここで,$\mat{K}_\mathrm{T}^\star$は熱移動支配方程式の接線剛性マトリックス (tangent stiffness matrix)である.
\end{subequations}
この関係を用いて,Newton-Raphson法による繰り返し回数$l$回での温度は下記のように求められる.
\begin{subequations}
  \begin{align}
    \evalub{\mat{K}_\mathrm{T}^\star}{n+1}{l} \adif{\vect{T}}_{n+1}^{l+1} & = - \evalub{\vect{\Phi}_\mathrm{T}}{n+1}{l} \label{Eq:NR-System}                                                                                                                                                                                              \\
    \evalub{\mat{K}_\mathrm{T}^\star}{n+1}{l}                             & = \adif{t}_{n}\pab{\evalub{\mat{K}_\mathrm{T}}{n+1}{l} + \evalub{\pdv{\mat{K}_\mathrm{T}}{\vect{T}}}{n+1}{l} \vect{T}_{n+1}^{l} - \evalub{\pdv{\vect{F}_\mathrm{T}}{\vect{T}}}{n+1}{l}} \notag                                                                \\
                                                                          & \quad + \pab{\evalub{\mat{C}_\mathrm{T}}{n+1}{l} + \evalub{\pdv{\mat{C}_\mathrm{T}}{\vect{T}}}{n+1}{l} \vect{T}_{n+1}^{l} - \evalub{\pdv{\mat{C}_\mathrm{T}}{\vect{T}}}{n+1}{l} \vect{T}_{n}} \label{Eq:Tangent-Matrix}                                       \\
    \evalub{\vect{\Phi}_\mathrm{T}}{n+1}{l}                               & = \adif{t}_{n}\evalub{\mat{K}_\mathrm{T}}{n+1}{l} \vect{T}_{n+1}^{l} + \evalub{\mat{C}_\mathrm{T}}{n+1}{l} \vect{T}_{n+1}^{l} - \evalub{\mat{C}_\mathrm{T}}{n+1}{l}\vect{T}_{n} - \adif{t}_{n}\evalub{\vect{F}_\mathrm{T}}{n+1}{l} \label{Eq:Residual-Vector} \\
    \vect{T}_{n+1}^{l+1}                                                  & = \vect{T}_{n+1}^l + \adif{\vect{T}}_{n+1}^{l+1}
  \end{align}
\end{subequations}