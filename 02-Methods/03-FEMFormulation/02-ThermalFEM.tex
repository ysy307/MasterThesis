\subsection{熱移動支配方程式の離散化}
\label{Sec:ThermalFEM}

熱移動に対する境界条件は以下のとおりである.
境界 $\partial V$ 上における外向き単位法線ベクトルを $\vect{n}$ としたとき,境界上の熱収支は全熱エネルギーフラックスの法線成分 $\vect{j}_\mathrm{E} \cdot \vect{n}$ を用いて記述される.

\begin{ConditionBox}{温度境界条件}{Thermal_BC}
  領域 $V$ における温度場 $T(\vect{x},t)$ は,境界 $\partial V$ 上において,
  以下に示す代表的な温度境界条件を満たすものとする.
  ここで,
  $h$ は熱伝達係数,
  $\varepsilon$ は放射率,
  $\sigma$ はステファン=ボルツマン定数,
  $F$ は形態係数を表す.
  さらに,
  $\alpha_\mathrm{HR}$ は熱放射項を線形化した等価熱伝達係数である.

  \begin{subequations}
    \label{Eq:Thermal-BC}
    \begin{description}
      \item[(a) 温度既定境界 (Dirichlet 境界条件)]
            境界 $\partial V_\mathrm{HD}$ 上において温度が既定される場合,
            \begin{equation}
              T\pab{\vect{x}, t} = T_\mathrm{D}\pab{\vect{x}}
              \qquad \text{for } \vect{x}\in\partial V_\mathrm{HD}
            \end{equation}

      \item[(b) 熱流束既定境界 (Neumann 境界条件)]
            境界 $\partial V_\mathrm{HN}$ 上において,全熱エネルギーフラックスの法線成分が既定される場合,
            \begin{equation}
              \vect{j}_\mathrm{E}\pab{\vect{x}, t} \cdot \vect{n}
              = q_\mathrm{TN}\pab{\vect{x}, t}
              \qquad \text{for } \vect{x}\in\partial V_\mathrm{HN}
            \end{equation}
            ここで $q_\mathrm{TN}$ は境界を通して流入・流出する正味の熱流束(既知量)である.

      \item[(c) Robin 境界条件]
            境界 $\partial V_\mathrm{HC}$ において,熱流束が温度の線形関数として与えられる場合,
            当該境界において水分(液状水・水蒸気)の出入りはないものと仮定すると,境界上のエネルギー収支は以下のように記述される.
            \begin{equation}
              \vect{j}_\mathrm{E}\pab{\vect{x}, t} \cdot \vect{n}
              = \beta\pab{\vect{x}}\,T\pab{\vect{x}, t} + \gamma\pab{\vect{x}}
              \qquad \text{for } \vect{x}\in\partial V_\mathrm{HC}
            \end{equation}
            一方,当該領域において,水分の出入りがある場合には,境界上のエネルギー収支は以下のように記述される.
            \begin{align}
               & \vect{j}_\mathrm{E}\pab{\vect{x}, t} \cdot \vect{n} \notag                                                      \\
               & = \quad \beta\pab{\vect{x}}\,T\pab{\vect{x}, t} + \gamma\pab{\vect{x}}
              + c_\mathrm{w} \rho_\mathrm{w} \vect{j}_\mathrm{WL}\pab{\vect{x}, t} \cdot \vect{n}\, T\pab{\vect{x}, t} \notag    \\
               & \qquad + c_\mathrm{v} \rho_\mathrm{w} \vect{j}_\mathrm{WV}\pab{\vect{x}, t} \cdot \vect{n}\, T\pab{\vect{x}, t}
              + \rho_\mathrm{w} L_\mathrm{v} \vect{j}_\mathrm{WV}\pab{\vect{x}, t} \cdot \vect{n}
              \qquad \text{for } \vect{x}\in\partial V_\mathrm{HC}
            \end{align}

      \item[(d) 熱伝達境界]
            境界 $\partial V_\mathrm{HH}$ において周囲環境との熱伝達(対流熱伝達)を考慮する場合,
            当該境界において水分の出入りはない(不透水境界)と仮定する.このとき,全熱エネルギーフラックスは熱伝達量と釣り合うため,以下のように記述される.
            \begin{equation}
              \vect{j}_\mathrm{E}\pab{\vect{x}, t} \cdot \vect{n}
              = h(\vect{x})
              \bab{T\pab{\vect{x}, t}-T_\mathrm{env}\pab{\vect{x}, t}}
              \qquad \text{for } \vect{x}\in\partial V_\mathrm{HH}
            \end{equation}

      \item[(e) 熱放射境界]
            境界 $\partial V_\mathrm{HR}$ において熱放射によるエネルギー交換を考慮する場合,
            同様に水分の出入りはないものと仮定すると,以下の条件が成立する.
            \begin{align}
              \vect{j}_\mathrm{E}\pab{\vect{x}, t} \cdot \vect{n}
               & = \varepsilon\sigma F
              \bab{T\pab{\vect{x}, t}^4 - T_\mathrm{r}\pab{\vect{x}, t}^4} \notag \\
               & \simeq \alpha_\mathrm{HR}(\vect{x})
              \bab{T\pab{\vect{x}, t}-T_\mathrm{r}\pab{\vect{x}, t}}
              \qquad \text{for } \vect{x}\in\partial V_\mathrm{HR}
            \end{align}
    \end{description}
  \end{subequations}
\end{ConditionBox}

支配方程式\ref{Eq:Energy_Continuity_Differential}を有限要素法で離散化することを考える.$V$を$\mathbb{R}^3$で有界な領域,重み関数を$\chi$とすれば弱形式は
\begin{equation}
  \label{Eq:Thermal-weak-form}
  \iiint_{V} \bab{\pdv{\mathcal{U}}{t} + \nabla\cdot \vect{j}_\mathrm{E} + S_\mathrm{T}}\chi \odif{V} = 0
\end{equation}
ただし$\chi$は任意な関数であるが,以下の条件を満たしている必要がある.
\begin{ConditionBox}{重み関数$\chi$の条件}{Weight_Function_Condition}
  \begin{itemize}
    \item 重み関数$\chi$は,有限要素の節点で連続である.
    \item 重み関数$\chi$は,Dirichlet境界条件において,値がゼロである.
    \item 重み関数$\chi$は,Neumann境界条件において,値が1で微分値がゼロである.
    \item 重み関数$\chi$は無次元である.
    \item 重み関数$\chi$は,互いに独立関数でなくてはならない.
  \end{itemize}
\end{ConditionBox}
まず,ベクトル解析の恒等式 (\cref{Form:Identity-Div-Product})を用いると,支配方程式の体積積分項はガウスの発散定理 (\ref{Form:Gauss-Divergence-Theorem-3D})より次のように変形できる.
\begin{align}
  \iiint_{V} \chi \nabla \cdot \vect{j}_\mathrm{E} \odif{V}
   & = \iiint_{V} \bab{\nabla \cdot \pab{\chi\vect{j}_\mathrm{E}} - \nabla\chi \cdot \vect{j}_\mathrm{E}} \odif{V} \notag              \\
  \label{Eq:Thermal-Weak-Form-Divergence}
   & = \oiint_{\partial V} \chi \vect{j}_\mathrm{E} \cdot \vect{n} \odif{S} - \iiint_{V} \nabla\chi \cdot \vect{j}_\mathrm{E} \odif{V}
\end{align}
ここで,右辺第一項の面積分への変換にガウスの発散定理を用いた.$\vect{n}$は境界$\partial V$上の外向き単位法線ベクトルである.
これより,\eqref{Eq:Thermal-weak-form}は次のように書き換えられる.
\begin{equation}
  \iiint_{V} \bab{\chi\pdv{\mathcal{U}}{t} - \nabla\chi \cdot \vect{j}_\mathrm{E} + \chi S_\mathrm{T}} \odif{V} + \oiint_{\partial V} \chi \vect{j}_\mathrm{E} \cdot \vect{n} \odif{S} = 0
\end{equation}
次に,境界積分項を評価する.全境界$\partial V$は,適用される境界条件の種類に応じて以下のように分割される.
\begin{equation}
  \partial V = \partial V_\mathrm{HD} \cup \partial V_\mathrm{HN} \cup \partial V_\mathrm{HC} \cup \partial V_\mathrm{HH} \cup \partial V_\mathrm{HR}
\end{equation}
ここで,\cref{Condition:Weight_Function_Condition}の重み関数の条件より,Dirichlet境界($\partial V_\mathrm{HD}$)上では$\chi=0$となるため,当該境界上での積分は消失する.
その他の境界については,式\eqref{Eq:Thermal-BC}で与えられる各条件を代入する.
各境界において,全熱エネルギーフラックスの法線成分 $\vect{j}_\mathrm{E} \cdot \vect{n}$ は,既定の熱流束 $q_\mathrm{N}$ あるいは温度に依存する関数($Q_\mathrm{HC}, Q_\mathrm{HH}, Q_\mathrm{HR}$)と釣り合うため,最終的な弱形式は以下のようになる.
\begin{align}
  \label{Eq:Thermal-weak-form-final}
   & \iiint_{V} \bab{\chi\pdv{\mathcal{U}}{t} - \nabla\chi \cdot \vect{j}_\mathrm{E} + \chi S_\mathrm{T}} \odif{V} \notag \\
   & \quad + \iint_{\partial V_\mathrm{HN}} \chi q_\mathrm{N} \odif{S}
  + \iint_{\partial V_\mathrm{HC}} \chi Q_\mathrm{HC} \odif{S} \notag                                                     \\
   & \quad + \iint_{\partial V_\mathrm{HH}} \chi Q_\mathrm{HH} \odif{S}
  + \iint_{\partial V_\mathrm{HR}} \chi Q_\mathrm{HR} \odif{S} = 0
\end{align}
ここで,$Q_\mathrm{HC}, Q_\mathrm{HH}, Q_\mathrm{HR}$ はそれぞれ式\eqref{Eq:Thermal-BC}におけるRobin 境界,熱伝達境界,熱放射境界の右辺項を表す.

\FloatBarrier