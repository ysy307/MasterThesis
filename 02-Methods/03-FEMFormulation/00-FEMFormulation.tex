\section{有限要素法による離散化}
\label{Sec:FEMFormulation}
\numberwithin{equation}{subsection}

自然界の物理現象の多くは偏微分方程式によって記述されるが,複雑な境界条件や非線形性を有する場合,厳密な解析解を得ることは極めて困難である.有限要素法(Finite Element Method: FEM)は,このような連続体の境界値問題を,計算機を用いて数値的に解くための最も汎用的な離散化手法の一つである.本手法の核心は,本来無限の自由度を持つ連続体上の未知関数を,有限個のパラメータ(節点値)と既知の関数(形状関数)の線形結合によって近似的に表現することにある.具体的には,解析領域を「要素」と呼ばれる単純な形状の小領域に分割し,各要素内での物理量の変化を低次の多項式などで補間する.数理的には,支配方程式(強形式)に対して重み付き残差法(Method of Weighted Residuals)を適用し,積分形式である弱形式 (Weak Form) へと変換するプロセスを経る.これにより,微分可能性の要求を緩和しつつ,代数的な連立一次方程式へと帰着させることが可能となる.本節では,\cref{Sec:GoverningEquationDerivation}で導出した支配方程式を有限要素法により離散化する手法を示す.

\subsection{時間積分法}
\label{Sec:TimeIntegration}

\subsubsection{時間積分法の概要}
\label{Sec:TimeIntegration_Overview}

有限要素法などにより空間方向を離散化すると,\cref{Sec:GoverningEquationDerivation}で導出した熱・水分移動の支配方程式系は,一般に次のような常微分方程式(あるいは微分代数方程式)系として表される.
\begin{equation}
  \label{Eq:Generic_ODE_System}
  \pdv{\vect{y}}{t} = \vect{F}\,\pab{\vect{y},t}
\end{equation}
ここで,$\vect{y}(t)$ は全自由度の集合,$\vect{F}$ は伝導・移流・相変化・供給項などを含む非線形作用素ベクトルである.数値計算では,この時間連続問題を,離散的な時刻列
\begin{equation}
  t_0 < t_1 < \dots < t_n < t_{n+1} < \dots < t_N, \qquad  \adif{t_n} \coloneq t_{n+1}-t_n
\end{equation}
上で定義された近似解列 $\Bab{\vect{y}^n}_{n=0}^N$ によって置き換える.このとき,各時間ステップでどのように $\vect{y}^{n+1}$ を求めるかを定める手続きが時間積分法である.

代表的な 1 ステップ時間積分法として,いわゆる $\theta$ 法が挙げられる.\eqref{Eq:Generic_ODE_System} に対して $\theta$ 法を適用すると,時刻 $t_n$ から $t_{n+1}$ への更新は
\begin{equation}
  \label{Eq:Theta_Method}
  \frac{\vect{y}^{n+1} - \vect{y}^{n}}{\adif{t_n}}
  = (1-\theta)\,\vect{F}\,\pab{\vect{y}^{n},t_n}
  + \theta\,\vect{F}\,\pab{\vect{y}^{n+1},t_{n+1}}
\end{equation}
のように記述される.ここで $\theta=0$ のときは陽解法(前進オイラー法),$\theta = 1/2$ のときはCrank–Nicolson法,$\theta=1$ のときは陰解法(後退オイラー法)となる.土壌凍結問題のように強い非線形性や硬い
% \footnote{偏微分方程式系が硬いとは,その固有値分布が広いことをいう.具体的には最小固有値$\lambda_\min$と最大固有値$\lambda_\max$の比である条件数$\kappa=\lambda_\max/\lambda_\min > 10^4$ 程度のものをさすことが多い}
特性をもつ系では,大きな時間刻みでも数値的安定性を保ちやすい陰解法($\theta \geq 1/2$),とりわけ $\theta=1$ の後退オイラー法が広く用いられる.

一方で,長期の凍結・融解過程を高精度に追跡するには,1 階精度の後退オイラー法だけでは時間離散化誤差が支配的となる場合がある.このため本研究では,陰的 1 ステップ法の枠組みを一般化した多段法である後退微分法(Backward Differentiation Formula, BDF)を採用する.BDF は,現在時刻 $t_{n+1}$ における時間微分を,現在値および過去 $k$ ステップ分の解を用いた線形結合として近似する多段陰解法であり,
\begin{equation}
  \label{Eq:BDF_k_Intro}
  \pdv{\vect{y}}{t}\bigg|_{t_{n+1}}
  \approx
  \frac{1}{\Delta t_n}
  \sum_{i=0}^{k} \alpha_i\,\vect{y}^{\,n+1-i}
\end{equation}
という形で表される.ここで $\alpha_i$ は BDF-$k$ に特有の係数であり,$k=1$ の場合には後退オイラー法に一致する.BDF は硬い方程式系に対して高い数値安定性を有し,$k$ を大きくすることで時間積分の高次精度化も可能である.

本研究の時間積分スキームでは,可変時間ステップおよび可変次数に対応した BDF 法を用いて,熱移動・水分移動の連成方程式系を時間方向に積分する.BDF 係数の具体的な導出方法や,可変ステップ・可変次数アルゴリズムの詳細については,\cref{Sec:BDF} にて述べる.

\subsubsection{可変時間ステップおよび可変次数における後退差分法}
\label{Sec:BDF}

後退差分法(Backward Differentiation Formula, BDF)は,時間積分において,過去の値を用いて現在の値を求める方法である.BDFは,特に支配方程式の系が硬い問題において有効であり,数値的な安定性が高い特徴を持つ.7次以上は零点安定性がなくなるので考える必要はなく,6次までのBDFを考えることとする.
時間微分項 $\pdv{\vect{y}}/{t}$ を近似するため,いくつかの過去の時点 $\pab{\vect{y}^i, t_i}$ を通るラグランジュ補間多項式 $Y\pab{t}$ を構築し,その現在時刻 $t_{n+1}$ での微分 $P'\pab{t_{n+1}}$ を用いる.
一般にLagrange補間多項式は,Lagrange基底関数 $l_i(x)$
\begin{align}
  l_i\pab{x} = \prod_{\substack{j=0\leq j \leq i \\[0.2mm] j \neq i}}\frac{x - x_j}{x_i - x_j}
\end{align}
と補間対象となる変数$y_i$との線形結合として与えられる.
\begin{equation}
  \label{Eq:bdf_lagrange}
  L\pab{x,t}\coloneq\sum_{i=0}^{j} l_i \pab{x} y_i \pab{t}
\end{equation}
計算を簡略化するため,局所的な時間座標 $\tau = t - t_{n+1}$ を導入する.この座標系では,現在時刻は $\tau=0$ となる.ただし,1次のBDFは単純な後退差分であるため,2次以上のBDFについて考える.本節では特に2次について詳述するとともに,6次までのBDFはそれまでと同様に導出できるため,その結果を示すだけとする.本章では$k$-次数のBDFをBDF-$k$と表記する.

\subsubsection{BDF-2における導出}
3つの点 $\pab{\vect{y}^{n+1},t_{n+1}}$,$\pab{\vect{y}^n, t_n}$,$\pab{\vect{y}^{n-1}, t_{n-1}}$ を通る2次のラグランジュ補間多項式 $P\pab{\tau}$ を考える.各点は局所座標で $\pab{\vect{y}^{n+1},\tau_0}, \pab{\vect{y}^n, \tau_1}, \pab{\vect{y}^{n-1}, \tau_2}$ と表せる.
\begin{subequations}
  \label{Eq:bdf2_lagrange_time}
  \begin{align}
    \tau_0 & = t_{n+1} - t_{n+1} = 0                                \\
    \tau_1 & = t_n - t_{n+1} = - \adif{t_n}                         \\
    \tau_2 & = t_{n-1} - t_{n+1} = -( \adif{t_n} +  \adif{t_{n-1}})
  \end{align}
\end{subequations}
ここで$P\pab{\tau}$が$k=2$の時のLagrange補間多項式であるとすると,\eqref{Eq:bdf_lagrange}より
\begin{align}
  P\pab{\tau} = l_0\pab{\tau} \vect{y}^{n+1} + l_1\pab{\tau} \vect{y}^n + l_2\pab{\tau} \vect{y}^{n-1}
\end{align}
ここで,各基底関数は次のように定義される.
\begin{subequations}
  \label{Eq:bdf2_lagrange_basis}
  \begin{align}
    l_0\pab{\tau} = \frac{\tau-\tau_1}{\tau_0-\tau_1} \frac{\tau-\tau_2}{\tau_0-\tau_2} \\[1mm]
    l_1\pab{\tau} = \frac{\tau-\tau_0}{\tau_1-\tau_0} \frac{\tau-\tau_2}{\tau_1-\tau_2} \\[1mm]
    l_2\pab{\tau} = \frac{\tau-\tau_0}{\tau_2-\tau_0} \frac{\tau-\tau_1}{\tau_2-\tau_1}
  \end{align}
\end{subequations}
ここで各基底関数の微分は\eqref{Eq:bdf2_lagrange_basis}より
\begin{subequations}
  \label{Eq:bdf2_lagrange_basis_diff}
  \begin{align}
    l'_0\pab{\tau} & = \frac{\tau-\tau_2}{\pab{\tau_0-\tau_1}\pab{\tau_0-\tau_2}} + \frac{\tau-\tau_1}{\pab{\tau_0-\tau_1}\pab{\tau_0-\tau_2}} = \frac{2\tau - \tau_1 - \tau_2}{\pab{\tau_0-\tau_1}\pab{\tau_0-\tau_2}} \\[1mm]
    l'_1\pab{\tau} & = \frac{\tau-\tau_2}{\pab{\tau_1-\tau_0}\pab{\tau_1-\tau_2}} + \frac{\tau-\tau_0}{\pab{\tau_1-\tau_0}\pab{\tau_1-\tau_2}} = \frac{2\tau - \tau_0 - \tau_2}{\pab{\tau_1-\tau_0}\pab{\tau_1-\tau_2}} \\[1mm]
    l'_2\pab{\tau} & = \frac{\tau-\tau_1}{\pab{\tau_2-\tau_0}\pab{\tau_2-\tau_1}} + \frac{\tau-\tau_0}{\pab{\tau_2-\tau_0}\pab{\tau_2-\tau_1}} = \frac{2\tau - \tau_0 - \tau_1}{\pab{\tau_2-\tau_0}\pab{\tau_2-\tau_1}}
  \end{align}
\end{subequations}
$P\pab{\tau}$ の $\tau=0$ における微分は,
\begin{equation}
  P'\pab{0} = \vect{y}^{n+1}l'_0(0) + \vect{y}^n l'_1(0) + \vect{y}^{n-1}l'_2(0)
\end{equation}
これに\eqref{Eq:bdf2_lagrange_basis_diff}および$\tau_1, \tau_2$ を代入し,時間ステップ比 $\rho_1 = \adif{t_n} / \adif{t_{n-1}}$ を用いて整理すると,
\begin{subequations}
  \begin{align}
    l'_0\pab{0} & = -\frac{\tau_1 + \tau_2}{\tau_1 \tau_2} = \frac{2\adif{t_n} + \adif{t_{n-1}}}{\adif{t_n} \pab{\adif{t_n} + \adif{t_{n-1}}}} = \frac{2\rho_1 + 1}{\rho_1 + 1}\frac{1}{\adif{t_n}} \\[2mm]
    l'_1\pab{0} & = -\frac{\tau_2}{\tau_1 \pab{\tau_1 - \tau_2}} = -\frac{\adif{t_n} + \adif{t_{n-1}}}{\adif{t_n} \adif{t_{n-1}}} = -\pab{\rho_1 + 1}\frac{1}{\adif{t_n}}                           \\[2mm]
    l'_2\pab{0} & = -\frac{\tau_1}{\tau_2 \pab{\tau_2 - \tau_1}} = \frac{\adif{t_n}}{\pab{\adif{t_n} + \adif{t_{n-1}}}\adif{t_{n-1}} } = \frac{\rho_1^2}{\rho_1 + 1}\frac{1}{\adif{t_n}}
  \end{align}
\end{subequations}
よって,時間微分の近似式は,
\begin{equation}
  \label{Eq:bdf2_approx}
  \pdv{\vect{y}}{t} \bigg|_{t_{n+1}} \approx \frac{1}{\adif{t_n}} \bab{\frac{2\rho_1 + 1}{\rho_1 + 1}\vect{y}^{n+1} - \pab{\rho_1 + 1} \vect{y}^n + \frac{\rho_1^2}{\rho_1 + 1} \vect{y}^{n-1}}
\end{equation}

\subsubsection{BDF-$k$における係数導出のための一般化}
\label{Sec:BDF_Generalization}

BDF-$k$においては,$k$個の過去の時点を用いて微分を近似する.
一般に,時間ステップをくくりだすことで,$k$次のBDFは次のように定義される.
\begin{equation}
  \label{Eq:bdf_k_approx}
  \pdv{\vect{y}}{t} \bigg|_{t_{n+1}} \approx \frac{1}{\adif{t_n}} \sum_{i=0}^{k} l'_i\pab{0} \vect{y}^{n+1-i}
\end{equation}
% ここで,$l_i$はBDF-$k$における係数であり,BDF-$k$の係数はCode Snippet\ref{prog:bdf_coeffs_final}を用いて求めることができる.

% % \begin{lstlisting}[caption=Back Differential Formula Coefficients, label=prog:bdf_coeffs_final]
% % import sympy
% % from IPython.display import display, Math

% % def display_bdf_coeffs(k_order):
% %     """
% %     指定された次数のBDF係数を導出し,LaTeX数式としてIPython環境で表示する.
% %     """
% %     print(f'--- BDF-{k_order} の係数 ---')

% %     # 1. LaTeX表示用のシンボルを定義
% %     dt_n_sym = sympy.Symbol(r'\Delta t_n', positive=True)
% %     dts = [dt_n_sym] + [
% %         sympy.Symbol(rf'\Delta t_{{n-{i}}}', positive=True) for i in range(1, k_order)
% %     ]
% %     rhos = [
% %         sympy.Symbol(rf'\rho_{{{i + 1}}}', positive=True) for i in range(k_order - 1)
% %     ]

% %     tau = sympy.Symbol(r'\tau')
% %     n_sym = sympy.Symbol('n')

% %     # 2. 補間点の座標を定義(k_order+1点)
% %     nodes = [0]
% %     for i in range(1, k_order + 1):
% %         nodes.append(-sum(dts[:i]))

% %     # 3. 各係数 l'_j(0) を計算
% %     for j in range(k_order + 1):
% %         num = 1
% %         den = 1
% %         for i in range(k_order + 1):
% %             if i != j:
% %                 num *= tau - nodes[i]
% %                 den *= nodes[j] - nodes[i]
% %         l_j = num / den

% %         l_j_prime = sympy.diff(l_j, tau)
% %         l_j_prime_at_0 = l_j_prime.subs(tau, 0)

% %         result_in_rho = l_j_prime_at_0
% %         if k_order > 1:
% %             subs_rules = {}
% %             rho_prod = 1
% %             for i in range(k_order - 1):
% %                 rho_prod *= rhos[i]
% %                 subs_rules[dts[i + 1]] = dts[0] / rho_prod
% %             result_in_rho = result_in_rho.subs(subs_rules)

% %         exponent_val = (n_sym + 1) - j
% %         if exponent_val == n_sym + 1:
% %             superscript = 'n+1'
% %         elif exponent_val == n_sym:
% %             superscript = 'n'
% %         else:
% %             superscript = f'n-{j - 1}'
% %         term_comment = rf'\quad \text{{(Coefficient for }} T^{{{superscript}}})'

% %         latex_str = sympy.latex(sympy.factor(result_in_rho))
% %         display(Math(rf"L'_{{{j}}}(0) = {latex_str} {term_comment}"))

% % if __name__ == '__main__':
% %     display_bdf_coeffs(6)
% % \end{lstlisting}

\FloatBarrier
\subsection{熱移動支配方程式の離散化}
\label{Sec:ThermalFEM}

熱移動に対する境界条件は以下のとおりである.
境界 $\partial V$ 上における外向き単位法線ベクトルを $\vect{n}$ としたとき,境界上の熱収支は全熱エネルギーフラックスの法線成分 $\vect{j}_\mathrm{E} \cdot \vect{n}$ を用いて記述される.

\begin{ConditionBox}{温度境界条件}{Thermal_BC}
  領域 $V$ における温度場 $T(\vect{x},t)$ は,境界 $\partial V$ 上において,
  以下に示す代表的な温度境界条件を満たすものとする.
  ここで,
  $h$ は熱伝達係数,
  $\varepsilon$ は放射率,
  $\sigma$ はステファン=ボルツマン定数,
  $F$ は形態係数を表す.
  さらに,
  $\alpha_\mathrm{HR}$ は熱放射項を線形化した等価熱伝達係数である.

  \begin{subequations}
    \label{Eq:Thermal-BC}
    \begin{description}
      \item[(a) 温度既定境界 (Dirichlet 境界条件)]
            境界 $\partial V_\mathrm{HD}$ 上において温度が既定される場合,
            \begin{equation}
              T\pab{\vect{x}, t} = T_\mathrm{D}\pab{\vect{x}}
              \qquad \text{for } \vect{x}\in\partial V_\mathrm{HD}
            \end{equation}

      \item[(b) 熱流束既定境界 (Neumann 境界条件)]
            境界 $\partial V_\mathrm{HN}$ 上において,全熱エネルギーフラックスの法線成分が既定される場合,
            \begin{equation}
              \vect{j}_\mathrm{E}\pab{\vect{x}, t} \cdot \vect{n}
              = q_\mathrm{TN}\pab{\vect{x}, t}
              \qquad \text{for } \vect{x}\in\partial V_\mathrm{HN}
            \end{equation}
            ここで $q_\mathrm{TN}$ は境界を通して流入・流出する正味の熱流束(既知量)である.

      \item[(c) Robin 境界条件]
            境界 $\partial V_\mathrm{HC}$ において,熱流束が温度の線形関数として与えられる場合,
            当該境界において水分(液状水・水蒸気)の出入りはないものと仮定すると,境界上のエネルギー収支は以下のように記述される.
            \begin{equation}
              \vect{j}_\mathrm{E}\pab{\vect{x}, t} \cdot \vect{n}
              = \beta\pab{\vect{x}}\,T\pab{\vect{x}, t} + \gamma\pab{\vect{x}}
              \qquad \text{for } \vect{x}\in\partial V_\mathrm{HC}
            \end{equation}
            一方,当該領域において,水分の出入りがある場合には,境界上のエネルギー収支は以下のように記述される.
            \begin{align}
               & \vect{j}_\mathrm{E}\pab{\vect{x}, t} \cdot \vect{n} \notag                                                      \\
               & = \quad \beta\pab{\vect{x}}\,T\pab{\vect{x}, t} + \gamma\pab{\vect{x}}
              + c_\mathrm{w} \rho_\mathrm{w} \vect{j}_\mathrm{WL}\pab{\vect{x}, t} \cdot \vect{n}\, T\pab{\vect{x}, t} \notag    \\
               & \qquad + c_\mathrm{v} \rho_\mathrm{w} \vect{j}_\mathrm{WV}\pab{\vect{x}, t} \cdot \vect{n}\, T\pab{\vect{x}, t}
              + \rho_\mathrm{w} L_\mathrm{v} \vect{j}_\mathrm{WV}\pab{\vect{x}, t} \cdot \vect{n}
              \qquad \text{for } \vect{x}\in\partial V_\mathrm{HC}
            \end{align}

      \item[(d) 熱伝達境界]
            境界 $\partial V_\mathrm{HH}$ において周囲環境との熱伝達(対流熱伝達)を考慮する場合,
            当該境界において水分の出入りはない(不透水境界)と仮定する.このとき,全熱エネルギーフラックスは熱伝達量と釣り合うため,以下のように記述される.
            \begin{equation}
              \vect{j}_\mathrm{E}\pab{\vect{x}, t} \cdot \vect{n}
              = h(\vect{x})
              \bab{T\pab{\vect{x}, t}-T_\mathrm{env}\pab{\vect{x}, t}}
              \qquad \text{for } \vect{x}\in\partial V_\mathrm{HH}
            \end{equation}

      \item[(e) 熱放射境界]
            境界 $\partial V_\mathrm{HR}$ において熱放射によるエネルギー交換を考慮する場合,
            同様に水分の出入りはないものと仮定すると,以下の条件が成立する.
            \begin{align}
              \vect{j}_\mathrm{E}\pab{\vect{x}, t} \cdot \vect{n}
               & = \varepsilon\sigma F
              \bab{T\pab{\vect{x}, t}^4 - T_\mathrm{r}\pab{\vect{x}, t}^4} \notag \\
               & \simeq \alpha_\mathrm{HR}(\vect{x})
              \bab{T\pab{\vect{x}, t}-T_\mathrm{r}\pab{\vect{x}, t}}
              \qquad \text{for } \vect{x}\in\partial V_\mathrm{HR}
            \end{align}
    \end{description}
  \end{subequations}
\end{ConditionBox}

支配方程式\ref{Eq:Energy_Continuity_Differential}を有限要素法で離散化することを考える.$V$を$\mathbb{R}^3$で有界な領域,重み関数を$\chi$とすれば弱形式は
\begin{equation}
  \label{Eq:Thermal-weak-form}
  \iiint_{V} \bab{\pdv{\mathcal{U}}{t} + \nabla\cdot \vect{j}_\mathrm{E} + S_\mathrm{T}}\chi \odif{V} = 0
\end{equation}
ただし$\chi$は任意な関数であるが,以下の条件を満たしている必要がある.
\begin{ConditionBox}{重み関数$\chi$の条件}{Weight_Function_Condition}
  \begin{itemize}
    \item 重み関数$\chi$は,有限要素の節点で連続である.
    \item 重み関数$\chi$は,Dirichlet境界条件において,値がゼロである.
    \item 重み関数$\chi$は,Neumann境界条件において,値が1で微分値がゼロである.
    \item 重み関数$\chi$は無次元である.
    \item 重み関数$\chi$は,互いに独立関数でなくてはならない.
  \end{itemize}
\end{ConditionBox}
まず,ベクトル解析の恒等式 (\cref{Form:Identity-Div-Product})を用いると,支配方程式の体積積分項はガウスの発散定理 (\ref{Form:Gauss-Divergence-Theorem-3D})より次のように変形できる.
\begin{align}
  \iiint_{V} \chi \nabla \cdot \vect{j}_\mathrm{E} \odif{V}
   & = \iiint_{V} \bab{\nabla \cdot \pab{\chi\vect{j}_\mathrm{E}} - \nabla\chi \cdot \vect{j}_\mathrm{E}} \odif{V} \notag              \\
  \label{Eq:Thermal-Weak-Form-Divergence}
   & = \oiint_{\partial V} \chi \vect{j}_\mathrm{E} \cdot \vect{n} \odif{S} - \iiint_{V} \nabla\chi \cdot \vect{j}_\mathrm{E} \odif{V}
\end{align}
ここで,右辺第一項の面積分への変換にガウスの発散定理を用いた.$\vect{n}$は境界$\partial V$上の外向き単位法線ベクトルである.
これより,\eqref{Eq:Thermal-weak-form}は次のように書き換えられる.
\begin{equation}
  \iiint_{V} \bab{\chi\pdv{\mathcal{U}}{t} - \nabla\chi \cdot \vect{j}_\mathrm{E} + \chi S_\mathrm{T}} \odif{V} + \oiint_{\partial V} \chi \vect{j}_\mathrm{E} \cdot \vect{n} \odif{S} = 0
\end{equation}
次に,境界積分項を評価する.全境界$\partial V$は,適用される境界条件の種類に応じて以下のように分割される.
\begin{equation}
  \partial V = \partial V_\mathrm{HD} \cup \partial V_\mathrm{HN} \cup \partial V_\mathrm{HC} \cup \partial V_\mathrm{HH} \cup \partial V_\mathrm{HR}
\end{equation}
ここで,\cref{Condition:Weight_Function_Condition}の重み関数の条件より,Dirichlet境界($\partial V_\mathrm{HD}$)上では$\chi=0$となるため,当該境界上での積分は消失する.
その他の境界については,式\eqref{Eq:Thermal-BC}で与えられる各条件を代入する.
各境界において,全熱エネルギーフラックスの法線成分 $\vect{j}_\mathrm{E} \cdot \vect{n}$ は,既定の熱流束 $q_\mathrm{N}$ あるいは温度に依存する関数($Q_\mathrm{HC}, Q_\mathrm{HH}, Q_\mathrm{HR}$)と釣り合うため,最終的な弱形式は以下のようになる.
\begin{align}
  \label{Eq:Thermal-weak-form-final}
   & \iiint_{V} \bab{\chi\pdv{\mathcal{U}}{t} - \nabla\chi \cdot \vect{j}_\mathrm{E} + \chi S_\mathrm{T}} \odif{V} \notag \\
   & \quad + \iint_{\partial V_\mathrm{HN}} \chi q_\mathrm{N} \odif{S}
  + \iint_{\partial V_\mathrm{HC}} \chi Q_\mathrm{HC} \odif{S} \notag                                                     \\
   & \quad + \iint_{\partial V_\mathrm{HH}} \chi Q_\mathrm{HH} \odif{S}
  + \iint_{\partial V_\mathrm{HR}} \chi Q_\mathrm{HR} \odif{S} = 0
\end{align}
ここで,$Q_\mathrm{HC}, Q_\mathrm{HH}, Q_\mathrm{HR}$ はそれぞれ式\eqref{Eq:Thermal-BC}におけるRobin 境界,熱伝達境界,熱放射境界の右辺項を表す.

\FloatBarrier
\subsection{水分移動支配方程式の離散化}
\label{Sec:HydraulicFEM}

水分移動に対する境界条件は,全水分質量フラックス $\vect{J}_\mathrm{m} = \rho_\mathrm{w} \pab{\vect{j}_\mathrm{WL} + \vect{j}_\mathrm{WV}}$(液状水および水蒸気の和)を用いて以下のように記述される.
\begin{ConditionBox}{圧力境界条件}{Hydraulic_BC}
  領域 $V$ における間隙水圧場 $P\,\pab{\vect{x},t}$ は,境界 $\partial V$ 上において,以下に示す代表的な圧力境界条件を満たすものとする.
  \begin{subequations}
    \label{Eq:water-BC}
    \begin{description}
      \item[(a) 圧力既定境界 (Dirichlet 境界条件)]
            境界 $\Gamma_\mathrm{HD}$ 上において間隙水圧が既定される場合,
            \begin{equation}
              P\,\pab{\vect{x}, t}
              = P_{\mathrm{D}}\pab{\vect{x}}
              \qquad \text{for } \vect{x}\in\Gamma_\mathrm{HD}
            \end{equation}
      \item[(b) 水分フラックス既定境界 (Neumann 境界条件)]
            境界 $\Gamma_\mathrm{HN}$ 上において,全水分質量フラックス $\vect{J}_\mathrm{m}$ の法線成分が既定される場合,
            \begin{equation}
              \vect{J}_\mathrm{m}\pab{\vect{x}, t} \cdot \vect{n}
              = q_{\mathrm{HN}}\pab{\vect{x},t}
              \qquad \text{for } \vect{x}\in\Gamma_\mathrm{HN}
            \end{equation}
            ここで,$q_{\mathrm{HN}}$ は境界を通して流出する正味の水分フラックスである.
      \item[(c) 大気境界条件]
            地表面境界 $\Gamma_\mathrm{ATM}$ では,気象条件と土壌の浸透・保水性に応じ,境界条件型を動的に切り替える系依存型境界条件を適用する.
            鉛直上向き正の $z$ 軸に対し,可能蒸発散量 $E_\mathrm{potential}$ および降水量 $q_\mathrm{rain}$ から,正味の可能水分フラックス $q_{\mathrm{potential}}$ を次式で定義する.
            \begin{equation}
              q_{\mathrm{potential}}\pab{\vect{x},t} = E_\mathrm{potential}\pab{\vect{x},t} - q_\mathrm{rain}\pab{\vect{x},t}
            \end{equation}
            ここでは外向き法線ベクトル $\vect{n}$ が $+z$ 方向であるため,正の値は蒸発として系外への流出し,負の値は降雨に伴う系内への流入を表す.
            実際の境界条件は,地表面の許容圧力範囲(乾燥限界 $P_{\min}$ および湛水限界 $P_{\max}$)に基づき,以下の制約を満たすように決定される.
            \begin{equation}
              \begin{cases}
                \abs{\vect{J}_\mathrm{m}\pab{\vect{x}, t} \cdot \vect{n}} \le \abs{q_{\mathrm{potential}}} \\[3pt]
                P_{\min} \le P\pab{\vect{x},t} \le P_{\max}
              \end{cases}
              \qquad \text{for } \vect{x}\in\Gamma_\mathrm{ATM}
            \end{equation}
            ここで,$P_{\max}$が$0$のときは,地表面に湛水せず,即座に流出する条件となる.一方,$P_{\max}>0$のときは,地表面にある一定の湛水深が許容される.
            この条件は物理的に以下の2つの状況を包含している.
            \begin{description}
              \item[1. 降雨・浸透過程]:
                    土壌の浸透能が降雨強度を上回る間,つまり $q_{\mathrm{potential}}<0$ のときは,降雨量すべてが流入するフラックス境界となる.浸透能を超過し地表面が飽和すると,圧力境界 $P = P_{\max}$ に切り替わり,差分の水量は表面流出(Surface runoff)として扱われる.
              \item[2. 蒸発・乾燥過程]:
                    土壌水分が十分に存在するとき,つまり $q_{\mathrm{potential}}>0$ のときは,可能蒸発散量が要求する値でのフラックス境界となる.土壌が乾燥し限界圧力 $P_{\min}$ に達すると,圧力境界 $P = P_{\min}$ に固定され,実際の蒸発量は土壌の水分供給能力によって制限される.
            \end{description}
    \end{description}
  \end{subequations}
\end{ConditionBox}

水分移動に関する支配方程式\eqref{Eq:Continuity_void}を有限要素法で離散化する.
形状関数ベクトルを $\vect{\psi}_\mathrm{H}$ とすれば,Galerkin法に基づく弱形式は以下のように記述される.
\begin{equation}
  \label{Eq:Hydraulic_weak_form}
  \iiint_{V} \vect{\psi}_\mathrm{H} \bab{\pdv{\rho_\mathrm{void}}{t} + \nabla\cdot \vect{J}_\mathrm{m} + S_\mathrm{H}} \odif{V} = \vect{0}
\end{equation}
\eqref{Eq:Thermal-Weak-Form-Divergence}と同様にガウスの発散定理を用いて空間微分項を変形し,境界条件を代入すると,弱形式は次のように書き換えられる.
\begin{align}
  \label{Eq:Hydraulic_weak_form_final}
   & \iiint_{V} \bab{ \vect{\psi}_\mathrm{H} \pdv{\rho_\mathrm{void}}{t} - \nabla\vect{\psi}_\mathrm{H}^\mathsf{T} \cdot \vect{J}_\mathrm{m} + \vect{\psi}_\mathrm{H} S_\mathrm{H} } \odif{V} \notag \\
   & \quad + \iint_{\Gamma_\mathrm{HN}} \vect{\psi}_\mathrm{H} q_{\mathrm{HN}} \odif{S} = 0
\end{align}

% 時間微分項 $\pdv{\rho_\mathrm{void}}/{t}$ に対して $k$ 次の BDF を適用し,時刻 $t_{n+1}$ における残差ベクトル $\vect{R}_\mathrm{H}$ を定義する.
% 残差ベクトルは,外力ベクトル,内力ベクトル,および貯留・慣性ベクトルの和として整理される.
% \begin{align}
%   \label{Eq:Residual-Water}
%   \vect{R}_\mathrm{H}\pab{\vect{T}_{n+1}^{m}, \vect{P}_{n+1}^{m}}
%    & = \vect{F}^\mathrm{H}_\text{external} - \vect{F}^\mathrm{H}_\text{internal} - \vect{F}^\mathrm{H}_\text{transient} = \vect{0}
% \end{align}

% \begin{enumerate}
%   \item 外力ベクトル $\vect{F}^\mathrm{H}_\text{external}$:
%         境界条件によって領域表面から流入する水分フラックスの寄与である.
%         \begin{equation}
%           \vect{F}^\mathrm{H}_\text{external} = - \iint_{\Gamma_\mathrm{HN}} j_{\mathrm{w},\mathrm{N}} \vect{\psi}_\mathrm{H} \odif{S}
%         \end{equation}
%         ここで,流出フラックス $j_{\mathrm{w},\mathrm{N}}$ に対して流入方向を正とするためマイナス符号が付く.

%   \item 内力ベクトル $\vect{F}^\mathrm{H}_\text{internal}$:
%         領域内部の水分移動および内部シンク項である.
%         全水分質量フラックス $\vect{J}_\mathrm{m}$ の各成分(液状水・水蒸気)を展開して記述すると以下のようになる.
%         \begin{align}
%           \vect{F}^\mathrm{H}_\text{internal}
%            & = - \iiint_{V} \nabla\vect{\psi}_\mathrm{H}^\mathsf{T} \cdot \vect{J}_\mathrm{m} \odif{V}
%           + \iiint_{V} \vect{\psi}_\mathrm{H} S_\mathrm{H} \odif{V} \notag                                                                                   \\
%            & = - \iiint_{V} \nabla\vect{\psi}_\mathrm{H}^\mathsf{T} \cdot \rho_\mathrm{w} \bigl( \vect{j}_\mathrm{WL} + \vect{j}_\mathrm{WV} \bigr) \odif{V}
%           + \iiint_{V} \vect{\psi}_\mathrm{H} S_\mathrm{H} \odif{V}
%         \end{align}
%         ここで,$\vect{j}_\mathrm{WL}$ および $\vect{j}_\mathrm{WV}$ は,それぞれ $P$ および $T$ の勾配に依存する(式\eqref{Eq:LiquidFlux_final3}, \eqref{Eq:VaporFlux_final}参照).

%   \item 貯留・慣性ベクトル $\vect{F}^\mathrm{H}_\text{transient}$:
%         間隙内の水分貯留量の時間変化項である.
%         \begin{align}
%           \vect{F}^\mathrm{H}_\text{transient}
%            & = \iiint_{V} \vect{\psi}_\mathrm{H} \pab{ \alpha_0 \rho_\mathrm{void}^{n+1} + \rho_\mathrm{void}^\mathrm{hist} } \odif{V}
%         \end{align}
%         ここで,$\rho_\mathrm{void} = \rho_\mathrm{w} \phi S_\mathrm{w} + \rho_\mathrm{v} \phi (1-S_\mathrm{w})$ は間隙中の総水分密度である.
% \end{enumerate}

% Newton-Raphson法による解法のため,残差ベクトル $\vect{R}_\mathrm{H}$ を未知変数 $\vect{P}$ および $\vect{T}$ で偏微分し,ヤコビアン(接線剛性行列)を導出する.
% \begin{equation}
%   \mathbf{J}_\mathrm{hydraulic} =
%   \begin{bmatrix}
%     \mathbf{J}_{HP} & \mathbf{J}_{HT}
%   \end{bmatrix}
%   =
%   \begin{bmatrix}
%     \pdv{\vect{R}_\mathrm{H}}{\vect{P}} & \pdv{\vect{R}_\mathrm{H}}{\vect{T}}
%   \end{bmatrix}
% \end{equation}

% \subsubsection{水-水ブロック ($\mat{J}_\mathrm{HH}$)}
% 間隙水圧の変化に対する水分収支の応答を表す項である.
% 水分容量(貯留項の微分)および透水・通気特性(フラックス項の微分)が含まれる.
% \begin{align}
%   \mat{J}_\mathrm{HH} = \pdv{\vect{R}_\mathrm{H}}{\vect{P}}
%    & = \iiint_{V} \alpha_0 C_{HP} \vect{\psi}_\mathrm{H} \vect{\psi}_\mathrm{H}^\mathsf{T} \odif{V} \notag
%    & \quad + \iiint_{V} \nabla\vect{\psi}_\mathrm{H}^\mathsf{T} \bigl( \rho_\mathrm{w} K_\mathrm{wP} + \rho_\mathrm{w} K_\mathrm{vP} \bigr) \nabla\vect{\psi}_\mathrm{H} \odif{V}
% \end{align}
% ここで,$C_{HP} = \pdv*{\rho_\mathrm{void}}{P}$ は有効水分容量,$K_\mathrm{wP}, K_\mathrm{vP}$ はそれぞれ液状水および水蒸気の水圧勾配に関する輸送係数である.
% なお,フラックス項の微分において $\vect{J}_\mathrm{m}$ が $-\nabla P$ に比例するため,微分後の符号は正($+$)となる.

% \subsubsection{水-熱ブロック ($\mat{J}_\mathrm{HT}$)}
% 温度変化が水分移動に与える影響を表す連成項である.
% 温度変化による密度変化や飽和度変化(容量項),および温度勾配による水分移動(Soret効果など)が含まれる.
% \begin{align}
%   \mat{J}_\mathrm{HT} = \pdv{\vect{R}_\mathrm{H}}{\vect{T}}
%    & = \iiint_{V} \alpha_0 C_{HT} \vect{\psi}_\mathrm{H} \vect{\psi}_\mathrm{T}^\mathsf{T} \odif{V} \notag                                                                               \\
%    & \quad + \iiint_{V} \nabla\vect{\psi}_\mathrm{H}^\mathsf{T} \bigl( \rho_\mathrm{w} K_\mathrm{wT} + \rho_\mathrm{w} K_\mathrm{vT} \bigr) \nabla\vect{\psi}_\mathrm{T} \odif{V} \notag \\
%    & \quad + \iiint_{V} \nabla\vect{\psi}_\mathrm{H}^\mathsf{T} \vect{V}_{HT} \vect{\psi}_\mathrm{T}^\mathsf{T} \odif{V}
% \end{align}
% ここで,$C_{HT} = \pdv*{\rho_\mathrm{void}}{T}$ は温度変化に伴う水分貯留量の変化率である.
% $K_\mathrm{wT}, K_\mathrm{vT}$ は温度勾配による水分輸送係数である.
% また,$\vect{V}_{HT}$ は物性値(水密度や透水係数など)の温度依存性に起因する補正項(移流的な寄与)であり,以下のように定義される.
% \begin{equation}
%   \vect{V}_{HT} = \pdv{\vect{J}_\mathrm{m}}{T} \bigg|_{\nabla P, \nabla T \text{ const}}
% \end{equation}
\subsection{非線形方程式の線形化}
\label{Sec:LinearizedSystem}

\cref{Sec:ThermalFEM,Sec:HydraulicFEM}で導出した熱移動・水分移動の有限要素離散化方程式は,非線形連立方程式系となる.これらを数値的に解くため,Newton-Raphson法による線形化を行う.

\subsubsection{Newton-Raphson法による定式化}
熱移動および水分移動の支配方程式に対し,残差ベクトル$\vect{R}_\mathrm{T}$,$\vect{R}_\mathrm{H}$を次のように定義する.
\begin{subequations}
  \begin{align}
    \vect{R}_\mathrm{T}\pab{\vect{T}, \vect{P}\,} & = \vect{0} \\[2mm]
    \vect{R}_\mathrm{H}\pab{\vect{T}, \vect{P}\,} & = \vect{0}
  \end{align}
\end{subequations}
ここで,$\vect{T}$および$\vect{P}$はそれぞれ節点温度および節点間隙水圧ベクトルである.
第$m$回目の反復ステップにおいて,残差ベクトルをTaylor展開し,2次の項を無視して線形化すると次式が得られる.
\begin{align}
  \vect{R}_\mathrm{T}\pab{\vect{T}_{n+1}^{m+1}, \vect{P}_{n+1}^{m+1}}
   & \approx \vect{R}_\mathrm{T}\pab{\vect{T}_{n+1}^{m}, \vect{P}_{n+1}^{m}}
  + \pdv{\vect{R}_\mathrm{T}}{\vect{T}} \bigg|_{m} \adif{\vect{T}_{n+1}^{m+1}}
  + \pdv{\vect{R}_\mathrm{T}}{\vect{P}} \bigg|_{m} \adif{\vect{P}_{n+1}^{m+1}} \notag \\
   & = \vect{0}
\end{align}
ここで,$\adif{\vect{T}_{n+1}^{m+1}} = \vect{T}_{n+1}^{m+1} - \vect{T}_{n+1}^{m}$ および $\adif{\vect{P}_{n+1}^{m+1}} = \vect{P}_{n+1}^{m+1} - \vect{P}_{n+1}^{m}$ は更新量を表す.
水分移動方程式についても同様に展開し,これを行列形式で整理すると以下の更新式が得られる.
\begin{equation}
  \begin{bmatrix}
    \mat{J}_\mathrm{TT} & \mat{J}_\mathrm{TP} \\
    \mat{J}_\mathrm{HT} & \mat{J}_\mathrm{HH}
  \end{bmatrix}_{n+1}^{m}
  \begin{Bmatrix}
    \adif{\vect{T}} \\
    \adif{\vect{P}}
  \end{Bmatrix}_{n+1}^{m+1}
  = -
  \begin{Bmatrix}
    \vect{R}_\mathrm{T} \\
    \vect{R}_\mathrm{H}
  \end{Bmatrix}_{n+1}^{m}
  \label{Eq:NR_System}
\end{equation}
ここで,ヤコビアン(接線剛性行列)の各ブロックは次のように定義される.
\begin{subequations}
  \begin{align}
    \label{Eq:Jacobian_Thermal_Blocks}
    \mat{J}_\mathrm{TT} & = \pdv{\vect{R}_\mathrm{T}}{\vect{T}} , \quad
    \mat{J}_\mathrm{TP} = \pdv{\vect{R}_\mathrm{T}}{\vect{P}}           \\[2mm]
    \label{Eq:Jacobian_Hydraulic_Blocks}
    \mat{J}_\mathrm{HT} & = \pdv{\vect{R}_\mathrm{H}}{\vect{T}} , \quad
    \mat{J}_\mathrm{HH} = \pdv{\vect{R}_\mathrm{H}}{\vect{P}}
  \end{align}
\end{subequations}
各反復ステップにおいて式\eqref{Eq:NR_System}を解き,残差ベクトルのノルムが許容値以下になるまで解を更新する.

\subsubsection{ヤコビアンおよび残差ベクトルの詳細}
各項の具体的な導出について述べる.
変数 $X$ の時刻 $t_{n+1}$ における時間微分項は,時間ステップ情報を含んだBDF係数 $\beta_k$ \unit{[\second^{-1}]} を用いて以下のように近似される.
\begin{equation}
  \eval{\pdv{X}{t}}{n+1} \approx \beta_0 X_{n+1} + X_\mathrm{hist}, \quad X_\mathrm{hist} = \sum_{k=1}^{k} \beta_k X_{n+1-k}
\end{equation}
ここで,$X_\mathrm{hist}$ は過去の時刻の既知量から定まる履歴項である.

残差ベクトルは,内力項,貯留・慣性項,および外力項の和として構成される.ここで,内力項に含まれるフラックス項は,発散定理(Green-Gaussの定理)により符号が反転することに注意する.
\begin{align}
  \vect{R}_\mathrm{T}\pab{\vect{T}_{n+1}^{m}, \vect{P}_{n+1}^{m}}
   & = \vect{F}^\mathrm{T}_\text{internal} + \vect{F}^\mathrm{T}_\text{transient} - \vect{F}^\mathrm{T}_\text{external} \notag                                                                                                                               \\
   & = \iiint_{V} \bab{-\nabla\vect{\psi}_\mathrm{T}^\mathsf{T} \cdot \vect{j}_\mathrm{E} + \vect{\psi}_\mathrm{T}^\mathsf{T} \pab{\beta_0 \mathcal{U}_{n+1} + \mathcal{U}_\mathrm{hist}} - \vect{\psi}_\mathrm{T}^\mathsf{T} S_\mathrm{T} } \odif{V} \notag \\
   & \quad - \iint_{\partial V} \vect{\psi}_\mathrm{T}^\mathsf{T} \vect{Q}_\mathrm{b} \odif{S}
  \label{Eq:Residual-Thermal}
\end{align}
\begin{align}
  \label{Eq:Residual-Water}
  \vect{R}_\mathrm{H}\pab{\vect{T}_{n+1}^{m}, \vect{P}_{n+1}^{m}}
   & = \vect{F}^\mathrm{H}_\text{internal} + \vect{F}^\mathrm{H}_\text{transient} - \vect{F}^\mathrm{H}_\text{external} \notag                                                                                                              \\
   & = \iiint_{V} \bab{-\nabla\vect{\psi}_\mathrm{H}^\mathsf{T} \cdot \vect{J}_\mathrm{m} + \vect{\psi}_\mathrm{H} \pab{\beta_0 \rho_\mathrm{void, n+1} + \rho_\mathrm{void, hist}} - \vect{\psi}_\mathrm{H} S_\mathrm{H} } \odif{V} \notag \\
   & \quad - \iint_{\Gamma_\mathrm{HN}} \vect{\psi}_\mathrm{H} q_{\mathrm{HN}} \odif{S}
\end{align}
ここで,各項の構成は以下の通りである.

\begin{description}
  \item [熱外力ベクトル:] $\vect{F}^\mathrm{T}_\text{external}$ \\
        境界からの既知の熱流入および定数的な熱源項である.
        \begin{align}
          \vect{F}^\mathrm{T}_\text{external}
          % Neumann (influx is -q_N if q_N is outward)
           & = - \iint_{\partial V_\mathrm{HN}} q_\mathrm{N} \vect{\psi}_\mathrm{T} \odif{S}
          % Robin constant part
          + \iint_{\partial V_\mathrm{HC}} \gamma \vect{\psi}_\mathrm{T} \odif{S} \notag              \\
          % Heat Transfer constant part
           & \quad + \iint_{\partial V_\mathrm{HH}} h  T_\mathrm{env} \vect{\psi}_\mathrm{T} \odif{S}
          % Radiation constant part
          + \iint_{\partial V_\mathrm{HR}} \beta_\mathrm{HR} T_\mathrm{r} \vect{\psi}_\mathrm{T} \odif{S}
          \label{Eq:F_external_expansion}
        \end{align}

  \item [熱内力ベクトル:] $\vect{F}^\mathrm{T}_\text{internal}$ \\
        熱伝導,移流,および温度依存の境界条件に由来する項である.$\vect{j}_\mathrm{E}$ に含まれる熱伝導項($-\tensor{R}\nabla T$)は,式\eqref{Eq:Residual-Thermal}の$-\nabla\vect{\psi}_\mathrm{T}^\mathsf{T} \cdot \vect{j}_\mathrm{E}$により正の剛性行列となる.
        \begin{align}
          \vect{F}^\mathrm{T}_\text{internal}
           & = \iiint_{V} \nabla\vect{\psi}_\mathrm{T}^\mathsf{T} \tensor{R} \nabla\vect{\psi}_\mathrm{T} \vect{T} \odif{V}
          - \iiint_{V} \nabla\vect{\psi}_\mathrm{T}^\mathsf{T} \pab{c_\mathrm{w} \rho_\mathrm{w} \vect{j}_\mathrm{WL} + c_\mathrm{v}\rho_\mathrm{w} \vect{j}_\mathrm{WV}} \vect{\psi}_\mathrm{T}^\mathsf{T} \vect{T} \odif{V} \notag \\
           & \quad - \iiint_{V} \nabla\vect{\psi}_\mathrm{T}^\mathsf{T} \pab{\rho_\mathrm{w} L_\mathrm{v} \vect{j}_\mathrm{WV}} \odif{V}
          - \iiint_{V} \vect{\psi}_\mathrm{T} S_\mathrm{T} \odif{V}\notag                                                                                                                                                            \\
           & \quad + \iint_{\partial V_\mathrm{HC}} \beta \vect{\psi}_\mathrm{T} \vect{\psi}_\mathrm{T}^\mathsf{T} \vect{T} \odif{S}
          + \iint_{\partial V_\mathrm{HH}} h \vect{\psi}_\mathrm{T} \vect{\psi}_\mathrm{T}^\mathsf{T} \vect{T} \odif{S}
          + \iint_{\partial V_\mathrm{HR}} \beta_\mathrm{HR} \vect{\psi}_\mathrm{T} \vect{\psi}_\mathrm{T}^\mathsf{T} \vect{T} \odif{S} \notag
        \end{align}

  \item [熱貯留・慣性ベクトル:] $\vect{F}^\mathrm{T}_\text{transient}$ \\
        内部エネルギーの時間変化項であり,BDF近似により次のように表される.
        \begin{equation}
          \vect{F}^\mathrm{T}_\text{transient}
          = \beta_0 \iiint_{V} \vect{\psi}_\mathrm{T}^\mathsf{T} \mathcal{U}\pab{\vect{T}_{n+1}, \vect{P}_{n+1}} \odif{V}
          + \iiint_{V} \vect{\psi}_\mathrm{T}^\mathsf{T} \mathcal{U}_\mathrm{hist} \odif{V}
        \end{equation}

  \item [水分外力ベクトル:] $\vect{F}^\mathrm{H}_\text{external}$ \\
        境界からの既知の水分流入および定数的な流出項である.
        \begin{equation}
          \vect{F}^\mathrm{H}_\text{external}
          = - \iint_{\Gamma_\mathrm{HN}} q_{\mathrm{HN}} \vect{\psi}_\mathrm{H} \odif{S}
          \label{Eq:F_H_external_expansion}
        \end{equation}

  \item [水分内力ベクトル:] $\vect{F}^\mathrm{H}_\text{internal}$ \\
        流体移動および圧力依存の境界条件に由来する項である.
        \begin{equation}
          \vect{F}^\mathrm{H}_\text{internal}
          = -\iiint_{V} \nabla\vect{\psi}_\mathrm{H}^\mathsf{T} \cdot \vect{J}_\mathrm{m} \odif{V}
          - \iiint_{V} \vect{\psi}_\mathrm{H} S_\mathrm{H} \odif{V}
        \end{equation}
  \item [水分貯留・慣性ベクトル:] $\vect{F}^\mathrm{H}_\text{transient}$ \\
        孔隙水量の時間変化項であり,BDF近似により次のように表される.
        \begin{equation}
          \vect{F}^\mathrm{H}_\text{transient}
          = \beta_0 \iiint_{V} \vect{\psi}_\mathrm{H} \rho_\mathrm{void}\pab{\vect{T}_{n+1}, \vect{P}_{n+1}} \odif{V}
          + \iiint_{V} \vect{\psi}_\mathrm{H} \rho_\mathrm{void, \mathrm{hist}} \odif{V}
        \end{equation}
\end{description}
\eqref{Eq:Jacobian_Thermal_Blocks}の各ヤコビアンブロックは,残差ベクトル\eqref{Eq:Residual-Thermal}の各項を変数$\vect{T}$および$\vect{P}$で偏微分することで得られる.
以下に,具体的な式を示す.
\begin{description}
  \item [熱-熱ブロック:] $\mat{J}_\mathrm{TT}$ \\
        温度変化に対する熱収支の応答を表す.式\eqref{Eq:J_TT_def}の各項は順に,熱容量,熱伝導,顕熱移流,および水蒸気移動に伴う潜熱輸送の寄与を示す.
        \begin{equation}
          \begin{split}
            \mat{J}_\mathrm{TT}
             & = \pdv{\vect{F}^\mathrm{T}_\text{internal}}{\vect{T}} + \pdv{\vect{F}^\mathrm{T}_\text{transient}}{\vect{T}} \notag                                                                                              \\
             & = \beta_0 \iiint_{V} C_\mathrm{TT} \vect{\psi}_\mathrm{T} \vect{\psi}_\mathrm{T}^\mathsf{T} \odif{V}
            + \iiint_{V} \nabla\vect{\psi}_\mathrm{T}^\mathsf{T} \tensor{R} \nabla\vect{\psi}_\mathrm{T} \odif{V}                                                                                                               \\
             & \quad - \iiint_{V} \nabla\vect{\psi}_\mathrm{T}^\mathsf{T} \pab{c_\mathrm{w} \rho_\mathrm{w} \vect{j}_\mathrm{WL} + c_\mathrm{v}\rho_\mathrm{w} \vect{j}_\mathrm{WV}} \vect{\psi}_\mathrm{T}^\mathsf{T} \odif{V}
            - \iiint_{V} \nabla\vect{\psi}_\mathrm{T}^\mathsf{T} \pab{\rho_\mathrm{w} L_\mathrm{v} K_\mathrm{vT}} \nabla\vect{\psi}_\mathrm{T} \odif{V}                                                                         \\
             & \quad + \iint_{\partial V_\mathrm{HC}} \beta \vect{\psi}_\mathrm{T} \vect{\psi}_\mathrm{T}^\mathsf{T} \odif{S}
            + \iint_{\partial V_\mathrm{HH}} h \vect{\psi}_\mathrm{T} \vect{\psi}_\mathrm{T}^\mathsf{T} \odif{S}
            + \iint_{\partial V_\mathrm{HR}} \beta_\mathrm{HR} \vect{\psi}_\mathrm{T} \vect{\psi}_\mathrm{T}^\mathsf{T} \odif{S}
          \end{split}
          \label{Eq:J_TT_def}
        \end{equation}
  \item [熱-水ブロック:] $\mat{J}_\mathrm{TH}$ \\
        間隙水圧の変化が熱収支に与える影響を表す.第1項は相変化等に伴う内部エネルギー変化,第2項以降は圧力勾配の変化に起因する流速変動が,顕熱および潜熱輸送へ及ぼす影響項である.
        \begin{equation}
          \begin{split}
            \mat{J}_\mathrm{TH}
             & = \pdv{\vect{F}^\mathrm{T}_\text{internal}}{\vect{P}} + \pdv{\vect{F}^\mathrm{T}_\text{transient}}{\vect{P}} \notag                                                                                                                                   \\
             & = \beta_0  \iiint_{V} C_\mathrm{TH} \vect{\psi}_\mathrm{T} \vect{\psi}_\mathrm{P}^\mathsf{T} \odif{V}                                                                                                                                                 \\
             & \quad + \iiint_{V} \nabla\vect{\psi}_\mathrm{T}^\mathsf{T} \bab{\pab{c_\mathrm{w} \rho_\mathrm{w} K_\mathrm{wP} + c_\mathrm{v} \rho_\mathrm{w} K_\mathrm{vP}} \pab{\vect{\psi}_\mathrm{T}^\mathsf{T} \vect{T}}} \nabla\vect{\psi}_\mathrm{P} \odif{V} \\
             & \quad + \iiint_{V} \nabla\vect{\psi}_\mathrm{T}^\mathsf{T} \pab{\rho_\mathrm{w} L_\mathrm{v} K_\mathrm{vP}} \nabla\vect{\psi}_\mathrm{P} \odif{V}
          \end{split}
          \label{Eq:J_TP_def}
        \end{equation}

  \item[水-水ブロック:] $\mat{J}_\mathrm{HH}$ \\
        間隙水圧の変化に対する水分収支の応答を表す項である.
        \begin{align}
          \mat{J}_\mathrm{HH} = \pdv{\vect{R}_\mathrm{H}}{\vect{P}}
           & = \iiint_{V} \beta_0 C_\mathrm{HH} \vect{\psi}_\mathrm{H} \vect{\psi}_\mathrm{H}^\mathsf{T} \odif{V} \notag                   \\
           & \quad + \iiint_{V} \nabla\vect{\psi}_\mathrm{H}^\mathsf{T} \rho_\mathrm{w} K_\mathrm{P} \nabla\vect{\psi}_\mathrm{H} \odif{V}
        \end{align}

  \item[水-熱ブロック:] $\mat{J}_\mathrm{HT}$ \\
        温度変化が水分移動に与える影響を表す連成項である.
        \begin{align}
          \mat{J}_\mathrm{HT} = \pdv{\vect{R}_\mathrm{H}}{\vect{T}}
           & = \iiint_{V} \beta_0 C_\mathrm{HT} \vect{\psi}_\mathrm{H} \vect{\psi}_\mathrm{T}^\mathsf{T} \odif{V} \notag                          \\
           & \quad + \iiint_{V} \nabla\vect{\psi}_\mathrm{H}^\mathsf{T} \rho_\mathrm{w} K_\mathrm{T} \nabla\vect{\psi}_\mathrm{T} \odif{V} \notag \\
           & \quad - \iiint_{V} \nabla\vect{\psi}_\mathrm{H}^\mathsf{T} \vect{V}_\mathrm{HT} \vect{\psi}_\mathrm{T}^\mathsf{T} \odif{V}
        \end{align}
        ここで,$C_\mathrm{HT}$ は温度変化に伴う水分貯留量の変化率である.
        $K_\mathrm{wT}, K_\mathrm{vT}$ は温度勾配による水分輸送係数である.
        また,$\vect{V}_\mathrm{HT}$ は物性値の温度依存性に起因する補正項であり,$\vect{R}_\mathrm{H}$のフラックス項が負であるため,ヤコビアンへの寄与は負となる.
        \begin{equation}
          \vect{V}_\mathrm{HT} = \pdv{\vect{J}_\mathrm{m}}{T} \bigg|_{\nabla P, \nabla T \text{ const}}
        \end{equation}
\end{description}

\subsection{修正Picard反復法による線形化}
\label{Sec:PicardLinearization}

本解析では,温度 $\vect{T}$ と間隙水圧 $\vect{P}$ を同時に解く完全連成解析(Monolithic Approach)を採用する.
非線形方程式系の解法には,接線剛性行列(Jacobian)の更新を必要としない修正Picard反復法を用いる.
この手法では,第 $m$ 回目の反復において係数行列内の物性値を既知の解($\vect{T}^{m}, \vect{P}^{m}$)で評価して固定し,線形化された方程式系を解くことで更新解 $\vect{T}^{m+1}, \vect{P}^{m+1}$ を得る.

離散化された線形方程式系は,以下のブロック行列形式で記述される.
\begin{equation}
  \label{Eq:Monolithic-Picard-System}
  \begin{bmatrix}
    \mat{K}_\mathrm{TT} & \mat{K}_\mathrm{TH} \\
    \mat{K}_\mathrm{HT} & \mat{K}_\mathrm{HH}
  \end{bmatrix}^{m}
  \begin{Bmatrix}
    \vect{T} \\
    \vect{P}
  \end{Bmatrix}^{m+1}
  =
  \begin{Bmatrix}
    \vect{F}_\mathrm{T} \\
    \vect{F}_\mathrm{H}
  \end{Bmatrix}^{m}
\end{equation}
ここで,$\mat{K}$は一般化されたシステム行列,$\vect{F}_\mathrm{RHS}$は一般化された外力ベクトルを表す.

\subsubsection{熱移動ブロックの詳細}
熱移動支配方程式に対応する上段のブロック($\mat{K}_\mathrm{TT}, \mat{K}_\mathrm{TH}$)および右辺ベクトル $\vect{F}_\mathrm{T}$ は,以下の通り定義される.
ここで,非線形性の強い移流項(顕熱・潜熱移動)は右辺ベクトルに組み込むことで,左辺行列の安定化を図っている.

\begin{description}
  \item[熱-熱ブロック:] $\mat{K}_\mathrm{TT}$ \\
        温度変化に対する慣性項(熱容量)と熱伝導項,および温度に依存する境界条件項から構成される.
        \begin{align}
          \mat{K}_\mathrm{TT}
           & = \iiint_{V} \beta_0 C_\mathrm{TT} \vect{\psi}_\mathrm{T} \vect{\psi}_\mathrm{T}^\mathsf{T} \odif{V}
          + \iiint_{V} \nabla\vect{\psi}_\mathrm{T}^\mathsf{T} \tensor{R} \nabla\vect{\psi}_\mathrm{T} \odif{V} \notag      \\
           & \quad + \iint_{\partial V_\mathrm{HC}} \beta \vect{\psi}_\mathrm{T} \vect{\psi}_\mathrm{T}^\mathsf{T} \odif{S}
          + \iint_{\partial V_\mathrm{HH}} h \vect{\psi}_\mathrm{T} \vect{\psi}_\mathrm{T}^\mathsf{T} \odif{S}
          + \iint_{\partial V_\mathrm{HR}} \beta_\mathrm{HR} \vect{\psi}_\mathrm{T} \vect{\psi}_\mathrm{T}^\mathsf{T} \odif{S}
        \end{align}

  \item[熱-水ブロック:] $\mat{K}_\mathrm{TH}$ \\
        水圧変化が熱収支に与える影響を表す連成項である.ここでは相変化や密度変化に伴う熱容量の連成成分 $C_\mathrm{TH}$ を考慮する.
        \begin{equation}
          \mat{K}_\mathrm{TH} = \iiint_{V} \beta_0 C_\mathrm{TH} \vect{\psi}_\mathrm{T} \vect{\psi}_\mathrm{P}^\mathsf{T} \odif{V}
        \end{equation}

  \item [熱外力ベクトル:] $\vect{F}_\mathrm{T}$ \\
        境界条件からの既知の流入出,内部発熱に加え,前回の反復値で評価された移流フラックス,および時間積分の履歴項から構成される.
        \begin{align}
          \vect{F}_\mathrm{T}
           & = - \iint_{\partial V_\mathrm{HN}} q_\mathrm{N} \vect{\psi}_\mathrm{T} \odif{S}
          - \iint_{\partial V_\mathrm{HC}} \gamma \vect{\psi}_\mathrm{T} \odif{S} \notag                                                       \\
           & \quad + \iint_{\partial V_\mathrm{HH}} h T_\mathrm{env} \vect{\psi}_\mathrm{T} \odif{S}
          + \iint_{\partial V_\mathrm{HR}} \beta_\mathrm{HR} T_\mathrm{r} \vect{\psi}_\mathrm{T} \odif{S} \notag                               \\
           & \quad + \iiint_{V} \nabla\vect{\psi}_\mathrm{T}^\mathsf{T} \cdot \vect{j}_\mathrm{adv} \odif{V}
          + \iiint_{V} \vect{\psi}_\mathrm{T} S_\mathrm{T} \odif{V} \notag                                                                     \\
           & \quad - \iiint_{V} \vect{\psi}_\mathrm{T}^\mathsf{T} \pab{C_\mathrm{TT} T_\mathrm{hist} + C_\mathrm{TH} P_\mathrm{hist}} \odif{V}
        \end{align}
        ここで,$\vect{j}_\mathrm{adv}$ は前回の反復値を用いて算定された,流体移動に伴う顕熱および潜熱フラックスの総和である.
        \begin{equation}
          \vect{j}_\mathrm{adv} = \pab{c_\mathrm{w} \rho_\mathrm{w} \vect{j}_\mathrm{WL} T + c_\mathrm{v} \rho_\mathrm{w} \vect{j}_\mathrm{WV} T + \rho_\mathrm{w} L_\mathrm{v} \vect{j}_\mathrm{WV}}
        \end{equation}
        また,$T_\mathrm{hist}, P_\mathrm{hist}$ はBDF法における時間微分の履歴項 $\sum \beta_k X_{n+1-k}$ に対応する.
\end{description}

\subsubsection{水分移動ブロックの詳細}
水分移動支配方程式に対応する下段のブロックおよび右辺ベクトルについても,熱移動と同様に構築される.
水分移動における非線形項(透水係数など)は,Picard反復により前回の反復値を用いて線形化される.

\begin{description}
  \item[水-水ブロック:] $\mat{K}_\mathrm{HH}$ \\
        \begin{equation}
          \mat{K}_\mathrm{HH}
          = \iiint_{V} \beta_0 C_\mathrm{HH} \vect{\psi}_\mathrm{H} \vect{\psi}_\mathrm{H}^\mathsf{T} \odif{V}
          + \iiint_{V} \nabla\vect{\psi}_\mathrm{H}^\mathsf{T} \tensor{K}_\mathrm{H} \nabla\vect{\psi}_\mathrm{H} \odif{V}
        \end{equation}
        ここで,$\tensor{K}_\mathrm{H} = \rho_\mathrm{w}(K_\mathrm{wP} + K_\mathrm{vP})$ は物質移動係数の総和である.

  \item[水-熱ブロック:] $\mat{K}_\mathrm{HT}$ \\
        \begin{equation}
          \mat{K}_\mathrm{HT}
          = \iiint_{V} \beta_0 C_\mathrm{HT} \vect{\psi}_\mathrm{H} \vect{\psi}_\mathrm{T}^\mathsf{T} \odif{V}
          + \iiint_{V} \nabla\vect{\psi}_\mathrm{H}^\mathsf{T} \tensor{K}_\mathrm{T} \nabla\vect{\psi}_\mathrm{T} \odif{V}
        \end{equation}
        ここで,$\tensor{K}_\mathrm{T} = \rho_\mathrm{w}(K_\mathrm{wT} + K_\mathrm{vT})$ は温度勾配に起因する水分移動係数である.

  \item[水外力ベクトル:] $\vect{F}_\mathrm{H}$ \\
        重力項およびソース項を右辺に配置する.重力項は発散定理により正の符号で加算される.
        \begin{align}
          \vect{F}_\mathrm{H}
           & = - \iint_{\Gamma_\mathrm{HN}} q_{\mathrm{HN}} \vect{\psi}_\mathrm{H} \odif{S}
          + \iiint_{V} \nabla\vect{\psi}_\mathrm{H}^\mathsf{T} \cdot \pab{\rho_\mathrm{w} K_\mathrm{wP} \vect{g}} \odif{V} \notag                                                                        \\
           & \quad + \iiint_{V} \vect{\psi}_\mathrm{H} S_\mathrm{H} \odif{V} - \iiint_{V} \vect{\psi}_\mathrm{H}^\mathsf{T} \pab{C_\mathrm{HH} P_\mathrm{hist} + C_\mathrm{HT} T_\mathrm{hist}} \odif{V}
        \end{align}
\end{description}

\numberwithin{equation}{section}

\FloatBarrier