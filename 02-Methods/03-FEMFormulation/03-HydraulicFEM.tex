\subsection{水分移動支配方程式の離散化}
\label{Sec:HydraulicFEM}

水分移動に対する境界条件は,全水分質量フラックス $\vect{J}_\mathrm{m} = \rho_\mathrm{w} \pab{\vect{j}_\mathrm{WL} + \vect{j}_\mathrm{WV}}$(液状水および水蒸気の和)を用いて以下のように記述される.
\begin{ConditionBox}{圧力境界条件}{Hydraulic_BC}
  領域 $V$ における間隙水圧場 $P\,\pab{\vect{x},t}$ は,境界 $\partial V$ 上において,以下に示す代表的な圧力境界条件を満たすものとする.
  \begin{subequations}
    \label{Eq:water-BC}
    \begin{description}
      \item[(a) 圧力既定境界 (Dirichlet 境界条件)]
            境界 $\Gamma_\mathrm{HD}$ 上において間隙水圧が既定される場合,
            \begin{equation}
              P\,\pab{\vect{x}, t}
              = P_{\mathrm{D}}\pab{\vect{x}}
              \qquad \text{for } \vect{x}\in\Gamma_\mathrm{HD}
            \end{equation}
      \item[(b) 水分フラックス既定境界 (Neumann 境界条件)]
            境界 $\Gamma_\mathrm{HN}$ 上において,全水分質量フラックス $\vect{J}_\mathrm{m}$ の法線成分が既定される場合,
            \begin{equation}
              \vect{J}_\mathrm{m}\pab{\vect{x}, t} \cdot \vect{n}
              = q_{\mathrm{HN}}\pab{\vect{x},t}
              \qquad \text{for } \vect{x}\in\Gamma_\mathrm{HN}
            \end{equation}
            ここで,$q_{\mathrm{HN}}$ は境界を通して流出する正味の水分フラックスである.
      \item[(c) 大気境界条件]
            地表面境界 $\Gamma_\mathrm{ATM}$ では,気象条件と土壌の浸透・保水性に応じ,境界条件型を動的に切り替える系依存型境界条件を適用する.
            鉛直上向き正の $z$ 軸に対し,可能蒸発散量 $E_\mathrm{potential}$ および降水量 $q_\mathrm{rain}$ から,正味の可能水分フラックス $q_{\mathrm{potential}}$ を次式で定義する.
            \begin{equation}
              q_{\mathrm{potential}}\pab{\vect{x},t} = E_\mathrm{potential}\pab{\vect{x},t} - q_\mathrm{rain}\pab{\vect{x},t}
            \end{equation}
            ここでは外向き法線ベクトル $\vect{n}$ が $+z$ 方向であるため,正の値は蒸発として系外への流出し,負の値は降雨に伴う系内への流入を表す.
            実際の境界条件は,地表面の許容圧力範囲(乾燥限界 $P_{\min}$ および湛水限界 $P_{\max}$)に基づき,以下の制約を満たすように決定される.
            \begin{equation}
              \begin{cases}
                \abs{\vect{J}_\mathrm{m}\pab{\vect{x}, t} \cdot \vect{n}} \le \abs{q_{\mathrm{potential}}} \\[3pt]
                P_{\min} \le P\pab{\vect{x},t} \le P_{\max}
              \end{cases}
              \qquad \text{for } \vect{x}\in\Gamma_\mathrm{ATM}
            \end{equation}
            ここで,$P_{\max}$が$0$のときは,地表面に湛水せず,即座に流出する条件となる.一方,$P_{\max}>0$のときは,地表面にある一定の湛水深が許容される.
            この条件は物理的に以下の2つの状況を包含している.
            \begin{description}
              \item[1. 降雨・浸透過程]:
                    土壌の浸透能が降雨強度を上回る間,つまり $q_{\mathrm{potential}}<0$ のときは,降雨量すべてが流入するフラックス境界となる.浸透能を超過し地表面が飽和すると,圧力境界 $P = P_{\max}$ に切り替わり,差分の水量は表面流出(Surface runoff)として扱われる.
              \item[2. 蒸発・乾燥過程]:
                    土壌水分が十分に存在するとき,つまり $q_{\mathrm{potential}}>0$ のときは,可能蒸発散量が要求する値でのフラックス境界となる.土壌が乾燥し限界圧力 $P_{\min}$ に達すると,圧力境界 $P = P_{\min}$ に固定され,実際の蒸発量は土壌の水分供給能力によって制限される.
            \end{description}
    \end{description}
  \end{subequations}
\end{ConditionBox}

水分移動に関する支配方程式\eqref{Eq:Continuity_void}を有限要素法で離散化する.
形状関数ベクトルを $\vect{\psi}_\mathrm{H}$ とすれば,Galerkin法に基づく弱形式は以下のように記述される.
\begin{equation}
  \label{Eq:Hydraulic_weak_form}
  \iiint_{V} \vect{\psi}_\mathrm{H} \bab{\pdv{\rho_\mathrm{void}}{t} + \nabla\cdot \vect{J}_\mathrm{m} + S_\mathrm{H}} \odif{V} = \vect{0}
\end{equation}
\eqref{Eq:Thermal-Weak-Form-Divergence}と同様にガウスの発散定理を用いて空間微分項を変形し,境界条件を代入すると,弱形式は次のように書き換えられる.
\begin{align}
  \label{Eq:Hydraulic_weak_form_final}
   & \iiint_{V} \bab{ \vect{\psi}_\mathrm{H} \pdv{\rho_\mathrm{void}}{t} - \nabla\vect{\psi}_\mathrm{H}^\mathsf{T} \cdot \vect{J}_\mathrm{m} + \vect{\psi}_\mathrm{H} S_\mathrm{H} } \odif{V} \notag \\
   & \quad + \iint_{\Gamma_\mathrm{HN}} \vect{\psi}_\mathrm{H} q_{\mathrm{HN}} \odif{S} = 0
\end{align}

% 時間微分項 $\pdv{\rho_\mathrm{void}}/{t}$ に対して $k$ 次の BDF を適用し,時刻 $t_{n+1}$ における残差ベクトル $\vect{R}_\mathrm{H}$ を定義する.
% 残差ベクトルは,外力ベクトル,内力ベクトル,および貯留・慣性ベクトルの和として整理される.
% \begin{align}
%   \label{Eq:Residual-Water}
%   \vect{R}_\mathrm{H}\pab{\vect{T}_{n+1}^{m}, \vect{P}_{n+1}^{m}}
%    & = \vect{F}^\mathrm{H}_\text{external} - \vect{F}^\mathrm{H}_\text{internal} - \vect{F}^\mathrm{H}_\text{transient} = \vect{0}
% \end{align}

% \begin{enumerate}
%   \item 外力ベクトル $\vect{F}^\mathrm{H}_\text{external}$:
%         境界条件によって領域表面から流入する水分フラックスの寄与である.
%         \begin{equation}
%           \vect{F}^\mathrm{H}_\text{external} = - \iint_{\Gamma_\mathrm{HN}} j_{\mathrm{w},\mathrm{N}} \vect{\psi}_\mathrm{H} \odif{S}
%         \end{equation}
%         ここで,流出フラックス $j_{\mathrm{w},\mathrm{N}}$ に対して流入方向を正とするためマイナス符号が付く.

%   \item 内力ベクトル $\vect{F}^\mathrm{H}_\text{internal}$:
%         領域内部の水分移動および内部シンク項である.
%         全水分質量フラックス $\vect{J}_\mathrm{m}$ の各成分(液状水・水蒸気)を展開して記述すると以下のようになる.
%         \begin{align}
%           \vect{F}^\mathrm{H}_\text{internal}
%            & = - \iiint_{V} \nabla\vect{\psi}_\mathrm{H}^\mathsf{T} \cdot \vect{J}_\mathrm{m} \odif{V}
%           + \iiint_{V} \vect{\psi}_\mathrm{H} S_\mathrm{H} \odif{V} \notag                                                                                   \\
%            & = - \iiint_{V} \nabla\vect{\psi}_\mathrm{H}^\mathsf{T} \cdot \rho_\mathrm{w} \bigl( \vect{j}_\mathrm{WL} + \vect{j}_\mathrm{WV} \bigr) \odif{V}
%           + \iiint_{V} \vect{\psi}_\mathrm{H} S_\mathrm{H} \odif{V}
%         \end{align}
%         ここで,$\vect{j}_\mathrm{WL}$ および $\vect{j}_\mathrm{WV}$ は,それぞれ $P$ および $T$ の勾配に依存する(式\eqref{Eq:LiquidFlux_final3}, \eqref{Eq:VaporFlux_final}参照).

%   \item 貯留・慣性ベクトル $\vect{F}^\mathrm{H}_\text{transient}$:
%         間隙内の水分貯留量の時間変化項である.
%         \begin{align}
%           \vect{F}^\mathrm{H}_\text{transient}
%            & = \iiint_{V} \vect{\psi}_\mathrm{H} \pab{ \alpha_0 \rho_\mathrm{void}^{n+1} + \rho_\mathrm{void}^\mathrm{hist} } \odif{V}
%         \end{align}
%         ここで,$\rho_\mathrm{void} = \rho_\mathrm{w} \phi S_\mathrm{w} + \rho_\mathrm{v} \phi (1-S_\mathrm{w})$ は間隙中の総水分密度である.
% \end{enumerate}

% Newton-Raphson法による解法のため,残差ベクトル $\vect{R}_\mathrm{H}$ を未知変数 $\vect{P}$ および $\vect{T}$ で偏微分し,ヤコビアン(接線剛性行列)を導出する.
% \begin{equation}
%   \mathbf{J}_\mathrm{hydraulic} =
%   \begin{bmatrix}
%     \mathbf{J}_{HP} & \mathbf{J}_{HT}
%   \end{bmatrix}
%   =
%   \begin{bmatrix}
%     \pdv{\vect{R}_\mathrm{H}}{\vect{P}} & \pdv{\vect{R}_\mathrm{H}}{\vect{T}}
%   \end{bmatrix}
% \end{equation}

% \subsubsection{水-水ブロック ($\mat{J}_\mathrm{HH}$)}
% 間隙水圧の変化に対する水分収支の応答を表す項である.
% 水分容量(貯留項の微分)および透水・通気特性(フラックス項の微分)が含まれる.
% \begin{align}
%   \mat{J}_\mathrm{HH} = \pdv{\vect{R}_\mathrm{H}}{\vect{P}}
%    & = \iiint_{V} \alpha_0 C_{HP} \vect{\psi}_\mathrm{H} \vect{\psi}_\mathrm{H}^\mathsf{T} \odif{V} \notag
%    & \quad + \iiint_{V} \nabla\vect{\psi}_\mathrm{H}^\mathsf{T} \bigl( \rho_\mathrm{w} K_\mathrm{wP} + \rho_\mathrm{w} K_\mathrm{vP} \bigr) \nabla\vect{\psi}_\mathrm{H} \odif{V}
% \end{align}
% ここで,$C_{HP} = \pdv*{\rho_\mathrm{void}}{P}$ は有効水分容量,$K_\mathrm{wP}, K_\mathrm{vP}$ はそれぞれ液状水および水蒸気の水圧勾配に関する輸送係数である.
% なお,フラックス項の微分において $\vect{J}_\mathrm{m}$ が $-\nabla P$ に比例するため,微分後の符号は正($+$)となる.

% \subsubsection{水-熱ブロック ($\mat{J}_\mathrm{HT}$)}
% 温度変化が水分移動に与える影響を表す連成項である.
% 温度変化による密度変化や飽和度変化(容量項),および温度勾配による水分移動(Soret効果など)が含まれる.
% \begin{align}
%   \mat{J}_\mathrm{HT} = \pdv{\vect{R}_\mathrm{H}}{\vect{T}}
%    & = \iiint_{V} \alpha_0 C_{HT} \vect{\psi}_\mathrm{H} \vect{\psi}_\mathrm{T}^\mathsf{T} \odif{V} \notag                                                                               \\
%    & \quad + \iiint_{V} \nabla\vect{\psi}_\mathrm{H}^\mathsf{T} \bigl( \rho_\mathrm{w} K_\mathrm{wT} + \rho_\mathrm{w} K_\mathrm{vT} \bigr) \nabla\vect{\psi}_\mathrm{T} \odif{V} \notag \\
%    & \quad + \iiint_{V} \nabla\vect{\psi}_\mathrm{H}^\mathsf{T} \vect{V}_{HT} \vect{\psi}_\mathrm{T}^\mathsf{T} \odif{V}
% \end{align}
% ここで,$C_{HT} = \pdv*{\rho_\mathrm{void}}{T}$ は温度変化に伴う水分貯留量の変化率である.
% $K_\mathrm{wT}, K_\mathrm{vT}$ は温度勾配による水分輸送係数である.
% また,$\vect{V}_{HT}$ は物性値(水密度や透水係数など)の温度依存性に起因する補正項(移流的な寄与)であり,以下のように定義される.
% \begin{equation}
%   \vect{V}_{HT} = \pdv{\vect{J}_\mathrm{m}}{T} \bigg|_{\nabla P, \nabla T \text{ const}}
% \end{equation}