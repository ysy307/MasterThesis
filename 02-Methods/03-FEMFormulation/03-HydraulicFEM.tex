\subsection{水分移動支配方程式の離散化}
\label{Sec:HydraulicFEM}

水分移動に対する境界条件は,全水分質量フラックス $\vect{J}_\mathrm{m} = \rho_\mathrm{w} \pab{\vect{j}_\mathrm{WL} + \vect{j}_\mathrm{WV}}$(液状水および水蒸気の和)を用いて以下のように記述される.
\begin{ConditionBox}{圧力境界条件}{Hydraulic_BC}
  領域 $V$ における間隙水圧場 $P\,\pab{\vect{x},t}$ は,境界 $\partial V$ 上において,以下に示す代表的な圧力境界条件を満たすものとする.
  \begin{subequations}
    \label{Eq:water-BC}
    \begin{description}
      \item[(a) 圧力既定境界 (Dirichlet 境界条件)]
            境界 $\Gamma_\mathrm{HD}$ 上において間隙水圧が既定される場合,
            \begin{equation}
              P\,\pab{\vect{x}, t}
              = P_{\mathrm{D}}\pab{\vect{x}}
              \qquad \text{for } \vect{x}\in\Gamma_\mathrm{HD}
            \end{equation}
      \item[(b) 水分フラックス既定境界 (Neumann 境界条件)]
            境界 $\Gamma_\mathrm{HN}$ 上において,全水分質量フラックス $\vect{J}_\mathrm{m}$ の法線成分が既定される場合,
            \begin{equation}
              \vect{J}_\mathrm{m}\pab{\vect{x}, t} \cdot \vect{n}
              = q_{\mathrm{HN}}\pab{\vect{x},t}
              \qquad \text{for } \vect{x}\in\Gamma_\mathrm{HN}
            \end{equation}
            ここで,$q_{\mathrm{HN}}$ は境界を通して流出する正味の水分フラックスである.
      \item[(c) 大気境界条件]
            地表面境界 $\Gamma_\mathrm{ATM}$ では,気象条件と土壌の浸透・保水性に応じ,境界条件型を動的に切り替える系依存型境界条件を適用する.
            鉛直上向き正の $z$ 軸に対し,可能蒸発散量 $E_\mathrm{potential}$ および降水量 $q_\mathrm{rain}$ から,正味の可能水分フラックス $q_{\mathrm{potential}}$ を次式で定義する.
            \begin{equation}
              q_{\mathrm{potential}}\pab{\vect{x},t} = E_\mathrm{potential}\pab{\vect{x},t} - q_\mathrm{rain}\pab{\vect{x},t}
            \end{equation}
            ここでは外向き法線ベクトル $\vect{n}$ が $+z$ 方向であるため,正の値は蒸発として系外への流出し,負の値は降雨に伴う系内への流入を表す.
            実際の境界条件は,地表面の許容圧力範囲(乾燥限界 $P_{\min}$ および湛水限界 $P_{\max}$)に基づき,以下の制約を満たすように決定される.
            \begin{equation}
              \begin{cases}
                \abs{\vect{J}_\mathrm{m}\pab{\vect{x}, t} \cdot \vect{n}} \le \abs{q_{\mathrm{potential}}} \\[3pt]
                P_{\min} \le P\pab{\vect{x},t} \le P_{\max}
              \end{cases}
              \qquad \text{for } \vect{x}\in\Gamma_\mathrm{ATM}
            \end{equation}
            ここで,$P_{\max}$が$0$のときは,地表面に湛水せず,即座に流出する条件となる.一方,$P_{\max}>0$のときは,地表面にある一定の湛水深が許容される.
            この条件は物理的に以下の2つの状況を包含している.
            \begin{description}
              \item[1. 降雨・浸透過程]:
                    土壌の浸透能が降雨強度を上回る間,つまり $q_{\mathrm{potential}}<0$ のときは,降雨量すべてが流入するフラックス境界となる.浸透能を超過し地表面が飽和すると,圧力境界 $P = P_{\max}$ に切り替わり,差分の水量は表面流出(Surface runoff)として扱われる.
              \item[2. 蒸発・乾燥過程]:
                    土壌水分が十分に存在するとき,つまり $q_{\mathrm{potential}}>0$ のときは,可能蒸発散量が要求する値でのフラックス境界となる.土壌が乾燥し限界圧力 $P_{\min}$ に達すると,圧力境界 $P = P_{\min}$ に固定され,実際の蒸発量は土壌の水分供給能力によって制限される.
            \end{description}
    \end{description}
  \end{subequations}
\end{ConditionBox}

水分移動に関する支配方程式\eqref{Eq:Continuity_void}を有限要素法で離散化する.
形状関数ベクトルを $\vect{\psi}_\mathrm{H}$ とすれば,Galerkin法に基づく弱形式は以下のように記述される.
\begin{equation}
  \label{Eq:Hydraulic_weak_form}
  \iiint_{V} \vect{\psi}_\mathrm{H} \bab{\pdv{\rho_\mathrm{void}}{t} + \nabla\cdot \vect{J}_\mathrm{m} + S_\mathrm{H}} \odif{V} = \vect{0}
\end{equation}
\eqref{Eq:Thermal-Weak-Form-Divergence}と同様にガウスの発散定理を用いて空間微分項を変形し,境界条件を代入すると,弱形式は次のように書き換えられる.
\begin{align}
  \label{Eq:Hydraulic_weak_form_final}
   & \iiint_{V} \bab{ \vect{\psi}_\mathrm{H} \pdv{\rho_\mathrm{void}}{t} - \nabla\vect{\psi}_\mathrm{H}^\mathsf{T} \cdot \vect{J}_\mathrm{m} + \vect{\psi}_\mathrm{H} S_\mathrm{H} } \odif{V} \notag \\
   & \quad + \iint_{\Gamma_\mathrm{HN}} \vect{\psi}_\mathrm{H} q_{\mathrm{HN}} \odif{S} = 0
\end{align}

\FloatBarrier