\subsection{時間積分法}
\label{Sec:TimeIntegration}

\subsubsection{時間積分法の概要}
\label{Sec:TimeIntegration_Overview}

有限要素法などにより空間方向を離散化すると,\cref{Sec:GoverningEquationDerivation}で導出した熱・水分移動の支配方程式系は,一般に次のような常微分方程式(あるいは微分代数方程式)系として表される.
\begin{equation}
  \label{Eq:Generic_ODE_System}
  \pdv{\vect{y}}{t} = \vect{F}\,\pab{\vect{y},t}
\end{equation}
ここで,$\vect{y}(t)$ は全自由度の集合,$\vect{F}$ は伝導・移流・相変化・供給項などを含む非線形作用素ベクトルである.数値計算では,この時間連続問題を,離散的な時刻列
\begin{equation}
  t_0 < t_1 < \dots < t_n < t_{n+1} < \dots < t_N, \qquad  \adif{t_n} \coloneq t_{n+1}-t_n
\end{equation}
上で定義された近似解列 $\Bab{\vect{y}^n}_{n=0}^N$ によって置き換える.このとき,各時間ステップでどのように $\vect{y}^{n+1}$ を求めるかを定める手続きが時間積分法である.

代表的な 1 ステップ時間積分法として,いわゆる $\theta$ 法が挙げられる.\eqref{Eq:Generic_ODE_System} に対して $\theta$ 法を適用すると,時刻 $t_n$ から $t_{n+1}$ への更新は
\begin{equation}
  \label{Eq:Theta_Method}
  \frac{\vect{y}^{n+1} - \vect{y}^{n}}{\adif{t_n}}
  = (1-\theta)\,\vect{F}\,\pab{\vect{y}^{n},t_n}
  + \theta\,\vect{F}\,\pab{\vect{y}^{n+1},t_{n+1}}
\end{equation}
のように記述される.ここで $\theta=0$ のときは陽解法(前進オイラー法),$\theta = 1/2$ のときはCrank–Nicolson法,$\theta=1$ のときは陰解法(後退オイラー法)となる.土壌凍結問題のように強い非線形性や硬い
% \footnote{偏微分方程式系が硬いとは,その固有値分布が広いことをいう.具体的には最小固有値$\lambda_\min$と最大固有値$\lambda_\max$の比である条件数$\kappa=\lambda_\max/\lambda_\min > 10^4$ 程度のものをさすことが多い}
特性をもつ系では,大きな時間刻みでも数値的安定性を保ちやすい陰解法($\theta \geq 1/2$),とりわけ $\theta=1$ の後退オイラー法が広く用いられる.

一方で,長期の凍結・融解過程を高精度に追跡するには,1 階精度の後退オイラー法だけでは時間離散化誤差が支配的となる場合がある.このため本研究では,陰的 1 ステップ法の枠組みを一般化した多段法である後退微分法(Backward Differentiation Formula, BDF)を採用する.BDF は,現在時刻 $t_{n+1}$ における時間微分を,現在値および過去 $k$ ステップ分の解を用いた線形結合として近似する多段陰解法であり,
\begin{equation}
  \label{Eq:BDF_k_Intro}
  \pdv{\vect{y}}{t}\bigg|_{t_{n+1}}
  \approx
  \frac{1}{\Delta t_n}
  \sum_{i=0}^{k} \alpha_i\,\vect{y}^{\,n+1-i}
\end{equation}
という形で表される.ここで $\alpha_i$ は BDF-$k$ に特有の係数であり,$k=1$ の場合には後退オイラー法に一致する.BDF は硬い方程式系に対して高い数値安定性を有し,$k$ を大きくすることで時間積分の高次精度化も可能である.

本研究の時間積分スキームでは,可変時間ステップおよび可変次数に対応した BDF 法を用いて,熱移動・水分移動の連成方程式系を時間方向に積分する.BDF 係数の具体的な導出方法や,可変ステップ・可変次数アルゴリズムの詳細については,\cref{Sec:BDF} にて述べる.

\subsubsection{可変時間ステップおよび可変次数における後退差分法}
\label{Sec:BDF}

後退差分法(Backward Differentiation Formula, BDF)は,時間積分において,過去の値を用いて現在の値を求める方法である.BDFは,特に支配方程式の系が硬い問題において有効であり,数値的な安定性が高い特徴を持つ.7次以上は零点安定性がなくなるので考える必要はなく,6次までのBDFを考えることとする.
時間微分項 $\pdv{\vect{y}}/{t}$ を近似するため,いくつかの過去の時点 $\pab{\vect{y}^i, t_i}$ を通るラグランジュ補間多項式 $Y\pab{t}$ を構築し,その現在時刻 $t_{n+1}$ での微分 $P'\pab{t_{n+1}}$ を用いる.
一般にLagrange補間多項式は,Lagrange基底関数 $l_i(x)$
\begin{align}
  l_i\pab{x} = \prod_{\substack{j=0\leq j \leq i \\[0.2mm] j \neq i}}\frac{x - x_j}{x_i - x_j}
\end{align}
と補間対象となる変数$y_i$との線形結合として与えられる.
\begin{equation}
  \label{Eq:bdf_lagrange}
  L\pab{x,t}\coloneq\sum_{i=0}^{j} l_i \pab{x} y_i \pab{t}
\end{equation}
計算を簡略化するため,局所的な時間座標 $\tau = t - t_{n+1}$ を導入する.この座標系では,現在時刻は $\tau=0$ となる.ただし,1次のBDFは単純な後退差分であるため,2次以上のBDFについて考える.本節では特に2次について詳述するとともに,6次までのBDFはそれまでと同様に導出できるため,その結果を示すだけとする.本章では$k$-次数のBDFをBDF-$k$と表記する.

\subsubsection{BDF-2における導出}
3つの点 $\pab{\vect{y}^{n+1},t_{n+1}}$,$\pab{\vect{y}^n, t_n}$,$\pab{\vect{y}^{n-1}, t_{n-1}}$ を通る2次のラグランジュ補間多項式 $P\pab{\tau}$ を考える.各点は局所座標で $\pab{\vect{y}^{n+1},\tau_0}, \pab{\vect{y}^n, \tau_1}, \pab{\vect{y}^{n-1}, \tau_2}$ と表せる.
\begin{subequations}
  \label{Eq:bdf2_lagrange_time}
  \begin{align}
    \tau_0 & = t_{n+1} - t_{n+1} = 0                                \\
    \tau_1 & = t_n - t_{n+1} = - \adif{t_n}                         \\
    \tau_2 & = t_{n-1} - t_{n+1} = -( \adif{t_n} +  \adif{t_{n-1}})
  \end{align}
\end{subequations}
ここで$P\pab{\tau}$が$k=2$の時のLagrange補間多項式であるとすると,\eqref{Eq:bdf_lagrange}より
\begin{align}
  P\pab{\tau} = l_0\pab{\tau} \vect{y}^{n+1} + l_1\pab{\tau} \vect{y}^n + l_2\pab{\tau} \vect{y}^{n-1}
\end{align}
ここで,各基底関数は次のように定義される.
\begin{subequations}
  \label{Eq:bdf2_lagrange_basis}
  \begin{align}
    l_0\pab{\tau} = \frac{\tau-\tau_1}{\tau_0-\tau_1} \frac{\tau-\tau_2}{\tau_0-\tau_2} \\[1mm]
    l_1\pab{\tau} = \frac{\tau-\tau_0}{\tau_1-\tau_0} \frac{\tau-\tau_2}{\tau_1-\tau_2} \\[1mm]
    l_2\pab{\tau} = \frac{\tau-\tau_0}{\tau_2-\tau_0} \frac{\tau-\tau_1}{\tau_2-\tau_1}
  \end{align}
\end{subequations}
ここで各基底関数の微分は\eqref{Eq:bdf2_lagrange_basis}より
\begin{subequations}
  \label{Eq:bdf2_lagrange_basis_diff}
  \begin{align}
    l'_0\pab{\tau} & = \frac{\tau-\tau_2}{\pab{\tau_0-\tau_1}\pab{\tau_0-\tau_2}} + \frac{\tau-\tau_1}{\pab{\tau_0-\tau_1}\pab{\tau_0-\tau_2}} = \frac{2\tau - \tau_1 - \tau_2}{\pab{\tau_0-\tau_1}\pab{\tau_0-\tau_2}} \\[1mm]
    l'_1\pab{\tau} & = \frac{\tau-\tau_2}{\pab{\tau_1-\tau_0}\pab{\tau_1-\tau_2}} + \frac{\tau-\tau_0}{\pab{\tau_1-\tau_0}\pab{\tau_1-\tau_2}} = \frac{2\tau - \tau_0 - \tau_2}{\pab{\tau_1-\tau_0}\pab{\tau_1-\tau_2}} \\[1mm]
    l'_2\pab{\tau} & = \frac{\tau-\tau_1}{\pab{\tau_2-\tau_0}\pab{\tau_2-\tau_1}} + \frac{\tau-\tau_0}{\pab{\tau_2-\tau_0}\pab{\tau_2-\tau_1}} = \frac{2\tau - \tau_0 - \tau_1}{\pab{\tau_2-\tau_0}\pab{\tau_2-\tau_1}}
  \end{align}
\end{subequations}
$P\pab{\tau}$ の $\tau=0$ における微分は,
\begin{equation}
  P'\pab{0} = \vect{y}^{n+1}l'_0(0) + \vect{y}^n l'_1(0) + \vect{y}^{n-1}l'_2(0)
\end{equation}
これに\eqref{Eq:bdf2_lagrange_basis_diff}および$\tau_1, \tau_2$ を代入し,時間ステップ比 $\rho_1 = \adif{t_n} / \adif{t_{n-1}}$ を用いて整理すると,
\begin{subequations}
  \begin{align}
    l'_0\pab{0} & = -\frac{\tau_1 + \tau_2}{\tau_1 \tau_2} = \frac{2\adif{t_n} + \adif{t_{n-1}}}{\adif{t_n} \pab{\adif{t_n} + \adif{t_{n-1}}}} = \frac{2\rho_1 + 1}{\rho_1 + 1}\frac{1}{\adif{t_n}} \\[2mm]
    l'_1\pab{0} & = -\frac{\tau_2}{\tau_1 \pab{\tau_1 - \tau_2}} = -\frac{\adif{t_n} + \adif{t_{n-1}}}{\adif{t_n} \adif{t_{n-1}}} = -\pab{\rho_1 + 1}\frac{1}{\adif{t_n}}                           \\[2mm]
    l'_2\pab{0} & = -\frac{\tau_1}{\tau_2 \pab{\tau_2 - \tau_1}} = \frac{\adif{t_n}}{\pab{\adif{t_n} + \adif{t_{n-1}}}\adif{t_{n-1}} } = \frac{\rho_1^2}{\rho_1 + 1}\frac{1}{\adif{t_n}}
  \end{align}
\end{subequations}
よって,時間微分の近似式は,
\begin{equation}
  \label{Eq:bdf2_approx}
  \pdv{\vect{y}}{t} \bigg|_{t_{n+1}} \approx \frac{1}{\adif{t_n}} \bab{\frac{2\rho_1 + 1}{\rho_1 + 1}\vect{y}^{n+1} - \pab{\rho_1 + 1} \vect{y}^n + \frac{\rho_1^2}{\rho_1 + 1} \vect{y}^{n-1}}
\end{equation}

\subsubsection{BDF-$k$における係数導出のための一般化}
\label{Sec:BDF_Generalization}

BDF-$k$においては,$k$個の過去の時点を用いて微分を近似する.
一般に,時間ステップをくくりだすことで,$k$次のBDFは次のように定義される.
\begin{equation}
  \label{Eq:bdf_k_approx}
  \pdv{\vect{y}}{t} \bigg|_{t_{n+1}} \approx \frac{1}{\adif{t_n}} \sum_{i=0}^{k} l'_i\pab{0} \vect{y}^{n+1-i}
\end{equation}
% ここで,$l_i$はBDF-$k$における係数であり,BDF-$k$の係数はCode Snippet\ref{prog:bdf_coeffs_final}を用いて求めることができる.

% % \begin{lstlisting}[caption=Back Differential Formula Coefficients, label=prog:bdf_coeffs_final]
% % import sympy
% % from IPython.display import display, Math

% % def display_bdf_coeffs(k_order):
% %     """
% %     指定された次数のBDF係数を導出し,LaTeX数式としてIPython環境で表示する.
% %     """
% %     print(f'--- BDF-{k_order} の係数 ---')

% %     # 1. LaTeX表示用のシンボルを定義
% %     dt_n_sym = sympy.Symbol(r'\Delta t_n', positive=True)
% %     dts = [dt_n_sym] + [
% %         sympy.Symbol(rf'\Delta t_{{n-{i}}}', positive=True) for i in range(1, k_order)
% %     ]
% %     rhos = [
% %         sympy.Symbol(rf'\rho_{{{i + 1}}}', positive=True) for i in range(k_order - 1)
% %     ]

% %     tau = sympy.Symbol(r'\tau')
% %     n_sym = sympy.Symbol('n')

% %     # 2. 補間点の座標を定義(k_order+1点)
% %     nodes = [0]
% %     for i in range(1, k_order + 1):
% %         nodes.append(-sum(dts[:i]))

% %     # 3. 各係数 l'_j(0) を計算
% %     for j in range(k_order + 1):
% %         num = 1
% %         den = 1
% %         for i in range(k_order + 1):
% %             if i != j:
% %                 num *= tau - nodes[i]
% %                 den *= nodes[j] - nodes[i]
% %         l_j = num / den

% %         l_j_prime = sympy.diff(l_j, tau)
% %         l_j_prime_at_0 = l_j_prime.subs(tau, 0)

% %         result_in_rho = l_j_prime_at_0
% %         if k_order > 1:
% %             subs_rules = {}
% %             rho_prod = 1
% %             for i in range(k_order - 1):
% %                 rho_prod *= rhos[i]
% %                 subs_rules[dts[i + 1]] = dts[0] / rho_prod
% %             result_in_rho = result_in_rho.subs(subs_rules)

% %         exponent_val = (n_sym + 1) - j
% %         if exponent_val == n_sym + 1:
% %             superscript = 'n+1'
% %         elif exponent_val == n_sym:
% %             superscript = 'n'
% %         else:
% %             superscript = f'n-{j - 1}'
% %         term_comment = rf'\quad \text{{(Coefficient for }} T^{{{superscript}}})'

% %         latex_str = sympy.latex(sympy.factor(result_in_rho))
% %         display(Math(rf"L'_{{{j}}}(0) = {latex_str} {term_comment}"))

% % if __name__ == '__main__':
% %     display_bdf_coeffs(6)
% % \end{lstlisting}

\FloatBarrier