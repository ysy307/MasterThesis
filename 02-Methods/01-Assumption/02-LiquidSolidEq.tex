\subsection{液相・固相間の相平衡}
\label{sec:LiquidSolidEq}

単一系における液状水と氷の平衡条件は,各相のGibbs自由エネルギーが等しいことによって定義される.
ここで,Fig.~\ref{Fig:GCC_diagram}に示すように,氷相に水蒸気のみに一様に働く圧力を$P_\mathrm{ice}$,
液相に水蒸気のみに一様に働く圧力を$\Pw$とすれば,
両者は共通の大気圧$P_\mathrm{a}$を介して次式のように関係づけられる.
\begin{subequations}
  \begin{align}
    \label{Eq:Pressure_1}
    P_2 & = P_\mathrm{a} + \Pw   \\
    \label{Eq:Pressure_2}
    P_1 & = P_\mathrm{a} + P_\mathrm{ice}
  \end{align}
\end{subequations}
ここで,$P_1$,$P_2$はそれぞれ氷相および液相に作用する全圧である.
$T_\mathrm{f}$を三相が平衡状態にあるときの温度(凝固点)とする.

\begin{figure}[tbp]
  \centering
  \begin{tikzpicture}[scale=1.5] % スケール調整はお好みで
    \usetikzlibrary{calc, arrows.meta} % arrows.meta を追加 (calc は既存)

    % 共通の圧力矢印スタイルを定義
    \tikzset{pressurearrow/.style={
    -{LaTeX[length=4.0mm, width=3.6mm]}, % 矢印の先端形状とサイズ
    line width=1.2mm                    % 線の太さ
    }
    }

    % 座標やサイズを調整しやすくするための変数定義(任意)
    \def\waterHeight{2}
    \def\iceHeight{3}
    \def\blockY{0.2} % ブロックの下端のY座標
    \def\containerTop{5}
    \def\containerBottom{0}
    \def\containerLeft{0}
    \def\containerRight{7}
    \def\pressureArrowStartY{6} % 圧力矢印の開始Y座標
    \def\pressureLabelY{5.5}    % 圧力ラベルの共通Y座標 (お好みで調整)

    % 外側の容器の線
    \draw (\containerLeft,\containerBottom) rectangle (\containerRight,\containerTop);

    % 水の部分 (左下)
    \coordinate (waterSW) at (0.2,\blockY); % 左下隅の座標
    \coordinate (waterNE) at (3.5,\waterHeight); % 右上隅の座標
    \fill[pattern=north west lines, pattern color=blue!50] (waterSW) rectangle (waterNE);
    \draw (waterSW) rectangle (waterNE);
    \node[fill=white, text=black, font=\bfseries, inner sep=2pt, rounded corners=1pt] at ($(waterSW)!0.5!(waterNE)$) {Water};

    % 氷の部分 (右下)
    \coordinate (iceSW) at (3.5,\blockY); % 左下隅の座標
    \coordinate (iceNE) at (6.8,\iceHeight); % 右上隅の座標
    \fill[pattern=north east lines, pattern color=cyan!30] (iceSW) rectangle (iceNE);
    \draw (iceSW) rectangle (iceNE);
    \node[fill=white, text=black, font=\bfseries, inner sep=2pt, rounded corners=1pt] at ($(iceSW)!0.5!(iceNE)$) {Ice};

    % 蒸気の部分 (上部全体)
    \node at (3.5,3.5) {Air};
    % 蒸気と水の境界線
    \draw (0.2,\waterHeight) -- (3.5,\waterHeight);
    % 蒸気と氷の境界線
    \draw (3.5,\iceHeight) -- (6.8,\iceHeight);
    % 水と氷の間の境界線
    \draw (3.5,\blockY) -- (3.5, \waterHeight); % 水ブロックの高さまで

    % 容器の上部の線 (Vaporと接する部分)
    \draw (0.2,\waterHeight) -- (0.2,\containerTop-0.2);
    \draw (6.8,\iceHeight) -- (6.8,\containerTop-0.2);
    \draw (0.2,\containerTop-0.2) -- (6.8,\containerTop-0.2);

    % --- 圧力矢印とラベル ---
    % 圧力 P (全体にかかる)
    \draw[pressurearrow] (3.5,\pressureArrowStartY) -- (3.5,\containerTop-1.2);
    \node[right, xshift=3mm] at (3.5, \pressureLabelY) {$P_\mathrm{a}$};

    % 水面にかかる圧力 P_w
    \draw[pressurearrow] (1.75,\pressureArrowStartY) -- (1.75,\waterHeight+0.2);
    \node[left, xshift=-3mm] at (1.75, \pressureLabelY) {$\Pw$};
    \node[below] at (1.75,\containerBottom-0.3) {$P_2 = P_\mathrm{a} + \Pw$};

    % 氷面にかかる圧力 P_f
    \draw[pressurearrow] (5.25,\pressureArrowStartY) -- (5.25,\iceHeight+0.2);
    \node[right, xshift=3mm] at (5.25, \pressureLabelY) {$P_\mathrm{ice}$};
    \node[below] at (5.25,\containerBottom-0.3) {$P_1 = P_\mathrm{a} + P_\mathrm{ice}$};
  \end{tikzpicture}
  \caption{固相,液相,気相の三相それぞれに異なる圧力が加わっている状況における,各相間の平衡状態を表す概念図}
  \label{Fig:GCC_diagram}
\end{figure}
\noindent
液相と氷相が平衡状態にある場合,それぞれの比Gibbs自由エネルギーは等しくなる.
\begin{equation}
  \label{Eq:Gibbs_equal}
  G_1 = G_2
\end{equation}
ここで,$G_1$,$G_2$はそれぞれ氷相および液相のGibbs自由エネルギーである.
系に変化が生じた際には,これらの自由エネルギーの微分が等しくなければならない.
\begin{equation}
  \label{Eq:Gibbs_Equilibrium}
  \odif{G_1} = \odif{G_2}
\end{equation}
\eqref{Eq:Gibbs_Equilibrium}に対し,Gibbs自由エネルギーの全微分\eqref{Eq:dG_constN}を用いると,
\begin{equation}
  \label{Eq:Gibbs_Equilibrium_2}
  -S_1\odif{T} + \nu_1\odif{P_1} = -S_2\odif{T} + \nu_2\odif{P_2}
\end{equation}
が得られる.ここで,$\nu_1$,$\nu_2$はそれぞれ氷相,液相の比体積 \unit{[\meter^3.\kilogram^{-1}]}であり,
$\odif{P_1}$と$\odif{P_2}$はそれぞれ氷および水に加わる全圧力の変化である.
一般に$\odif{P_1}\neq\odif{P_2}$である.
\eqref{Eq:Gibbs_Equilibrium_2}に\eqref{Eq:Pressure_1},\eqref{Eq:Pressure_2}を代入して整理すると,
\begin{equation}
  \nu_1\pab{\odv{P_\mathrm{a}}{T} + \odv{P_\mathrm{ice}}{T}} - S_1
  = \nu_2\pab{\odv{P_\mathrm{a}}{T} + \odv{\Pw}{T}} - S_2
\end{equation}
となり,さらに次式を得る.
\begin{equation}
  \label{Eq:Gibbs_Equilibrium_3}
  \nu_2\odv{\Pw}{T} - \nu_1\odv{P_\mathrm{ice}}{T}
  + \odv{P_\mathrm{a}}{T}\pab{\nu_2 - \nu_1} = S_2 - S_1
\end{equation}
一般に液相から固相への相変化に伴うエンタルピー変化は,
\begin{equation}
  \label{Eq:Enthalpy_Change}
  H_\mathrm{fus}\pab{T; N} = T \Bab{S \pab{T; V_\mathrm{liquid}\pab{T; N}, N} - S\pab{T;V_\mathrm{solid}\pab{T; N}, N}}
\end{equation}
で与えられる.ここで,$H_\mathrm{fus}$は液相から固相への凝固潜熱である.
液状水から氷への相変化を考えれば,
\begin{equation}
  \label{Eq:Enthalpy_Change_2}
  S_2 - S_1 = \frac{L_\mathrm{f}}{T}
\end{equation}
となる.ただし,$L_\mathrm{f}$は水の凍結潜熱 \unit{[J.kg^{-1}]}である.
\eqref{Eq:Gibbs_Equilibrium_3}に\eqref{Eq:Enthalpy_Change_2}を代入し,比体積を密度に置き換えると次式を得る.

\begin{NoteBox}{Generalized Clausius-Clapeyron式}{GCCE}
  \eqref{Eq:GCC_main}は,凍土における相平衡の基礎となる一般化Clausius-Clapeyron式(Generalized Clausius-Clapeyron equation, GCC)である.
  \begin{equation}
    \label{Eq:GCC_main}
    \frac{1}{\rho_\mathrm{w}}\odv{\Pw}{T} - \frac{1}{\rho_\mathrm{ice}}\odv{P_\mathrm{ice}}{T} + \odv{P_\mathrm{a}}{T}\pab{\frac{1}{\rho_\mathrm{w}} - \frac{1}{\rho_\mathrm{ice}}} = \frac{L_\mathrm{f}}{T}
  \end{equation}
  ここで,$\rho_\mathrm{w}$,$\rho_\mathrm{ice}$はそれぞれ水と氷の密度 \unit{[\kilogram.\meter^{-3}]}である.
\end{NoteBox}
\noindent
特に本研究内では,以下の2つの場合に着目する.ただし,圧力の単位は \unit{\pascal}である.
\begin{enumerate}[label=Case \arabic*:, leftmargin=*, labelwidth=3.1em, labelsep=0.5em]
  \item 氷の圧力を一定とし,水の圧力のみが変化する場合.\\[4mm]
        氷の圧力微分$\displaystyle\odv{P_\mathrm{ice}}{T}=0$とすれば,\eqref{Eq:GCC_main}は
        \begin{equation}
          \label{Eq:GCC_ice_constant}
          \odv{\Pw}{T}
          = \frac{\rho_\mathrm{w} L_\mathrm{f}}{ T}
          + \odv{P_\mathrm{a}}{T}\pab{\frac{\rho_\mathrm{w}}{\rho_\mathrm{ice}} - 1}
        \end{equation}
        右辺第2項は通常非常に小さいため無視できるため,次式が得られる.
        \begin{equation}
          \label{Eq:GCC_ice_constant_approx}
          \odv{\Pw}{T} \approx \frac{L_\mathrm{f}\rho_\mathrm{w}}{T}
        \end{equation}
        大気圧が標準状態$P_\mathrm{a}=\qty{1013.25}{\hecto\pascal}$のとき,凝固点$T_\mathrm{f}^\ast=\qty{273.15}{\kelvin}$で凍結するとすれば,
        \eqref{Eq:GCC_ice_constant_approx}の積分形は次式で与えられる.
        \begin{align}
          \int_{0}^{\Pw}\odif{\Pw}
           & = \int_{T_\mathrm{f}^\ast}^{T^\ast}\frac{L_\mathrm{f}\rho_\mathrm{w}}{T}\odif{T} \notag \\
          \label{Eq:GCC_ice_constant_approx_integrated}
          \Pw
           & = L_\mathrm{f}\rho_\mathrm{w}\ln{\frac{T^\ast}{T_\mathrm{f}^\ast}}
        \end{align}

  \item 大気圧$P_\mathrm{a}$が温度に依存しない場合.\\
        このとき,\eqref{Eq:GCC_main}左辺第3項は消去されるため,
        \begin{equation}
          \label{Eq:GCC_vapor_constant}
          \frac{1}{\rho_\mathrm{w}}\odv{\Pw}{T} - \frac{1}{\rho_\mathrm{ice}}\odv{P_\mathrm{ice}}{T} = \frac{L_\mathrm{f}}{T}
        \end{equation}
        よって,氷の圧力変化は次式で表される.
        \begin{equation}
          \label{Eq:GCC_vapor_constant_2}
          \odv{P_\mathrm{ice}}{T}
          = \frac{\rho_\mathrm{ice}}{\rho_\mathrm{w}}\odv{\Pw}{T}
          - \frac{L_\mathrm{f}\rho_\mathrm{ice}}{T}
        \end{equation}
        Case 1 と同様に積分すれば,
        \begin{align}
          \int_{0}^{P_\mathrm{ice}}\odif{P_\mathrm{ice}}
           & = \int_{0}^{\Pw}\odif{\Pw}
          - \int_{T_\mathrm{f}^\ast}^{T^\ast}\frac{L_\mathrm{f}\rho_\mathrm{ice}}{T}\odif{T} \notag \\
          \label{Eq:GCC_vapor_constant_2_integrated}
          P_\mathrm{ice}
           & = \frac{\rho_\mathrm{ice}}{\rho_\mathrm{w}}\Pw
          - L_\mathrm{f}\rho_\mathrm{ice}\ln{\frac{T^\ast}{T_\mathrm{f}^\ast}}
        \end{align}
\end{enumerate}
