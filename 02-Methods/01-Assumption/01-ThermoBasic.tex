\subsection{熱力学的前提と記号定義}
\label{Sec:ThermoBasic}

\subsubsection{系の構成と相の定義}
\label{Sec:PhaseDefinition}

本研究で対象とする凍土は,熱・水分移動を同時に取り扱う必要があるため,多相混合体として定式化する.ここでは,土粒子(solid),液相水(liquid),氷(ice),および気相(air)の四相からなる系を考える.
\begin{figure}[tbp]
  \centering
  \begin{tikzpicture}[scale=0.95]
    % 座標定義
    \coordinate (p11) at (5,0); \coordinate (p12) at (5,3);
    \coordinate (p13) at (5,5.25); \coordinate (p14) at (5,7.5);
    \coordinate (p15) at (5,9); \coordinate (p21) at (9,0);
    \coordinate (p22) at (9,3); \coordinate (p23) at (9,5.25);
    \coordinate (p24) at (9,7.5); \coordinate (p25) at (9,9);

    % ▼▼▼ グレー背景と模様の設定 ▼▼▼

    % --- Soil (土粒子): 20%グレー + 右上がり斜線 ---
    \fill[graySoil] (p11) rectangle (p22);
    \fill[pattern={Lines[distance=4pt, line width=0.4pt, angle=45]}] (p11) rectangle (p22);
    % 文字背景を白く抜く (fill=white)
    \node[fill=white, inner sep=2pt] at ($(p11)!0.5!(p22)$) {\Large Soil};

    % --- Water (水): 10%グレー + 水平線 ---
    \fill[grayWater] (p12) rectangle (p23);
    \fill[pattern={Lines[distance=4pt, line width=0.4pt, angle=0]}] (p12) rectangle (p23);
    \node[fill=white, inner sep=2pt] at ($(p12)!0.5!(p23)$) {\Large Water};

    % --- Ice (氷): 3%グレー(ほぼ白) + 左上がり斜線 ---
    \fill[grayIce] (p13) rectangle (p24);
    \fill[pattern={Lines[distance=4pt, line width=0.4pt, angle=-45]}] (p13) rectangle (p24);
    \node[fill=white, inner sep=2pt] at ($(p13)!0.5!(p24)$) {\Large Ice};

    % --- Air (空気): 白 + 模様なし ---
    \node at ($(p14)!0.5!(p25)$) {\Large Air};

    % ▲▲▲ 設定ここまで ▲▲▲

    % 枠線
    \draw[line width = 0.5mm] (p11) rectangle (p25);
    \draw[line width = 0.5mm] (p12) -- (p22);
    \draw[line width = 0.5mm] (p13) -- (p23);
    \draw[line width = 0.5mm] (p14) -- (p24);

    % --- Volume (体積) ---
    \coordinate (v11) at (0,0); \coordinate (v12) at (0,9);
    \coordinate (v21) at (4,0); \coordinate (v22) at (4,9);
    \coordinate (v31) at (1.5,3); \coordinate (v32) at (4,3);
    \coordinate (v41) at (3.5,0); \coordinate (v42) at (3.5,3);
    \coordinate (v43) at (3.5,5.25); \coordinate (v44) at (3.5,7.5);
    \coordinate (v45) at (3.5,9); \coordinate (v51) at (3,5.25);
    \coordinate (v52) at (3,7.5);

    \node at ($(v12)!0.5!(v22)+(0,0.5)$) {\LARGE Volume};
    \draw[line width = 0.5mm] (v11) -- (v21);
    \draw[line width = 0.5mm] (v12) -- (v22);
    \draw[line width = 0.5mm] (v31) -- (v32);
    \draw[line width = 0.5mm] (v51) -- ++(1,0);
    \draw[line width = 0.5mm] (v52) -- ++(1,0);
    \draw[{Latex[length=3mm]}-{Latex[length=3mm]},line width = 0.5mm] ($(v11)+(0.5,0)$) -- ($(v12)+(0.5,0)$) node[midway, left] {\Large $V$};

    \draw[{Latex[length=3mm]}-{Latex[length=3mm]},line width = 0.5mm] ($(v12)+(2,0)$) -- ++(0,-6) node[midway, left] {\Large $\Vvoid$};
    \draw[{Latex[length=3mm]}-{Latex[length=3mm]},line width = 0.5mm] (v41) -- (v42)  node[midway, left] {\Large $\Vs$};
    \draw[{Latex[length=3mm]}-{Latex[length=3mm]},line width = 0.5mm] (v42) -- (v43)  node[midway, left] {\Large $\Vw$};
    \draw[{Latex[length=3mm]}-{Latex[length=3mm]},line width = 0.5mm] (v43) -- (v44)  node[midway, left] {\Large $\Vice$};
    \draw[{Latex[length=3mm]}-{Latex[length=3mm]},line width = 0.5mm] (v44) -- (v45)  node[midway, left] {\Large $\Va$};

    % --- Mass (質量) ---
    \coordinate (m11) at (10,0); \coordinate (m12) at (10,3);
    \coordinate (m13) at (10,5.25); \coordinate (m14) at (10,7.5);
    \coordinate (m15) at (10,9); \coordinate (m21) at (14,9);
    \coordinate (m31) at (13,0);

    \node at ($(m15)!0.5!(m21)+(0,0.5)$) {\LARGE Mass};
    \draw[line width = 0.5mm] (m11) -- ++(4,0);
    \draw[line width = 0.5mm] (m12) -- ++(1,0);
    \draw[line width = 0.5mm] (m13) -- ++(1,0);
    \draw[line width = 0.5mm] (m14) -- ++(1,0);
    \draw[line width = 0.5mm] (m15) -- ++(4,0);

    \draw[{Latex[length=3mm]}-{Latex[length=3mm]},line width = 0.5mm] ($(m11)+(0.5,0)$) -- ++(0,3)  node[midway, right] {\Large $m_\mathrm{s}$};
    \draw[{Latex[length=3mm]}-{Latex[length=3mm]},line width = 0.5mm] ($(m12)+(0.5,0)$) -- ++(0,2.25)  node[midway, right] {\Large $m_\mathrm{w}$};
    \draw[{Latex[length=3mm]}-{Latex[length=3mm]},line width = 0.5mm] ($(m13)+(0.5,0)$) -- ++(0,2.25)  node[midway, right] {\Large $m_\mathrm{ice}$};
    \draw[{Latex[length=3mm]}-{Latex[length=3mm]},line width = 0.5mm] ($(m14)+(0.5,0)$) -- ++(0,1.5)  node[midway, right] {\Large $m_\mathrm{a}=0$};
    \draw[{Latex[length=3mm]}-{Latex[length=3mm]},line width = 0.5mm] (m31) -- ++(0,9)  node[midway, right] {\Large $m$};

  \end{tikzpicture}
  \caption{凍土中における四相地盤構造}\label{fig:4Phase_Soil}
\end{figure}
間隙比$e$ \unit{[-]} および間隙率$\phiv$ \unit{[-]} は次式で定義される.
\begin{subequations}
  \label{Eq:Void_Definition}
  \begin{align}
    \label{Eq:Void_Definition_e}
    e     & = \frac{\Vvoid}{\Vs} \\
    \label{Eq:Void_Definition_phi}
    \phiv & = \frac{\Vvoid}{V}
  \end{align}
\end{subequations}
両者の関係は次式で表される.
\begin{subequations}
  \label{Eq:Void_Porosity}
  \begin{align}
    \label{Eq:Void_Porosity_e}
    \phiv   & = \frac{e}{1+e} \\
    \label{Eq:Void_Porosity_phi}
    1-\phiv & = \frac{1}{1+e}
  \end{align}
\end{subequations}
間隙体積に対する各相の体積比(飽和度)\unit{[-]} を次式で定義する.
\begin{subequations}
  \label{Eq:Saturation_Def}
  \begin{align}
    \label{Eq:Saturation_Water_Def}
    \Sw   & = \frac{\Vw}{\Vvoid}   \\
    \label{Eq:Saturation_Ice_Def}
    \Sice & = \frac{\Vice}{\Vvoid} \\
    \label{Eq:Saturation_Air_Def}
    \Sa   & = \frac{\Va}{\Vvoid}   \\
    \label{Eq:Saturation_Vapor_Def}
    \Sv   & = \frac{\Vv}{\Vvoid}
  \end{align}
\end{subequations}
このとき,次の関係が成り立つ.
\begin{equation}
  \label{Eq:Saturation_Sum}
  \Sw + \Sice + \Sa = 1
\end{equation}
全体体積に対する体積分率$\theta_\alpha$ \unit{[\meter^3.\meter^{-3}]} は次式で定義される.
ただし,$\alpha = n\!:\text{soil},\ w\!:\text{water},\ i\!:\text{ice},\ a\!:\text{air},\ v\!:\text{vapor}$である.
\begin{subequations}
  \label{Eq:Theta_Def}
  \begin{align}
    \label{Eq:Theta_Solid_Def}
    \Qn   & = \frac{\Vs}{V}   \\
    \label{Eq:Theta_Water_Def}
    \Qw   & = \frac{\Vw}{V}   \\
    \label{Eq:Theta_Ice_Def}
    \Qice & = \frac{\Vice}{V} \\
    \label{Eq:Theta_Air_Def}
    \Qa   & = \frac{\Va}{V}   \\
    \label{Eq:Theta_Vapor_Def}
    \Qv   & = \frac{\Vv}{V}
  \end{align}
\end{subequations}
体積保存条件は次式となる.
\begin{equation}
  \label{Eq:VolumeSum}
  \Qn + \Qw + \Qice + \Qa = 1
\end{equation}
さらに,間隙における有効密度$\rho_\mathrm{e}$ \unit{[\kilogram.\meter^{-3}]}を次式で定義する.
\begin{align}
  \label{Eq:Density_effective}
  \rho_\mathrm{e}
   & = \phiv (\rho_\mathrm{w} \Sw + \rho_\mathrm{ice} \Sice + \rho_\mathrm{a} \Sa) \notag    \\
   & = \frac{e (\rho_\mathrm{w} \Sw + \rho_\mathrm{ice} \Sice + \rho_\mathrm{a} \Sa)}{1 + e}
\end{align}
以上により,四相地盤の体積構成および基本変数の定義が確定する.この定義に基づき,次節では各相の状態変数と熱力学関係式を整理する.

\subsubsection{状態変数と熱力学関係式}
\label{Sec:ThermoRelation}

各相(固相,液相,気相)における熱力学状態は,温度$T$,圧力$p$,および比体積$\nu$を基本変数として定義する.本章の表記は\textcite{Tasaki-Thermodynamics}に従う.
内部エネルギー$U$,エントロピー$S$,エンタルピー$H$,Helmholtzの自由エネルギー$F$,Gibbsの自由エネルギー$G$は熱力学ポテンシャルと呼ばれる.ここで,$T$は熱力学的温度,$V$は体積,$p$は圧力,$N$は物質量(モル数または質量)である.
熱力学の基本関係式より,内部エネルギー$U$の全微分は次式で与えられる.
\begin{equation}
  \label{Eq:dU}
  \odif{U} = T \odif{S} - p \odif{V} + \mu \odif{N}
\end{equation}
ここで,$\mu$は化学ポテンシャルである.他の熱力学ポテンシャルは,$U$からルジャンドル変換により,独立変数を$S \to T$または$V \to p$(あるいはその両方)へ変更することで定義される.
まず,Helmholtzの自由エネルギー$F$は次式で定義される.
\begin{equation}
  \label{Eq:Helmholtz}
  F = U - TS
\end{equation}
ただし,Helmholtzの自由エネルギー$F$が$T$について微分可能とする.その全微分は,\eqref{Eq:dU}式および\eqref{Eq:Helmholtz}式を用いて次式で導かれる.
\begin{align}
  \odif{F}
   & = \odif{U} - \odif{(TS)} \notag                                                       \\
   & = \pab{T \odif{S} - p \odif{V} + \mu \odif{N}} - \pab{S \odif{T} + T \odif{S}} \notag \\
  \label{Eq:dF}
   & = -S \odif{T} - p \odif{V} + \mu \odif{N}
\end{align}
また,状態量としての圧力はHelmholtzの自由エネルギー$F$の体積微分として表せるので,
\begin{equation}
  \label{Eq:Pressure_differential}
  \pdv{F}{V} = -p
\end{equation}
同様に物質量$N$についても同様の量である化学ポテンシャル$\mu$を考える.
\begin{equation}
  \label{Eq:Chemical_potential_differential}
  \pdv{F}{N} = \mu
\end{equation}
次に,エンタルピー$H$は次式で定義される.
\begin{equation}
  \label{Eq:Enthalpy_def}
  H = U + pV
\end{equation}
その全微分は次式で与えられる.
\begin{align}
  \odif{H}
   & = \odif{U} + \odif{\pab{pV}} \notag                                                   \\
   & = \pab{T \odif{S} - p \odif{V} + \mu \odif{N}} + \pab{V \odif{p} + p \odif{V}} \notag \\
  \label{Eq:dH}
   & = T \odif{S} + V \odif{p} + \mu \odif{N}
\end{align}
さらに,Gibbsの自由エネルギー $G$ は次式で定義される.
\begin{equation}
  \label{Eq:Gibbs_def}
  G = U - TS + pV
\end{equation}
$G$ は $F$ または $H$ を用いて次のようにも表される.
\begin{equation}
  \label{Eq:ThermoRelation}
  G = F + pV = H - TS
\end{equation}
$G$ の全微分は次式で与えられる.
\begin{align}
  \odif{G}
   & = \odif{F} + \odif{(pV)} \notag                                                        \\
   & = \pab{-S \odif{T} - p \odif{V} + \mu \odif{N}} + \pab{V \odif{p} + p \odif{V}} \notag \\
  \label{Eq:dG}
   & = -S \odif{T} + V \odif{p} + \mu \odif{N}
\end{align}
物質量 $N$ が一定($\odif{N}=0$)の場合,\eqref{Eq:dG} 式は次式のように書ける.
\begin{equation}
  \label{Eq:dG_constN}
  \eval{\odif{G}}{{N=\text{const}}} = -S \odif{T} + V \odif{p}
\end{equation}

各相の物性は温度と圧力の関数として次式で表される.
\begin{subequations}
  \begin{align}
    \label{Eq:Density_TP}
    \rho_\alpha           & = \rho_\alpha\pab{T, p}           \\
    \label{Eq:Specific_Heat_TP}
    c_{\mathrm{p},\alpha} & = c_{\mathrm{p},\alpha}\pab{T, p}
  \end{align}
\end{subequations}
特に後章で導入する体積熱容量$C$や熱伝導率$\lambda$の導出において,これらの関数形が用いられる.
