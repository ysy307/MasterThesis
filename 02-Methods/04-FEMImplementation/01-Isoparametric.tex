\subsection{2次元アイソパラメトリック要素の統一的記述}
\label{Sec:2D_Isoparametric_Elements}

本解析では,2次元領域の空間離散化において,アイソパラメトリック要素(Isoparametric Element)を採用する.
これは,解析対象になる物理量の補間と,座標などの要素形状の定義に同一の基底関数を用いる手法である.
本節では,解析に用いる以下の4種類の要素を統一的に記述する.
\begin{itemize}
  \item 三角形一次要素 (Tri3) および 二次要素 (Tri6)
  \item 四角形一次要素 (Quad4) および 二次要素 (Quad8)
\end{itemize}
まず,要素内の任意の点における物理量 $T$ および 物理座標 $\vect{x} = (x, y)^\top$ は,要素内の節点値を用いて次式で近似される.
\begin{subequations}
  \label{Eq:Iso-Interp}
  \begin{align}
    T\pab{\xi, \eta}        & = \sum_{i=1}^{N_n} \psi_i\pab{\xi, \eta} T_i              \\
    \Pw\pab{\xi, \eta}      & = \sum_{i=1}^{N_n} \psi_i\pab{\xi, \eta} P_{\mathrm{w},i} \\
    \vect{x}\pab{\xi, \eta} & = \sum_{i=1}^{N_n} \psi_i\pab{\xi, \eta} \vect{x}_i
  \end{align}
\end{subequations}
ここで,$N_n$ は要素あたりの節点数,$\psi_i$ は正規化座標系 $\pab{\xi, \eta}$ で定義された形状関数である.
この定式化により,要素の幾何形状が歪んでいる場合でも,正規化座標系上での積分計算に帰着させることが可能となる.
座標変換に伴うJacobian行列 $\mat{J}$ は,要素タイプに関わらず次式で統一的に定義される.
\begin{equation}
  \label{Eq:General_Jacobian}
  \mat{J} =
  \begin{bmatrix}
    \displaystyle \sum \pdv{\psi_i}{\xi} x_i & \displaystyle \sum \pdv{\psi_i}{\eta} x_i \\[3mm]
    \displaystyle \sum \pdv{\psi_i}{\xi} y_i & \displaystyle \sum \pdv{\psi_i}{\eta} y_i
  \end{bmatrix}
\end{equation}

\FloatBarrier