\subsection{数値積分と物性値の評価方法}

本解析では,土壌の熱・水分物性値(容量係数,拡散テンソル,輸送ベクトル等)が温度 $T$ および間隙水圧 $\Pw$ に強く依存する非線形性を有する.
そのため,係数自体を節点値から形状関数で補間する手法は用いず,各ガウス積分点において補間された状態量に基づき,構成則を用いて直接評価する手法を採用する.

まず,要素内のガウス積分点 $p$(局所座標 $\xi_p, \eta_p$)における未知変数の値($T_p, P_{\mathrm{w},p}$)およびその勾配は,節点値 $\vect{T}^e, \vect{P}_\mathrm{w}^e$ と形状関数 $\vect{\psi}$,勾配行列 $\mat{B}$ を用いて以下のように計算される.
\begin{equation}
  \label{Eq:GaussPoint_Interp}
  \begin{aligned}
    T_p & = \vect{\psi}\pab{\xi_p, \eta_p} \vect{T}^e,            & \nabla T_p & = \mat{B}\pab{\xi_p, \eta_p}^\top \vect{T}^e   \\
    P_p & = \vect{\psi}\pab{\xi_p, \eta_p} \vect{P}_\mathrm{w}^e, & \nabla P_p & = \mat{B}\pab{\xi_p, \eta_p}^\top \vect{\Pw}^e
  \end{aligned}
\end{equation}
ここで,$\mat{B} = \nabla \vect{\psi}$ は形状関数の空間微分行列である。
これらを用いて,積分点 $p$ における物理係数(スカラー $A$,テンソル $\mat{M}$,ベクトル $\vect{V}$)は,構成則関数 $\mathcal{F}$ を用いて以下のように決定される.
\begin{equation}
  A_p = \mathcal{F}_A\pab{T_p, P_p}, \quad
  \mat{M}_p = \mathcal{F}_M\pab{T_p, P_p}, \quad
  \vect{V}_p = \mathcal{F}_V\pab{T_p, P_p, \nabla T_p, \dots}
\end{equation}

これを踏まえ,各要素行列の数値積分計算式を以下のように記述する.

\subsubsection{容量型行列(分布スカラー係数 $A$)}
質量行列や熱容量行列などがこれに該当する.積分点ごとの係数値 $A_p$ を用いて計算される.
\begin{align}
  \mat{K}_1^e
   & = \iint_{\Omega^e} \vect{\psi}^\top A\pab{T, \Pw} \vect{\psi} \det\mat{J} \odif{\xi}\odif{\eta} \notag                \\
   & \approx \sum_{p=1}^{N_\mathrm{int}} w_p \vect{\psi}\pab{\xi_p}^\top A_p \vect{\psi}\pab{\xi_p} \det\mat{J}\pab{\xi_p}
\end{align}

\subsubsection{拡散型行列(分布テンソル係数 $\mat{M}$)}
熱伝導行列や透水行列などがこれに該当する.積分点ごとのテンソル値 $\mat{M}_p$ を用いて計算される.
\begin{align}
  \mat{K}_2^e
   & = \iint_{\Omega^e} \mat{B}^\top \mat{M}\pab{T, \Pw} \mat{B} \det\mat{J} \odif{\xi}\odif{\eta} \notag                \\
   & \approx \sum_{p=1}^{N_\mathrm{int}} w_p \mat{B}\pab{\xi_p}^\top \mat{M}_p \mat{B}\pab{\xi_p} \det\mat{J}\pab{\xi_p}
\end{align}

\subsubsection{混合型行列(分布ベクトル係数 $\vect{V}$)}
移流項などがこれに該当する.積分点ごとのベクトル値 $\vect{V}_p$ を用いて計算される.
(ここでは $\nabla \psi \cdot \vect{V} \psi$ の型を例示する)
\begin{align}
  \mat{K}_3^e
   & = \iint_{\Omega^e} \pab{\mat{B}^\top \vect{V}\pab{T, \Pw}} \vect{\psi} \det\mat{J} \odif{\xi}\odif{\eta} \notag                \\
   & \approx \sum_{p=1}^{N_\mathrm{int}} w_p \pab{\mat{B}\pab{\xi_p}^\top \vect{V}_p} \vect{\psi}\pab{\xi_p} \det\mat{J}\pab{\xi_p}
\end{align}

\FloatBarrier