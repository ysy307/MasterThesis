\section{研究背景}
\label{Sec:Background}

\subsection{人工地盤凍結工法}
\label{Sec:AGF_Introduction}

建設技術の高度化と都市空間の深部利用が進む現代において、土壌物理学や地盤工学が直面する課題は複雑化の一途を辿っている。
特に、大深度における高水圧下の帯水層や軟弱地盤における掘削工事、あるいは汚染物質の封じ込めといった極限的な状況において、地盤の強度増加と完全な遮水性を同時に、かつ可逆的に実現する技術への社会的な要求は高まっている。
こうした要請に応えうる技術として、人工地盤凍結工法(Artificial Ground Freezing Method, AGF)が存在する\cite{Alzoubi-2021, Diego-2022}。

人工地盤凍結工法とは、地盤中に一定間隔で埋設した凍結管に、冷凍機で冷却されたブライン(塩化カルシウム水溶液などの不凍液)や液体窒素等の冷媒を循環させ、管周囲の地盤熱を奪うことで土中の間隙水を凍結させる技術である。この工法が他の地盤改良技術と決定的に異なる点は、土粒子そのものを化学的に固結させるのではなく、間隙水を固体である氷へと相変化させる物理現象を利用する点にある。この物理的相変化を用いることで、以下の工学的利点が存在する。
\begin{itemize}
      \item 均質かつ高強度な改良体の造成: \\
            土の種類や粒度分布に依存せず、含水さえしていれば凍結が可能である。凍土の強度は温度に依存して一義的に定まり、マイナス10度程度でコンクリートの約3分の1、マイナス40度からマイナス162度(LNGタンク周辺)といった極低温下ではコンクリートと同等以上の圧縮強度を発現する\cite{Alzoubi-2021, Diego-2022, Nishimura-2022}。
      \item 完全な遮水性: \\
            間隙が氷で充填されるため、透水係数は実質的にゼロとなり、完全な不透水層(遮水壁)を形成する。
      \item 可逆性と無公害性: \\
            工事終了後に冷却を停止すれば地盤は融解し、元の状態に戻る。薬液による地下水汚染のリスクがないため、環境保全性が求められる現代の建設プロジェクトにおいて再評価されている。
\end{itemize}
一方で、水が氷になる際の体積膨張(約9\%)や、未凍結部からの水分吸引に伴う凍上(Frost Heave)、および融解時の融解沈下(Thaw Settlement)といった負の側面も併せ持つ。これらの挙動をいかに精緻に予測し制御するかが、本工法の設計・施工における核心的課題となる。
\section{数値解析の背景}

凍土工学における数値シミュレーションは、地盤の温度変化、水分移動、および力学的変形の複雑な相互作用を記述する手法として発展してきた。近年では、計算機の演算能力向上と定式化の高度化により、熱・水・力学(Thermal-Hydraulic-Mechanical: THM)を包括的に扱う連成解析や、氷レンズの生成を捉える不連続面モデルが実用化されている\cite{Nishimura-2009}。

凍土の物理挙動を精度良く予測するためには、相変化に伴う熱と物質の輸送、およびそれに起因する地盤の変形を同時に解く連成解析が不可欠である。従来の解析手法では不飽和土力学の枠組みを応用した有効応力概念が用いられてきたが、近年の高度な定式化においては、氷圧、不凍水圧(液圧)、および全応力を独立した状態変数として扱う手法が提案されている\cite{Zhou-2013}。
このアプローチでは、一般化されたクラウジウス・クラペイロンの式に基づき、液相と固相の間の熱力学的平衡条件を考慮することが可能となった。これにより、温度変化が凍結吸引力(サクション)の変化を引き起こし、それが未凍結領域から凍結前線への水分移動を駆動して地盤の膨張をもたらすプロセスが、物理的根拠に基づいて記述されている。

数値解析において、凍土の物理特性を規定する最も重要な構成則は、負温と不凍水含有量の関係を示す土壌凍結特性曲線(SFCC)である\cite{Talamucci-2003}。解析においては、van Genuchtenモデル等の水分特性曲線を温度領域に拡張したモデルが広く用いられ、相変化に伴う潜熱の出入りはエネルギー保存則に基づくエンタルピー法等によって処理される\cite{Bao-2016}。
また、近年の研究では、土粒子骨格と氷の相互作用を考慮した速度依存型の構成則(Creep/Secondary consolidation)が導入されており、人工地盤凍結(AGF)工法における長期間の施工プロセスや、地表面の凍上挙動をより正確に再現する試みがなされている。

\subsection{アイスレンズ形成と不連続面アプローチ}
凍上現象の主要因であるアイスレンズ(分離氷)の形成は、地盤内のマクロな不連続面としての成長プロセスである。これに対し、アイスレンズを物理的な不連続面として直接モデリングする手法が進展している。この手法では、破壊力学的な基準に基づいてアイスレンズの核形成と成長を記述し、複数のレンズが周期的に形成される挙動を再現することが可能となっている。
さらに、相の分布をスカラー場で表現するフェーズフィールド法(Phase-field Method)も、複雑な形状の氷層が合流・分岐するプロセスを解析する新たなアプローチとして注目されている。

\subsection{大規模計算基盤の活用と並列化技術}
広域な永久凍土の変動予測や、都市インフラにおける大規模な人工地盤凍結解析においては、計算負荷の増大が課題となる。スーパーコンピュータ「富岳」に搭載されたA64FXプロセッサや、高速なネットワーク(TofuインターコネクトD)に最適化されたMPIおよびOpenMPによる並列有限要素法コードの開発により、数万ノード規模での高解像度な3次元解析が現実のものとなりつつある。

これらの技術により、地下水流による熱移流効果や地表面の複雑な境界条件を詳細に考慮したシミュレーションが実現している。このような大規模計算技術は、施工リスクの事前評価やインフラ維持管理の高度化において不可欠な基盤となっている。