\section{ベクトル解析の基礎について}
\label{sec:VectorAnalysis}
本付録では,本修論を読むために助けになるベクトル解析の知識について解説する.本修論では,すべて右手系の直交座標系を用いている.そのときの正規直交基底$\vect{e}$は次のように定義される.
\begin{equation}
  \vect{e}_x=\begin{bmatrix}
    1 \\0\\0
  \end{bmatrix}, \quad
  \vect{e}_y=\begin{bmatrix}
    0 \\1\\0
  \end{bmatrix}, \quad
  \vect{e}_z=\begin{bmatrix}
    0 \\0\\1
  \end{bmatrix}
\end{equation}
\begin{FormulaBox}{Kronecker delta}{Kronecker-delta}
  Kronecker deltaは,$i$と$j$が等しいときに$1$,それ以外のときは$0$となる関数である.
  \begin{equation}
    \label{eq:Kronecker-delta}
    \delta_{ij}=\begin{cases}
      1 & \pab{i=j}     \\
      0 & \pab{i\neq j}
    \end{cases}
  \end{equation}
  Kronecker deltaは,$\vect{e}$の内積を用いて次のように表すことができる.
  \begin{equation}
    \label{eq:Kronecker-delta-inner-product}
    \delta_{ij}=\vect{e}_i\cdot\vect{e}_j
  \end{equation}
  Kronecker deltaは,行列の対角成分を表すのに便利である.
\end{FormulaBox}
\begin{FormulaBox}{Levi-Civita symbol}{Levi-Civita-symbol}
  Levi-Civita symbolは,添字$i,j,k$の並び方に応じて次のような値を取るものとして定義される.
  \begin{equation}
    \label{eq:Levi-Civita-symbol}
    \epsilon_{ijk}=\begin{cases}
      +1 & \pab{\pab{i,j,k}\in\Bab{\pab{1,2,3}, \pab{2,3,1}, \pab{3,1,2}}} \\
      -1 & \pab{\pab{i,j,k}\in\Bab{\pab{1,3,2}, \pab{3,2,1}, \pab{2,1,3}}} \\
      0  & \pab{\text{otherwise}}
    \end{cases}
  \end{equation}
  添字$ijk$ が, $\pab{1,2,3}$ の偶置換である場合は $+1$ ,奇置換である場合は $-1$ ,それ以外は$0$となる.正規直交基底のスカラー三重積で表示することもできる.
  \begin{equation}
    \label{eq:Levi-Civita-ScalarTripleProduct}
    \epsilon_{ijk}=\pab{\vect{e}_i\times\vect{e}_j}\cdot\vect{e}_k
  \end{equation}
  また,Levi-Civita symbolは循環性がある.$i\rightarrow j$,$j\rightarrow k$,$k\rightarrow i$のとき,$\epsilon_{ijk}=\epsilon_{jki}=\epsilon_{kij}$が成り立つ.
\end{FormulaBox}
ここで適当なベクトル$\vect{a}$,$\vect{b}$,$\vect{c}$を次のように定義する.
\begin{equation}
  \vect{a}=\begin{bmatrix}
    a_1 \\a_2\\a_3
  \end{bmatrix}, \quad
  \vect{b}=\begin{bmatrix}
    b_1 \\b_2\\b_3
  \end{bmatrix}, \quad
  \vect{c}=\begin{bmatrix}
    c_1 \\c_2\\c_3
  \end{bmatrix}
\end{equation}

\begin{FormulaBox}{ベクトルの内積}{Vector-InnerProduct}
  ベクトル$\vect{a}$と$\vect{b}$の内積は次のように定義される.
  \begin{equation}
    \label{eq:Vector-InnerProduct}
    \vect{a}\cdot\vect{b}=\sum_{i=1}^{3}a_ib_i=\pab{\vect{a}\cdot\vect{b}}_i
  \end{equation}
  また,内積はKronecker deltaを用いて次のように表すことができる.
  \begin{equation}
    \label{eq:Vector-InnerProduct-KroneckerDelta}
    \vect{a}\cdot\vect{b}=\delta_{ij}a_ib_j
  \end{equation}
\end{FormulaBox}
\begin{FormulaBox}{ベクトルの外積}{Vector-OuterProduct}
  ベクトル$\vect{a}$と$\vect{b}$の外積
  \begin{equation}
    \vect{a}\times\vect{b}=\begin{vmatrix}
      \vect{e}_x & \vect{e}_y & \vect{e}_z \\
      a_1        & a_2        & a_3        \\
      b_1        & b_2        & b_3
    \end{vmatrix}=\begin{bmatrix}
      a_2b_3-a_3b_2 \\
      a_3b_1-a_1b_3 \\
      a_1b_2-a_2b_1
    \end{bmatrix}
  \end{equation}
  の各成分はLevi-Civita symbolを用いて次のように表すことができる.
  \begin{equation}
    \label{eq:Levi-Civita-OuterProduct}
    \pab{\vect{a}\times\vect{b}}_i=\sum_{j=1}^{3}\sum_{k=1}^{3}\epsilon_{ijk}a_jb_k
  \end{equation}
\end{FormulaBox}
\begin{proof}
  \eqref{eq:Levi-Civita-symbol}を用いると
  \begin{align*}
      & \sum_{j=1}^{3}\sum_{k=1}^{3}\epsilon_{1jk}a_jb_k\notag                \\
    = & \epsilon_{111}a_1b_1+\epsilon_{112}a_1b_2+\epsilon_{113}a_1b_3\notag  \\
      & +\epsilon_{121}a_2b_1+\epsilon_{122}a_2b_2+\epsilon_{123}a_2b_3\notag \\
      & +\epsilon_{131}a_3b_1+\epsilon_{132}a_3b_2+\epsilon_{133}a_3b_3\notag \\
    = & a_2b_3-a_3b_2\notag                                                   \\
    = & \pab{\vect{a}\times\vect{b}}_1
  \end{align*}
  同様に
  \begin{align*}
      & \sum_{j=1}^{3}\sum_{k=1}^{3}\epsilon_{2jk}a_jb_k\notag                \\
    = & \epsilon_{211}a_1b_1+\epsilon_{212}a_1b_2+\epsilon_{213}a_1b_3\notag  \\
      & +\epsilon_{221}a_2b_1+\epsilon_{222}a_2b_2+\epsilon_{223}a_2b_3\notag \\
      & +\epsilon_{231}a_3b_1+\epsilon_{232}a_3b_2+\epsilon_{233}a_3b_3\notag \\
    = & a_3b_1-a_1b_3\notag                                                   \\
    = & \pab{\vect{a}\times\vect{b}}_2
  \end{align*}
  \begin{align*}
      & \sum_{j=1}^{3}\sum_{k=1}^{3}\epsilon_{3jk}a_jb_k\notag                \\
    = & \epsilon_{311}a_1b_1+\epsilon_{312}a_1b_2+\epsilon_{313}a_1b_3\notag  \\
      & +\epsilon_{321}a_2b_1+\epsilon_{322}a_2b_2+\epsilon_{323}a_2b_3\notag \\
      & +\epsilon_{331}a_3b_1+\epsilon_{332}a_3b_2+\epsilon_{333}a_3b_3\notag \\
    = & a_1b_2-a_2b_1\notag                                                   \\
    = & \pab{\vect{a}\times\vect{b}}_3
  \end{align*}
\end{proof}
\begin{FormulaBox}{Levi-Civita symbolの恒等式}{Levi-Civita-symbol-Identity}
  $n=3$のLevi-Civita symbolには次の恒等式が成り立つ.
  \begin{equation}
    \label{eq:Levi-Civita-Identity}
    \sum_{i=1}^{3}\epsilon_{ijk}\epsilon_{ilm}=\delta_{jl}\delta_{km}-\delta_{jm}\delta_{kl}
  \end{equation}
\end{FormulaBox}
\begin{proof}
  \eqref{eq:Levi-Civita-ScalarTripleProduct}を用いると,\eqref{eq:Levi-Civita-Identity}の左辺は次のように表される.
  \begin{align}
    \label{eq:Levi-Civita-Identity-proof-1}
    \sum_{i=1}^{3}\epsilon_{ijk}\epsilon_{ilm} & =\sum_{i=1}^{3}\bab[big]{\pab[big]{\vect{e}_i\times\vect{e}_j}\cdot\vect{e}_k}\bab[big]{\pab[big]{\vect{e}_i\times\vect{e}_l}\cdot\vect{e}_m}
  \end{align}
  一般にスカラー三重積は,それを構成する三つのベクトルを列ベクトルとする行列の行列式に等しい.よって\eqref{eq:Levi-Civita-Identity-proof-1}の右辺に現れたスカラー三重積を行列式に書き換えると,
  \begin{align}
    \label{eq:Levi-Civita-Identity-proof-2}
    \sum_{i=1}^{3}\epsilon_{ijk}\epsilon_{ilm} & =\sum_{i=1}^{3}\bab[big]{\pab[big]{\vect{e}_i\times\vect{e}_j}\cdot\vect{e}_k}\bab[big]{\pab[big]{\vect{e}_i\times\vect{e}_l}\cdot\vect{e}_m}\notag \\
                                               & = \sum_{i=1}^{3}
    \det\begin{bmatrix}
          \vect{e}_i & \vect{e}_j & \vect{e}_k
        \end{bmatrix}
    \det\begin{bmatrix}
          \vect{e}_i & \vect{e}_l & \vect{e}_m
        \end{bmatrix}
  \end{align}
  一般に転置行列の行列式はもとの行列の行列式と等しく,行列の積の行列式はそれぞれの行列の行列式の積に等しいので,\eqref{eq:Levi-Civita-Identity-proof-2}は次のように書き換えることができる.
  \begin{align}
    \label{eq:Levi-Civita-Identity-proof-3}
    \sum_{i=1}^{3}\epsilon_{ijk}\epsilon_{ilm} & =\sum_{i=1}^{3}
    \det\begin{bmatrix}
          \vect{e}_i^\top \\ \vect{e}_j^\top \\ \vect{e}_k^\top
        \end{bmatrix}
    \det\begin{bmatrix}
          \vect{e}_i & \vect{e}_l & \vect{e}_m
        \end{bmatrix}\notag                                                                                                               \\
                                               & =\sum_{i=1}^{3}\det~\pab{\begin{bmatrix}
                                                                              \vect{e}_i^\top \\ \vect{e}_j^\top \\ \vect{e}_k^\top\\
                                                                            \end{bmatrix}
    \begin{bmatrix}
        \vect{e}_i & \vect{e}_l & \vect{e}_m
      \end{bmatrix}}\notag                                                                                                                   \\
                                               & =\sum_{i=1}^{3}\det\begin{bmatrix}
                                                                      \vect{e}_i^\top \vect{e}_i & \vect{e}_i^\top \vect{e}_l & \vect{e}_i^\top \vect{e}_m \\
                                                                      \vect{e}_j^\top \vect{e}_i & \vect{e}_j^\top \vect{e}_l & \vect{e}_j^\top \vect{e}_m \\
                                                                      \vect{e}_k^\top \vect{e}_i & \vect{e}_k^\top \vect{e}_l & \vect{e}_k^\top \vect{e}_m \\
                                                                    \end{bmatrix}
  \end{align}
  ここで,ベクトルの内積が$\vect{a}\cdot\vect{b}=\vect{a}^\top\vect{b}$であり,$\vect{e}$が正規直交基底をなすことに注意すれば,
  \begin{align}
    \label{eq:Levi-Civita-Identity-proof-4}
    \sum_{i=1}^{3}\epsilon_{ijk}\epsilon_{ilm} & =\sum_{i=1}^{3}\det\begin{bmatrix}
                                                                      \vect{e}_i\cdot\vect{e}_i & \vect{e}_i\cdot\vect{e}_l & \vect{e}_i\cdot\vect{e}_m \\
                                                                      \vect{e}_j\cdot\vect{e}_i & \vect{e}_j\cdot\vect{e}_l & \vect{e}_j\cdot\vect{e}_m \\
                                                                      \vect{e}_k\cdot\vect{e}_i & \vect{e}_k\cdot\vect{e}_l & \vect{e}_k\cdot\vect{e}_m \\
                                                                    \end{bmatrix}\notag \\
                                               & =\sum_{i=1}^{3}\det\begin{bmatrix}
                                                                      \delta_{ii} & \delta_{il} & \delta_{im} \\
                                                                      \delta_{ji} & \delta_{jl} & \delta_{jm} \\
                                                                      \delta_{ki} & \delta_{kl} & \delta_{km} \\
                                                                    \end{bmatrix}\notag                                           \\
                                               & =\sum_{i=1}^{3}\det\begin{bmatrix}
                                                                      1           & \delta_{il} & \delta_{im} \\
                                                                      \delta_{ji} & \delta_{jl} & \delta_{jm} \\
                                                                      \delta_{ki} & \delta_{kl} & \delta_{km} \\
                                                                    \end{bmatrix}
  \end{align}
  \eqref{eq:Levi-Civita-Identity-proof-4}の行列式を余因子展開すれば,
  \begin{align}
    \label{eq:Levi-Civita-Identity-proof-5}
    \sum_{i=1}^{3}\epsilon_{ijk}\epsilon_{ilm} & =\sum_{i=1}^{3}\det\begin{bmatrix}
                                                                      1           & \delta_{il} & \delta_{im} \\
                                                                      \delta_{ji} & \delta_{jl} & \delta_{jm} \\
                                                                      \delta_{ki} & \delta_{kl} & \delta_{km} \\
                                                                    \end{bmatrix}\notag                                                                                                                                          \\
                                               & =\sum_{i=1}^{3}\Bab{\begin{vmatrix}
                                                                         \delta_{jl} & \delta_{jm} \\
                                                                         \delta_{kl} & \delta_{km} \\
                                                                       \end{vmatrix}+\delta_{il}\begin{vmatrix}
                                                                                                  \delta_{jm} & \delta_{ji} \\
                                                                                                  \delta_{km} & \delta_{ki} \\
                                                                                                \end{vmatrix}+\delta_{im}\begin{vmatrix}
                                                                                                                           \delta_{ji} & \delta_{jl} \\
                                                                                                                           \delta_{ki} & \delta_{kl} \\
                                                                                                                         \end{vmatrix}}\notag                                                                                                     \\
                                               & =\sum_{i=1}^{3}\bab{\pab{\delta_{jl}\delta_{km}-\delta_{jm}\delta_{kl}}+\delta_{il}\pab{\delta_{jm}\delta_{ki}-\delta_{ji}\delta_{km}}+\delta_{im}\pab{\delta_{ji}\delta_{kl}-\delta_{jl}\delta_{ki}}}
  \end{align}
  Kronecker-deltaの定義に注意し売れば,\eqref{eq:Levi-Civita-Identity-proof-5}の右辺は次のように書き換えることができる.
  \begin{subequations}
    \begin{equation}
      \label{eq:Levi-Civita-Identity-proof-6}
      \sum_{i=1}^{3}\delta_{jl}\delta_{km}-\delta_{jm}\delta_{kl}=3\pab{\delta_{jl}\delta_{km}-\delta_{jm}\delta_{kl}}
    \end{equation}
    \begin{equation}
      \label{eq:Levi-Civita-Identity-proof-7}
      \sum_{i=1}^{3}\delta_{il}\pab{\delta_{jm}\delta_{ki}-\delta_{ji}\delta_{km}}=\delta_{jm}\delta_{kl}-\delta_{jl}\delta_{km}
    \end{equation}
    \begin{equation}
      \label{eq:Levi-Civita-Identity-proof-8}
      \sum_{i=1}^{3}\delta_{im}\pab{\delta_{ji}\delta_{kl}-\delta_{jl}\delta_{ki}}=\delta_{jm}\delta_{kl}-\delta_{jl}\delta_{km}
    \end{equation}
  \end{subequations}
  よって,
  \eqref{eq:Levi-Civita-Identity-proof-5}は次のように書き換えることができる.
  \begin{align}
    \label{eq:Levi-Civita-Identity-proof-9}
    \sum_{i=1}^{3}\epsilon_{ijk}\epsilon_{ilm} & =\sum_{i=1}^{3}\bab{\pab{\delta_{jl}\delta_{km}-\delta_{jm}\delta_{kl}}+\delta_{il}\pab{\delta_{jm}\delta_{ki}-\delta_{ji}\delta_{km}}+\delta_{im}\pab{\delta_{ji}\delta_{kl}-\delta_{jl}\delta_{ki}}}\notag \\
                                               & =3\pab{\delta_{jl}\delta_{km}-\delta_{jm}\delta_{kl}}+\delta_{jm}\delta_{kl}-\delta_{jl}\delta_{km}+\delta_{jm}\delta_{kl}-\delta_{jl}\delta_{km}\notag                                                      \\
                                               & =\delta_{jl}\delta_{km}-\delta_{jm}\delta_{kl}
  \end{align}
\end{proof}

\begin{FormulaBox}{発散(Divergence)}{Divergence-Definition}
  ベクトル場$\vect{a}\pab{x, y, z}$の各成分の空間微分によって定義されるスカラー場を,$\vect{a}$の発散(Divergence)と呼び,$\nabla\cdot\vect{a}$あるいは$\div\vect{a}$と表記する.
  \begin{equation}
    \label{eq:Divergence-Def}
    \nabla\cdot\vect{a} = \pdv{a_1}{x} + \pdv{a_2}{y} + \pdv{a_3}{z} = \sum_{i=1}^{3}\pdv{a_i}{x_i}
  \end{equation}
  ここで,$\nabla$はナブラ演算子
  \begin{equation*}
    \nabla = \vect{e}_x \pdv{}{x} + \vect{e}_y \pdv{}{y} + \vect{e}_z \pdv{}{z}
  \end{equation*}
  を表し,形式的にナブラ演算子とベクトルの内積として扱うことができる.
\end{FormulaBox}

\begin{FormulaBox}{ガウスの発散定理(3次元)}{Gauss-Divergence-Theorem-3D}
  空間内の閉領域$V$とその境界である閉曲面$\partial V$を考える.ベクトル場$\vect{a}$が$V$を含む領域で連続な偏導関数を持つとき,以下の等式が成り立つ.
  \begin{equation}
    \label{eq:Gauss-Divergence-Theorem-3D}
    \int_{V} \nabla\cdot\vect{a} \odif{V} = \int_{\partial V} \vect{a}\cdot\vect{n} \odif{S}
  \end{equation}
  ここで,$\vect{n}$は閉曲面$\partial V$上の外向き単位法線ベクトル,$dV$は体積要素,$dS$は面積要素である.
\end{FormulaBox}

\begin{proof}
  領域$V$が$x, y, z$の各軸方向に関して単純な領域(任意の軸に平行な直線が,境界と高々2点で交わる領域)であると仮定する.
  まず,$\nabla\cdot\vect{a}$の第3項($z$微分成分)の体積積分を考える.$V$の$xy$平面への正射影を$D$とし,$V$の境界$\partial V$のうち,上側の面を$z=z_{top}\pab{x,y}$,下側の面を$z=z_{btm}\pab{x,y}$とする.
  \begin{align}
    \label{eq:Gauss-Proof-3D-1}
    \int_{V} \frac{\partial a_3}{\partial z} \, dV 
     & = \iint_{D} \bab{\int_{z_{btm}\pab{x,y}}^{z_{top}\pab{x,y}} \frac{\partial a_3}{\partial z} \, dz} \, dxdy \notag \\
     & = \iint_{D} \bab{a_3\pab{x,y,z_{top}} - a_3\pab{x,y,z_{btm}}} \, dxdy
  \end{align}
  ここで,上側の面における外向き法線ベクトル$\vect{n}_{top}$と$\vect{e}_z$のなす角を$\gamma_{top}$とすると,面素の関係$dxdy = \cos\gamma_{top} dS = \pab{\vect{n}_{top}\cdot\vect{e}_z}dS$が成り立つ.同様に下側の面では,外向き法線$\vect{n}_{btm}$が下を向くため,$\vect{n}_{btm}\cdot\vect{e}_z$は負となり,$dxdy = -\cos\gamma_{btm} dS = -\pab{\vect{n}_{btm}\cdot\vect{e}_z}dS$となる.
  これらを\eqref{eq:Gauss-Proof-3D-1}に代入し,閉曲面$\partial V$全体での積分に書き換えると,
  \begin{align}
    \label{eq:Gauss-Proof-3D-2}
    \iint_{D} a_3\pab{x,y,z_{top}} \, dxdy + \iint_{D} -a_3\pab{x,y,z_{btm}} \, dxdy 
     & = \int_{\partial V_{top}} a_3 \pab{\vect{n}\cdot\vect{e}_z} \, dS + \int_{\partial V_{btm}} a_3 \pab{\vect{n}\cdot\vect{e}_z} \, dS \notag \\
     & = \int_{\partial V} a_3 n_3 \, dS
  \end{align}
  となる.$x, y$成分についても同様の議論を行うことで,
  \begin{equation}
    \int_{V} \frac{\partial a_1}{\partial x} \, dV = \int_{\partial V} a_1 n_1 \, dS, \quad
    \int_{V} \frac{\partial a_2}{\partial y} \, dV = \int_{\partial V} a_2 n_2 \, dS
  \end{equation}
  が得られる.これら3つの式の和をとることで,\eqref{eq:Gauss-Divergence-Theorem-3D}が得られる.
  なお,複雑な形状の領域であっても,単純な領域の和に分割することで,本定理は同様に成立する.
\end{proof}

\begin{FormulaBox}{ガウスの発散定理(2次元)}{Gauss-Divergence-Theorem-2D}
  平面内の閉領域$S$とその境界である閉曲線$\partial S$を考える.2次元ベクトル場$\vect{a}=\begin{bmatrix}a_1 & a_2\end{bmatrix}^\top$について,以下の等式が成り立つ.
  \begin{equation}
    \label{eq:Gauss-Divergence-Theorem-2D}
    \int_{S} \nabla\cdot\vect{a} \, dS = \oint_{\partial S} \vect{a}\cdot\vect{n} \, dl
  \end{equation}
  ここで,$\nabla\cdot\vect{a} = \dfrac{\partial a_1}{\partial x} + \frac{\partial a_2}{\partial y}$であり,$\vect{n}$は境界$\partial S$上の外向き単位法線ベクトル,$l$は弧長パラメータである.
\end{FormulaBox}

\begin{proof}
  3次元の場合と同様に,$S$を$y$軸に平行な線分で切ったときに境界と高々2点で交わる単純な領域と仮定する.$\nabla\cdot\vect{a}$の第2項($y$微分成分)の面積分を考える.
  境界$\partial S$を,上側の曲線$y=y_{top}\pab{x}$と下側の曲線$y=y_{btm}\pab{x}$(定義域は$x_1 \le x \le x_2$)に分割する.
  \begin{align}
    \label{eq:Gauss-Proof-2D-1}
    \int_{S} \frac{\partial a_2}{\partial y} \, dS 
     & = \int_{x_1}^{x_2} \bab{\int_{y_{btm}\pab{x}}^{y_{top}\pab{x}} \frac{\partial a_2}{\partial y} \, dy} \, dx \notag \\
     & = \int_{x_1}^{x_2} \bab{a_2\pab{x,y_{top}} - a_2\pab{x,y_{btm}}} \, dx
  \end{align}
  ここで,境界上の線素ベクトル$d\vect{l}$を反時計回りに取ると,外向き法線ベクトル$\vect{n}$を用いて$\vect{n}\,dl = \pab{dy, -dx}^\top$の関係がある(あるいは幾何学的に,$dx = n_y dl / \sin\theta$等の関係を用いる).
  上側の曲線では$x$が増加するにつれて積分路は右へ進むが,$\vect{n}$の$y$成分は正であるため,$dx = \frac{dx}{dl}dl$に対し$n_2 dl \approx dx$の関係(厳密には$n_2 dl = dx$)が成り立つ.下側の曲線では積分路の向きと$x$の増分が逆になること等を考慮し,線積分へ書き換えると以下のようになる.
  \begin{equation}
    \int_{x_1}^{x_2} a_2\pab{x,y_{top}} \, dx - \int_{x_1}^{x_2} a_2\pab{x,y_{btm}} \, dx 
    = \int_{\partial S_{top}} a_2 n_2 \, dl + \int_{\partial S_{btm}} a_2 n_2 \, dl 
    = \oint_{\partial S} a_2 n_2 \, dl
  \end{equation}
  $x$成分についても同様に$\int_{S} \frac{\partial a_1}{\partial x} \, dS = \oint_{\partial S} a_1 n_1 \, dl$が示せるため,両者の和をとることで\eqref{eq:Gauss-Divergence-Theorem-2D}が得られる.
\end{proof}

\begin{FormulaBox}{発散の積の微分}{Identity-Div-Product}
  スカラー場$f$とベクトル場$\vect{A}$の積の発散について,以下の恒等式が成り立つ.
  \begin{equation}
    \label{eq:Identity-Div-Product}
    \nabla \cdot \pab{f \vect{A}} = \nabla f \cdot \vect{A} + f \nabla \cdot \vect{A}
  \end{equation}
  これは,1変数の積の微分公式 $\pab{uv}' = u'v + uv'$ の多次元(ベクトル解析)版に相当する.
\end{FormulaBox}

\begin{proof}
  直交座標系における成分計算によって示す.$\nabla$演算子の$i$成分を$\partial_i = \frac{\partial}{\partial x_i}$,$\vect{A}$の第$i$成分を$A_i$と表記し,アインシュタインの縮約記法を用いる.
  左辺を展開し,積の微分公式を適用すると以下のようになる.
  \begin{align}
    \nabla \cdot \pab{f \vect{A}} & = \partial_i \pab{f A_i} \notag                   \\
                                  & = \pab{\partial_i f} A_i + f \pab{\partial_i A_i}
  \end{align}
  ここで,第1項は勾配ベクトル$\nabla f$と$\vect{A}$の内積 $\pab{\nabla f} \cdot \vect{A}$ であり,第2項は$f$と発散 $\nabla \cdot \vect{A}$ の積であるため,
  \begin{equation}
    = \nabla f \cdot \vect{A} + f \nabla \cdot \vect{A}
  \end{equation}
  となり,等式が示された.
\end{proof}