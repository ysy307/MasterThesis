\section{グラフ理論に基づいた彩色アルゴリズム}
グラフとは、頂点 (Node)とそれらをつなぐ辺 (Edge)によって構成される。通常有限要素法や有限差分法で出てくるグラフは、分割した節点に対して、要素分割と同じように構築する節点接続グラフや要素同士の接続によって構成される要素接続グラフなどが構れる。これらはもちろん、均等に分割されていればグリッド状のグラフにはなるが、有限要素法においては、数値解析として非構造メッシュを用いることが容易であり、多くの場合複雑な形状に対して、複雑なメッシュ分割が行われる。ただし、いずれの場合においても、無向グラフを構築することで事足りる。本節では、特に並列処理のための要素彩色について説明する。

\subsection{グラフについて}
グラフ$G(V,E)$は$n$個の頂点によって構成される頂点集合$V$と$m$個の辺によって構成される辺集合$E$によって定義される。

\begin{dfn}{a}{Def-Graph}
  もし、$\Bab{u,v}\in E$ならば、頂点$u$と$v$は隣接している (adjacent)といえる。頂点$u$と$v$が接続している (incident)とき、$\Bab{u,v}\in E$である。もし、$\Bab{u,v}\notin E$ならば、頂点$u$と$v$は隣接していない (nonadjacent)といえる。
\end{dfn}
\begin{dfn}{b}{Def-Graph-Degree}
  頂点$v$の近傍を$\Gamma_G\pab{v}$と書けば、それはグラフ$G$における$v$に隣接している頂点の集合である。これは$\Gamma_G\pab{v}=\Bab{u\in V:\Bab{v,u}\in E}$である。頂点$v$における次数 (degree)は近傍集合の濃度 (cardinality)\footnote{濃度は集合の個数を一般化したものである。ある集合$A$がn個の有限集合である場合、濃度は$n$であるという。本論文内で扱う集合はすべて有限集合なので、集合の個数と読み替えてもよい。}$\abs{\Gamma_G\pab{v}}$であり、たいてい$\mathrm{deg}_G\pab{v}$と表記する。ただし、簡便のため添え字の$G$を省略して、$\Gamma\pab{v}$、$\mathrm{deg}\pab{v}$と書く。
\end{dfn}
\begin{dfn}{}{Def-Graph-Density}
  グラフ$G=\pab{V,E}$の密度は頂点のペアの数と辺の数の比率で定義される。ループがない単純グラフなら密度は$m/((n(n−1))/2)$と計算される。密度が低いグラフは疎グラフ (sparse graph)、密度が高いグラフは密グラフ (dense graph)と呼ばれる。
\end{dfn}
\begin{dfn}{}{}
\end{dfn}
