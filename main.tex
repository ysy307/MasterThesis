%============================================================================================================
% MasterThesis main.tex
%============================================================================================================
% 概要:
%   修士論文の主ファイル(LuaLaTeX + luatexja を想定)
%   このファイルは各章を \input で読み込み、目次・図表・参考文献を出力します。
%============================================================================================================
\documentclass[a4paper,12pt]{ltjsarticle}
\usepackage{luatexja}
\usepackage{luatexja-fontspec}
\usepackage[lualatex]{luatexja-preset}
\usepackage{unicode-math}
%============================================================================================================
% 章・節見出しの設定
%============================================================================================================
\usepackage{titlesec}
\usepackage{titletoc}

% 欧文本文(Serif)
\setmainfont{TeX Gyre Termes}
% 和文本文(明朝)
\setmainjfont[
  UprightFont = SourceHanSerif-Regular.otf,
  BoldFont    = SourceHanSerif-Bold.otf
]{Source Han Serif}

% 和文明朝(個別指定用)
\newjfontfamily\jmincho[
  UprightFont = SourceHanSerif-Regular.otf,
  BoldFont    = SourceHanSerif-Bold.otf
]{Source Han Serif}

% 和文ゴシック(見出し用)
\newjfontfamily\jgothic[
  UprightFont = SourceHanSans-Regular.otf,
  BoldFont    = SourceHanSans-Medium.otf
]{Source Han Sans}

% 欧文見出しフォント(Serifのまま)
\newfontfamily\esectionfont[
  UprightFont = SourceHanSans-Regular.otf,
  BoldFont    = SourceHanSans-Medium.otf
]{Source Han Sans}

% 見出し用統合フォント
\newcommand{\headingfont}{\esectionfont\jgothic\bfseries}
\newcommand{\headingtocfont}{\esectionfont\jmincho}

%============================================================================================================
% part の設定
%============================================================================================================
\newcommand{\partlabel}{第\thepart 部}
\titleformat{\part}[block]
  {\headingfont\fontsize{40pt}{70pt}\selectfont\centering}
  {\partlabel}
  {0pt}
  {\\[1ex]}
\titlespacing*{\part}{0pt}{0.35\textheight}{40pt}
%============================================================================================================
% section 以下の設定
%============================================================================================================
\titleformat{\section}[block]
  {\headingfont\LARGE}
  {\thesection}{1em}{}
  [\titlerule]

\titleformat{\subsection}
  {\headingfont\Large}{\thesubsection}{1em}{}

\titleformat{\subsubsection}
  {\headingfont\large}{\thesubsubsection}{1em}{}

\titlespacing*{\section}{0pt}{3.5ex plus 1ex minus .2ex}{2.3ex plus .2ex}
\titlespacing*{\subsection}{0pt}{3.25ex plus 1ex minus .2ex}{1.5ex plus .2ex}
\titlespacing*{\subsubsection}{0pt}{3.25ex plus 1ex minus .2ex}{1.5ex plus .2ex}
%============================================================================================================
% 目次設定
%============================================================================================================
\titlecontents{part}
  [0pt]
  {\addvspace{10pt}\headingfont}
  {\makebox[3em][l]{第\thecontentslabel 部\enspace}}
  {}
  {\hfill\contentspage}

\titlecontents{section}
  [1.5em]
  {\headingtocfont}
  {\makebox[3em][l]{\thecontentslabel\enspace}}
  {\hspace*{1.5em}}
  {\titlerule*[.5pc]{.}\contentspage}

\titlecontents{subsection}
  [3em]
  {\headingtocfont}
  {\makebox[4em][l]{\thecontentslabel\enspace}} 
  {\hspace*{3.8em}}
  {\titlerule*[.5pc]{.}\contentspage}

\titlecontents{subsubsection}
  [4.5em]
  {\headingtocfont}
  {\makebox[5em][l]{\thecontentslabel\enspace}}
  {\hspace*{7.0em}}
  {\titlerule*[.5pc]{.}\contentspage}

%============================================================================================================
% 章番号の設定
%============================================================================================================
\usepackage{secdot}
\sectiondot{subsection}
\renewcommand{\thepart}{\arabic{part}}
\renewcommand{\thesection}{\thepart.\arabic{section}}
\renewcommand{\thesubsection}{\thesection.\arabic{subsection}}
\makeatletter
\@addtoreset{section}{part}
\makeatother
\setcounter{secnumdepth}{3}
%============================================================================================================
% マージン設定
%============================================================================================================
\usepackage[top=30truemm,bottom=30truemm,left=30truemm,right=20truemm]{geometry}
% 全角文字幅を基準にしたマージン設定
% 35字×40行
\setlength{\baselineskip}{22.35pt}
\ltjsetparameter{jacharrange={-2}}
\setlength{\zw}{12.95pt}

% 基本パッケージ
\usepackage{cancel}
%============================================================================================================
% URL・ハイパーリンク
%============================================================================================================
\usepackage{xurl}
\urlstyle{same}
\usepackage{hyperref}
\hypersetup{
  luatex,
  pdfencoding=auto,
	setpagesize=false,
	bookmarks=true,
	bookmarksdepth=tocdepth,
	bookmarksnumbered=true,
	colorlinks=true,
	linkcolor=black,
	citecolor=black,
	urlcolor=black,
	filecolor=black,
	pdftitle={地下水流存在下の地盤凍結過程の水分・熱の同時解析手法の確立に向けて},
	pdfsubject={Toward the Establishment of a Method for Simultaneous Analysis of Water and Heat in Ground Freezing Processes in the Presence of Groundwater Flow},
	pdfauthor={菊地 駿},
	pdfkeywords={土壌凍結, 地盤凍結, 地下水流, 水・熱移動, 数値解析}
}

%============================================================================================================
% Bibliography settings
%============================================================================================================
\usepackage[
  block=par,
  backend=biber,
  style=../ref/biblatex/jpa, 
  hyperref=true,        % ハイパーリンクを有効化
  natbib=true,          % 互換性を高める
  labeldateparts=true,  % 日付のリンク精度を上げる
  uniquename=full,
  uniquelist=true
]{biblatex}
\DeclareNameFormat{labelname}{%
  \ifhyperref
    {\bibhyperref{%
       \usebibmacro{labelname:doname}%
         {\namepartfamily}%
         {\namepartfamilyi}%
         {\namepartgiven}%
         {\namepartgiveni}%
         {\namepartprefix}%
         {\namepartprefixi}%
         {\namepartsuffix}%
         {\namepartsuffixi}}}
    {\usebibmacro{labelname:doname}%
       {\namepartfamily}%
       {\namepartfamilyi}%
       {\namepartgiven}%
       {\namepartgiveni}%
       {\namepartprefix}%
       {\namepartprefixi}%
       {\namepartsuffix}%
       {\namepartsuffixi}}%
}
\renewbibmacro*{cite:plabelyear+extradate}{%
  \ifhyperref
    {\bibhyperref{%
       \usebibmacro{cite:plabelyear+extradate:inner}}}
    {\usebibmacro{cite:plabelyear+extradate:inner}}%
}

% 元の処理を内部マクロとして定義(jpa.cbxの内容をラップ)
\newbibmacro*{cite:plabelyear+extradate:inner}{%
  \iffieldundef{year}{%
    \iffieldundef{pubstate}{\printlabeldateextra}%
    {\printfield{pubstate}}%
  }{%
    \ifthenelse{\boolean{japanese}}{%
      \iffieldundef{origyear}{}{%
        \printorigdate%
        \addspace}%
      \iffieldequalstr{relatedtype}{translationof}{%
        \entrydata*{\thefield{related}}{\printdateextra}%
        \addspace}{}%
      \ifnameundef{translator}{}{%
        \printnames{translator}%
        \iffieldundef{translatortype}{\printtext{訳}\space}%
        {\printfield{translatortype}\space}%
      }%
      \printlabeldateextra%
    }{%
      \iffieldundef{origyear}{}{\printorigdate\setunit*{\addslash}}%
      \iffieldequalstr{relatedtype}{translationof}{%
        \entrydata*{\thefield{related}}{\printlabeldateextra}%
        \setunit*{\addslash}%
      }{} %
      \printlabeldateextra%
    }%
  }%
}
\addbibresource{references.bib}

%============================================================================================================
% 画像設定
%============================================================================================================
\usepackage{graphicx}
\usepackage{subcaption}
\usepackage{float}
\usepackage{wrapfig}
\usepackage{placeins}

\graphicspath{{/workspace/Images/Master/}}
%============================================================================================================
% TikZ settings
%============================================================================================================
\usepackage{tikz}
\usepackage{tikz-3dplot}
\usetikzlibrary{intersections,calc,arrows.meta,patterns,patterns.meta, decorations.markings}
%============================================================================================================
% Table settings
%============================================================================================================
\usepackage{booktabs}
\usepackage{longtable}
\usepackage{multicol}
\usepackage{multirow}



%============================================================================================================
% Math settings
%============================================================================================================
\usepackage{amsthm}
\usepackage{mathtools}
\usepackage{physics2}
\usephysicsmodule{ab}
\usepackage{derivative}
\usepackage{ifthen}

\allowdisplaybreaks[4]
\numberwithin{equation}{section}
\usepackage{../ref/mymathdef}

% Chemical formulas
\usepackage[version=3]{mhchem}
%============================================================================================================
% Caption settings
%============================================================================================================
\usepackage{cleveref}
% 数式番号
\crefformat{equation}{#2(#1) 式#3}
\crefrangeformat{equation}{#3(#1)#4~#5(#2)#6 式}
\crefmultiformat{equation}
  {#2(#1) 式#3}
  {,#2(#1) 式#3}
  {,#2(#1) 式#3}
  {,#2(#1) 式#3}
\crefrangemultiformat{equation}
  {#3(#1)#4~#5(#2)#6 式}
  {,#3(#1)#4~#5(#2)#6 式}
  {,#3(#1)#4~#5(#2)#6 式}
  {,#3(#1)#4~#5(#2)#6 式}

% 図番号
\crefformat{figure}{#2Fig.~#1#3}
\crefrangeformat{figure}{#3Figs.~#1#4--#5#2#6}
\crefmultiformat{figure}
  {#2Figs.~#1#3}
  {, #2#1#3}
  {, #2#1#3}
  {, #2#1#3}
\crefrangemultiformat{figure}
  {#3Figs.~#1#4--#5#2#6}
  {, #3#1#4--#5#2#6}
  {, #3#1#4--#5#2#6}
  {, #3#1#4--#5#2#6}

% subfigure番号
\crefformat{subfigure}{#2Fig.~#1#3}
\crefrangeformat{subfigure}{#3Figs.~#1#4--#5#2#6}
\crefmultiformat{subfigure}
  {#2Figs.~#1#3}
  {, #2#1#3}
  {, #2#1#3}
  {, #2#1#3}
\crefrangemultiformat{subfigure}
  {#3Figs.~#1#4--#5#2#6}
  {, #3#1#4--#5#2#6}
  {, #3#1#4--#5#2#6}
  {, #3#1#4--#5#2#6}

% 表番号
\crefformat{table}{#2Table~#1#3}
\crefrangeformat{table}{#3Tables~#1#4--#5#2#6}
\crefmultiformat{table}
  {#2Tables~#1#3}
  {, #2#1#3}
  {, #2#1#3}
  {, #2#1#3}
\crefrangemultiformat{table}
  {#3Tables~#1#4--#5#2#6}
  {, #3#1#4--#5#2#6}
  {, #3#1#4--#5#2#6}
  {, #3#1#4--#5#2#6}

% part番号
\crefformat{part}{#2第#1部#3}
\crefrangeformat{part}{#3第#1部#4~#5第#2部#6}
\crefmultiformat{part}
  {#2第#1部#3}
  {,#2第#1部#3}
  {,#2第#1部#3}
  {,#2第#1部#3}
\crefrangemultiformat{part}
  {#3第#1部#4~#5第#2部#6}
  {,#3第#1部#4~#5第#2部#6}
  {,#3第#1部#4~#5第#2部#6}
  {,#3第#1部#4~#5第#2部#6}

% section番号
\crefformat{section}{#2#1 節#3}
\crefrangeformat{section}{#3#1 節#4~#5#2 節#6}
\crefmultiformat{section}
  {#2#1 節#3}
  {,#2#1 節#3}
  {,#2#1 節#3}
  {,#2#1 節#3}
\crefrangemultiformat{section}
  {#3#1 節#4~#5#2 節#6}
  {,#3#1 節#4~#5#2 節#6}
  {,#3#1 節#4~#5#2 節#6}
  {,#3#1 節#4~#5#2 節#6}

% subsection番号
\crefformat{subsection}{#2#1 小節#3}
\crefrangeformat{subsection}{#3#1 小節#4~#5#2 小節#6}
\crefmultiformat{subsection}
  {#2#1 小節#3}
  {,#2#1 小節#3}
  {,#2#1 小節#3}
  {,#2#1 小節#3}
\crefrangemultiformat{subsection}
  {#3#1 小節#4~#5#2 小節#6}
  {,#3#1 小節#4~#5#2 小節#6}
  {,#3#1 小節#4~#5#2 小節#6}
  {,#3#1 小節#4~#5#2 小節#6}

% subsubsection番号
\crefformat{subsubsection}{#2#1 小小節#3}
\crefrangeformat{subsubsection}{#3#1 小小節#4~#5#2 小小節#6}
\crefmultiformat{subsubsection}
  {#2#1 小小節#3}
  {,#2#1 小小節#3}
  {,#2#1 小小節#3}
  {,#2#1 小小節#3}
\crefrangemultiformat{subsubsection}
  {#3#1 小小節#4~#5#2 小小節#6}
  {,#3#1 小小節#4~#5#2 小小節#6}
  {,#3#1 小小節#4~#5#2 小小節#6}
  {,#3#1 小小節#4~#5#2 小小節#6}

\let\eqref\cref

\usepackage{caption}
\captionsetup{
	format=plain,
	justification=centering,
	labelformat=simple,
	labelsep=quad,
  font=normalsize
}
\captionsetup[figure]{name=Fig.,skip=2pt}
\captionsetup[table]{name=Table}
\captionsetup[lstlisting]{name=Code}

% --- 図表の間隔設定 ---
\setlength{\textfloatsep}{6pt plus 2pt minus 2pt} % ページ上下にある図と本文の間
\setlength{\intextsep}{6pt plus 2pt minus 2pt}    % [h]などで文中に置いた図の前後
\setlength{\floatsep}{6pt plus 2pt minus 2pt}     % 図と図の間
\setlength{\abovecaptionskip}{3pt}                % 図とキャプションの間
\setlength{\belowcaptionskip}{2pt}                % キャプションの下(本文との間)

\captionsetup[subfigure]{
  font=normalsize,
  labelfont=normalsize,
  labelformat=parens
}
\renewcommand{\thesubfigure}{\alph{subfigure}}

%============================================================================================================
% TOC Settings
%============================================================================================================
\usepackage{tocloft}

% =========================================
%  1. Title (TOC, LOF, LOT) Settings
% =========================================
\renewcommand{\cfttoctitlefont}{\LARGE\headingfont}
\renewcommand{\cftloftitlefont}{\LARGE\headingfont}
\renewcommand{\cftlottitlefont}{\LARGE\headingfont}

% =========================================
%  2. Figure List Settings
% =========================================
% Add "Fig." prefix
\renewcommand{\cftfigpresnum}{Fig.~}

% Set width (calculated with standard font size)
\settowidth{\cftfignumwidth}{Fig.~00\quad} 

% Font settings: all standard (\normalfont)
\renewcommand{\cftfigfont}{\normalfont}
\renewcommand{\cftfigpagefont}{\normalfont}


% =========================================
%  3. Table List Settings
% =========================================
\renewcommand{\cfttabpresnum}{Table~}

% Set width (calculated with standard font size)
\settowidth{\cfttabnumwidth}{Table~00\quad}

% Font settings: all standard (\normalfont)
\renewcommand{\cfttabfont}{\normalfont}
\renewcommand{\cfttabpagefont}{\normalfont}

% --- Part (部) ---
\renewcommand{\cftpartfont}{\large\headingfont}
\renewcommand{\cftpartpagefont}{\headingfont}
\addtolength{\cftpartnumwidth}{1.5em}

% --- Section (節・付録) ---
% 標準フォント(\normalfont)に戻す
\renewcommand{\cftsecfont}{\normalfont}
\renewcommand{\cftsecpagefont}{\normalfont}
\addtolength{\cftsecnumwidth}{3em}

% --- Subsection (小節) ---
% 標準フォント(\normalfont)に戻す
\renewcommand{\cftsubsecfont}{\normalfont}
\renewcommand{\cftsubsecpagefont}{\normalfont}
\addtolength{\cftsubsecnumwidth}{2em}

%============================================================================================================
% SI Unit Settings
%============================================================================================================
\usepackage{siunitx}
\sisetup{
  % --- リスト (qtylist) ---
  list-separator       = {,},
  list-final-separator = {,},
  list-pair-separator  = {,},
  list-units           = single,
  % --- 範囲 (qtyrange) ---
  range-phrase         = {〜},
  range-units          = repeat,
  % --- 積 (qtyproduct) ---
  product-units        = repeat
}
\let\originalunit\unit
\renewcommand{\unit}[2][]{\relax[\originalunit[#1]{#2}]}
%============================================================================================================
% Layout Settings
%============================================================================================================
\usepackage{balance}
\usepackage{indentfirst}

%============================================================================================================
% 箇条書き設定
%============================================================================================================
\usepackage{enumitem}
\setlist[itemize]{font=\esectionfont\jgothic\bfseries}
\setlist[enumerate]{font=\esectionfont\jgothic\bfseries}
\setlist[description]{font=\esectionfont\jgothic\bfseries}

\usepackage{../ref/mycolorbox}
%============================================================================================================
% Color Definitions
%============================================================================================================
\definecolor{soil}{HTML}{996600}
\definecolor{water}{HTML}{0033CC}
\definecolor{ice}{HTML}{CCFFFF}
\definecolor{air}{HTML}{93FFAB}
\definecolor{vsoil}{HTML}{FFC7AF}

\definecolor{graySoil}{gray}{0.80}
\definecolor{grayWater}{gray}{0.90}
\definecolor{grayIce}{gray}{0.97}

%============================================================================================================
% Appendix Settings
%============================================================================================================
\usepackage{appendix}
\renewcommand{\appendixname}{付録}
\renewcommand{\appendixtocname}{付録}
\renewcommand{\appendixpagename}{付録}

%============================================================================================================
% Code Snippet Settings
%============================================================================================================
\usepackage{listings}
% 参考先
% https://whitecat-student.hatenablog.com/entry/2016/09/05/180705
\lstset{
  language = Python,
  backgroundcolor={\color[gray]{.90}},
  breaklines = true,
  breakindent = 10pt,
  basicstyle = \ttfamily\footnotesize,
  commentstyle = {\itshape \color[cmyk]{1,0.4,1,0}},
  classoffset = 0,
  keywordstyle = {\bfseries \color[cmyk]{0,1,0,0}},
  stringstyle = {\ttfamily \color[rgb]{0,0,1}},
  frame = TBrl,
  framesep = 5pt,
  numbers = left,
  stepnumber = 1,
  numberstyle = \tiny,
  tabsize = 4,
  captionpos = t,
  showspaces = false, 
  showstringspaces = false,
}

%============================================================================================================
% Algorithm Settings
%============================================================================================================
\usepackage{algorithm}
\usepackage[noend]{algpseudocode}
\algrenewcommand\algorithmicdo{}

\usepackage{lineno}
\newcommand*\patchAmsMathEnvironmentForLineno[1]{
  \expandafter\let\csname old#1\expandafter\endcsname\csname #1\endcsname
  \expandafter\let\csname oldend#1\expandafter\endcsname\csname end#1\endcsname
  \renewenvironment{#1}
     {\linenomath\csname old#1\endcsname}
     {\csname oldend#1\endcsname\endlinenomath}}
\newcommand*\patchBothAmsMathEnvironmentsForLineno[1]{
  \patchAmsMathEnvironmentForLineno{#1}
  \patchAmsMathEnvironmentForLineno{#1*}}
\AtBeginDocument{
\patchBothAmsMathEnvironmentsForLineno{equation}
\patchBothAmsMathEnvironmentsForLineno{align}
\patchBothAmsMathEnvironmentsForLineno{flalign}
\patchBothAmsMathEnvironmentsForLineno{alignat}
\patchBothAmsMathEnvironmentsForLineno{gather}
\patchBothAmsMathEnvironmentsForLineno{multline}
}
\linenumbers

%============================================================================================================
% Header/Footer Settings
%============================================================================================================
\usepackage{fancyhdr}
\renewcommand{\footnoterule}{%
  \kern -3pt
  \hrule width \columnwidth height 0.4pt
  \kern 2.6pt 
}
%============================================================================================================
\begin{document}
%============================================================================================================
% FRONT MATTER
%============================================================================================================
\begin{titlepage}
  \begin{center}
    % --- ロゴ ---
    \includegraphics[width=10cm]{logo/logotype4_c.pdf} % ロゴ画像を挿入
    \vspace{0.5cm}

    {\Large 東京農工大学大学院 農学府}

    \vspace{1.6cm}

    {\Large 令和7年度 修士論文}

    % --- 中央(タイトル) ---
    \vfill
    {\huge\bfseries 地下水流存在下の地盤凍結過程の\\水分・熱の同時解析手法の確立に向けて\par}
    \vspace{1cm}
    {\Large Toward the Establishment of a Method for Simultaneous Analysis of Water and Heat in Ground Freezing Processes in the Presence of Groundwater Flow\par}
    \vfill

    % --- 下部 ---
    \begin{tabular}{lcl}
      {\large 指導教員} & : & {\large 斎藤 広隆 教授}  \\
      \multicolumn{3}{c}{\vspace{2em}}       \\ % 縦のスペース
      {\large 研究室}  & : & {\large 地水環境工学研究室} \\
      \multicolumn{3}{c}{\vspace{2em}}       \\
      {\large 学籍番号} & : & {\large 24517003}  \\
      {\large 氏名}   & : & {\large 菊地 駿}
    \end{tabular}

    \vspace{2cm} % 最下部からのマージン

  \end{center}
\end{titlepage}
\pagenumbering{roman}
\pagestyle{fancy}
\fancyhf{}
\fancyhead[R]{\sffamily\thepage}
\renewcommand{\headrulewidth}{0pt}
\fancypagestyle{plain}{
  \fancyhf{}
  \fancyhead[R]{\sffamily\thepage}
  \renewcommand{\headrulewidth}{0pt}
}
\clearpage
\begin{center}
  {\LARGE\headingfont\bfseries 要旨}
\end{center}

\vspace{1.5\baselineskip}

\normalsize
\setlength{\baselineskip}{22.35pt}

近年,建設技術の高度化および都市空間の深部利用の進展に伴い,大深度高水圧下の帯水層や軟弱地盤における掘削工事,さらには汚染物質の封じ込めといった難条件への対応が必要となっている.
そのため,地盤の強度増加と遮水性を同時に,かつ可逆的に確保可能な地盤改良技術が強く求められている.これらの要求に応え得る有効な工法の一つとして,人工地盤凍結工法が挙げられる.
本工法は,地盤中に一定間隔で設置した凍結管内に,冷媒を循環させることで地盤から熱を奪い,間隙水を凍結させる地盤改良技術である.
土粒子を化学的に固結させる従来工法とは異なり,間隙水の相変化という物理現象を利用する点が特徴である.
これにより,均質で高強度な凍結地盤を形成できること,実質的な完全遮水性,ならびに地盤を原状回復できる可逆性といった利点を有する.
一方で,地下水流れが速い地盤では流れによって凍土の造成が阻害され,その適用可能性には限りがある.
既往の解析的な検討において,閉塞可能な上限流速である限界流速は,凍土壁の規模を示す代表長さによって支配される.
しかし,凍土壁閉塞に影響するのは凍土周辺の局所的な範囲に限られるため,慣例的に凍土壁全長を代表長さとして用いた場合,実際の限界流速は過小評価される.
そのため,実際には造成可能な地下水流速でも凍土が閉塞しないと判定され,凍土壁全長を代表長さとみなして限界流速式を適用し,工法の可否を判定する従来手法には課題が残る.
そこで本研究では開発した凍結・融解を含む飽和二次元熱・水移動数値解析ソルバーを用いた数値解析によって,凍結管配置に基づく代表長さの算出式を構築した.

本研究で開発した数値解析ソルバーは,エネルギー保存則に基づく熱移動支配方程式と間隙水質量保存則に基づく水分支配方程式を連成して構築した.
液状水と氷の相平衡については一般化クラジウス・クラペイロン式を用い,算出された液状水の圧力を水分保持関数に代入することで,ある温度条件下での不凍水分量を計算した.
なお,本解析では解析領域を凍結範囲に対して十分に大きく設定し,境界条件の影響を排除するとともに,相変化に伴う体積変化は無視した.

ここで,解析対象として,凍結管が配置されている空間内のある一定の深さにおける\qtyproduct{30 x 30}{\meter}の断面領域をとり,初期地下水流速を \qtyrange{0.0}{1.2}{\meter.\day^{-1}}の範囲で19通り設定した.
表面温度が\qty{-30}{\degreeCelsius}である凍結管を解析領域の左端から\qty{10}{\meter}の地点に上下対称に2,4,8,12本の管をそれぞれ\qtylist{0.6;0.8;1.0}{m}の一定間隔で設置し,得られた温度分布の経時変化に基づき,凍土壁が閉塞するまでの所要時間を算出した.
まず,それぞれの凍結管間隔・本数に対し,初期流速と凍土が閉塞するまでの時間について双曲線フィッティングを行った.
凍結管本数が多い条件では,初期流速が閉塞成立の限界に近づくにつれて閉塞時間が急増する傾向があったが,双曲線フィッティングは全体的な挙動を概ね良好にとらえた.
特に凍結管が2本の場合,フィッティングされた双曲線の漸近値は理論的な限界流速とよく一致した.
これより,数値解析結果のフィッティングより得られる漸近値は,理論上の閉塞限界と等価であるとみなせる.
したがって,限界流速式で算出される限界流速をこの数値的な漸近値に置き換えることで,経験的パラメータである代表長さを数値計算によって逆算することが可能となる.
このようにして数値的に求めた代表長さと,慣例的に代表長さとして用いられてきた凍土壁全長との比較を行った.
その結果を踏まえ,凍結管間隔,凍結管密度およびそれらの相互作用項を説明変数とする重回帰分析を行い,新たな代表長さの算定式を構築した.これにより,凍結管配置を考慮した代表長さを良好に整理でき,本手法の有効性を示した.
本研究で構築した代表長さの算定式は,凍結管間隔や凍結管密度など,あらかじめ決定可能な幾何学的条件のみを説明変数としている.
このため,従来設計者の判断に委ねられていた代表長さを,客観的かつ定量的に決定することが可能となる.本研究の成果は,解析的設計手法の信頼性および実用性を向上させ,地下水流動を伴う人工地盤凍結工事の設計において,過度に保守的な設計や不要な地盤改良の解消に寄与することが期待される.


%---------------------------------
%	Table of Contents
%---------------------------------
% 目次
\clearpage
\tableofcontents
% 図目次
\clearpage
\listoffigures
\clearpage
\listoftables
\clearpage


%============================================================================================================
% MAIN BODY
%============================================================================================================
\pagenumbering{arabic}
\pagestyle{fancy}
\fancyhead[L]{\leftmark}
\renewcommand{\headrulewidth}{0.4pt}
\fancypagestyle{plain}{
  \fancyhf{}
  \fancyhead[R]{\thepage}
  \renewcommand{\headrulewidth}{0pt}
}

% Introduction
\part{序論}
\label{Part:Introduction}
\clearpage

\section{研究背景}
\label{Sec:Background}

\subsection{人工地盤凍結工法}
\label{Sec:AGF_Introduction}

建設技術の高度化と都市空間の深部利用が進む現代において、土壌物理学や地盤工学が直面する課題は複雑化の一途を辿っている。
特に、大深度における高水圧下の帯水層や軟弱地盤における掘削工事、あるいは汚染物質の封じ込めといった極限的な状況において、地盤の強度増加と完全な遮水性を同時に、かつ可逆的に実現する技術への社会的な要求は高まっている。
こうした要請に応えうる技術として、人工地盤凍結工法(Artificial Ground Freezing Method, AGF)が存在する\cite{Alzoubi-2021, Diego-2022}。

人工地盤凍結工法とは、地盤中に一定間隔で埋設した凍結管に、冷凍機で冷却されたブライン(塩化カルシウム水溶液などの不凍液)や液体窒素等の冷媒を循環させ、管周囲の地盤熱を奪うことで土中の間隙水を凍結させる技術である。この工法が他の地盤改良技術と決定的に異なる点は、土粒子そのものを化学的に固結させるのではなく、間隙水を固体である氷へと相変化させる物理現象を利用する点にある。この物理的相変化を用いることで、以下の工学的利点が存在する。
\begin{itemize}
      \item 均質かつ高強度な改良体の造成: \\
            土の種類や粒度分布に依存せず、含水さえしていれば凍結が可能である。凍土の強度は温度に依存して一義的に定まり、マイナス10度程度でコンクリートの約3分の1、マイナス40度からマイナス162度(LNGタンク周辺)といった極低温下ではコンクリートと同等以上の圧縮強度を発現する\cite{Alzoubi-2021, Diego-2022, Nishimura-2022}。
      \item 完全な遮水性: \\
            間隙が氷で充填されるため、透水係数は実質的にゼロとなり、完全な不透水層(遮水壁)を形成する。
      \item 可逆性と無公害性: \\
            工事終了後に冷却を停止すれば地盤は融解し、元の状態に戻る。薬液による地下水汚染のリスクがないため、環境保全性が求められる現代の建設プロジェクトにおいて再評価されている。
\end{itemize}
一方で、水が氷になる際の体積膨張(約9\%)や、未凍結部からの水分吸引に伴う凍上(Frost Heave)、および融解時の融解沈下(Thaw Settlement)といった負の側面も併せ持つ。これらの挙動をいかに精緻に予測し制御するかが、本工法の設計・施工における核心的課題となる。
\section{数値解析の背景}

凍土工学における数値シミュレーションは、地盤の温度変化、水分移動、および力学的変形の複雑な相互作用を記述する手法として発展してきた。近年では、計算機の演算能力向上と定式化の高度化により、熱・水・力学(Thermal-Hydraulic-Mechanical: THM)を包括的に扱う連成解析や、氷レンズの生成を捉える不連続面モデルが実用化されている\cite{Nishimura-2009}。

凍土の物理挙動を精度良く予測するためには、相変化に伴う熱と物質の輸送、およびそれに起因する地盤の変形を同時に解く連成解析が不可欠である。従来の解析手法では不飽和土力学の枠組みを応用した有効応力概念が用いられてきたが、近年の高度な定式化においては、氷圧、不凍水圧(液圧)、および全応力を独立した状態変数として扱う手法が提案されている\cite{Zhou-2013}。
このアプローチでは、一般化されたクラウジウス・クラペイロンの式に基づき、液相と固相の間の熱力学的平衡条件を考慮することが可能となった。これにより、温度変化が凍結吸引力(サクション)の変化を引き起こし、それが未凍結領域から凍結前線への水分移動を駆動して地盤の膨張をもたらすプロセスが、物理的根拠に基づいて記述されている。

数値解析において、凍土の物理特性を規定する最も重要な構成則は、負温と不凍水含有量の関係を示す土壌凍結特性曲線(SFCC)である\cite{Talamucci-2003}。解析においては、van Genuchtenモデル等の水分特性曲線を温度領域に拡張したモデルが広く用いられ、相変化に伴う潜熱の出入りはエネルギー保存則に基づくエンタルピー法等によって処理される\cite{Bao-2016}。
また、近年の研究では、土粒子骨格と氷の相互作用を考慮した速度依存型の構成則(Creep/Secondary consolidation)が導入されており、人工地盤凍結(AGF)工法における長期間の施工プロセスや、地表面の凍上挙動をより正確に再現する試みがなされている。

\subsection{アイスレンズ形成と不連続面アプローチ}
凍上現象の主要因であるアイスレンズ(分離氷)の形成は、地盤内のマクロな不連続面としての成長プロセスである。これに対し、アイスレンズを物理的な不連続面として直接モデリングする手法が進展している。この手法では、破壊力学的な基準に基づいてアイスレンズの核形成と成長を記述し、複数のレンズが周期的に形成される挙動を再現することが可能となっている。
さらに、相の分布をスカラー場で表現するフェーズフィールド法(Phase-field Method)も、複雑な形状の氷層が合流・分岐するプロセスを解析する新たなアプローチとして注目されている。

\subsection{大規模計算基盤の活用と並列化技術}
広域な永久凍土の変動予測や、都市インフラにおける大規模な人工地盤凍結解析においては、計算負荷の増大が課題となる。スーパーコンピュータ「富岳」に搭載されたA64FXプロセッサや、高速なネットワーク(TofuインターコネクトD)に最適化されたMPIおよびOpenMPによる並列有限要素法コードの開発により、数万ノード規模での高解像度な3次元解析が現実のものとなりつつある。

これらの技術により、地下水流による熱移流効果や地表面の複雑な境界条件を詳細に考慮したシミュレーションが実現している。このような大規模計算技術は、施工リスクの事前評価やインフラ維持管理の高度化において不可欠な基盤となっている。

% Methods
\section{数理解析モデルの基本仮定}
\numberwithin{equation}{subsection}
\subsection{熱力学的前提と記号定義}
\label{Sec:ThermoBasic}

\subsubsection{系の構成と相の定義}
\label{Sec:PhaseDefinition}

本研究で対象とする凍土は,熱・水分移動を同時に取り扱う必要があるため,多相混合体として定式化する.ここでは,土粒子(solid),液相水(liquid),氷(ice),および気相(air)の四相からなる系を考える.
\begin{figure}[tbp]
  \centering
  \begin{tikzpicture}[scale=0.95]
    % 座標定義
    \coordinate (p11) at (5,0); \coordinate (p12) at (5,3);
    \coordinate (p13) at (5,5.25); \coordinate (p14) at (5,7.5);
    \coordinate (p15) at (5,9); \coordinate (p21) at (9,0);
    \coordinate (p22) at (9,3); \coordinate (p23) at (9,5.25);
    \coordinate (p24) at (9,7.5); \coordinate (p25) at (9,9);

    % ▼▼▼ グレー背景と模様の設定 ▼▼▼

    % --- Soil (土粒子): 20%グレー + 右上がり斜線 ---
    \fill[graySoil] (p11) rectangle (p22);
    \fill[pattern={Lines[distance=4pt, line width=0.4pt, angle=45]}] (p11) rectangle (p22);
    % 文字背景を白く抜く (fill=white)
    \node[fill=white, inner sep=2pt] at ($(p11)!0.5!(p22)$) {\Large Soil};

    % --- Water (水): 10%グレー + 水平線 ---
    \fill[grayWater] (p12) rectangle (p23);
    \fill[pattern={Lines[distance=4pt, line width=0.4pt, angle=0]}] (p12) rectangle (p23);
    \node[fill=white, inner sep=2pt] at ($(p12)!0.5!(p23)$) {\Large Water};

    % --- Ice (氷): 3%グレー(ほぼ白) + 左上がり斜線 ---
    \fill[grayIce] (p13) rectangle (p24);
    \fill[pattern={Lines[distance=4pt, line width=0.4pt, angle=-45]}] (p13) rectangle (p24);
    \node[fill=white, inner sep=2pt] at ($(p13)!0.5!(p24)$) {\Large Ice};

    % --- Air (空気): 白 + 模様なし ---
    \node at ($(p14)!0.5!(p25)$) {\Large Air};

    % ▲▲▲ 設定ここまで ▲▲▲

    % 枠線
    \draw[line width = 0.5mm] (p11) rectangle (p25);
    \draw[line width = 0.5mm] (p12) -- (p22);
    \draw[line width = 0.5mm] (p13) -- (p23);
    \draw[line width = 0.5mm] (p14) -- (p24);

    % --- Volume (体積) ---
    \coordinate (v11) at (0,0); \coordinate (v12) at (0,9);
    \coordinate (v21) at (4,0); \coordinate (v22) at (4,9);
    \coordinate (v31) at (1.5,3); \coordinate (v32) at (4,3);
    \coordinate (v41) at (3.5,0); \coordinate (v42) at (3.5,3);
    \coordinate (v43) at (3.5,5.25); \coordinate (v44) at (3.5,7.5);
    \coordinate (v45) at (3.5,9); \coordinate (v51) at (3,5.25);
    \coordinate (v52) at (3,7.5);

    \node at ($(v12)!0.5!(v22)+(0,0.5)$) {\LARGE Volume};
    \draw[line width = 0.5mm] (v11) -- (v21);
    \draw[line width = 0.5mm] (v12) -- (v22);
    \draw[line width = 0.5mm] (v31) -- (v32);
    \draw[line width = 0.5mm] (v51) -- ++(1,0);
    \draw[line width = 0.5mm] (v52) -- ++(1,0);
    \draw[{Latex[length=3mm]}-{Latex[length=3mm]},line width = 0.5mm] ($(v11)+(0.5,0)$) -- ($(v12)+(0.5,0)$) node[midway, left] {\Large $V$};

    \draw[{Latex[length=3mm]}-{Latex[length=3mm]},line width = 0.5mm] ($(v12)+(2,0)$) -- ++(0,-6) node[midway, left] {\Large $\Vvoid$};
    \draw[{Latex[length=3mm]}-{Latex[length=3mm]},line width = 0.5mm] (v41) -- (v42)  node[midway, left] {\Large $\Vs$};
    \draw[{Latex[length=3mm]}-{Latex[length=3mm]},line width = 0.5mm] (v42) -- (v43)  node[midway, left] {\Large $\Vw$};
    \draw[{Latex[length=3mm]}-{Latex[length=3mm]},line width = 0.5mm] (v43) -- (v44)  node[midway, left] {\Large $\Vice$};
    \draw[{Latex[length=3mm]}-{Latex[length=3mm]},line width = 0.5mm] (v44) -- (v45)  node[midway, left] {\Large $\Va$};

    % --- Mass (質量) ---
    \coordinate (m11) at (10,0); \coordinate (m12) at (10,3);
    \coordinate (m13) at (10,5.25); \coordinate (m14) at (10,7.5);
    \coordinate (m15) at (10,9); \coordinate (m21) at (14,9);
    \coordinate (m31) at (13,0);

    \node at ($(m15)!0.5!(m21)+(0,0.5)$) {\LARGE Mass};
    \draw[line width = 0.5mm] (m11) -- ++(4,0);
    \draw[line width = 0.5mm] (m12) -- ++(1,0);
    \draw[line width = 0.5mm] (m13) -- ++(1,0);
    \draw[line width = 0.5mm] (m14) -- ++(1,0);
    \draw[line width = 0.5mm] (m15) -- ++(4,0);

    \draw[{Latex[length=3mm]}-{Latex[length=3mm]},line width = 0.5mm] ($(m11)+(0.5,0)$) -- ++(0,3)  node[midway, right] {\Large $m_\mathrm{s}$};
    \draw[{Latex[length=3mm]}-{Latex[length=3mm]},line width = 0.5mm] ($(m12)+(0.5,0)$) -- ++(0,2.25)  node[midway, right] {\Large $m_\mathrm{w}$};
    \draw[{Latex[length=3mm]}-{Latex[length=3mm]},line width = 0.5mm] ($(m13)+(0.5,0)$) -- ++(0,2.25)  node[midway, right] {\Large $m_\mathrm{ice}$};
    \draw[{Latex[length=3mm]}-{Latex[length=3mm]},line width = 0.5mm] ($(m14)+(0.5,0)$) -- ++(0,1.5)  node[midway, right] {\Large $m_\mathrm{a}=0$};
    \draw[{Latex[length=3mm]}-{Latex[length=3mm]},line width = 0.5mm] (m31) -- ++(0,9)  node[midway, right] {\Large $m$};

  \end{tikzpicture}
  \caption{凍土中における四相地盤構造}\label{fig:4Phase_Soil}
\end{figure}
間隙比$e$ \unit{[-]} および間隙率$\phiv$ \unit{[-]} は次式で定義される.
\begin{subequations}
  \label{Eq:Void_Definition}
  \begin{align}
    \label{Eq:Void_Definition_e}
    e     & = \frac{\Vvoid}{\Vs} \\
    \label{Eq:Void_Definition_phi}
    \phiv & = \frac{\Vvoid}{V}
  \end{align}
\end{subequations}
両者の関係は次式で表される.
\begin{subequations}
  \label{Eq:Void_Porosity}
  \begin{align}
    \label{Eq:Void_Porosity_e}
    \phiv   & = \frac{e}{1+e} \\
    \label{Eq:Void_Porosity_phi}
    1-\phiv & = \frac{1}{1+e}
  \end{align}
\end{subequations}
間隙体積に対する各相の体積比(飽和度)\unit{[-]} を次式で定義する.
\begin{subequations}
  \label{Eq:Saturation_Def}
  \begin{align}
    \label{Eq:Saturation_Water_Def}
    \Sw   & = \frac{\Vw}{\Vvoid}   \\
    \label{Eq:Saturation_Ice_Def}
    \Sice & = \frac{\Vice}{\Vvoid} \\
    \label{Eq:Saturation_Air_Def}
    \Sa   & = \frac{\Va}{\Vvoid}   \\
    \label{Eq:Saturation_Vapor_Def}
    \Sv   & = \frac{\Vv}{\Vvoid}
  \end{align}
\end{subequations}
このとき,次の関係が成り立つ.
\begin{equation}
  \label{Eq:Saturation_Sum}
  \Sw + \Sice + \Sa = 1
\end{equation}
全体体積に対する体積分率$\theta_\alpha$ \unit{[\meter^3.\meter^{-3}]} は次式で定義される.
ただし,$\alpha = n\!:\text{soil},\ w\!:\text{water},\ i\!:\text{ice},\ a\!:\text{air},\ v\!:\text{vapor}$である.
\begin{subequations}
  \label{Eq:Theta_Def}
  \begin{align}
    \label{Eq:Theta_Solid_Def}
    \Qn   & = \frac{\Vs}{V}   \\
    \label{Eq:Theta_Water_Def}
    \Qw   & = \frac{\Vw}{V}   \\
    \label{Eq:Theta_Ice_Def}
    \Qice & = \frac{\Vice}{V} \\
    \label{Eq:Theta_Air_Def}
    \Qa   & = \frac{\Va}{V}   \\
    \label{Eq:Theta_Vapor_Def}
    \Qv   & = \frac{\Vv}{V}
  \end{align}
\end{subequations}
体積保存条件は次式となる.
\begin{equation}
  \label{Eq:VolumeSum}
  \Qn + \Qw + \Qice + \Qa = 1
\end{equation}
さらに,間隙における有効密度$\rho_\mathrm{e}$ \unit{[\kilogram.\meter^{-3}]}を次式で定義する.
\begin{align}
  \label{Eq:Density_effective}
  \rho_\mathrm{e}
   & = \phiv (\rho_\mathrm{w} \Sw + \rho_\mathrm{ice} \Sice + \rho_\mathrm{a} \Sa) \notag    \\
   & = \frac{e (\rho_\mathrm{w} \Sw + \rho_\mathrm{ice} \Sice + \rho_\mathrm{a} \Sa)}{1 + e}
\end{align}
以上により,四相地盤の体積構成および基本変数の定義が確定する.この定義に基づき,次節では各相の状態変数と熱力学関係式を整理する.

\subsubsection{状態変数と熱力学関係式}
\label{Sec:ThermoRelation}

各相(固相,液相,気相)における熱力学状態は,温度$T$,圧力$p$,および比体積$\nu$を基本変数として定義する.本章の表記は\textcite{Tasaki-Thermodynamics}に従う.
内部エネルギー$U$,エントロピー$S$,エンタルピー$H$,Helmholtzの自由エネルギー$F$,Gibbsの自由エネルギー$G$は熱力学ポテンシャルと呼ばれる.ここで,$T$は熱力学的温度,$V$は体積,$p$は圧力,$N$は物質量(モル数または質量)である.
熱力学の基本関係式より,内部エネルギー$U$の全微分は次式で与えられる.
\begin{equation}
  \label{Eq:dU}
  \odif{U} = T \odif{S} - p \odif{V} + \mu \odif{N}
\end{equation}
ここで,$\mu$は化学ポテンシャルである.他の熱力学ポテンシャルは,$U$からルジャンドル変換により,独立変数を$S \to T$または$V \to p$(あるいはその両方)へ変更することで定義される.
まず,Helmholtzの自由エネルギー$F$は次式で定義される.
\begin{equation}
  \label{Eq:Helmholtz}
  F = U - TS
\end{equation}
ただし,Helmholtzの自由エネルギー$F$が$T$について微分可能とする.その全微分は,\eqref{Eq:dU}式および\eqref{Eq:Helmholtz}式を用いて次式で導かれる.
\begin{align}
  \odif{F}
   & = \odif{U} - \odif{(TS)} \notag                                                       \\
   & = \pab{T \odif{S} - p \odif{V} + \mu \odif{N}} - \pab{S \odif{T} + T \odif{S}} \notag \\
  \label{Eq:dF}
   & = -S \odif{T} - p \odif{V} + \mu \odif{N}
\end{align}
また,状態量としての圧力はHelmholtzの自由エネルギー$F$の体積微分として表せるので,
\begin{equation}
  \label{Eq:Pressure_differential}
  \pdv{F}{V} = -p
\end{equation}
同様に物質量$N$についても同様の量である化学ポテンシャル$\mu$を考える.
\begin{equation}
  \label{Eq:Chemical_potential_differential}
  \pdv{F}{N} = \mu
\end{equation}
次に,エンタルピー$H$は次式で定義される.
\begin{equation}
  \label{Eq:Enthalpy_def}
  H = U + pV
\end{equation}
その全微分は次式で与えられる.
\begin{align}
  \odif{H}
   & = \odif{U} + \odif{\pab{pV}} \notag                                                   \\
   & = \pab{T \odif{S} - p \odif{V} + \mu \odif{N}} + \pab{V \odif{p} + p \odif{V}} \notag \\
  \label{Eq:dH}
   & = T \odif{S} + V \odif{p} + \mu \odif{N}
\end{align}
さらに,Gibbsの自由エネルギー $G$ は次式で定義される.
\begin{equation}
  \label{Eq:Gibbs_def}
  G = U - TS + pV
\end{equation}
$G$ は $F$ または $H$ を用いて次のようにも表される.
\begin{equation}
  \label{Eq:ThermoRelation}
  G = F + pV = H - TS
\end{equation}
$G$ の全微分は次式で与えられる.
\begin{align}
  \odif{G}
   & = \odif{F} + \odif{(pV)} \notag                                                        \\
   & = \pab{-S \odif{T} - p \odif{V} + \mu \odif{N}} + \pab{V \odif{p} + p \odif{V}} \notag \\
  \label{Eq:dG}
   & = -S \odif{T} + V \odif{p} + \mu \odif{N}
\end{align}
物質量 $N$ が一定($\odif{N}=0$)の場合,\eqref{Eq:dG} 式は次式のように書ける.
\begin{equation}
  \label{Eq:dG_constN}
  \eval{\odif{G}}{{N=\text{const}}} = -S \odif{T} + V \odif{p}
\end{equation}

各相の物性は温度と圧力の関数として次式で表される.
\begin{subequations}
  \begin{align}
    \label{Eq:Density_TP}
    \rho_\alpha           & = \rho_\alpha\pab{T, p}           \\
    \label{Eq:Specific_Heat_TP}
    c_{\mathrm{p},\alpha} & = c_{\mathrm{p},\alpha}\pab{T, p}
  \end{align}
\end{subequations}
特に後章で導入する体積熱容量$C$や熱伝導率$\lambda$の導出において,これらの関数形が用いられる.

\subsection{液相・固相間の相平衡}
\label{sec:LiquidSolidEq}

単一系における液状水と氷の平衡条件は,各相のGibbs自由エネルギーが等しいことによって定義される.
ここで,Fig.~\ref{Fig:GCC_diagram}に示すように,氷相に水蒸気のみに一様に働く圧力を$P_\mathrm{ice}$,
液相に水蒸気のみに一様に働く圧力を$\Pw$とすれば,
両者は共通の大気圧$P_\mathrm{a}$を介して次式のように関係づけられる.
\begin{subequations}
  \begin{align}
    \label{Eq:Pressure_1}
    P_2 & = P_\mathrm{a} + \Pw   \\
    \label{Eq:Pressure_2}
    P_1 & = P_\mathrm{a} + P_\mathrm{ice}
  \end{align}
\end{subequations}
ここで,$P_1$,$P_2$はそれぞれ氷相および液相に作用する全圧である.
$T_\mathrm{f}$を三相が平衡状態にあるときの温度(凝固点)とする.

\begin{figure}[tbp]
  \centering
  \begin{tikzpicture}[scale=1.5] % スケール調整はお好みで
    \usetikzlibrary{calc, arrows.meta} % arrows.meta を追加 (calc は既存)

    % 共通の圧力矢印スタイルを定義
    \tikzset{pressurearrow/.style={
    -{LaTeX[length=4.0mm, width=3.6mm]}, % 矢印の先端形状とサイズ
    line width=1.2mm                    % 線の太さ
    }
    }

    % 座標やサイズを調整しやすくするための変数定義(任意)
    \def\waterHeight{2}
    \def\iceHeight{3}
    \def\blockY{0.2} % ブロックの下端のY座標
    \def\containerTop{5}
    \def\containerBottom{0}
    \def\containerLeft{0}
    \def\containerRight{7}
    \def\pressureArrowStartY{6} % 圧力矢印の開始Y座標
    \def\pressureLabelY{5.5}    % 圧力ラベルの共通Y座標 (お好みで調整)

    % 外側の容器の線
    \draw (\containerLeft,\containerBottom) rectangle (\containerRight,\containerTop);

    % 水の部分 (左下)
    \coordinate (waterSW) at (0.2,\blockY); % 左下隅の座標
    \coordinate (waterNE) at (3.5,\waterHeight); % 右上隅の座標
    \fill[pattern=north west lines, pattern color=blue!50] (waterSW) rectangle (waterNE);
    \draw (waterSW) rectangle (waterNE);
    \node[fill=white, text=black, font=\bfseries, inner sep=2pt, rounded corners=1pt] at ($(waterSW)!0.5!(waterNE)$) {Water};

    % 氷の部分 (右下)
    \coordinate (iceSW) at (3.5,\blockY); % 左下隅の座標
    \coordinate (iceNE) at (6.8,\iceHeight); % 右上隅の座標
    \fill[pattern=north east lines, pattern color=cyan!30] (iceSW) rectangle (iceNE);
    \draw (iceSW) rectangle (iceNE);
    \node[fill=white, text=black, font=\bfseries, inner sep=2pt, rounded corners=1pt] at ($(iceSW)!0.5!(iceNE)$) {Ice};

    % 蒸気の部分 (上部全体)
    \node at (3.5,3.5) {Air};
    % 蒸気と水の境界線
    \draw (0.2,\waterHeight) -- (3.5,\waterHeight);
    % 蒸気と氷の境界線
    \draw (3.5,\iceHeight) -- (6.8,\iceHeight);
    % 水と氷の間の境界線
    \draw (3.5,\blockY) -- (3.5, \waterHeight); % 水ブロックの高さまで

    % 容器の上部の線 (Vaporと接する部分)
    \draw (0.2,\waterHeight) -- (0.2,\containerTop-0.2);
    \draw (6.8,\iceHeight) -- (6.8,\containerTop-0.2);
    \draw (0.2,\containerTop-0.2) -- (6.8,\containerTop-0.2);

    % --- 圧力矢印とラベル ---
    % 圧力 P (全体にかかる)
    \draw[pressurearrow] (3.5,\pressureArrowStartY) -- (3.5,\containerTop-1.2);
    \node[right, xshift=3mm] at (3.5, \pressureLabelY) {$P_\mathrm{a}$};

    % 水面にかかる圧力 P_w
    \draw[pressurearrow] (1.75,\pressureArrowStartY) -- (1.75,\waterHeight+0.2);
    \node[left, xshift=-3mm] at (1.75, \pressureLabelY) {$\Pw$};
    \node[below] at (1.75,\containerBottom-0.3) {$P_2 = P_\mathrm{a} + \Pw$};

    % 氷面にかかる圧力 P_f
    \draw[pressurearrow] (5.25,\pressureArrowStartY) -- (5.25,\iceHeight+0.2);
    \node[right, xshift=3mm] at (5.25, \pressureLabelY) {$P_\mathrm{ice}$};
    \node[below] at (5.25,\containerBottom-0.3) {$P_1 = P_\mathrm{a} + P_\mathrm{ice}$};
  \end{tikzpicture}
  \caption{固相,液相,気相の三相それぞれに異なる圧力が加わっている状況における,各相間の平衡状態を表す概念図}
  \label{Fig:GCC_diagram}
\end{figure}
\noindent
液相と氷相が平衡状態にある場合,それぞれの比Gibbs自由エネルギーは等しくなる.
\begin{equation}
  \label{Eq:Gibbs_equal}
  G_1 = G_2
\end{equation}
ここで,$G_1$,$G_2$はそれぞれ氷相および液相のGibbs自由エネルギーである.
系に変化が生じた際には,これらの自由エネルギーの微分が等しくなければならない.
\begin{equation}
  \label{Eq:Gibbs_Equilibrium}
  \odif{G_1} = \odif{G_2}
\end{equation}
\eqref{Eq:Gibbs_Equilibrium}に対し,Gibbs自由エネルギーの全微分\eqref{Eq:dG_constN}を用いると,
\begin{equation}
  \label{Eq:Gibbs_Equilibrium_2}
  -S_1\odif{T} + \nu_1\odif{P_1} = -S_2\odif{T} + \nu_2\odif{P_2}
\end{equation}
が得られる.ここで,$\nu_1$,$\nu_2$はそれぞれ氷相,液相の比体積 \unit{[\meter^3.\kilogram^{-1}]}であり,
$\odif{P_1}$と$\odif{P_2}$はそれぞれ氷および水に加わる全圧力の変化である.
一般に$\odif{P_1}\neq\odif{P_2}$である.
\eqref{Eq:Gibbs_Equilibrium_2}に\eqref{Eq:Pressure_1},\eqref{Eq:Pressure_2}を代入して整理すると,
\begin{equation}
  \nu_1\pab{\odv{P_\mathrm{a}}{T} + \odv{P_\mathrm{ice}}{T}} - S_1
  = \nu_2\pab{\odv{P_\mathrm{a}}{T} + \odv{\Pw}{T}} - S_2
\end{equation}
となり,さらに次式を得る.
\begin{equation}
  \label{Eq:Gibbs_Equilibrium_3}
  \nu_2\odv{\Pw}{T} - \nu_1\odv{P_\mathrm{ice}}{T}
  + \odv{P_\mathrm{a}}{T}\pab{\nu_2 - \nu_1} = S_2 - S_1
\end{equation}
一般に液相から固相への相変化に伴うエンタルピー変化は,
\begin{equation}
  \label{Eq:Enthalpy_Change}
  H_\mathrm{fus}\pab{T; N} = T \Bab{S \pab{T; V_\mathrm{liquid}\pab{T; N}, N} - S\pab{T;V_\mathrm{solid}\pab{T; N}, N}}
\end{equation}
で与えられる.ここで,$H_\mathrm{fus}$は液相から固相への凝固潜熱である.
液状水から氷への相変化を考えれば,
\begin{equation}
  \label{Eq:Enthalpy_Change_2}
  S_2 - S_1 = \frac{L_\mathrm{f}}{T}
\end{equation}
となる.ただし,$L_\mathrm{f}$は水の凍結潜熱 \unit{[J.kg^{-1}]}である.
\eqref{Eq:Gibbs_Equilibrium_3}に\eqref{Eq:Enthalpy_Change_2}を代入し,比体積を密度に置き換えると次式を得る.

\begin{NoteBox}{Generalized Clausius-Clapeyron式}{GCCE}
  \eqref{Eq:GCC_main}は,凍土における相平衡の基礎となる一般化Clausius-Clapeyron式(Generalized Clausius-Clapeyron equation, GCC)である.
  \begin{equation}
    \label{Eq:GCC_main}
    \frac{1}{\rho_\mathrm{w}}\odv{\Pw}{T} - \frac{1}{\rho_\mathrm{ice}}\odv{P_\mathrm{ice}}{T} + \odv{P_\mathrm{a}}{T}\pab{\frac{1}{\rho_\mathrm{w}} - \frac{1}{\rho_\mathrm{ice}}} = \frac{L_\mathrm{f}}{T}
  \end{equation}
  ここで,$\rho_\mathrm{w}$,$\rho_\mathrm{ice}$はそれぞれ水と氷の密度 \unit{[\kilogram.\meter^{-3}]}である.
\end{NoteBox}
\noindent
特に本研究内では,以下の2つの場合に着目する.ただし,圧力の単位は \unit{\pascal}である.
\begin{enumerate}[label=Case \arabic*:, leftmargin=*, labelwidth=3.1em, labelsep=0.5em]
  \item 氷の圧力を一定とし,水の圧力のみが変化する場合.\\[4mm]
        氷の圧力微分$\displaystyle\odv{P_\mathrm{ice}}{T}=0$とすれば,\eqref{Eq:GCC_main}は
        \begin{equation}
          \label{Eq:GCC_ice_constant}
          \odv{\Pw}{T}
          = \frac{\rho_\mathrm{w} L_\mathrm{f}}{ T}
          + \odv{P_\mathrm{a}}{T}\pab{\frac{\rho_\mathrm{w}}{\rho_\mathrm{ice}} - 1}
        \end{equation}
        右辺第2項は通常非常に小さいため無視できるため,次式が得られる.
        \begin{equation}
          \label{Eq:GCC_ice_constant_approx}
          \odv{\Pw}{T} \approx \frac{L_\mathrm{f}\rho_\mathrm{w}}{T}
        \end{equation}
        大気圧が標準状態$P_\mathrm{a}=\qty{1013.25}{\hecto\pascal}$のとき,凝固点$T_\mathrm{f}^\ast=\qty{273.15}{\kelvin}$で凍結するとすれば,
        \eqref{Eq:GCC_ice_constant_approx}の積分形は次式で与えられる.
        \begin{align}
          \int_{0}^{\Pw}\odif{\Pw}
           & = \int_{T_\mathrm{f}^\ast}^{T^\ast}\frac{L_\mathrm{f}\rho_\mathrm{w}}{T}\odif{T} \notag \\
          \label{Eq:GCC_ice_constant_approx_integrated}
          \Pw
           & = L_\mathrm{f}\rho_\mathrm{w}\ln{\frac{T^\ast}{T_\mathrm{f}^\ast}}
        \end{align}

  \item 大気圧$P_\mathrm{a}$が温度に依存しない場合.\\
        このとき,\eqref{Eq:GCC_main}左辺第3項は消去されるため,
        \begin{equation}
          \label{Eq:GCC_vapor_constant}
          \frac{1}{\rho_\mathrm{w}}\odv{\Pw}{T} - \frac{1}{\rho_\mathrm{ice}}\odv{P_\mathrm{ice}}{T} = \frac{L_\mathrm{f}}{T}
        \end{equation}
        よって,氷の圧力変化は次式で表される.
        \begin{equation}
          \label{Eq:GCC_vapor_constant_2}
          \odv{P_\mathrm{ice}}{T}
          = \frac{\rho_\mathrm{ice}}{\rho_\mathrm{w}}\odv{\Pw}{T}
          - \frac{L_\mathrm{f}\rho_\mathrm{ice}}{T}
        \end{equation}
        Case 1 と同様に積分すれば,
        \begin{align}
          \int_{0}^{P_\mathrm{ice}}\odif{P_\mathrm{ice}}
           & = \int_{0}^{\Pw}\odif{\Pw}
          - \int_{T_\mathrm{f}^\ast}^{T^\ast}\frac{L_\mathrm{f}\rho_\mathrm{ice}}{T}\odif{T} \notag \\
          \label{Eq:GCC_vapor_constant_2_integrated}
          P_\mathrm{ice}
           & = \frac{\rho_\mathrm{ice}}{\rho_\mathrm{w}}\Pw
          - L_\mathrm{f}\rho_\mathrm{ice}\ln{\frac{T^\ast}{T_\mathrm{f}^\ast}}
        \end{align}
\end{enumerate}

\subsection{GCC式を用いた不凍水量の表現}
\label{Sec:GCC_UnfrozenWater}

飽和土の凍結・融解過程における液状水・氷分布と,脱水・給水過程における液状水・空気分布が等しいと仮定する\parencite{Williams-1964}.
このとき液状水と空気の吸引力 $P_\mathrm{aw}=P_\mathrm{a}-\Pw$ と不凍水と氷の cryogenic suction $P_\mathrm{iw}=P_\mathrm{ice}-\Pw$ を等しいとすれば,水分保持関数(Water Retention Function, WRF)を用いて $T^\ast$ の関数として不凍水量を推定できる\parencite{Black-1989}.
たとえば,WRF として van Genuchten (vG) 式\parencite{van-Genuchten-1980}を選べば,凍土存在下での液状水飽和度 $\Sw$ は次式で与えられる.
\begin{equation}
  \label{Eq:van_Genuchten_cryogenic}
  \Sw = \pab{1 + \abs{\alpha P_\mathrm{iw}}^n}^m
\end{equation}
ここで $\alpha, n, m$ は vG パラメータである.
$P_\mathrm{iw}$ は \eqref{Eq:GCC_vapor_constant_2_integrated} 式により
\begin{equation}
  \label{Eq:P_iw}
  P_\mathrm{iw} = \pab{\frac{\rho_\mathrm{ice}}{\rho_\mathrm{w}} - 1} \Pw - L_\mathrm{f} \rho_\mathrm{ice} \ln \frac{T^\ast}{T^\ast_\mathrm{f}}
\end{equation}
\begin{figure}[hbp]
  \centering
  \includegraphics[width=.98\linewidth,pagebox=cropbox,clip]{schematic-gcc-withEXP.pdf}
  \caption{土中の液状水の動態に関する概念図.(a) 脱水過程,(b) 凍結過程.}\label{Fig:GCC}
\end{figure}
\begin{figure}[hbp]
  \includegraphics[width=.98\linewidth,pagebox=cropbox,clip]{GCC_vs_T.pdf}
  \caption{異なる条件下でのCryogenic Suction,\eqref{Eq:P_iw}との関係}\label{Fig:Piw}
\end{figure}
と表される.
これを \eqref{Eq:van_Genuchten_cryogenic} に代入すれば
\begin{equation}
  \label{Eq:van_Genuchten_cryogenic_2}
  \Sw = \Bab{1+\abs{\alpha \bab{\pab{\frac{\rho_\mathrm{ice}}{\rho_\mathrm{w}}-1}\Pw - L_\mathrm{f} \rho_\mathrm{ice} \ln{\frac{T^{{\ast}}}{T^{{\ast}}_\mathrm{f}}}}}^n}^{m}
\end{equation}
\eqref{Eq:van_Genuchten_cryogenic_2}を用いることで,不凍水量を求めることで間接的に含氷量を求めることができる.他のWRFを用いることも可能であるが,その詳細については別節で述べる.
定式化に伴い飽和度$\Sw$の微分が必要になるため,GCC式$\mathcal{F}_\mathrm{GCC}$と水分保持関数$\mathcal{F}_\mathrm{WRF}$を微分形に変形する.
\begin{align}
  \Sw & = \mathcal{F}_\mathrm{WRF}\pab{P_\mathrm{ice}-\mathcal{F}_\mathrm{GCC}\pab{T^{\ast}, \Pw, P_\mathrm{ice}}}\notag \\
  \label{Eq:Sw_form}
      & = \mathcal{F}_\mathrm{WRF}\pab{P_\mathrm{iw}}
\end{align}
GCC式の全微分は,温度微分と圧力微分の和として表される.
\begin{equation}
  \label{Eq:GCC_differential}
  \odif{\Sw} = \pdv{\Sw}{\Pw} \odif{\Pw} +\pdv{\Sw}{P_\mathrm{ice}} \odif{P_\mathrm{ice}} + \pdv{\Sw}{T^{\ast}} \odif{T^{\ast}}
\end{equation}
\eqref{Eq:Sw_form}を考慮すれば,\eqref{Eq:GCC_differential}は次のように表される.
\begin{align}
  \label{Eq:GCC_differential_form}
  \odif{\Sw} & = \pdv{\mathcal{F}_\mathrm{WRF}}{P_\mathrm{iw}}\pdv{P_\mathrm{iw}}{\Pw} \odif{\Pw} + \pdv{\mathcal{F}_\mathrm{WRF}}{P_\mathrm{iw}}\pdv{P_\mathrm{iw}}{P_\mathrm{ice}} \odif{P_\mathrm{ice}} \notag                  \\
             & + \pdv{\mathcal{F}_\mathrm{WRF}}{P_\mathrm{iw}}\pdv{P_\mathrm{iw}}{T^{\ast}} \odif{T^{\ast}}\notag                                                                                                                                    \\
             & = \pdv{\mathcal{F}_\mathrm{WRF}}{P_\mathrm{iw}}\pab{\frac{\rho_\mathrm{ice}}{\rho_\mathrm{w}}-1} \odif{\Pw} - \pdv{\mathcal{F}_\mathrm{WRF}}{P_\mathrm{iw}}\frac{L_\mathrm{f} \rho_\mathrm{ice}}{T^{{\ast}}} \odif{T^{\ast}}
\end{align}

これにより,GCC 式とWRFを組み合わせて不凍水量を評価でき,後続章で輸送式に利用可能である.

\clearpage
\subsection{液相・気相間の相平衡}
\label{Sec:LiquidVaporEquilibrium}

\indent
本節では熱力学的な平衡論に基づいて土壌中の液・気相間の相平衡をマトリックポテンシャルを用いて定式化することを目的とする.
土壌表面などある基準面を考えれば,その上部に存在する水は毛管力によって引き上げられ,基準面よりも高い位置にある水は負圧状態になる.
ここで,水のポテンシャルを基準面からの水柱の高さで表したものがマトリックポテンシャル $h$ \unit{\meter} である.
このとき,あるマトリックポテンシャル $h$ をもつ液相の水の比自由エネルギー $\Delta f_\mathrm{w}$ \unit{\joule.\kilogram^{-1}} は,次のようになる.
\begin{equation}
  \label{Eq:LiquidFreeEnergy}
  \Delta f_\mathrm{w} = h g
\end{equation}
ここで,$g$は重力加速度 \unit{\meter.\second^{-2}} である.
一方,同じ高さにおける水蒸気の比自由エネルギー $\Delta f_\mathrm{v}$ \unit{\joule.\kilogram^{-1}} は,その場所の蒸気圧$P_\mathrm{v}$ \unit{\pascal}と基準面$h=0$での飽和蒸気圧$P_\mathrm{v}^\mathrm{sat}$ \unit{\pascal} との比で決定される.
\begin{equation}
  \label{Eq:VaporFreeEnergy}
  \Delta f_\mathrm{v} = R_\mathrm{v} T^\ast \ln{\frac{P_\mathrm{v}}{P_\mathrm{v}^\mathrm{sat}}}
\end{equation}
ここで,$R_\mathrm{v}$は水蒸気の気体定数(\qty{461.5}{\joule.\kilogram^{-1}.\kelvin^{-1}})である.水蒸気の気体定数は,一般気体定数$R=\qty{8.314}{\joule.\mole^{-1}.\kelvin^{-1}}$を水のモル質量$M_\mathrm{w}=\qty{18.015}{\gram.\mole^{-1}}$で割ったものである.すなわち,$R_\mathrm{v} = R / M_\mathrm{w}$である.
考えている土壌の系が熱力学的平衡状態にあるとき,液相と気相の単位質量あたりのHelmholtz自由エネルギーは等しい.すなわち,$\Delta f_\mathrm{w} = \Delta f_\mathrm{v}$が成り立つ.これを用いれば,\eqref{Eq:LiquidFreeEnergy}と\eqref{Eq:VaporFreeEnergy}よりKelvin方程式が得られる.
\begin{equation}
  \label{Eq:LiquidVaporEquilibrium}
  h g =\frac{R T^\ast}{M_\mathrm{w}} \ln{\frac{P_\mathrm{v}}{P_\mathrm{v}^\mathrm{sat}}}
\end{equation}
ここで,相対湿度$H_\mathrm{r}$ \unit{-}は,空気中の水蒸気分圧と飽和水蒸気圧の比で表すことができる.
\begin{equation}
  \label{Eq:RelativeHumidity}
  H_\mathrm{r} = \frac{P_\mathrm{v}}{P_\mathrm{v}^\mathrm{sat}}
\end{equation}
よって,相対湿度は\eqref{Eq:LiquidVaporEquilibrium}より次のように表される.
\begin{align}
  \label{Eq:RelativeHumidityFinal}
  H_\mathrm{r} & = \exp{\pab{\frac{h M_\mathrm{w} g}{R T^\ast}}}
\end{align}
\begin{figure}[tbp]
  \centering
  \includegraphics[width=\linewidth,pagebox=cropbox,clip]{2-1/Hr_soil.pdf}
  \caption{相対湿度$H_\mathrm{r}$とマトリックポテンシャルの関係}\label{Fig:Hr_soil}
\end{figure}
ここで,水蒸気密度$\rho_\mathrm{v}$は,飽和水蒸気密度$\rho_\mathrm{v}^\mathrm{sat}$ \unit{\kilogram.\meter^{-3}} と相対湿度$H_\mathrm{r}$の積で表される.
\begin{align}
  \label{Eq:VaporDensity}
  \rho_\mathrm{v} & = \rho_\mathrm{v}^\mathrm{sat} H_\mathrm{r} = \rho_\mathrm{v}^\mathrm{sat} \exp{\pab{\frac{h M_\mathrm{w} g}{R T^\ast}}}
\end{align}
また,水蒸気を液水換算したときの体積水蒸気量 $\Qv$ \unit{-} は,水蒸気密度 $\rho_\mathrm{v}$ と気相率 $\Qa$ を用いて次のように定義される.
\begin{equation}
  \label{Eq:VolumetricVaporContent}
  \Qv = \frac{\rho_\mathrm{v} \Qa}{\rho_\mathrm{w}} = \rho_\mathrm{v}^\mathrm{sat} H_\mathrm{r} \frac{\Qa}{\rho_\mathrm{w}}
\end{equation}
$\rho_\mathrm{v}^\mathrm{sat}$は$T$の関数,$H_\mathrm{r}$は$h$と$T$の関数であり,水分保持関数によって$h$は$\Qw$について一価関数になることを考慮すれば,$\rho_\mathrm{v}$の空間方向の微分は,積の微分法則より次のようになる.
\begin{align}
  \label{Eq:VaporDensityGradient}
  \nabla \rho_\mathrm{v} & = H_\mathrm{r} \nabla \rho_\mathrm{v}^\mathrm{sat} + \rho_\mathrm{v}^\mathrm{sat} \nabla H_\mathrm{r} \notag                                                           \\
                         & = H_\mathrm{r} \odv{\rho_\mathrm{v}^\mathrm{sat}}{T} \nabla T + \rho_\mathrm{v}^\mathrm{sat} \pab{\pdv{H_\mathrm{r}}{T} \nabla T + \pdv{H_\mathrm{r}}{\Qw} \nabla \Qw}
\end{align}
\eqref{Eq:VaporDensityGradient}は $\nabla \rho_\mathrm{v}$ を $\nabla T$ と $\nabla \Qw$ で厳密に表現している.
ここで,この式をより実用的な形に単純化するため,相対湿度 $H_\mathrm{r}$ が $h$ と $T$ に対してどのように応答するか,その特性を次に考察する.
\eqref{Eq:RelativeHumidityFinal}より明らかなように,$H_\mathrm{r}$は $h$ の指数関数となっている.
$h$ は不飽和帯において負の値をとるため,土壌が乾燥するにしたがって $h$ は負の方向により大きな値をとる.
ここで例えば温度を$\qty{298.15}{\kelvin}=\qty{25.0}{\degreeCelsius}$と固定すれば,\eqref{Eq:RelativeHumidityFinal}の指数部の$h$以外$\pab{M_\mathrm{w} g / \pab{R T^\ast}}$を定数として計算でき,その定数部はおよそ $\qty{7.13e-5}{\meter^{-1}}$ となる.
$h>\qty{-1e3}{\meter}$であれば,指数部の絶対値が比較的小さいため,$H_\mathrm{r}$は$1$に非常に近い値をとる.
しかし,$h$ が $\qty{-1e3}{\meter}$ より小さくなる(負の方向に大きくなる)と,指数部の$h$の影響が大きくなり,$H_\mathrm{r}$は減少し始める.$h<-\qty{e6}{\meter}$ になると$h$が支配的になり,$H_\mathrm{r}$は$0$に漸近する.この傾向は \cref{Fig:Hr_soil} と一致する.
一方,絶対温度 $T^\ast$は\eqref{Eq:RelativeHumidityFinal}の指数の分母にある.$h$ は負であるため,温度 $T^\ast$ が上昇すると,$H_\mathrm{r}$は大きくなる.
しかし,$T^\ast$の変動幅が $h$ の変動幅に比べて小さいため,\cref{Fig:Hr_soil_contour}に示す通り,$H_\mathrm{r}$の$T^\ast$に対する感度は $h$ に対する感度と比べて小さい.
よって,$H_\mathrm{r}$ の値は,主にマトリックポテンシャル $h$および,それと水分保持関数を介して関連する液状水量 $\Qw$によって支配的に決定されると言える.
$H_\mathrm{r}$ の特性についての考察から,$H_\mathrm{r}$ の温度 $T^\ast$ に対する依存性は,マトリックポテンシャル $h$ に対する依存性と比較して小さいことが示された.
この知見に基づき,\eqref{Eq:VaporDensityGradient}を単純化するため,$H_\mathrm{r}$ の温度依存性は無視できると仮定する.
土壌が極めて乾燥している状態(例えば $h < \qty{-2e3}{\meter}$)や塩分濃度が高い状態を除けば,$H_\mathrm{r} \approx 1$ の近似を用いることができる.
前者の条件下では温度依存性を無視しうるため,\cref{Fig:Hr_soil_validation} に示すように,温度 $\qty{15}{\degreeCelsius}$ における $H_\mathrm{r}$ と $h$ の関係を代表例として用いることができる.
このとき,基準マトリックポテンシャルを $\psi_\mathrm{m,ref} = \qty{-2e3}{\meter}$ のように適切に設定すれば,$h > \psi_\mathrm{m,ref}$ の範囲において $H_\mathrm{r} \approx 1$ と近似できることが確認される.
この近似の成立には数パーセント程度の誤差を許容する必要があるが,例えば $H_\mathrm{r} = 0.99$ に相当する $h$ はおおよそ $\qty{-1383}{\meter}$ であり,実用上は十分な精度を保つ.

\begin{figure}[tbp]
  \centering
  \includegraphics[width=.8\linewidth,pagebox=cropbox,clip]{2-1/Hr_contour_plot.pdf}
  \caption{温度およびマトリックポテンシャルに対する相対湿度の応答曲面}\label{Fig:Hr_soil_contour}
\end{figure}
\begin{figure}[tbp]
  \centering
  \includegraphics[width=.8\linewidth,pagebox=cropbox,clip]{2-1/Hr_soil_validation.pdf}
  \caption{温度$\qty{15}{\degreeCelsius}$における相対湿度とマトリックポテンシャルの関係および近似のための基準マトリックポテンシャル}\label{Fig:Hr_soil_validation}
\end{figure}

\FloatBarrier
\subsection{構成方程式}
\label{Sec:ConstitutiveRelations}
本節では,熱輸送および水分移動の計算に必要となる構成方程式を定義する.

\subsubsection{熱物性値の定義}
\label{Sec:ThermalProperties}

土壌全体の体積熱容量 $C_\mathrm{p}$ \unit{[\joule.\meter^{-3}.\kelvin^{-1}]} は,各相体積熱容量の体積率による加重平均として次のように定義される\parencite{Jury-2004}.
\begin{equation}
  \label{Eq:Volumetric_Heat_Capacity_Total}
  C_\mathrm{p} \coloneq C_\mathrm{s}\pab{1-\phiv} + C_\mathrm{w}\Qw + C_\mathrm{ice}\Qice + C_\mathrm{v}\Qv + C_\mathrm{o}\Qo
\end{equation}
ここで,土粒子,間隙水,間隙氷,間隙空気,有機物の各相の体積熱容量 $C_\mathrm{s}$, $C_\mathrm{w}$, $C_\mathrm{ice}$, $C_\mathrm{v}$, $C_\mathrm{o}$ \unit{[\joule.\meter^{-3}.\kelvin^{-1}]} を,それぞれの比熱と密度の積として次のように定義する.
\begin{subequations}
  \label{Eq:Volumetric_Heat_Capacity}
  \begin{align}
    C_\mathrm{s}   & \coloneq \rho_\mathrm{s} c_\mathrm{s}     \\
    C_\mathrm{w}   & \coloneq \rho_\mathrm{w} c_\mathrm{w}     \\
    C_\mathrm{ice} & \coloneq \rho_\mathrm{ice} c_\mathrm{ice} \\
    C_\mathrm{v}   & \coloneq \rho_\mathrm{v} c_\mathrm{v}     \\
    C_\mathrm{o}   & \coloneq \rho_\mathrm{o} c_\mathrm{o}
  \end{align}
\end{subequations}
ここで,$c_\alpha$ は比熱 \unit{[\joule.\kilogram^{-1}.\kelvin^{-1}]},$\rho_\alpha$ は密度 \unit{[\kilogram.\meter^{-3}]} を表す.

さらに,土壌熱伝導率 $\lambda$ \unit{[\watt.\meter{-1}.\kelvin^{-1}]} は,特に土壌飽和条件下では各相の熱伝導率の幾何平均としてあらわすことができる\parencite{Cote-2005}.
\begin{equation}
  \label{Eq:Thermal_Conductivity_Saturated}
  \lambda_0 \coloneq \lambda_\mathrm{s}^{1-\phiv} \lambda_\mathrm{w}^{\Qw} \lambda_\mathrm{ice}^{\Qice}
\end{equation}
ここで,$\lambda_\mathrm{s}$,$\lambda_\mathrm{w}$,$\lambda_\mathrm{ice}$ はそれぞれ土粒子,間隙水,間隙氷の熱伝導率\unit{[\watt.\meter^{-1}.\kelvin^{-1}]}を表す.
一方,不飽和土壌における熱伝導率は,非線形性が強く,様々な経験式が提案されている.本研究では,\textcite{Campbell-1985}を凍土に拡張した\textcite{Hansson-2004}の式を採用する.
\begin{equation}
  \label{Eq:Thermal_Conductivity_Unsaturated}
  \lambda_0 = C_1 + C_2 \pab{\Qw + F \Qice} - \pab{C_1 - C_4}\exp\Bab{-\bab{C_3 \pab{\Qw + F \Qice}}^{C_5}}
\end{equation}
ここで,$C_1$,$C_2$,$C_3$,$C_4$,$C_5$ は土壌の種類に依存する経験定数であり,通常$C_5=4$とする.さらに$F$ は次式で表される.
\begin{equation}
  \label{Eq:Factor_Fice}
  F = 1 + F_1 \Qice^{F_2}
\end{equation}
ここで,$F_1$,$F_2$ は経験的パラメータであり,それぞれ$F_1=13.05$,$F_2=1.06$とすることがある\parencite{Watanabe-2007}.
熱伝導率の等方性を仮定しない場合,\eqref{Eq:Thermal_Conductivity_Saturated}または\eqref{Eq:Thermal_Conductivity_Unsaturated}を用いて見かけの熱伝導率テンソル$\lambda_{ij}$は次式で与えられる.
\begin{equation}
  \label{eq:apparent_thermal_conductivity}
  \lambda_{ij}\pab{\Qw, \Qice} = \lambda_\mathrm{T} C_\mathrm{w} \norm{\vect{q}}_2 \delta_{ij} + \pab{\lambda_\mathrm{L}-\lambda_\mathrm{T}} C_\mathrm{w}\frac{q_j q_i}{\norm{\vect{q}}_2}+\lambda_0\pab{\Qw, \Qice}\delta_{ij}
\end{equation}
ここで,$\lambda_\mathrm{L}$,$\lambda_\mathrm{T}$ はそれぞれ縦方向(longitudinal)および横方向(transverse)の熱分散長 \unit{[\meter]} を表す.$\norm{\vect{q}}_2$ はダルシー流束密度の大きさ \unit{[\meter.\second^{-1}]},$\delta_{ij}$ はクロネッカーのデルタである.記号 $\norm{\cdot}_2$ はユークリッドノルムを意味する.
\begin{align*}
  \norm{\vect{q}}_2 & = \sqrt{q_x^2+q_y^2}       & \text{(2次元の場合)} \\
  \norm{\vect{q}}_2 & = \sqrt{q_x^2+q_y^2+q_z^2} & \text{(3次元の場合)}
\end{align*}
すると,見かけの熱伝導率テンソル成分は次のようになる.
\begin{equation}
  \begin{cases}
    \lambda_{xx} = \lambda_\mathrm{L} C_\mathrm{w}\dfrac{q_x^2}{\norm{\vect{q}}_2} + \lambda_\mathrm{T} C_\mathrm{w}\dfrac{q_y^2}{\norm{\vect{q}}_2} + \lambda_\mathrm{T} C_\mathrm{w}\dfrac{q_z^2}{\norm{\vect{q}}_2} + \lambda_0 \\[3mm]
    \lambda_{yy} = \lambda_\mathrm{L} C_\mathrm{w}\dfrac{q_y^2}{\norm{\vect{q}}_2} + \lambda_\mathrm{T} C_\mathrm{w}\dfrac{q_z^2}{\norm{\vect{q}}_2} + \lambda_\mathrm{T} C_\mathrm{w}\dfrac{q_x^2}{\norm{\vect{q}}_2} + \lambda_0 \\[3mm]
    \lambda_{zz} = \lambda_\mathrm{L} C_\mathrm{w}\dfrac{q_z^2}{\norm{\vect{q}}_2} + \lambda_\mathrm{T} C_\mathrm{w}\dfrac{q_x^2}{\norm{\vect{q}}_2} + \lambda_\mathrm{T} C_\mathrm{w}\dfrac{q_y^2}{\norm{\vect{q}}_2} + \lambda_0 \\[3mm]
    \lambda_{xy} = \pab{\lambda_\mathrm{L} -\lambda_\mathrm{T}} C_\mathrm{w}\dfrac{q_x q_y}{\norm{\vect{q}}_2}                                                                                                                     \\[3mm]
    \lambda_{yz} = \pab{\lambda_\mathrm{L} -\lambda_\mathrm{T}} C_\mathrm{w}\dfrac{q_y q_z}{\norm{\vect{q}}_2}                                                                                                                     \\[3mm]
    \lambda_{zx} = \pab{\lambda_\mathrm{L} -\lambda_\mathrm{T}} C_\mathrm{w}\dfrac{q_z q_x}{\norm{\vect{q}}_2}                                                                                                                     \\
  \end{cases}
\end{equation}

\subsubsection{水分物性値の定義}
\label{Sec:WaterProperties}

水分保持関数 (Water Retention Function, WRF) は,土壌水分量と土壌水分ポテンシャルとの関係を記述する関数であり,土壌水分ポテンシャルから土壌水分量を計算するために用いられる.本項では,主要な水分特性モデルについて,その定義と特性を概説する.
ここで, $\Sw$ は,体積含水率 $\Qw$,残留体積含水率 $\theta_\mathrm{r}$,飽和体積含水率 $\theta_\mathrm{s}$ を用いて次式で定義される正規化飽和度である.
\begin{equation}
  \Sw = \dfrac{\Qw - \theta_\mathrm{r}}{\theta_\mathrm{s} - \theta_\mathrm{r}}
\end{equation}
また,不飽和土壌における透水係数 $K_\mathrm{w}$ \unit{[\meter.\second^{-1}]} は,飽和透水係数 $K_\mathrm{s}$ \unit{[\meter.\second^{-1}]} と相対透水係数 $k_\mathrm{r}$ \unit{[-]} の積として次式で与えられる.
\begin{equation}
  \label{Eq:Hydraulic_Conductivity}
  K_\mathrm{w} = K_\mathrm{s} k_\mathrm{r}
\end{equation}
この $k_\mathrm{r}$ を記述する関数が透水係数関数 (Hydraulic Conductivity Function, HCF) と呼ばれる.

\begin{enumerate}[label=\arabic*:, leftmargin=*, labelwidth=1.5em, labelsep=0.5em]
  \item Brooks-Corey (BC) モデル \\
        \textcite{Brooks-1964}によるWRFおよびHCFは,以下で与えられる.
        \begin{equation}
          \Sw =
          \begin{cases}
            \pab{\dfrac{\alpha}{h}}^n & h < \alpha    \\
            1                         & h \geq \alpha
          \end{cases}
        \end{equation}
        \begin{equation}
          k_\mathrm{r} = \Sw^{l+2+2/n}
        \end{equation}
        ここで,$\alpha$ は空気侵入圧 \unit{[\meter^{-1}]},$n$ は間隙径分布指数 \unit{[-]},$l$ は間隙の連結性を示すパラメータ \unit{[-]} であり,極度の乾燥状態を除いて$2.0$と仮定される.
  \item van Genuchten-Mualem (vG) モデル \\
        \textcite{van-Genuchten-1980}によるWRFおよびHCFは,以下で与えられる.
        \begin{equation}
          \Sw =
          \begin{cases}
            \pab{1+\abs{\alpha h}^{n}}^{-m} & h < 0    \\
            1                               & h \geq 0 \\
          \end{cases}
        \end{equation}
        \begin{equation}
          k_\mathrm{r} = \Sw^{l}\bab{1-\pab{1-\Sw^{1/m}}^m}^2
        \end{equation}
        ここで,$m = 1 - 1/n$ ($n>1$) の関係が用いられる.
  \item Kosugi (KO) モデル \\
        \textcite{Kosugi-1996}は,$\Sw$ が対数正規分布に従うと仮定し,以下のWRFおよびHCFを提案した.
        \begin{equation}
          \Sw =
          \begin{cases}
            \dfrac{1}{2}\erfc\Bab{\dfrac{\ln\pab{h/\alpha}}{\sqrt{2}n}} & h < 0    \\
            1                                                           & h \geq 0
          \end{cases}
        \end{equation}
        \begin{equation}
          k_\mathrm{r} =
          \begin{cases}
            \Sw^{0.5}\Bab{\dfrac{1}{2}\erfc\bab{\dfrac{\ln\pab{h/\alpha}}{\sqrt{2}n}+\dfrac{n}{\sqrt{2}}}}^2 & h < 0    \\
            1                                                                                                & h \geq 0
          \end{cases}
        \end{equation}
        ここで,$\erfc\pab{x}$ は相補誤差関数である.
  \item Durner モデル \\
        ここまでに示したモデルでは捉えきれない団粒構造などを表現するため,\textcite{Durner-1994}は複数のWRFを線形結合する手法を提案した.
        \begin{equation}
          \label{eq:Linear_Conmination}
          S_\mathrm{w} = \sum_{i=1}^{k} w_i S_{\mathrm{w}_i}
        \end{equation}
        Durnerモデルでは2つのvGモデルを結合し,WRFとHCFを以下のように定義する.
        \begin{equation}
          \Sw = w_1\pab{1+\abs{\alpha_1 h}^{n_1}}^{-m_1} + w_2\pab{1+\abs{\alpha_2 h}^{n_2}}^{-m_2}
        \end{equation}
        \begin{equation}
          K_\mathrm{r} = \dfrac{\pab{w_1 S_\mathrm{w_1} + w_2 S_\mathrm{w_2}}^l\pab{w_1 \alpha_1\bab{1-\pab{1-S_\mathrm{w_1}^{1/m_1}}^{m_1}} + w_2 \alpha_2\bab{1-\pab{1-S_\mathrm{w_2}^{1/m_2}}^{m_2}}}^2}{\pab{w_1 \alpha_1 + w_2 \alpha_2}^2}
        \end{equation}
        ここで $w_i$ は各サブモデルの重み係数($\sum w_i = 1$),$S_\mathrm{w_i}$ は $i$番目のサブモデルの有効水分量である.
  \item Dual-vG-CH モデル \\
        \textcite{Seki-2022}は,Durnerモデルのパラメータを削減するため,各vGサブモデルで共通のパラメータ $\alpha$ を用いるDual-vG-CHモデルを提案した.
        \begin{equation}
          \Sw = w\pab{1+\abs{\alpha h}^{n_1}}^{-m_1} + \pab{1-w}\pab{1+\abs{\alpha h}^{n_2}}^{-m_2}
        \end{equation}
        \begin{equation}
          K_\mathrm{r} = \dfrac{\bab{w S_\mathrm{w_1} + \pab{1-w} S_\mathrm{w_2}}^l\Bab{w \alpha\bab{1-\pab{1-S_\mathrm{w_1}^{1/m_1}}^{m_1}} + \pab{1-w} \alpha\bab{1-\pab{1-S_\mathrm{w_2}^{1/m_2}}^{m_2}}}^2}{\pab{w \alpha + \pab{1-w} \alpha}^2}
        \end{equation}
\end{enumerate}

また,液状水が凍結する際に,氷が形成されることにより,間隙水量が減少する.この現象を考慮するため,間隙氷量 $\Qice$ を用いて,透水係数は次式で修正される(!TODO:引用入れる).
\begin{equation}
  \label{Eq:Hydraulic_Conductivity_Ice}
  K_\mathrm{flh} = K_{\mathrm{s}} 10^{-\Omega \Qice}
\end{equation}
ここで,$K_\mathrm{flh}$は凍結水含有状態における透水係数 [\unit{\meter.\second^{-1}}],$\Omega$ は経験的な定数 [--]であり,本研究では $\Omega = 10$ を採用した.

% \subsubsection{力学的熱物性値の定義}

% また,間隙水および間隙氷の密度は,それぞれの圧力 $\Pw$, $P_\mathrm{ice}$ に依存するとし,その体積圧縮係数をそれぞれ $K_\mathrm{w}$,$K_\mathrm{ice}$ とすると,以下の関係が成り立つ.
% \begin{subequations}
%   \label{Eq:Compressibility}
%   \begin{align}
%     \odif{\rho_\mathrm{w}}   & = \frac{\rho_\mathrm{w}}{K_\mathrm{w}} \odif{\Pw}                \\
%     \odif{\rho_\mathrm{ice}} & = \frac{\rho_\mathrm{ice}}{K_\mathrm{ice}} \odif{P_\mathrm{ice}}
%   \end{align}
% \end{subequations}

% \FloatBarrier
\subsection{輸送過程と保存則}
\label{Sec:TransportAndGoverning}

\subsubsection{液相の輸送}
% Darcy則に基づく液相フラックス
本節では,土壌中の液相水移動である液相水フラックスを,マトリックポテンシャル勾配を駆動力とするDarcyの法則に基づいてモデル化する.
多孔質媒体中の液相水フラックスは,Darcyの法則に従い,全水圧の勾配 $\nabla \Pw$ に比例すると仮定される.
\begin{equation}
  \label{Eq:LiquidFlux}
  \vect{j}_\mathrm{WL} = - K_\mathrm{w} \nabla \Pw
\end{equation}
ここで,$\vect{j}_\mathrm{WL}$ は液相水フラックス \unit{[\meter.\second^{-1}]}である.
不飽和土中の液相水の全水圧 $\Pw$ は,溶質を無視すればマトリックポテンシャル $h$ と重力ポテンシャル $z$ の和として表される.
\begin{equation}
  \label{Eq:LiquidPressure}
  \Pw = h + z
\end{equation}
ここで,重力ポテンシャル $z$ は,基準高さを $z_0$ \unit{[\meter]} とした場合,次式で与えられる.
\begin{equation}
  \label{Eq:GravitationalPotential}
  z = \pab{z_\text{soil} - z_0}
\end{equation}
ここで$z_\text{soil}$ は土中水の高さの鉛直座標 \unit{[\meter]} である.
液相水フラックス \eqref{Eq:LiquidFlux} に,液相水圧 \eqref{Eq:LiquidPressure}を代入し,さらに \eqref{Eq:GravitationalPotential}を用いて $\nabla \Pw$ を展開すると,次のように表される.
\begin{equation}
  \label{Eq:LiquidFlux_final}
  \vect{j}_\mathrm{WL} = - K_\mathrm{w} \nabla \pab{h + z}
\end{equation}
ここで,マトリックポテンシャル $h$ は,毛管力によって引き上げられた水の位置エネルギーを表すので,大気圧面を基準とした場合,$h$ は負の値をとる.
\begin{equation}
  \label{Eq:MatricPotential}
  h = - \frac{2 \sigma \cos\alpha}{\rho_\mathrm{w} g r_\mathrm{cap}}
\end{equation}
ここで,$\sigma$ は土壌間隙水の表面張力 \unit{[\newton.\meter^{-1}]},$\alpha$ は接触角 \unit{[\radian]},$r_\mathrm{cap}$ は毛管半径 \unit{[\meter]} である.
$\sigma$は温度の関数として次式で表される.
\begin{equation}
  \label{Eq:SurfaceTension}
  \sigma\pab{T} = 75.6 - 0.1425 T - 2.38 \times 10^{-4} T^2
\end{equation}
マトリックポテンシャル $h$ の温度依存性を調べるため,$\sigma$と$\rho_\mathrm{w}$が温度に依存するとし,\eqref{Eq:MatricPotential}を$T$ で偏微分すると,次式が得られる.
\begin{align}
  \label{Eq:MatricPotentialGradient}
  \pdv{h}{T} & = - \frac{2 \cos\alpha}{g r_\mathrm{cap}} \frac{\displaystyle \rho_\mathrm{w}\pdv{\sigma}{T} - \sigma\pdv{\rho_\mathrm{w}}{T}}{\rho_\mathrm{w}^2} \notag \\
             & = \frac{\rho_\mathrm{w} h}{\sigma} \frac{\displaystyle \rho_\mathrm{w}\pdv{\sigma}{T} - \sigma\pdv{\rho_\mathrm{w}}{T}}{\rho_\mathrm{w}^2} \notag        \\
             & = h \pab{\frac{1}{\sigma}\pdv{\sigma}{T} - \frac{1}{\rho_\mathrm{w}}\pdv{\rho_\mathrm{w}}{T}} \notag                                                     \\
             & \approx h \frac{1}{\sigma}\pdv{\sigma}{T}
\end{align}
よって $\vect{j}_\mathrm{WL}$は,温度勾配 $\nabla T$ とマトリックポテンシャル勾配 $\nabla h$ の関数として次式で表される.
\begin{equation}
  \label{Eq:LiquidFlux_final2}
  \vect{j}_\mathrm{WL} = - K_\mathrm{w} \pab{\frac{h}{\sigma}\pdv{\sigma}{T} \nabla T + \nabla h + \nabla z}
\end{equation}
ここで,温度勾配に起因する液相水フラックスの透水係数 $K_\mathrm{wT}$ \unit{[\meter^{2}.\second^{-1}.\kelvin^{-1}]} と,マトリックポテンシャル勾配に起因する液相水フラックスの透水係数  $K_\mathrm{wP}$ \unit{[\meter.\second^{-1}]} を,それぞれ次式で定義する.
\begin{align}
  \label{Eq:Kwh_def}
  K_\mathrm{wP} & = K_\mathrm{w}                                                 \\
  \label{Eq:KwT_def}
  K_\mathrm{wT} & = K_\mathrm{w} h G_\mathrm{wT} \frac{1}{\sigma}\pdv{\sigma}{T}
\end{align}
ここで,$G_\mathrm{wT}$は液相水の温度勾配に起因するフラックスの増進係数 \unit{[-]} であり,砂で$7$となる.
\eqref{Eq:Kwh_def} と \eqref{Eq:KwT_def} を用いると,液相水フラックス \eqref{Eq:LiquidFlux_final2} は次のように簡略化できる.
\begin{equation}
  \label{Eq:LiquidFlux_final3}
  \vect{j}_\mathrm{WL} = -K_\mathrm{wP} \pab{\nabla h + \nabla z} - K_\mathrm{wT} \nabla T
\end{equation}

\subsubsection{気相の輸送}
\label{Sec:VaporTransport}
% Fick則に基づく水蒸気輸送

本節では,土壌中の水蒸気移動である水蒸気密度フラックスを,水蒸気密度の勾配を駆動力とするFickの法則に基づいてモデル化する.
水蒸気密度フラックスは,Fickの法則に従い,水蒸気密度 $\rho_\mathrm{v}$ の勾配 $\nabla \rho_\mathrm{v}$ に比例すると仮定される.
したがって,当面の目標は,この水蒸気密度 $\rho_\mathrm{v}$ を熱力学変数で表し,その勾配 $\nabla \rho_\mathrm{v}$ を導出することである.
最終的なゴールは,水蒸気密度フラックス を観測可能な変数である温度勾配 $\nabla T$ とマトリックポテンシャル勾配 $\nabla h$ の関数として表現することである.

\cref{Sec:LiquidVaporEquilibrium}での仮定を用い,\eqref{Eq:VaporDensityGradient}を,$\nabla h$ を用いて書き直すと,次のように近似できる.
\begin{align}
  \label{Eq:VaporDensityGradient_approx}
  \nabla \rho_\mathrm{v} & = H_\mathrm{r} \odv{\rho_\mathrm{v}^\mathrm{sat}}{T} \nabla T + \rho_\mathrm{v}^\mathrm{sat} \pab{\pdv{H_\mathrm{r}}{T} \nabla T + \pdv{H_\mathrm{r}}{h} \nabla h} \notag  \\
                         & \approx H_\mathrm{r} \odv{\rho_\mathrm{v}^\mathrm{sat}}{T} \nabla T + \rho_\mathrm{v}^\mathrm{sat} \pdv{H_\mathrm{r}}{h} \nabla h \notag                                   \\
                         & = H_\mathrm{r} \odv{\rho_\mathrm{v}^\mathrm{sat}}{T} \nabla T + \rho_\mathrm{v}^\mathrm{sat} \pdv{}{h}\bab{\exp{\pab{\frac{h M_\mathrm{w} g}{R T^\ast}}} } \nabla h \notag \\
                         & = H_\mathrm{r} \odv{\rho_\mathrm{v}^\mathrm{sat}}{T} \nabla T + \rho_\mathrm{v}^\mathrm{sat} \pab{ H_\mathrm{r} \frac{M_\mathrm{w} g}{R T^\ast} } \nabla h
\end{align}
この\eqref{Eq:VaporDensityGradient_approx}が, 水蒸気密度フラックスを計算するための $\nabla \rho_\mathrm{v}$ の近似形となる.

% \subsubsection{土壌中の水蒸気拡散係数}
ここで,飽和水蒸気密度$\rho_\mathrm{v}^\mathrm{sat}$ \unit{[\kilogram.\meter^{-3}]} は温度$T$の関数であるので,次のように表される.
\begin{equation}
  \label{Eq:VaporSaturationDensity}
  \rho_\mathrm{v}^\mathrm{sat} = 10^{-3} \dfrac{\exp\pab{31.3716-\dfrac{6014.79}{T} - 0.00792495 T}}{T}
\end{equation}
水蒸気密度フラックスも他のガスと同様に,Fickの法則にしたがうと仮定すれば,次のように表される.
\begin{equation}
  \label{Eq:VaporFlux}
  \vect{j}_\mathrm{WV} = - \frac{\Qa \tau v D_\mathrm{atm}}{\rho_\mathrm{w}} \nabla \rho_\mathrm{v}
\end{equation}
ここで,$\vect{j}_\mathrm{WV}$は水蒸気密度フラックス\unit{[\meter.\second^{-1}]},$D_\mathrm{atm}$は大気中での水蒸気相互拡散係数\unit{[\meter^2.\second^{-1}]},$\tau$は屈曲度\unit{[-]},$v$は水蒸気の一方拡散による促進を示すマスフローファクター\unit{[-]}である.
一般に$D_\mathrm{atm}$は温度と土中空気全圧の関数となり,次のように表される\parencite{Campbell-1985}.
\begin{equation}
  \label{Eq:VaporDiffusionCoefficient}
  D_\mathrm{atm} \pab{T,P} = D_\mathrm{atm,0} \pab{\frac{T}{T_0}}^n \pab{\frac{P_0}{P}}
\end{equation}
ここで,$D_\mathrm{atm,0} = \qty{2.12e-5}{\meter^2.\second^{-1}}$は基準温度$T_0 = \qty{273.16}{\kelvin}$,基準圧力$P_0 = \qty{101.3}{\kilo\pascal}$における水蒸気の拡散係数であり,指数$n$は一般に水蒸気の場合$2$とされる.
土壌中の液状水表面からのみ水蒸気が発生するとし,氷などの固体によって気泡として閉じ込められた空気中の水蒸気は考慮しないとすれば,液状水表面は何らかの通路をたどって大気に接することができるので,$P_0/P\approx 1$の近似ができる.
よって,\eqref{Eq:VaporDiffusionCoefficient}は次のように近似できる.
\begin{equation}
  \label{Eq:VaporDiffusionCoefficient_approx}
  D_\mathrm{atm} \pab{T} \approx \num{2.12e-5} \pab{\frac{T}{273.15}}^2
\end{equation}
また,屈曲度$\tau$は,気相率$\Qa$の関数として次のように表される\parencite{Millington-1961}.
\begin{equation}
  \label{Eq:Tortuosity}
  \tau = \frac{\Qa^{7/3}}{\Qs^2}
\end{equation}
\textcite{Millington-1961} では,土壌中の水蒸気拡散係数 $D_\mathrm{v}$を $D_\mathrm{v} = D_\mathrm{atm} (\Qa^{10/3} / \Qs^2)$ のように $\Qa$ の $10/3$ 乗で表している.
しかし,本文では \eqref{Eq:VaporFlux} および \eqref{Eq:SoilVaporDiffusionCoefficient} で $D_\mathrm{v} = \Qa \tau D_\mathrm{atm}$ と定義しているため,これと整合させるために $\tau$ の定義 \eqref{Eq:Tortuosity} では $\Qa$ の指数が $7/3$ となっている点に注意されたい.
また,マスフローファクター$v$は,常温ではほぼ$1$とされる.

\eqref{Eq:VaporFlux}から\eqref{Eq:Tortuosity}までの議論は,土壌間隙中の気相をガスが単純に拡散する(Fickの法則)というモデルに基づいている.
しかし,単純な気相拡散モデルでは土壌中の熱的蒸気フラックス(温度勾配によるフラックス)を過小評価することが \textcite{Philip-1957} によって指摘されている.
これは,間隙中の液体の島 (liquid-island) を介した水蒸気と液状水の相互作用(相変化を伴う高速移動)や,熱伝導率の違いによる局所的な温度勾配の増加に起因する.
この効果を考慮するため,熱的蒸気フラックスを補正する促進係数 (enhancement factor) $\eta_e$ \unit{[-]} を導入する\parencite{Cass-1984, Campbell-1985}.
\begin{equation}
  \label{Eq:EnhancementFactor}
  \eta_e = 9.5 + 3 \frac{\Qw}{\Qs} - 8.5 \exp\Bab{-\bab{\pab{1+\frac{2.6}{\sqrt{f_c}}}\frac{\Qw}{\Qs}}^4}
\end{equation}
ここで,$f_c$は土壌中の質量粘土分率 \unit{[-]} である.

Fickの法則 \eqref{Eq:VaporFlux} に,近似した水蒸気密度勾配 \eqref{Eq:VaporDensityGradient_approx} を代入し,さらに \eqref{Eq:VaporDensityGradient} の第一項(温度勾配項)に対して熱的蒸気フラックスの促進係数 $\eta_e$ \eqref{Eq:EnhancementFactor} を適用することで,最終的な液相換算水蒸気フラックス密度は次のように表される.
\begin{align}
  \label{Eq:VaporFlux_final}
  \vect{j}_\mathrm{WV} & = - \frac{\Qa \tau v D_\mathrm{atm}}{\rho_\mathrm{w}} \pab{\eta_e H_\mathrm{r} \odv{\rho_\mathrm{v}^\mathrm{sat}}{T} \nabla T + \rho_\mathrm{v}^\mathrm{sat} \frac{M_\mathrm{w} g}{R T^\ast} H_\mathrm{r} \nabla h} \notag \\
                       & = - K_\mathrm{vT} \nabla T - K_\mathrm{vP} \nabla h
\end{align}
ここで,$K_\mathrm{vT}$は温度勾配に起因する水蒸気密度フラックスの有効拡散係数 \unit{[\meter^{2}.\second^{-1}.\kelvin^{-1}]},$K_\mathrm{vP}$はマトリックポテンシャル勾配に起因する水蒸気密度フラックスの有効拡散係数 \unit{[\meter.\second^{-1}]} であり,それぞれ
\begin{align}
  \label{Eq:Kvh_def}
  K_\mathrm{vP} & = \frac{D_\mathrm{v}}{\rho_\mathrm{w}} \rho_\mathrm{v}^\mathrm{sat} \frac{M_\mathrm{w} g}{R T^\ast} H_\mathrm{r} \\
  \label{Eq:KvT_def}
  K_\mathrm{vT} & = \frac{D_\mathrm{v}}{\rho_\mathrm{w}} \eta_e H_\mathrm{r} \odv{\rho_\mathrm{v}^\mathrm{sat}}{T}
\end{align}
と定義される.ここで,$D_\mathrm{v}$は土壌中の水蒸気拡散係数 \unit{[\meter^2.\second^{-1}]} であり,次のように表される.
\begin{equation}
  \label{Eq:SoilVaporDiffusionCoefficient}
  D_\mathrm{v} = \Qa \tau D_\mathrm{atm}
\end{equation}

\subsubsection{エネルギー保存則の基礎式}
\label{Sec:EnergyConservationBasic}
% 熱伝導と潜熱項

まず,領域$V$に含まれる土壌全体の全内部エネルギー$\mathcal{E}\pab{t}$は,ある点$\vect{r}\pab{x_1,x_2,x_3}$の単位体積あたり内部エネルギー$\mathcal{U}\pab{\vect{r},t}$の体積積分で表される.
\begin{equation}
  \label{Eq:Total_Energy}
  \mathcal{E}\pab{t} = \iiint_V \mathcal{U}\pab{\vect{r},t} \odif{V}
\end{equation}
ここで,$\mathcal{U}$は単位体積あたりの土壌全体の内部エネルギー \unit{[\joule.\meter^{-3}]} であり,各相の内部エネルギー$\mathcal{U}_i$の体積分率平均に加え,相変化に伴う内部エネルギー差を含む.
\begin{equation}
  \label{Eq:Total_InnerEnergy_Def}
  \mathcal{U} =
  \frac{1}{1 + e} \mathcal{U}_\mathrm{s}
  + \frac{e \Sw}{1 + e} \mathcal{U}_\mathrm{w}
  + \frac{e \Sice}{1 + e} \mathcal{U}_\mathrm{ice}
  + \pab{\frac{e \Sv}{1 + e} \frac{\rho_\mathrm{v}}{\rho_\mathrm{w}}} \mathcal{U}_\mathrm{v}^\star
  - \frac{e \Sice}{1 + e} \rho_\mathrm{ice} L_\mathrm{f}
  + \pab{\frac{e \Sv}{1 + e} \frac{\rho_\mathrm{v}}{\rho_\mathrm{w}}} \rho_\mathrm{w} L_\mathrm{v}
\end{equation}
また,各相の内部エネルギー$\mathcal{U}_i$ \unit{[\joule.\meter^{-3}]} は以下で与えられる.
\begin{subequations}
  \label{Eq:InnerEnergy_Component_Def}
  \begin{align}
    \label{Eq:InnerEnergy_Soil}
    \mathcal{U}_\mathrm{s}       & = c_\mathrm{s}T \rho_\mathrm{s}     \\
    \label{Eq:InnerEnergy_Liquid}
    \mathcal{U}_\mathrm{w}       & = c_\mathrm{w}T \rho_\mathrm{w}     \\
    \label{Eq:InnerEnergy_Ice}
    \mathcal{U}_\mathrm{ice}     & = c_\mathrm{ice}T \rho_\mathrm{ice} \\
    \label{Eq:InnerEnergy_Vapor_Eq}
    \mathcal{U}_\mathrm{v}^\star & = c_\mathrm{v}T \rho_\mathrm{w}
  \end{align}
\end{subequations}
ここで,$\mathcal{U}^\ast_\mathrm{v}$は液相換算水蒸気の内部エネルギーであるので注意されたい.
次に,領域の表面$S$を通過する単位時間あたりのエネルギー流束(エネルギーフラックス密度)$\vect{j}_E$ \unit{[\joule.\second^{-1}.\meter^{-2}]} を定義する.
エネルギーの輸送は,(1) 熱伝導による拡散と,(2) 液状水・水蒸気の移動に伴う移流の2形態で生じる.
\begin{equation}
  \label{Eq:Energy_Flux_Def}
  \vect{j}_E = \vect{j}_\mathrm{H} + \mathcal{U}_\mathrm{w} \vect{j}_\mathrm{WL} + \pab{\mathcal{U}_\mathrm{v}^\star + \rho_\mathrm{w} L_\mathrm{v}} \vect{j}_\mathrm{WV}
\end{equation}
ここで,$\vect{j}_\mathrm{H}$は熱伝導による熱フラックス密度,$\vect{j}_\mathrm{WL}$と$\vect{j}_\mathrm{WV}$はそれぞれ液状水と水蒸気のフラックス密度である.
エネルギー保存則より,領域$V$内のエネルギー変化量$\odif{E}$は,表面$S$を通じて流出するエネルギーと,領域内のエネルギー流出量$S_\mathrm{T}$ \unit{[\joule.\second^{-1}.\meter^{-3}]} の和と等しくなる.
ある微小時間$\odif{t}$の間に$S$全体から外向きに流出するエネルギーの総量は,フラックス$\vect{j}_E$の面積分で表される.
\begin{equation}
  \odif{\mathcal{E}\pab{t}} + \pab{\iint_S \vect{j}_E \cdot \odif{\vect{S}} + \iiint_V S_\mathrm{T} \odif{V}} \odif{t} = 0
\end{equation}
質量保存則の導出 \eqref{Eq:vel_Volume} と同様に,ガウスの発散定理を用いると,表面$S$からの流出項は体積積分に変換できる.
\begin{equation}
  \iint_S \vect{j}_E \cdot \odif{\vect{S}} = \iiint_V \div\pab{\vect{j}_E} \odif{V}
\end{equation}
これを上式に代入し,両辺を$\odif{t}\pab{>0}$で割ると,
\begin{equation}
  \label{Eq:Energy_Continuity_Integral}
  \odv{\mathcal{E}\pab{t}}{t} + \iiint_V \bab{\div\pab{\vect{j}_E} + S_\mathrm{T}} \odif{V} = 0
\end{equation}
\eqref{Eq:Total_Energy}を \eqref{Eq:Energy_Continuity_Integral}に代入し,時間微分を積分の内側に入れると,
\begin{align}
  \label{Eq:Energy_Continuity_Integral_Full}
   & \odv{}{t} \pab{\iiint_V \mathcal{U} \odif{V}} + \iiint_V \bab{\div\pab{\vect{j}_E} + S_\mathrm{T}} \odif{V} \notag \\
   & = \iiint_V \bab{\pdv{\mathcal{U}}{t} + \div\pab{\vect{j}_E} + S_\mathrm{T}} \odif{V} = 0
\end{align}
この\eqref{Eq:Energy_Continuity_Integral_Full}は任意の領域$V$について成り立つので,被積分関数は0でなければならない.
\begin{equation}
  \label{Eq:Energy_Continuity_Differential}
  \pdv{\mathcal{U}}{t} + \div\pab{\vect{j}_E} + S_\mathrm{T} = 0
\end{equation}
\eqref{Eq:Energy_Continuity_Differential}に,\eqref{Eq:Total_InnerEnergy_Def}と\eqref{Eq:Energy_Flux_Def}を代入することで,土壌中のエネルギー保存則の最終的な支配方程式が得られる.
\begin{equation}
  \label{Eq:EnergyConservation_Final}
  \begin{split}
     & \pdv{}{t} \Bab{\frac{1}{1 + e} \bab{\mathcal{U}_\mathrm{s} + e \Sw \mathcal{U}_\mathrm{w}+ e \Sice \mathcal{U}_\mathrm{ice} + \pab{e \Sv \frac{\rho_\mathrm{v}}{\rho_\mathrm{w}}} \mathcal{U}_\mathrm{v}^\star - e \Sice \rho_\mathrm{ice} L_\mathrm{f} + \pab{e \Sv \frac{\rho_\mathrm{v}}{\rho_\mathrm{w}}} \rho_\mathrm{w} L_\mathrm{v}}} \\
     & + \div\bab{\vect{j}_\mathrm{T} + \mathcal{U}_\mathrm{w} \vect{j}_\mathrm{WL} + \pab{\mathcal{U}_\mathrm{v}^\star + \rho_\mathrm{w} L_\mathrm{v}} \vect{j}_\mathrm{WV}} + S_\mathrm{T} = 0
  \end{split}
\end{equation}

\subsubsection{質量保存則の基礎式}
\label{Sec:MassConservationBasic}
% 質量保存

ある領域$V$に含まれる流体の質量$M\pab{t}$はある点$\vect{r}\pab{x_1,x_2,x_3}$の流体の密度$\rho\pab{\vect{r},t}$とその点の体積$\odif{V}$の積分で表される.
\begin{equation}
  \label{Eq:Density}
  M\pab{t} = \iiint_V \rho\pab{\vect{r},t}\odif{V}
\end{equation}
領域$V$の表面$S$について,面積要素$\odif{S}$,$S$に対して垂直で外向きの単位法線ベクトル$\vect{n}$を導入し,面積要素ベクトル$\odif{\vect{S}}$を定める.
\begin{equation}
  \odif{\vect{S}} = \vect{n}\odif{S}
\end{equation}
微小面積$\odif{S}$を通って流出する流体の体積はそのときの流体の速度$\vect{v}\pab{\vect{r},t}$を用いて表される.
\begin{equation}
  \label{Eq:vel_Volume}
  \abs{\vect{v}\pab{\vect{r},t}}\odif{S}\cos\theta = \vect{v}\pab{\vect{r},t}\cdot\odif{\vect{S}}
\end{equation}
ここで$\theta$は$\vect{v}$と$\vect{n}$のなす角である.この体積に密度をかけたものがある単位時間あたりに流出する流体の質量となる.$S$全体である微小時間$\odif{t}$の間に外向きに流出する流体の質量は\eqref{Eq:vel_Volume}を領域$S$全体で積分することで求められる.これをガウスの発散定理で表すと,
\begin{equation}
  \iint_S \rho\pab{\vect{r},t} \vect{v}\pab{\vect{r},t}\cdot\odif{\vect{S}}\odif{t} = \iiint_V\nabla\cdot\pab{\rho\pab{\vect{r},t}\vect{v}\pab{\vect{r},t}}\odif{V}\odif{t} = \iiint_V\div\pab{\rho\pab{\vect{r},t}\vect{v}\pab{\vect{r},t}}\odif{V}
\end{equation}
質量保存則が成り立つようにすれば,領域$V$内の流体の質量変化量$\odif{M}$は,表面$S$を通じて流出する流体の質量と流体の湧き出し量$S_\mathrm{H}$と等しくなる.
\begin{equation}
  \odif{M\pab{t}}+\iiint_V\bab{\div\pab{\rho\pab{\vect{r},t}\vect{v}\pab{\vect{r},t}} + S_\mathrm{H}}\odif{V}\odif{t} = 0
\end{equation}
両辺を$\odif{t}\pab{>0}$で割って,\eqref{Eq:Density}を用いて整理すると
\begin{align}
  \label{Eq:Continuity_Integral}
   & \odv{M\pab{t}}{t}+\iiint_V\bab{\div\pab{\rho\pab{\vect{r},t}\vect{v}\pab{\vect{r},t}}+ S_\mathrm{H}}\odif{V}\notag            \\
   & =\iiint_V\bab{\pdv{\rho\pab{\vect{r},t}}{t}+\div\pab{\rho\pab{\vect{r},t}\vect{v}\pab{\vect{r},t}} + S_\mathrm{H}} \odif{V}=0
\end{align}
\eqref{Eq:Continuity_Integral}は任意の$\vect{r}$についていたるところで成り立つので,積分の中身は0となる.よって,連続の式は
\begin{equation}
  \label{Eq:Continuity}
  \pdv{\rho}{t} + \div\pab{\rho\vect{v}} + S_\mathrm{H} = 0
\end{equation}
\eqref{Eq:Continuity}は連続の式と呼ばれ,流体の質量保存則を表す.ここで連続の式を土壌間隙に適用する.
このとき,間隙に対する水の密度$\rho_\mathrm{void}$ \unit{[\kilogram.\meter^{-3}]} は次のように定義される.
\begin{equation}
  \label{Eq:rho_void}
  \rho_\mathrm{void} \coloneq \frac{e}{1+e} \pab{\rho_\mathrm{w} \Sw + \rho_\mathrm{ice} \Sice + \rho_\mathrm{v} \Sv}
\end{equation}
この時,領域$V$の表面$S$からの流出は液状水および水蒸気であり氷は不動であるとすれば,\eqref{Eq:Continuity}は\eqref{Eq:rho_void}と液状水フラックス密度$\vect{j}_\mathrm{WL}$と水蒸気フラックス密度$\vect{j}_\mathrm{WV}$を用いて次のように表される.
\begin{equation}
  \label{Eq:Continuity_void}
  \pdv{\rho_\mathrm{void}}{t} + \div\bab{\rho_\mathrm{w} \pab{\vect{j}_\mathrm{WL} + \vect{j}_\mathrm{WV}}} + S_\mathrm{H} = 0
\end{equation}
\eqref{Eq:rho_void} と \eqref{Eq:Continuity_void} により,間隙中の水の質量保存則が表される.
\begin{equation}
  \label{Eq:MassConservation_Water}
  \pdv{}{t} \bab{\frac{e}{1+e} \pab{\rho_\mathrm{w} \Sw + \rho_\mathrm{ice} \Sice + \rho_\mathrm{v} \Sv}} + \div\bab{\rho_\mathrm{w} \pab{\vect{j}_\mathrm{WL} + \vect{j}_\mathrm{WV}}} + S_\mathrm{H} = 0
\end{equation}

\FloatBarrier
\numberwithin{equation}{section}

\FloatBarrier
\section{支配方程式の導出}
\label{Sec:GoverningEquationDerivation}
\numberwithin{equation}{subsection}

これまでに,土壌凍結現象を支配する熱・水分の基礎方程式を導出した.
以降では,これらの方程式を数値計算に適した形(支配方程式)へと変形し,まとめる.

\numberwithin{equation}{subsection}
\subsection{熱移動における支配方程式}
\label{Sec:ThermalGoverning}

\cref{Sec:EnergyConservationBasic}で導出したエネルギー保存則の基礎式 (\eqref{Eq:EnergyConservation_Final})に,各相の内部エネルギーを\eqref{Eq:InnerEnergy_Component_Def}として代入し,さらに熱フラックス$\vect{j}_\mathrm{T}$をFourier則に基づいて表しせば,次の支配方程式が得られる.ただし,圧力は全水頭として考えるものとする.
\begin{equation}
      \label{Eq:ThermalGoverningEquation}
      \begin{split}
             & \pdv{}{t} \pab[bigg]{c_\mathrm{s}\rho_\mathrm{s}\Qn T +  c_\mathrm{w} \rho_\mathrm{w} \Qw T + c_\mathrm{ice}\rho_\mathrm{ice} \Qice T + c_\mathrm{v} \rho_\mathrm{w} \Qv^\star T - \rho_\mathrm{ice} L_\mathrm{f} \Qice + \rho_\mathrm{w} L_\mathrm{v} \Qv^\star} \\
             & + \div\bab[big]{-\lambda \grad T + c_\mathrm{w} \rho_\mathrm{w} \vect{j}_\mathrm{WL} T + c_\mathrm{v}\rho_\mathrm{w} \vect{j}_\mathrm{WV} T + \rho_\mathrm{w} L_\mathrm{v} \vect{j}_\mathrm{WV}} + S_\mathrm{T} = 0
      \end{split}
\end{equation}
ここで,$\Qv^\star$は気相水蒸気の液相換算含有率であり,\eqref{Eq:Qvstar}で定義される.
\begin{equation}
      \label{Eq:Qvstar}
      \Qv^\star =\frac{e \Sv}{1 + e}\frac{\rho_\mathrm{v}}{\rho_\mathrm{w}}
\end{equation}
まず,\eqref{Eq:ThermalGoverningEquation}の時間微分項について整理する.
\begin{itemize}
      \item 土粒子項
            \begin{align}
                  \label{Eq:ThermalGoverning_Time_Soil}
                  \pdv{}{t} \pab{c_\mathrm{s}\rho_\mathrm{s}\Qn T} = c_\mathrm{s}\rho_\mathrm{s}\Qn \pdv{T}{t}
            \end{align}
      \item 液相水項
            \begin{align}
                  \pdv{}{t} \pab{c_\mathrm{w} \rho_\mathrm{w} \Qw T}
                   & = c_\mathrm{w} \Qw T \pdv{\rho_\mathrm{w}}{t} + c_\mathrm{w} \rho_\mathrm{w} T \pdv{\Qw}{t} + c_\mathrm{w} \rho_\mathrm{w} \Qw \pdv{T}{t}   \notag     \\
                   & = c_\mathrm{w} \Qw T \pab{\pdv{\rho_\mathrm{w}}{T}\pdv{T}{t} + \pdv{\rho_\mathrm{w}}{\Pw}\pdv{\Pw}{t}}                                         \notag  \\
                   & \quad + c_\mathrm{w} \rho_\mathrm{w} T \pab{\pdv{\Qw}{T}\pdv{T}{t} + \pdv{\Qw}{\Pw}\pdv{\Pw}{t}} + c_\mathrm{w} \rho_\mathrm{w} \Qw \pdv{T}{t} \notag  \\
                   & = \pab{c_\mathrm{w} \Qw T \pdv{\rho_\mathrm{w}}{T} + c_\mathrm{w} \rho_\mathrm{w} T \pdv{\Qw}{T} + c_\mathrm{w} \rho_\mathrm{w} \Qw} \pdv{T}{t} \notag \\
                  \label{Eq:ThermalGoverning_Time_Water}
                   & \quad + \pab{c_\mathrm{w} \Qw T \pdv{\rho_\mathrm{w}}{\Pw} + c_\mathrm{w} \rho_\mathrm{w} T \pdv{\Qw}{\Pw}} \pdv{\Pw}{t}
            \end{align}
      \item 氷項
            \begin{align}
                  \pdv{}{t} \pab{c_\mathrm{ice} \rho_\mathrm{ice} \Qice T}
                   & = c_\mathrm{ice} \Qice T \pdv{\rho_\mathrm{ice}}{t} + c_\mathrm{ice} \rho_\mathrm{ice} T \pdv{\Qice}{t} + c_\mathrm{ice} \rho_\mathrm{ice} \Qice \pdv{T}{t}                    \notag \\
                   & = c_\mathrm{ice} \Qice T \pab{\pdv{\rho_\mathrm{ice}}{T}\pdv{T}{t} + \pdv{\rho_\mathrm{ice}}{\Pw}\pdv{\Pw}{t}}                                               \notag                   \\
                   & \quad + c_\mathrm{ice} \rho_\mathrm{ice} T \pab{\pdv{\Qice}{T}\pdv{T}{t} + \pdv{\Qice}{\Pw}\pdv{\Pw}{t}} + c_\mathrm{ice} \rho_\mathrm{ice} \Qice \pdv{T}{t} \notag                   \\
                   & = \pab{c_\mathrm{ice} \Qice T \pdv{\rho_\mathrm{ice}}{T} + c_\mathrm{ice} \rho_\mathrm{ice} T \pdv{\Qice}{T} + c_\mathrm{ice} \rho_\mathrm{ice} \Qice} \pdv{T}{t}              \notag \\
                  \label{Eq:ThermalGoverning_Time_Ice}
                   & \quad + \pab{c_\mathrm{ice} \Qice T \pdv{\rho_\mathrm{ice}}{\Pw} + c_\mathrm{ice} \rho_\mathrm{ice} T \pdv{\Qice}{\Pw}} \pdv{\Pw}{t}
            \end{align}
      \item 気相水蒸気項
            \begin{align}
                  \pdv{}{t}  \pab{c_\mathrm{v} \rho_\mathrm{w} \Qv^\star T}
                   & = c_\mathrm{v} \pab{\Qv^\star T \pdv{\rho_\mathrm{w}}{t} + \rho_\mathrm{w} T \pdv{\Qv^\star}{t} + \rho_\mathrm{w} \Qv^\star \pdv{T}{t}}                                                    \notag \\
                   & = c_\mathrm{v} \left[\Qv^\star T \pab{\pdv{\rho_\mathrm{w}}{T}\pdv{T}{t} + \pdv{\rho_\mathrm{w}}{\Pw}\pdv{\Pw}{t}} \right.                                               \notag                   \\
                   & + \left. \rho_\mathrm{w} T \pab{\pdv{\Qv^\star}{T}\pdv{T}{t} + \pdv{\Qv^\star}{\Pw}\pdv{\Pw}{t}} + \rho_\mathrm{w} \Qv^\star \pdv{T}{t} \right]                          \notag                   \\
                   & = \pab{c_\mathrm{v} \Qv^\star T \pdv{\rho_\mathrm{w}}{T} + c_\mathrm{v} \rho_\mathrm{w} T \pdv{\Qv^\star}{T} + c_\mathrm{v} \rho_\mathrm{w} \Qv^\star} \pdv{T}{t}                          \notag \\
                   & + \pab{c_\mathrm{v} \Qv^\star T \pdv{\rho_\mathrm{w}}{\Pw} + c_\mathrm{v} \rho_\mathrm{w} T \pdv{\Qv^\star}{\Pw}} \pdv{\Pw}{t}  \label{Eq:ThermalGoverning_Time_Vapor}
            \end{align}
      \item 潜熱項
            \begin{align}
                  \pdv{}{t} & \pab{- \rho_\mathrm{ice} L_\mathrm{f} \Qice + \rho_\mathrm{w} L_\mathrm{v} \Qv^\star} \notag                                                                                                                                           \\
                            & = - L_\mathrm{f} \Qice \pdv{\rho_\mathrm{ice}}{t} - \rho_\mathrm{ice} L_\mathrm{f} \pdv{\Qice}{t} + L_\mathrm{v} \Qv^\star \pdv{\rho_\mathrm{w}}{t} + \rho_\mathrm{w} L_\mathrm{v} \pdv{\Qv^\star}{t} \notag                           \\
                            & = - L_\mathrm{f} \Qice \pab{\pdv{\rho_\mathrm{ice}}{T}\pdv{T}{t} + \pdv{\rho_\mathrm{ice}}{\Pw}\pdv{\Pw}{t}} \notag                                                                                                                    \\
                            & \quad - \rho_\mathrm{ice} L_\mathrm{f} \pab{\pdv{\Qice}{T}\pdv{T}{t} + \pdv{\Qice}{\Pw}\pdv{\Pw}{t}} \notag                                                                                                                            \\
                            & \quad + L_\mathrm{v} \Qv^\star \pab{\pdv{\rho_\mathrm{w}}{T}\pdv{T}{t} + \pdv{\rho_\mathrm{w}}{\Pw}\pdv{\Pw}{t}} \notag                                                                                                                \\
                            & \quad + \rho_\mathrm{w} L_\mathrm{v} \pab{\pdv{\Qv^\star}{T}\pdv{T}{t} + \pdv{\Qv^\star}{\Pw}\pdv{\Pw}{t}} \notag                                                                                                                      \\
                            & = \pab{- L_\mathrm{f} \Qice \pdv{\rho_\mathrm{ice}}{T} - \rho_\mathrm{ice} L_\mathrm{f} \pdv{\Qice}{T} + L_\mathrm{v} \Qv^\star \pdv{\rho_\mathrm{w}}{T} + \rho_\mathrm{w} L_\mathrm{v} \pdv{\Qv^\star}{T}} \pdv{T}{t} \notag          \\
                  \label{Eq:ThermalGoverning_Time_Latent}
                            & \quad + \pab{- L_\mathrm{f} \Qice \pdv{\rho_\mathrm{ice}}{\Pw} - \rho_\mathrm{ice} L_\mathrm{f} \pdv{\Qice}{\Pw} + L_\mathrm{v} \Qv^\star \pdv{\rho_\mathrm{w}}{\Pw} + \rho_\mathrm{w} L_\mathrm{v} \pdv{\Qv^\star}{\Pw}} \pdv{\Pw}{t}
            \end{align}
\end{itemize}
したがって,時間微分項は次式で表される.
\begin{equation}
      \label{Eq:ThermalGoverning_Time_Total}
      \begin{split}
             & \text{Time Term} =                                                                                                                                                                                                                                                    \\
             & \biggl[ c_\mathrm{s}\rho_\mathrm{s}\Qn + c_\mathrm{w} \Qw T \pdv{\rho_\mathrm{w}}{T} + c_\mathrm{w} \rho_\mathrm{w} T \pdv{\Qw}{T} + c_\mathrm{w} \rho_\mathrm{w} \Qw                                                                                                 \\
             & \quad + c_\mathrm{ice} \Qice T \pdv{\rho_\mathrm{ice}}{T} + c_\mathrm{ice} \rho_\mathrm{ice} T \pdv{\Qice}{T} + c_\mathrm{ice} \rho_\mathrm{ice} \Qice + c_\mathrm{v} \Qv^\star T \pdv{\rho_\mathrm{w}}{T} + c_\mathrm{v} \rho_\mathrm{w} T \pdv{\Qv^\star}{T}        \\
             & \quad + c_\mathrm{v} \rho_\mathrm{w} \Qv^\star - L_\mathrm{f} \Qice \pdv{\rho_\mathrm{ice}}{T} - \rho_\mathrm{ice} L_\mathrm{f} \pdv{\Qice}{T} + L_\mathrm{v} \Qv^\star \pdv{\rho_\mathrm{w}}{T} + \rho_\mathrm{w} L_\mathrm{v} \pdv{\Qv^\star}{T} \biggr] \pdv{T}{t} \\
             & + \biggl( c_\mathrm{w} \Qw T \pdv{\rho_\mathrm{w}}{\Pw} + c_\mathrm{w} \rho_\mathrm{w} T \pdv{\Qw}{\Pw} + c_\mathrm{ice} \Qice T \pdv{\rho_\mathrm{ice}}{\Pw} + c_\mathrm{ice} \rho_\mathrm{ice} T \pdv{\Qice}{\Pw}                                                   \\
             & \quad + c_\mathrm{v} \Qv^\star T \pdv{\rho_\mathrm{w}}{\Pw} + c_\mathrm{v} \rho_\mathrm{w} T \pdv{\Qv^\star}{\Pw} - L_\mathrm{f} \Qice \pdv{\rho_\mathrm{ice}}{\Pw} - \rho_\mathrm{ice} L_\mathrm{f} \pdv{\Qice}{\Pw}                                                 \\
             & \quad + L_\mathrm{v} \Qv^\star \pdv{\rho_\mathrm{w}}{\Pw} + \rho_\mathrm{w} L_\mathrm{v} \pdv{\Qv^\star}{\Pw} \biggr) \pdv{\Pw}{t}
      \end{split}
\end{equation}
以上より,エネルギー式における時間微分項は,温度 $T$ および水圧 $\Pw$ の時間変化に対して次式のように整理される.
\begin{equation}
      \label{Eq:ThermalGoverning_Time_Simplified}
      \text{Time Term} = C_\mathrm{TT} \pdv{T}{t} + C_\mathrm{TH} \pdv{\Pw}{t}
\end{equation}
ここで,$C_\mathrm{TT}$ および $C_\mathrm{TH}$ はそれぞれ温度および圧力に対する一般化された熱容量項であり,次式で定義される.
\begin{subequations}
      \begin{align}
            \label{Eq:Def_CTT}
            C_\mathrm{TT} & = c_\mathrm{s}\rho_\mathrm{s}\Qn + c_\mathrm{w} \Qw T \pdv{\rho_\mathrm{w}}{T} + c_\mathrm{w} \rho_\mathrm{w} T \pdv{\Qw}{T} + c_\mathrm{w} \rho_\mathrm{w} \Qw \notag                                                                                                \\
                          & \quad + c_\mathrm{ice} \Qice T \pdv{\rho_\mathrm{ice}}{T} + c_\mathrm{ice} \rho_\mathrm{ice} T \pdv{\Qice}{T} + c_\mathrm{ice} \rho_\mathrm{ice} \Qice + c_\mathrm{v} \Qv^\star T \pdv{\rho_\mathrm{v}}{T} + c_\mathrm{v} \rho_\mathrm{w} T \pdv{\Qv^\star}{T} \notag \\
                          & \quad + c_\mathrm{v} \rho_\mathrm{w} \Qv^\star - L_\mathrm{f} \Qice \pdv{\rho_\mathrm{ice}}{T} - \rho_\mathrm{ice} L_\mathrm{f} \pdv{\Qice}{T} + L_\mathrm{v} \Qv^\star \pdv{\rho_\mathrm{w}}{T} + \rho_\mathrm{w} L_\mathrm{v} \pdv{\Qv^\star}{T}                    \\[10pt]
            \label{Eq:Def_CTP}
            C_\mathrm{TH} & = c_\mathrm{w} \Qw T \pdv{\rho_\mathrm{w}}{\Pw} + c_\mathrm{w} \rho_\mathrm{w} T \pdv{\Qw}{\Pw} \notag                                                                                                                                                                \\
                          & \quad + c_\mathrm{ice} \Qice T \pdv{\rho_\mathrm{ice}}{\Pw} + c_\mathrm{ice} \rho_\mathrm{ice} T \pdv{\Qice}{\Pw} + c_\mathrm{v} \Qv^\star T \pdv{\rho_\mathrm{w}}{\Pw} + c_\mathrm{v} \rho_\mathrm{w} T \pdv{\Qv^\star}{\Pw} \notag                                  \\
                          & \quad - L_\mathrm{f} \Qice \pdv{\rho_\mathrm{ice}}{\Pw} - \rho_\mathrm{ice} L_\mathrm{f} \pdv{\Qice}{\Pw} + L_\mathrm{v} \Qv^\star \pdv{\rho_\mathrm{w}}{\Pw} + \rho_\mathrm{w} L_\mathrm{v} \pdv{\Qv^\star}{\Pw}
      \end{align}
\end{subequations}
したがって,熱移動支配方程式は次式で表される.
\begin{equation}
      \label{Eq:ThermalGoverning_Final}
      C_\mathrm{TT} \pdv{T}{t} + C_\mathrm{TH} \pdv{\Pw}{t} + \nabla\cdot\bab[bigg]{-\lambda \nabla T + c_\mathrm{w} \rho_\mathrm{w} \vect{j}_\mathrm{WL} T + c_\mathrm{v}\rho_\mathrm{w} \vect{j}_\mathrm{WV} T + \rho_\mathrm{w} L_\mathrm{v} \vect{j}_\mathrm{WV}} + S_\mathrm{T} = 0
\end{equation}

\FloatBarrier
\subsection{水分移動における支配方程式}
\label{Sec:HydrauilicGoverning}

\cref{Sec:MassConservationBasic}で導出した質量保存則の基礎式 (\eqref{Eq:MassConservation_Water})に対して,\eqref{Eq:Theta_Def,Eq:Qvstar}を用いれば
\begin{equation}
  \label{Eq:MassConservation_Water_Expanded}
  \pdv{}{t} \bab{\pab{\rho_\mathrm{w} \Qw + \rho_\mathrm{ice} \Qice + \rho_\mathrm{w} \Qv^\star}} + \div\bab{\rho_\mathrm{w} \pab{\vect{j}_\mathrm{WL} + \vect{j}_\mathrm{WV}}} + S_\mathrm{H} = 0
\end{equation}
時間微分項を展開すると,
\begin{align}
  \label{Eq:MassConservation_Water_TimeDerivative}
  \pdv{}{t} & \pab{\rho_\mathrm{w} \Qw + \rho_\mathrm{ice} \Qice + \rho_\mathrm{w} \Qv^\star} \notag                                                                                                                                                                                                                            \\
            & = \rho_\mathrm{w} \pdv{\Qw}{t} + \Qw \pdv{\rho_\mathrm{w}}{t} + \rho_\mathrm{ice} \pdv{\Qice}{t} + \Qice \pdv{\rho_\mathrm{ice}}{t} + \rho_\mathrm{w} \pdv{\Qv^\star}{t} + \Qv^\star \pdv{\rho_\mathrm{w}}{t} \notag                                                                                              \\
            & = \rho_\mathrm{w} \pab{\pdv{\Qw}{T} \pdv{T}{t} + \pdv{\Qw}{\Pw} \pdv{\Pw}{t}} + \Qw \pab{\pdv{\rho_\mathrm{w}}{T} \pdv{T}{t} + \pdv{\rho_\mathrm{w}}{\Pw} \pdv{\Pw}{t}} \notag                                                                                                \\
            & \quad + \rho_\mathrm{ice} \pab{\pdv{\Qice}{T} \pdv{T}{t} + \pdv{\Qice}{\Pw} \pdv{\Pw}{t}} + \Qice \pab{\pdv{\rho_\mathrm{ice}}{T} \pdv{T}{t} + \pdv{\rho_\mathrm{ice}}{\Pw} \pdv{\Pw}{t}} \notag                                                                              \\
            & \quad + \rho_\mathrm{w} \pab{\pdv{\Qv^\star}{T} \pdv{T}{t} + \pdv{\Qv^\star}{\Pw} \pdv{\Pw}{t}} + \Qv^\star \pab{\pdv{\rho_\mathrm{w}}{T} \pdv{T}{t} + \pdv{\rho_\mathrm{w}}{\Pw} \pdv{\Pw}{t}} \notag                                                                        \\
            & = \pab{\rho_\mathrm{w} \pdv{\Qw}{T} + \rho_\mathrm{ice} \pdv{\Qice}{T} + \rho_\mathrm{w} \pdv{\Qv^\star}{T} + \Qw \pdv{\rho_\mathrm{w}}{T} + \Qice \pdv{\rho_\mathrm{ice}}{T} + \Qv^\star \pdv{\rho_\mathrm{w}}{T}} \pdv{T}{t} \notag                                                                             \\
            & \quad + \pab{\rho_\mathrm{w} \pdv{\Qw}{\Pw} + \rho_\mathrm{ice} \pdv{\Qice}{\Pw} + \rho_\mathrm{w} \pdv{\Qv^\star}{\Pw} + \Qw \pdv{\rho_\mathrm{w}}{\Pw} + \Qice \pdv{\rho_\mathrm{ice}}{\Pw} + \Qv^\star \pdv{\rho_\mathrm{w}}{\Pw}} \pdv{\Pw}{t}
\end{align}
となる.ここで,次のように変数を整理する.
\begin{subequations}
  \label{Eq:MassConservation_Water_Coefficient_T}
  \begin{equation}
    \label{Eq:MassConservation_Water_Coefficient_T_Def}
    C_\mathrm{HT} = \rho_\mathrm{w} \pdv{\Qw}{T} + \rho_\mathrm{ice} \pdv{\Qice}{T} + \rho_\mathrm{w} \pdv{\Qv^\star}{T} + \Qw \pdv{\rho_\mathrm{w}}{T} + \Qice \pdv{\rho_\mathrm{ice}}{T} + \Qv^\star \pdv{\rho_\mathrm{w}}{T}
  \end{equation}
  \begin{equation}
    \label{Eq:MassConservation_Water_Coefficient_P}
    C_\mathrm{HH} = \rho_\mathrm{w} \pdv{\Qw}{\Pw} + \rho_\mathrm{ice} \pdv{\Qice}{\Pw} + \rho_\mathrm{w} \pdv{\Qv^\star}{\Pw} + \Qw \pdv{\rho_\mathrm{w}}{\Pw} + \Qice \pdv{\rho_\mathrm{ice}}{\Pw} + \Qv^\star \pdv{\rho_\mathrm{w}}{\Pw}
  \end{equation}
\end{subequations}

また,\eqref{Eq:LiquidFlux_final3,Eq:VaporFlux_final}に示した水分フラックス密度の式を用いれば,発散項は次式で表される.
\begin{align}
  \nabla\cdot & \bab{\rho_\mathrm{w} \pab{\vect{j}_\mathrm{WL} + \vect{j}_\mathrm{WV}}} \notag                                                                                                                                     \\
              & = \nabla\cdot\bab{-\rho_\mathrm{w} \pab{K_\mathrm{wP}\nabla \Pw - K_\mathrm{wP} \rho_\mathrm{w} g \nabla z + K_\mathrm{wT} \nabla T + K_\mathrm{vP} \nabla \Pw + K_\mathrm{vT} \nabla T}} \notag \\
              & = \nabla\cdot\bab{-\rho_\mathrm{w} \pab{\pab{K_\mathrm{wP} + K_\mathrm{vP}} \nabla \Pw - K_\mathrm{wP} \rho_\mathrm{w} g \nabla z + \pab{K_\mathrm{wT} + K_\mathrm{vT}} \nabla T}} \notag                 \\
              & = \nabla\cdot\bab{-\rho_\mathrm{w} \pab{K_\mathrm{P} \nabla \Pw - K_\mathrm{wP} \rho_\mathrm{w} g \nabla z + K_\mathrm{T} \nabla T}} \label{Eq:MassConservation_Water_Divergence}
\end{align}
ここで,液相および気相水分フラックスはいずれも圧力勾配および温度勾配に対して線形に応答するため,同一の駆動力に対応する項をまとめ,有効な透水係数として再定義する.
\begin{subequations}
  \label{Eq:MassConservation_Water_CombinedConductivity}
  \begin{align}
    \label{Eq:MassConservation_Water_CombinedConductivity_P}
    K_\mathrm{P} & \coloneq K_\mathrm{wP} + K_\mathrm{vP} \\
    \label{Eq:MassConservation_Water_CombinedConductivity_T}
    K_\mathrm{T} & \coloneq K_\mathrm{wT} + K_\mathrm{vT}
  \end{align}
\end{subequations}
以上より,水分移動に関する支配方程式は\eqref{Eq:MassConservation_Water_Expanded,Eq:MassConservation_Water_TimeDerivative,Eq:MassConservation_Water_Divergence}を組み合わせて得られる.
\begin{equation}
  \label{Eq:HydraulicGoverning_Final}
  C_\mathrm{HT} \pdv{T}{t} + C_\mathrm{HH} \pdv{\Pw}{t} - \nabla\cdot\bab{\rho_\mathrm{w} \pab{K_\mathrm{P} \nabla \Pw - K_\mathrm{wP} \rho_\mathrm{w} g \nabla z + K_\mathrm{T} \nabla T}} + S_\mathrm{H} = 0
\end{equation}
\numberwithin{equation}{section}
\section{有限要素法による離散化}
\label{Sec:FEMFormulation}
\numberwithin{equation}{subsection}

自然界の物理現象の多くは偏微分方程式によって記述されるが,複雑な境界条件や非線形性を有する場合,厳密な解析解を得ることは極めて困難である.有限要素法(Finite Element Method: FEM)は,このような連続体の境界値問題を,計算機を用いて数値的に解くための最も汎用的な離散化手法の一つである.本手法の核心は,本来無限の自由度を持つ連続体上の未知関数を,有限個のパラメータ(節点値)と既知の関数(形状関数)の線形結合によって近似的に表現することにある.具体的には,解析領域を「要素」と呼ばれる単純な形状の小領域に分割し,各要素内での物理量の変化を低次の多項式などで補間する.数理的には,支配方程式(強形式)に対して重み付き残差法(Method of Weighted Residuals)を適用し,積分形式である弱形式 (Weak Form) へと変換するプロセスを経る.これにより,微分可能性の要求を緩和しつつ,代数的な連立一次方程式へと帰着させることが可能となる.本節では,\cref{Sec:GoverningEquationDerivation}で導出した支配方程式を有限要素法により離散化する手法を示す.

\subsection{時間積分法}
\label{Sec:TimeIntegration}

\subsubsection{時間積分法の概要}
\label{Sec:TimeIntegration_Overview}

有限要素法などにより空間方向を離散化すると,\cref{Sec:GoverningEquationDerivation}で導出した熱・水分移動の支配方程式系は,一般に次のような常微分方程式(あるいは微分代数方程式)系として表される.
\begin{equation}
  \label{Eq:Generic_ODE_System}
  \pdv{\vect{y}}{t} = \vect{F}\,\pab{\vect{y},t}
\end{equation}
ここで,$\vect{y}(t)$ は全自由度の集合,$\vect{F}$ は伝導・移流・相変化・供給項などを含む非線形作用素ベクトルである.数値計算では,この時間連続問題を,離散的な時刻列
\begin{equation}
  t_0 < t_1 < \dots < t_n < t_{n+1} < \dots < t_N, \qquad  \adif{t_n} \coloneq t_{n+1}-t_n
\end{equation}
上で定義された近似解列 $\Bab{\vect{y}^n}_{n=0}^N$ によって置き換える.このとき,各時間ステップでどのように $\vect{y}^{n+1}$ を求めるかを定める手続きが時間積分法である.

代表的な 1 ステップ時間積分法として,いわゆる $\theta$ 法が挙げられる.\eqref{Eq:Generic_ODE_System} に対して $\theta$ 法を適用すると,時刻 $t_n$ から $t_{n+1}$ への更新は
\begin{equation}
  \label{Eq:Theta_Method}
  \frac{\vect{y}^{n+1} - \vect{y}^{n}}{\adif{t_n}}
  = (1-\theta)\,\vect{F}\,\pab{\vect{y}^{n},t_n}
  + \theta\,\vect{F}\,\pab{\vect{y}^{n+1},t_{n+1}}
\end{equation}
のように記述される.ここで $\theta=0$ のときは陽解法(前進オイラー法),$\theta = 1/2$ のときはCrank–Nicolson法,$\theta=1$ のときは陰解法(後退オイラー法)となる.土壌凍結問題のように強い非線形性や硬い
% \footnote{偏微分方程式系が硬いとは,その固有値分布が広いことをいう.具体的には最小固有値$\lambda_\min$と最大固有値$\lambda_\max$の比である条件数$\kappa=\lambda_\max/\lambda_\min > 10^4$ 程度のものをさすことが多い}
特性をもつ系では,大きな時間刻みでも数値的安定性を保ちやすい陰解法($\theta \geq 1/2$),とりわけ $\theta=1$ の後退オイラー法が広く用いられる.

一方で,長期の凍結・融解過程を高精度に追跡するには,1 階精度の後退オイラー法だけでは時間離散化誤差が支配的となる場合がある.このため本研究では,陰的 1 ステップ法の枠組みを一般化した多段法である後退微分法(Backward Differentiation Formula, BDF)を採用する.BDF は,現在時刻 $t_{n+1}$ における時間微分を,現在値および過去 $k$ ステップ分の解を用いた線形結合として近似する多段陰解法であり,
\begin{equation}
  \label{Eq:BDF_k_Intro}
  \pdv{\vect{y}}{t}\bigg|_{t_{n+1}}
  \approx
  \frac{1}{\Delta t_n}
  \sum_{i=0}^{k} \alpha_i\,\vect{y}^{\,n+1-i}
\end{equation}
という形で表される.ここで $\alpha_i$ は BDF-$k$ に特有の係数であり,$k=1$ の場合には後退オイラー法に一致する.BDF は硬い方程式系に対して高い数値安定性を有し,$k$ を大きくすることで時間積分の高次精度化も可能である.

本研究の時間積分スキームでは,可変時間ステップおよび可変次数に対応した BDF 法を用いて,熱移動・水分移動の連成方程式系を時間方向に積分する.BDF 係数の具体的な導出方法や,可変ステップ・可変次数アルゴリズムの詳細については,\cref{Sec:BDF} にて述べる.

\subsubsection{可変時間ステップおよび可変次数における後退差分法}
\label{Sec:BDF}

後退差分法(Backward Differentiation Formula, BDF)は,時間積分において,過去の値を用いて現在の値を求める方法である.BDFは,特に支配方程式の系が硬い問題において有効であり,数値的な安定性が高い特徴を持つ.7次以上は零点安定性がなくなるので考える必要はなく,6次までのBDFを考えることとする.
時間微分項 $\pdv{\vect{y}}/{t}$ を近似するため,いくつかの過去の時点 $\pab{\vect{y}^i, t_i}$ を通るラグランジュ補間多項式 $Y\pab{t}$ を構築し,その現在時刻 $t_{n+1}$ での微分 $P'\pab{t_{n+1}}$ を用いる.
一般にLagrange補間多項式は,Lagrange基底関数 $l_i(x)$
\begin{align}
  l_i\pab{x} = \prod_{\substack{j=0\leq j \leq i \\[0.2mm] j \neq i}}\frac{x - x_j}{x_i - x_j}
\end{align}
と補間対象となる変数$y_i$との線形結合として与えられる.
\begin{equation}
  \label{Eq:bdf_lagrange}
  L\pab{x,t}\coloneq\sum_{i=0}^{j} l_i \pab{x} y_i \pab{t}
\end{equation}
計算を簡略化するため,局所的な時間座標 $\tau = t - t_{n+1}$ を導入する.この座標系では,現在時刻は $\tau=0$ となる.ただし,1次のBDFは単純な後退差分であるため,2次以上のBDFについて考える.本節では特に2次について詳述するとともに,6次までのBDFはそれまでと同様に導出できるため,その結果を示すだけとする.本章では$k$-次数のBDFをBDF-$k$と表記する.

\subsubsection{BDF-2における導出}
3つの点 $\pab{\vect{y}^{n+1},t_{n+1}}$,$\pab{\vect{y}^n, t_n}$,$\pab{\vect{y}^{n-1}, t_{n-1}}$ を通る2次のラグランジュ補間多項式 $P\pab{\tau}$ を考える.各点は局所座標で $\pab{\vect{y}^{n+1},\tau_0}, \pab{\vect{y}^n, \tau_1}, \pab{\vect{y}^{n-1}, \tau_2}$ と表せる.
\begin{subequations}
  \label{Eq:bdf2_lagrange_time}
  \begin{align}
    \tau_0 & = t_{n+1} - t_{n+1} = 0                                \\
    \tau_1 & = t_n - t_{n+1} = - \adif{t_n}                         \\
    \tau_2 & = t_{n-1} - t_{n+1} = -( \adif{t_n} +  \adif{t_{n-1}})
  \end{align}
\end{subequations}
ここで$P\pab{\tau}$が$k=2$の時のLagrange補間多項式であるとすると,\eqref{Eq:bdf_lagrange}より
\begin{align}
  P\pab{\tau} = l_0\pab{\tau} \vect{y}^{n+1} + l_1\pab{\tau} \vect{y}^n + l_2\pab{\tau} \vect{y}^{n-1}
\end{align}
ここで,各基底関数は次のように定義される.
\begin{subequations}
  \label{Eq:bdf2_lagrange_basis}
  \begin{align}
    l_0\pab{\tau} = \frac{\tau-\tau_1}{\tau_0-\tau_1} \frac{\tau-\tau_2}{\tau_0-\tau_2} \\[1mm]
    l_1\pab{\tau} = \frac{\tau-\tau_0}{\tau_1-\tau_0} \frac{\tau-\tau_2}{\tau_1-\tau_2} \\[1mm]
    l_2\pab{\tau} = \frac{\tau-\tau_0}{\tau_2-\tau_0} \frac{\tau-\tau_1}{\tau_2-\tau_1}
  \end{align}
\end{subequations}
ここで各基底関数の微分は\eqref{Eq:bdf2_lagrange_basis}より
\begin{subequations}
  \label{Eq:bdf2_lagrange_basis_diff}
  \begin{align}
    l'_0\pab{\tau} & = \frac{\tau-\tau_2}{\pab{\tau_0-\tau_1}\pab{\tau_0-\tau_2}} + \frac{\tau-\tau_1}{\pab{\tau_0-\tau_1}\pab{\tau_0-\tau_2}} = \frac{2\tau - \tau_1 - \tau_2}{\pab{\tau_0-\tau_1}\pab{\tau_0-\tau_2}} \\[1mm]
    l'_1\pab{\tau} & = \frac{\tau-\tau_2}{\pab{\tau_1-\tau_0}\pab{\tau_1-\tau_2}} + \frac{\tau-\tau_0}{\pab{\tau_1-\tau_0}\pab{\tau_1-\tau_2}} = \frac{2\tau - \tau_0 - \tau_2}{\pab{\tau_1-\tau_0}\pab{\tau_1-\tau_2}} \\[1mm]
    l'_2\pab{\tau} & = \frac{\tau-\tau_1}{\pab{\tau_2-\tau_0}\pab{\tau_2-\tau_1}} + \frac{\tau-\tau_0}{\pab{\tau_2-\tau_0}\pab{\tau_2-\tau_1}} = \frac{2\tau - \tau_0 - \tau_1}{\pab{\tau_2-\tau_0}\pab{\tau_2-\tau_1}}
  \end{align}
\end{subequations}
$P\pab{\tau}$ の $\tau=0$ における微分は,
\begin{equation}
  P'\pab{0} = \vect{y}^{n+1}l'_0(0) + \vect{y}^n l'_1(0) + \vect{y}^{n-1}l'_2(0)
\end{equation}
これに\eqref{Eq:bdf2_lagrange_basis_diff}および$\tau_1, \tau_2$ を代入し,時間ステップ比 $\rho_1 = \adif{t_n} / \adif{t_{n-1}}$ を用いて整理すると,
\begin{subequations}
  \begin{align}
    l'_0\pab{0} & = -\frac{\tau_1 + \tau_2}{\tau_1 \tau_2} = \frac{2\adif{t_n} + \adif{t_{n-1}}}{\adif{t_n} \pab{\adif{t_n} + \adif{t_{n-1}}}} = \frac{2\rho_1 + 1}{\rho_1 + 1}\frac{1}{\adif{t_n}} \\[2mm]
    l'_1\pab{0} & = -\frac{\tau_2}{\tau_1 \pab{\tau_1 - \tau_2}} = -\frac{\adif{t_n} + \adif{t_{n-1}}}{\adif{t_n} \adif{t_{n-1}}} = -\pab{\rho_1 + 1}\frac{1}{\adif{t_n}}                           \\[2mm]
    l'_2\pab{0} & = -\frac{\tau_1}{\tau_2 \pab{\tau_2 - \tau_1}} = \frac{\adif{t_n}}{\pab{\adif{t_n} + \adif{t_{n-1}}}\adif{t_{n-1}} } = \frac{\rho_1^2}{\rho_1 + 1}\frac{1}{\adif{t_n}}
  \end{align}
\end{subequations}
よって,時間微分の近似式は,
\begin{equation}
  \label{Eq:bdf2_approx}
  \pdv{\vect{y}}{t} \bigg|_{t_{n+1}} \approx \frac{1}{\adif{t_n}} \bab{\frac{2\rho_1 + 1}{\rho_1 + 1}\vect{y}^{n+1} - \pab{\rho_1 + 1} \vect{y}^n + \frac{\rho_1^2}{\rho_1 + 1} \vect{y}^{n-1}}
\end{equation}

\subsubsection{BDF-$k$における係数導出のための一般化}
\label{Sec:BDF_Generalization}

BDF-$k$においては,$k$個の過去の時点を用いて微分を近似する.
一般に,時間ステップをくくりだすことで,$k$次のBDFは次のように定義される.
\begin{equation}
  \label{Eq:bdf_k_approx}
  \pdv{\vect{y}}{t} \bigg|_{t_{n+1}} \approx \frac{1}{\adif{t_n}} \sum_{i=0}^{k} l'_i\pab{0} \vect{y}^{n+1-i}
\end{equation}
% ここで,$l_i$はBDF-$k$における係数であり,BDF-$k$の係数はCode Snippet\ref{prog:bdf_coeffs_final}を用いて求めることができる.

% % \begin{lstlisting}[caption=Back Differential Formula Coefficients, label=prog:bdf_coeffs_final]
% % import sympy
% % from IPython.display import display, Math

% % def display_bdf_coeffs(k_order):
% %     """
% %     指定された次数のBDF係数を導出し,LaTeX数式としてIPython環境で表示する.
% %     """
% %     print(f'--- BDF-{k_order} の係数 ---')

% %     # 1. LaTeX表示用のシンボルを定義
% %     dt_n_sym = sympy.Symbol(r'\Delta t_n', positive=True)
% %     dts = [dt_n_sym] + [
% %         sympy.Symbol(rf'\Delta t_{{n-{i}}}', positive=True) for i in range(1, k_order)
% %     ]
% %     rhos = [
% %         sympy.Symbol(rf'\rho_{{{i + 1}}}', positive=True) for i in range(k_order - 1)
% %     ]

% %     tau = sympy.Symbol(r'\tau')
% %     n_sym = sympy.Symbol('n')

% %     # 2. 補間点の座標を定義(k_order+1点)
% %     nodes = [0]
% %     for i in range(1, k_order + 1):
% %         nodes.append(-sum(dts[:i]))

% %     # 3. 各係数 l'_j(0) を計算
% %     for j in range(k_order + 1):
% %         num = 1
% %         den = 1
% %         for i in range(k_order + 1):
% %             if i != j:
% %                 num *= tau - nodes[i]
% %                 den *= nodes[j] - nodes[i]
% %         l_j = num / den

% %         l_j_prime = sympy.diff(l_j, tau)
% %         l_j_prime_at_0 = l_j_prime.subs(tau, 0)

% %         result_in_rho = l_j_prime_at_0
% %         if k_order > 1:
% %             subs_rules = {}
% %             rho_prod = 1
% %             for i in range(k_order - 1):
% %                 rho_prod *= rhos[i]
% %                 subs_rules[dts[i + 1]] = dts[0] / rho_prod
% %             result_in_rho = result_in_rho.subs(subs_rules)

% %         exponent_val = (n_sym + 1) - j
% %         if exponent_val == n_sym + 1:
% %             superscript = 'n+1'
% %         elif exponent_val == n_sym:
% %             superscript = 'n'
% %         else:
% %             superscript = f'n-{j - 1}'
% %         term_comment = rf'\quad \text{{(Coefficient for }} T^{{{superscript}}})'

% %         latex_str = sympy.latex(sympy.factor(result_in_rho))
% %         display(Math(rf"L'_{{{j}}}(0) = {latex_str} {term_comment}"))

% % if __name__ == '__main__':
% %     display_bdf_coeffs(6)
% % \end{lstlisting}

\FloatBarrier
\subsection{熱移動支配方程式の離散化}
\label{Sec:ThermalFEM}

熱移動に対する境界条件は以下のとおりである.
境界 $\partial V$ 上における外向き単位法線ベクトルを $\vect{n}$ としたとき,境界上の熱収支は全熱エネルギーフラックスの法線成分 $\vect{j}_\mathrm{E} \cdot \vect{n}$ を用いて記述される.

\begin{ConditionBox}{温度境界条件}{Thermal_BC}
  領域 $V$ における温度場 $T(\vect{x},t)$ は,境界 $\partial V$ 上において,
  以下に示す代表的な温度境界条件を満たすものとする.
  ここで,
  $h$ は熱伝達係数,
  $\varepsilon$ は放射率,
  $\sigma$ はステファン=ボルツマン定数,
  $F$ は形態係数を表す.
  さらに,
  $\alpha_\mathrm{HR}$ は熱放射項を線形化した等価熱伝達係数である.

  \begin{subequations}
    \label{Eq:Thermal-BC}
    \begin{description}
      \item[(a) 温度既定境界 (Dirichlet 境界条件)]
            境界 $\partial V_\mathrm{HD}$ 上において温度が既定される場合,
            \begin{equation}
              T\pab{\vect{x}, t} = T_\mathrm{D}\pab{\vect{x}}
              \qquad \text{for } \vect{x}\in\partial V_\mathrm{HD}
            \end{equation}

      \item[(b) 熱流束既定境界 (Neumann 境界条件)]
            境界 $\partial V_\mathrm{HN}$ 上において,全熱エネルギーフラックスの法線成分が既定される場合,
            \begin{equation}
              \vect{j}_\mathrm{E}\pab{\vect{x}, t} \cdot \vect{n}
              = q_\mathrm{TN}\pab{\vect{x}, t}
              \qquad \text{for } \vect{x}\in\partial V_\mathrm{HN}
            \end{equation}
            ここで $q_\mathrm{TN}$ は境界を通して流入・流出する正味の熱流束(既知量)である.

      \item[(c) Robin 境界条件]
            境界 $\partial V_\mathrm{HC}$ において,熱流束が温度の線形関数として与えられる場合,
            当該境界において水分(液状水・水蒸気)の出入りはないものと仮定すると,境界上のエネルギー収支は以下のように記述される.
            \begin{equation}
              \vect{j}_\mathrm{E}\pab{\vect{x}, t} \cdot \vect{n}
              = \beta\pab{\vect{x}}\,T\pab{\vect{x}, t} + \gamma\pab{\vect{x}}
              \qquad \text{for } \vect{x}\in\partial V_\mathrm{HC}
            \end{equation}
            一方,当該領域において,水分の出入りがある場合には,境界上のエネルギー収支は以下のように記述される.
            \begin{align}
               & \vect{j}_\mathrm{E}\pab{\vect{x}, t} \cdot \vect{n} \notag                                                      \\
               & = \quad \beta\pab{\vect{x}}\,T\pab{\vect{x}, t} + \gamma\pab{\vect{x}}
              + c_\mathrm{w} \rho_\mathrm{w} \vect{j}_\mathrm{WL}\pab{\vect{x}, t} \cdot \vect{n}\, T\pab{\vect{x}, t} \notag    \\
               & \qquad + c_\mathrm{v} \rho_\mathrm{w} \vect{j}_\mathrm{WV}\pab{\vect{x}, t} \cdot \vect{n}\, T\pab{\vect{x}, t}
              + \rho_\mathrm{w} L_\mathrm{v} \vect{j}_\mathrm{WV}\pab{\vect{x}, t} \cdot \vect{n}
              \qquad \text{for } \vect{x}\in\partial V_\mathrm{HC}
            \end{align}

      \item[(d) 熱伝達境界]
            境界 $\partial V_\mathrm{HH}$ において周囲環境との熱伝達(対流熱伝達)を考慮する場合,
            当該境界において水分の出入りはない(不透水境界)と仮定する.このとき,全熱エネルギーフラックスは熱伝達量と釣り合うため,以下のように記述される.
            \begin{equation}
              \vect{j}_\mathrm{E}\pab{\vect{x}, t} \cdot \vect{n}
              = h(\vect{x})
              \bab{T\pab{\vect{x}, t}-T_\mathrm{env}\pab{\vect{x}, t}}
              \qquad \text{for } \vect{x}\in\partial V_\mathrm{HH}
            \end{equation}

      \item[(e) 熱放射境界]
            境界 $\partial V_\mathrm{HR}$ において熱放射によるエネルギー交換を考慮する場合,
            同様に水分の出入りはないものと仮定すると,以下の条件が成立する.
            \begin{align}
              \vect{j}_\mathrm{E}\pab{\vect{x}, t} \cdot \vect{n}
               & = \varepsilon\sigma F
              \bab{T\pab{\vect{x}, t}^4 - T_\mathrm{r}\pab{\vect{x}, t}^4} \notag \\
               & \simeq \alpha_\mathrm{HR}(\vect{x})
              \bab{T\pab{\vect{x}, t}-T_\mathrm{r}\pab{\vect{x}, t}}
              \qquad \text{for } \vect{x}\in\partial V_\mathrm{HR}
            \end{align}
    \end{description}
  \end{subequations}
\end{ConditionBox}

支配方程式\ref{Eq:Energy_Continuity_Differential}を有限要素法で離散化することを考える.$V$を$\mathbb{R}^3$で有界な領域,重み関数を$\chi$とすれば弱形式は
\begin{equation}
  \label{Eq:Thermal-weak-form}
  \iiint_{V} \bab{\pdv{\mathcal{U}}{t} + \nabla\cdot \vect{j}_\mathrm{E} + S_\mathrm{T}}\chi \odif{V} = 0
\end{equation}
ただし$\chi$は任意な関数であるが,以下の条件を満たしている必要がある.
\begin{ConditionBox}{重み関数$\chi$の条件}{Weight_Function_Condition}
  \begin{itemize}
    \item 重み関数$\chi$は,有限要素の節点で連続である.
    \item 重み関数$\chi$は,Dirichlet境界条件において,値がゼロである.
    \item 重み関数$\chi$は,Neumann境界条件において,値が1で微分値がゼロである.
    \item 重み関数$\chi$は無次元である.
    \item 重み関数$\chi$は,互いに独立関数でなくてはならない.
  \end{itemize}
\end{ConditionBox}
まず,ベクトル解析の恒等式 (\cref{Form:Identity-Div-Product})を用いると,支配方程式の体積積分項はガウスの発散定理 (\ref{Form:Gauss-Divergence-Theorem-3D})より次のように変形できる.
\begin{align}
  \iiint_{V} \chi \nabla \cdot \vect{j}_\mathrm{E} \odif{V}
   & = \iiint_{V} \bab{\nabla \cdot \pab{\chi\vect{j}_\mathrm{E}} - \nabla\chi \cdot \vect{j}_\mathrm{E}} \odif{V} \notag              \\
  \label{Eq:Thermal-Weak-Form-Divergence}
   & = \oiint_{\partial V} \chi \vect{j}_\mathrm{E} \cdot \vect{n} \odif{S} - \iiint_{V} \nabla\chi \cdot \vect{j}_\mathrm{E} \odif{V}
\end{align}
ここで,右辺第一項の面積分への変換にガウスの発散定理を用いた.$\vect{n}$は境界$\partial V$上の外向き単位法線ベクトルである.
これより,\eqref{Eq:Thermal-weak-form}は次のように書き換えられる.
\begin{equation}
  \iiint_{V} \bab{\chi\pdv{\mathcal{U}}{t} - \nabla\chi \cdot \vect{j}_\mathrm{E} + \chi S_\mathrm{T}} \odif{V} + \oiint_{\partial V} \chi \vect{j}_\mathrm{E} \cdot \vect{n} \odif{S} = 0
\end{equation}
次に,境界積分項を評価する.全境界$\partial V$は,適用される境界条件の種類に応じて以下のように分割される.
\begin{equation}
  \partial V = \partial V_\mathrm{HD} \cup \partial V_\mathrm{HN} \cup \partial V_\mathrm{HC} \cup \partial V_\mathrm{HH} \cup \partial V_\mathrm{HR}
\end{equation}
ここで,\cref{Condition:Weight_Function_Condition}の重み関数の条件より,Dirichlet境界($\partial V_\mathrm{HD}$)上では$\chi=0$となるため,当該境界上での積分は消失する.
その他の境界については,式\eqref{Eq:Thermal-BC}で与えられる各条件を代入する.
各境界において,全熱エネルギーフラックスの法線成分 $\vect{j}_\mathrm{E} \cdot \vect{n}$ は,既定の熱流束 $q_\mathrm{N}$ あるいは温度に依存する関数($Q_\mathrm{HC}, Q_\mathrm{HH}, Q_\mathrm{HR}$)と釣り合うため,最終的な弱形式は以下のようになる.
\begin{align}
  \label{Eq:Thermal-weak-form-final}
   & \iiint_{V} \bab{\chi\pdv{\mathcal{U}}{t} - \nabla\chi \cdot \vect{j}_\mathrm{E} + \chi S_\mathrm{T}} \odif{V} \notag \\
   & \quad + \iint_{\partial V_\mathrm{HN}} \chi q_\mathrm{N} \odif{S}
  + \iint_{\partial V_\mathrm{HC}} \chi Q_\mathrm{HC} \odif{S} \notag                                                     \\
   & \quad + \iint_{\partial V_\mathrm{HH}} \chi Q_\mathrm{HH} \odif{S}
  + \iint_{\partial V_\mathrm{HR}} \chi Q_\mathrm{HR} \odif{S} = 0
\end{align}
ここで,$Q_\mathrm{HC}, Q_\mathrm{HH}, Q_\mathrm{HR}$ はそれぞれ式\eqref{Eq:Thermal-BC}におけるRobin 境界,熱伝達境界,熱放射境界の右辺項を表す.

\FloatBarrier
\subsection{水分移動支配方程式の離散化}
\label{Sec:HydraulicFEM}

水分移動に対する境界条件は,全水分質量フラックス $\vect{J}_\mathrm{m} = \rho_\mathrm{w} \pab{\vect{j}_\mathrm{WL} + \vect{j}_\mathrm{WV}}$(液状水および水蒸気の和)を用いて以下のように記述される.
\begin{ConditionBox}{圧力境界条件}{Hydraulic_BC}
  領域 $V$ における間隙水圧場 $P\,\pab{\vect{x},t}$ は,境界 $\partial V$ 上において,以下に示す代表的な圧力境界条件を満たすものとする.
  \begin{subequations}
    \label{Eq:water-BC}
    \begin{description}
      \item[(a) 圧力既定境界 (Dirichlet 境界条件)]
            境界 $\Gamma_\mathrm{HD}$ 上において間隙水圧が既定される場合,
            \begin{equation}
              P\,\pab{\vect{x}, t}
              = P_{\mathrm{D}}\pab{\vect{x}}
              \qquad \text{for } \vect{x}\in\Gamma_\mathrm{HD}
            \end{equation}
      \item[(b) 水分フラックス既定境界 (Neumann 境界条件)]
            境界 $\Gamma_\mathrm{HN}$ 上において,全水分質量フラックス $\vect{J}_\mathrm{m}$ の法線成分が既定される場合,
            \begin{equation}
              \vect{J}_\mathrm{m}\pab{\vect{x}, t} \cdot \vect{n}
              = q_{\mathrm{HN}}\pab{\vect{x},t}
              \qquad \text{for } \vect{x}\in\Gamma_\mathrm{HN}
            \end{equation}
            ここで,$q_{\mathrm{HN}}$ は境界を通して流出する正味の水分フラックスである.
      \item[(c) 大気境界条件]
            地表面境界 $\Gamma_\mathrm{ATM}$ では,気象条件と土壌の浸透・保水性に応じ,境界条件型を動的に切り替える系依存型境界条件を適用する.
            鉛直上向き正の $z$ 軸に対し,可能蒸発散量 $E_\mathrm{potential}$ および降水量 $q_\mathrm{rain}$ から,正味の可能水分フラックス $q_{\mathrm{potential}}$ を次式で定義する.
            \begin{equation}
              q_{\mathrm{potential}}\pab{\vect{x},t} = E_\mathrm{potential}\pab{\vect{x},t} - q_\mathrm{rain}\pab{\vect{x},t}
            \end{equation}
            ここでは外向き法線ベクトル $\vect{n}$ が $+z$ 方向であるため,正の値は蒸発として系外への流出し,負の値は降雨に伴う系内への流入を表す.
            実際の境界条件は,地表面の許容圧力範囲(乾燥限界 $P_{\min}$ および湛水限界 $P_{\max}$)に基づき,以下の制約を満たすように決定される.
            \begin{equation}
              \begin{cases}
                \abs{\vect{J}_\mathrm{m}\pab{\vect{x}, t} \cdot \vect{n}} \le \abs{q_{\mathrm{potential}}} \\[3pt]
                P_{\min} \le P\pab{\vect{x},t} \le P_{\max}
              \end{cases}
              \qquad \text{for } \vect{x}\in\Gamma_\mathrm{ATM}
            \end{equation}
            ここで,$P_{\max}$が$0$のときは,地表面に湛水せず,即座に流出する条件となる.一方,$P_{\max}>0$のときは,地表面にある一定の湛水深が許容される.
            この条件は物理的に以下の2つの状況を包含している.
            \begin{description}
              \item[1. 降雨・浸透過程]:
                    土壌の浸透能が降雨強度を上回る間,つまり $q_{\mathrm{potential}}<0$ のときは,降雨量すべてが流入するフラックス境界となる.浸透能を超過し地表面が飽和すると,圧力境界 $P = P_{\max}$ に切り替わり,差分の水量は表面流出(Surface runoff)として扱われる.
              \item[2. 蒸発・乾燥過程]:
                    土壌水分が十分に存在するとき,つまり $q_{\mathrm{potential}}>0$ のときは,可能蒸発散量が要求する値でのフラックス境界となる.土壌が乾燥し限界圧力 $P_{\min}$ に達すると,圧力境界 $P = P_{\min}$ に固定され,実際の蒸発量は土壌の水分供給能力によって制限される.
            \end{description}
    \end{description}
  \end{subequations}
\end{ConditionBox}

水分移動に関する支配方程式\eqref{Eq:Continuity_void}を有限要素法で離散化する.
形状関数ベクトルを $\vect{\psi}_\mathrm{H}$ とすれば,Galerkin法に基づく弱形式は以下のように記述される.
\begin{equation}
  \label{Eq:Hydraulic_weak_form}
  \iiint_{V} \vect{\psi}_\mathrm{H} \bab{\pdv{\rho_\mathrm{void}}{t} + \nabla\cdot \vect{J}_\mathrm{m} + S_\mathrm{H}} \odif{V} = \vect{0}
\end{equation}
\eqref{Eq:Thermal-Weak-Form-Divergence}と同様にガウスの発散定理を用いて空間微分項を変形し,境界条件を代入すると,弱形式は次のように書き換えられる.
\begin{align}
  \label{Eq:Hydraulic_weak_form_final}
   & \iiint_{V} \bab{ \vect{\psi}_\mathrm{H} \pdv{\rho_\mathrm{void}}{t} - \nabla\vect{\psi}_\mathrm{H}^\mathsf{T} \cdot \vect{J}_\mathrm{m} + \vect{\psi}_\mathrm{H} S_\mathrm{H} } \odif{V} \notag \\
   & \quad + \iint_{\Gamma_\mathrm{HN}} \vect{\psi}_\mathrm{H} q_{\mathrm{HN}} \odif{S} = 0
\end{align}

% 時間微分項 $\pdv{\rho_\mathrm{void}}/{t}$ に対して $k$ 次の BDF を適用し,時刻 $t_{n+1}$ における残差ベクトル $\vect{R}_\mathrm{H}$ を定義する.
% 残差ベクトルは,外力ベクトル,内力ベクトル,および貯留・慣性ベクトルの和として整理される.
% \begin{align}
%   \label{Eq:Residual-Water}
%   \vect{R}_\mathrm{H}\pab{\vect{T}_{n+1}^{m}, \vect{P}_{n+1}^{m}}
%    & = \vect{F}^\mathrm{H}_\text{external} - \vect{F}^\mathrm{H}_\text{internal} - \vect{F}^\mathrm{H}_\text{transient} = \vect{0}
% \end{align}

% \begin{enumerate}
%   \item 外力ベクトル $\vect{F}^\mathrm{H}_\text{external}$:
%         境界条件によって領域表面から流入する水分フラックスの寄与である.
%         \begin{equation}
%           \vect{F}^\mathrm{H}_\text{external} = - \iint_{\Gamma_\mathrm{HN}} j_{\mathrm{w},\mathrm{N}} \vect{\psi}_\mathrm{H} \odif{S}
%         \end{equation}
%         ここで,流出フラックス $j_{\mathrm{w},\mathrm{N}}$ に対して流入方向を正とするためマイナス符号が付く.

%   \item 内力ベクトル $\vect{F}^\mathrm{H}_\text{internal}$:
%         領域内部の水分移動および内部シンク項である.
%         全水分質量フラックス $\vect{J}_\mathrm{m}$ の各成分(液状水・水蒸気)を展開して記述すると以下のようになる.
%         \begin{align}
%           \vect{F}^\mathrm{H}_\text{internal}
%            & = - \iiint_{V} \nabla\vect{\psi}_\mathrm{H}^\mathsf{T} \cdot \vect{J}_\mathrm{m} \odif{V}
%           + \iiint_{V} \vect{\psi}_\mathrm{H} S_\mathrm{H} \odif{V} \notag                                                                                   \\
%            & = - \iiint_{V} \nabla\vect{\psi}_\mathrm{H}^\mathsf{T} \cdot \rho_\mathrm{w} \bigl( \vect{j}_\mathrm{WL} + \vect{j}_\mathrm{WV} \bigr) \odif{V}
%           + \iiint_{V} \vect{\psi}_\mathrm{H} S_\mathrm{H} \odif{V}
%         \end{align}
%         ここで,$\vect{j}_\mathrm{WL}$ および $\vect{j}_\mathrm{WV}$ は,それぞれ $P$ および $T$ の勾配に依存する(式\eqref{Eq:LiquidFlux_final3}, \eqref{Eq:VaporFlux_final}参照).

%   \item 貯留・慣性ベクトル $\vect{F}^\mathrm{H}_\text{transient}$:
%         間隙内の水分貯留量の時間変化項である.
%         \begin{align}
%           \vect{F}^\mathrm{H}_\text{transient}
%            & = \iiint_{V} \vect{\psi}_\mathrm{H} \pab{ \alpha_0 \rho_\mathrm{void}^{n+1} + \rho_\mathrm{void}^\mathrm{hist} } \odif{V}
%         \end{align}
%         ここで,$\rho_\mathrm{void} = \rho_\mathrm{w} \phi S_\mathrm{w} + \rho_\mathrm{v} \phi (1-S_\mathrm{w})$ は間隙中の総水分密度である.
% \end{enumerate}

% Newton-Raphson法による解法のため,残差ベクトル $\vect{R}_\mathrm{H}$ を未知変数 $\vect{P}$ および $\vect{T}$ で偏微分し,ヤコビアン(接線剛性行列)を導出する.
% \begin{equation}
%   \mathbf{J}_\mathrm{hydraulic} =
%   \begin{bmatrix}
%     \mathbf{J}_{HP} & \mathbf{J}_{HT}
%   \end{bmatrix}
%   =
%   \begin{bmatrix}
%     \pdv{\vect{R}_\mathrm{H}}{\vect{P}} & \pdv{\vect{R}_\mathrm{H}}{\vect{T}}
%   \end{bmatrix}
% \end{equation}

% \subsubsection{水-水ブロック ($\mat{J}_\mathrm{HH}$)}
% 間隙水圧の変化に対する水分収支の応答を表す項である.
% 水分容量(貯留項の微分)および透水・通気特性(フラックス項の微分)が含まれる.
% \begin{align}
%   \mat{J}_\mathrm{HH} = \pdv{\vect{R}_\mathrm{H}}{\vect{P}}
%    & = \iiint_{V} \alpha_0 C_{HP} \vect{\psi}_\mathrm{H} \vect{\psi}_\mathrm{H}^\mathsf{T} \odif{V} \notag
%    & \quad + \iiint_{V} \nabla\vect{\psi}_\mathrm{H}^\mathsf{T} \bigl( \rho_\mathrm{w} K_\mathrm{wP} + \rho_\mathrm{w} K_\mathrm{vP} \bigr) \nabla\vect{\psi}_\mathrm{H} \odif{V}
% \end{align}
% ここで,$C_{HP} = \pdv*{\rho_\mathrm{void}}{P}$ は有効水分容量,$K_\mathrm{wP}, K_\mathrm{vP}$ はそれぞれ液状水および水蒸気の水圧勾配に関する輸送係数である.
% なお,フラックス項の微分において $\vect{J}_\mathrm{m}$ が $-\nabla P$ に比例するため,微分後の符号は正($+$)となる.

% \subsubsection{水-熱ブロック ($\mat{J}_\mathrm{HT}$)}
% 温度変化が水分移動に与える影響を表す連成項である.
% 温度変化による密度変化や飽和度変化(容量項),および温度勾配による水分移動(Soret効果など)が含まれる.
% \begin{align}
%   \mat{J}_\mathrm{HT} = \pdv{\vect{R}_\mathrm{H}}{\vect{T}}
%    & = \iiint_{V} \alpha_0 C_{HT} \vect{\psi}_\mathrm{H} \vect{\psi}_\mathrm{T}^\mathsf{T} \odif{V} \notag                                                                               \\
%    & \quad + \iiint_{V} \nabla\vect{\psi}_\mathrm{H}^\mathsf{T} \bigl( \rho_\mathrm{w} K_\mathrm{wT} + \rho_\mathrm{w} K_\mathrm{vT} \bigr) \nabla\vect{\psi}_\mathrm{T} \odif{V} \notag \\
%    & \quad + \iiint_{V} \nabla\vect{\psi}_\mathrm{H}^\mathsf{T} \vect{V}_{HT} \vect{\psi}_\mathrm{T}^\mathsf{T} \odif{V}
% \end{align}
% ここで,$C_{HT} = \pdv*{\rho_\mathrm{void}}{T}$ は温度変化に伴う水分貯留量の変化率である.
% $K_\mathrm{wT}, K_\mathrm{vT}$ は温度勾配による水分輸送係数である.
% また,$\vect{V}_{HT}$ は物性値(水密度や透水係数など)の温度依存性に起因する補正項(移流的な寄与)であり,以下のように定義される.
% \begin{equation}
%   \vect{V}_{HT} = \pdv{\vect{J}_\mathrm{m}}{T} \bigg|_{\nabla P, \nabla T \text{ const}}
% \end{equation}
\subsection{非線形方程式の線形化}
\label{Sec:LinearizedSystem}

\cref{Sec:ThermalFEM,Sec:HydraulicFEM}で導出した熱移動・水分移動の有限要素離散化方程式は,非線形連立方程式系となる.これらを数値的に解くため,Newton-Raphson法による線形化を行う.

\subsubsection{Newton-Raphson法による定式化}
熱移動および水分移動の支配方程式に対し,残差ベクトル$\vect{R}_\mathrm{T}$,$\vect{R}_\mathrm{H}$を次のように定義する.
\begin{subequations}
  \begin{align}
    \vect{R}_\mathrm{T}\pab{\vect{T}, \vect{P}\,} & = \vect{0} \\[2mm]
    \vect{R}_\mathrm{H}\pab{\vect{T}, \vect{P}\,} & = \vect{0}
  \end{align}
\end{subequations}
ここで,$\vect{T}$および$\vect{P}$はそれぞれ節点温度および節点間隙水圧ベクトルである.
第$m$回目の反復ステップにおいて,残差ベクトルをTaylor展開し,2次の項を無視して線形化すると次式が得られる.
\begin{align}
  \vect{R}_\mathrm{T}\pab{\vect{T}_{n+1}^{m+1}, \vect{P}_{n+1}^{m+1}}
   & \approx \vect{R}_\mathrm{T}\pab{\vect{T}_{n+1}^{m}, \vect{P}_{n+1}^{m}}
  + \pdv{\vect{R}_\mathrm{T}}{\vect{T}} \bigg|_{m} \adif{\vect{T}_{n+1}^{m+1}}
  + \pdv{\vect{R}_\mathrm{T}}{\vect{P}} \bigg|_{m} \adif{\vect{P}_{n+1}^{m+1}} \notag \\
   & = \vect{0}
\end{align}
ここで,$\adif{\vect{T}_{n+1}^{m+1}} = \vect{T}_{n+1}^{m+1} - \vect{T}_{n+1}^{m}$ および $\adif{\vect{P}_{n+1}^{m+1}} = \vect{P}_{n+1}^{m+1} - \vect{P}_{n+1}^{m}$ は更新量を表す.
水分移動方程式についても同様に展開し,これを行列形式で整理すると以下の更新式が得られる.
\begin{equation}
  \begin{bmatrix}
    \mat{J}_\mathrm{TT} & \mat{J}_\mathrm{TP} \\
    \mat{J}_\mathrm{HT} & \mat{J}_\mathrm{HH}
  \end{bmatrix}_{n+1}^{m}
  \begin{Bmatrix}
    \adif{\vect{T}} \\
    \adif{\vect{P}}
  \end{Bmatrix}_{n+1}^{m+1}
  = -
  \begin{Bmatrix}
    \vect{R}_\mathrm{T} \\
    \vect{R}_\mathrm{H}
  \end{Bmatrix}_{n+1}^{m}
  \label{Eq:NR_System}
\end{equation}
ここで,ヤコビアン(接線剛性行列)の各ブロックは次のように定義される.
\begin{subequations}
  \begin{align}
    \label{Eq:Jacobian_Thermal_Blocks}
    \mat{J}_\mathrm{TT} & = \pdv{\vect{R}_\mathrm{T}}{\vect{T}} , \quad
    \mat{J}_\mathrm{TP} = \pdv{\vect{R}_\mathrm{T}}{\vect{P}}           \\[2mm]
    \label{Eq:Jacobian_Hydraulic_Blocks}
    \mat{J}_\mathrm{HT} & = \pdv{\vect{R}_\mathrm{H}}{\vect{T}} , \quad
    \mat{J}_\mathrm{HH} = \pdv{\vect{R}_\mathrm{H}}{\vect{P}}
  \end{align}
\end{subequations}
各反復ステップにおいて式\eqref{Eq:NR_System}を解き,残差ベクトルのノルムが許容値以下になるまで解を更新する.

\subsubsection{ヤコビアンおよび残差ベクトルの詳細}
各項の具体的な導出について述べる.
変数 $X$ の時刻 $t_{n+1}$ における時間微分項は,時間ステップ情報を含んだBDF係数 $\beta_k$ \unit{[\second^{-1}]} を用いて以下のように近似される.
\begin{equation}
  \eval{\pdv{X}{t}}{n+1} \approx \beta_0 X_{n+1} + X_\mathrm{hist}, \quad X_\mathrm{hist} = \sum_{k=1}^{k} \beta_k X_{n+1-k}
\end{equation}
ここで,$X_\mathrm{hist}$ は過去の時刻の既知量から定まる履歴項である.

残差ベクトルは,内力項,貯留・慣性項,および外力項の和として構成される.ここで,内力項に含まれるフラックス項は,発散定理(Green-Gaussの定理)により符号が反転することに注意する.
\begin{align}
  \vect{R}_\mathrm{T}\pab{\vect{T}_{n+1}^{m}, \vect{P}_{n+1}^{m}}
   & = \vect{F}^\mathrm{T}_\text{internal} + \vect{F}^\mathrm{T}_\text{transient} - \vect{F}^\mathrm{T}_\text{external} \notag                                                                                                                               \\
   & = \iiint_{V} \bab{-\nabla\vect{\psi}_\mathrm{T}^\mathsf{T} \cdot \vect{j}_\mathrm{E} + \vect{\psi}_\mathrm{T}^\mathsf{T} \pab{\beta_0 \mathcal{U}_{n+1} + \mathcal{U}_\mathrm{hist}} - \vect{\psi}_\mathrm{T}^\mathsf{T} S_\mathrm{T} } \odif{V} \notag \\
   & \quad - \iint_{\partial V} \vect{\psi}_\mathrm{T}^\mathsf{T} \vect{Q}_\mathrm{b} \odif{S}
  \label{Eq:Residual-Thermal}
\end{align}
\begin{align}
  \label{Eq:Residual-Water}
  \vect{R}_\mathrm{H}\pab{\vect{T}_{n+1}^{m}, \vect{P}_{n+1}^{m}}
   & = \vect{F}^\mathrm{H}_\text{internal} + \vect{F}^\mathrm{H}_\text{transient} - \vect{F}^\mathrm{H}_\text{external} \notag                                                                                                              \\
   & = \iiint_{V} \bab{-\nabla\vect{\psi}_\mathrm{H}^\mathsf{T} \cdot \vect{J}_\mathrm{m} + \vect{\psi}_\mathrm{H} \pab{\beta_0 \rho_\mathrm{void, n+1} + \rho_\mathrm{void, hist}} - \vect{\psi}_\mathrm{H} S_\mathrm{H} } \odif{V} \notag \\
   & \quad - \iint_{\Gamma_\mathrm{HN}} \vect{\psi}_\mathrm{H} q_{\mathrm{HN}} \odif{S}
\end{align}
ここで,各項の構成は以下の通りである.

\begin{description}
  \item [熱外力ベクトル:] $\vect{F}^\mathrm{T}_\text{external}$ \\
        境界からの既知の熱流入および定数的な熱源項である.
        \begin{align}
          \vect{F}^\mathrm{T}_\text{external}
          % Neumann (influx is -q_N if q_N is outward)
           & = - \iint_{\partial V_\mathrm{HN}} q_\mathrm{N} \vect{\psi}_\mathrm{T} \odif{S}
          % Robin constant part
          + \iint_{\partial V_\mathrm{HC}} \gamma \vect{\psi}_\mathrm{T} \odif{S} \notag              \\
          % Heat Transfer constant part
           & \quad + \iint_{\partial V_\mathrm{HH}} h  T_\mathrm{env} \vect{\psi}_\mathrm{T} \odif{S}
          % Radiation constant part
          + \iint_{\partial V_\mathrm{HR}} \beta_\mathrm{HR} T_\mathrm{r} \vect{\psi}_\mathrm{T} \odif{S}
          \label{Eq:F_external_expansion}
        \end{align}

  \item [熱内力ベクトル:] $\vect{F}^\mathrm{T}_\text{internal}$ \\
        熱伝導,移流,および温度依存の境界条件に由来する項である.$\vect{j}_\mathrm{E}$ に含まれる熱伝導項($-\tensor{R}\nabla T$)は,式\eqref{Eq:Residual-Thermal}の$-\nabla\vect{\psi}_\mathrm{T}^\mathsf{T} \cdot \vect{j}_\mathrm{E}$により正の剛性行列となる.
        \begin{align}
          \vect{F}^\mathrm{T}_\text{internal}
           & = \iiint_{V} \nabla\vect{\psi}_\mathrm{T}^\mathsf{T} \tensor{R} \nabla\vect{\psi}_\mathrm{T} \vect{T} \odif{V}
          - \iiint_{V} \nabla\vect{\psi}_\mathrm{T}^\mathsf{T} \pab{c_\mathrm{w} \rho_\mathrm{w} \vect{j}_\mathrm{WL} + c_\mathrm{v}\rho_\mathrm{w} \vect{j}_\mathrm{WV}} \vect{\psi}_\mathrm{T}^\mathsf{T} \vect{T} \odif{V} \notag \\
           & \quad - \iiint_{V} \nabla\vect{\psi}_\mathrm{T}^\mathsf{T} \pab{\rho_\mathrm{w} L_\mathrm{v} \vect{j}_\mathrm{WV}} \odif{V}
          - \iiint_{V} \vect{\psi}_\mathrm{T} S_\mathrm{T} \odif{V}\notag                                                                                                                                                            \\
           & \quad + \iint_{\partial V_\mathrm{HC}} \beta \vect{\psi}_\mathrm{T} \vect{\psi}_\mathrm{T}^\mathsf{T} \vect{T} \odif{S}
          + \iint_{\partial V_\mathrm{HH}} h \vect{\psi}_\mathrm{T} \vect{\psi}_\mathrm{T}^\mathsf{T} \vect{T} \odif{S}
          + \iint_{\partial V_\mathrm{HR}} \beta_\mathrm{HR} \vect{\psi}_\mathrm{T} \vect{\psi}_\mathrm{T}^\mathsf{T} \vect{T} \odif{S} \notag
        \end{align}

  \item [熱貯留・慣性ベクトル:] $\vect{F}^\mathrm{T}_\text{transient}$ \\
        内部エネルギーの時間変化項であり,BDF近似により次のように表される.
        \begin{equation}
          \vect{F}^\mathrm{T}_\text{transient}
          = \beta_0 \iiint_{V} \vect{\psi}_\mathrm{T}^\mathsf{T} \mathcal{U}\pab{\vect{T}_{n+1}, \vect{P}_{n+1}} \odif{V}
          + \iiint_{V} \vect{\psi}_\mathrm{T}^\mathsf{T} \mathcal{U}_\mathrm{hist} \odif{V}
        \end{equation}

  \item [水分外力ベクトル:] $\vect{F}^\mathrm{H}_\text{external}$ \\
        境界からの既知の水分流入および定数的な流出項である.
        \begin{equation}
          \vect{F}^\mathrm{H}_\text{external}
          = - \iint_{\Gamma_\mathrm{HN}} q_{\mathrm{HN}} \vect{\psi}_\mathrm{H} \odif{S}
          \label{Eq:F_H_external_expansion}
        \end{equation}

  \item [水分内力ベクトル:] $\vect{F}^\mathrm{H}_\text{internal}$ \\
        流体移動および圧力依存の境界条件に由来する項である.
        \begin{equation}
          \vect{F}^\mathrm{H}_\text{internal}
          = -\iiint_{V} \nabla\vect{\psi}_\mathrm{H}^\mathsf{T} \cdot \vect{J}_\mathrm{m} \odif{V}
          - \iiint_{V} \vect{\psi}_\mathrm{H} S_\mathrm{H} \odif{V}
        \end{equation}
  \item [水分貯留・慣性ベクトル:] $\vect{F}^\mathrm{H}_\text{transient}$ \\
        孔隙水量の時間変化項であり,BDF近似により次のように表される.
        \begin{equation}
          \vect{F}^\mathrm{H}_\text{transient}
          = \beta_0 \iiint_{V} \vect{\psi}_\mathrm{H} \rho_\mathrm{void}\pab{\vect{T}_{n+1}, \vect{P}_{n+1}} \odif{V}
          + \iiint_{V} \vect{\psi}_\mathrm{H} \rho_\mathrm{void, \mathrm{hist}} \odif{V}
        \end{equation}
\end{description}
\eqref{Eq:Jacobian_Thermal_Blocks}の各ヤコビアンブロックは,残差ベクトル\eqref{Eq:Residual-Thermal}の各項を変数$\vect{T}$および$\vect{P}$で偏微分することで得られる.
以下に,具体的な式を示す.
\begin{description}
  \item [熱-熱ブロック:] $\mat{J}_\mathrm{TT}$ \\
        温度変化に対する熱収支の応答を表す.式\eqref{Eq:J_TT_def}の各項は順に,熱容量,熱伝導,顕熱移流,および水蒸気移動に伴う潜熱輸送の寄与を示す.
        \begin{equation}
          \begin{split}
            \mat{J}_\mathrm{TT}
             & = \pdv{\vect{F}^\mathrm{T}_\text{internal}}{\vect{T}} + \pdv{\vect{F}^\mathrm{T}_\text{transient}}{\vect{T}} \notag                                                                                              \\
             & = \beta_0 \iiint_{V} C_\mathrm{TT} \vect{\psi}_\mathrm{T} \vect{\psi}_\mathrm{T}^\mathsf{T} \odif{V}
            + \iiint_{V} \nabla\vect{\psi}_\mathrm{T}^\mathsf{T} \tensor{R} \nabla\vect{\psi}_\mathrm{T} \odif{V}                                                                                                               \\
             & \quad - \iiint_{V} \nabla\vect{\psi}_\mathrm{T}^\mathsf{T} \pab{c_\mathrm{w} \rho_\mathrm{w} \vect{j}_\mathrm{WL} + c_\mathrm{v}\rho_\mathrm{w} \vect{j}_\mathrm{WV}} \vect{\psi}_\mathrm{T}^\mathsf{T} \odif{V}
            - \iiint_{V} \nabla\vect{\psi}_\mathrm{T}^\mathsf{T} \pab{\rho_\mathrm{w} L_\mathrm{v} K_\mathrm{vT}} \nabla\vect{\psi}_\mathrm{T} \odif{V}                                                                         \\
             & \quad + \iint_{\partial V_\mathrm{HC}} \beta \vect{\psi}_\mathrm{T} \vect{\psi}_\mathrm{T}^\mathsf{T} \odif{S}
            + \iint_{\partial V_\mathrm{HH}} h \vect{\psi}_\mathrm{T} \vect{\psi}_\mathrm{T}^\mathsf{T} \odif{S}
            + \iint_{\partial V_\mathrm{HR}} \beta_\mathrm{HR} \vect{\psi}_\mathrm{T} \vect{\psi}_\mathrm{T}^\mathsf{T} \odif{S}
          \end{split}
          \label{Eq:J_TT_def}
        \end{equation}
  \item [熱-水ブロック:] $\mat{J}_\mathrm{TH}$ \\
        間隙水圧の変化が熱収支に与える影響を表す.第1項は相変化等に伴う内部エネルギー変化,第2項以降は圧力勾配の変化に起因する流速変動が,顕熱および潜熱輸送へ及ぼす影響項である.
        \begin{equation}
          \begin{split}
            \mat{J}_\mathrm{TH}
             & = \pdv{\vect{F}^\mathrm{T}_\text{internal}}{\vect{P}} + \pdv{\vect{F}^\mathrm{T}_\text{transient}}{\vect{P}} \notag                                                                                                                                   \\
             & = \beta_0  \iiint_{V} C_\mathrm{TH} \vect{\psi}_\mathrm{T} \vect{\psi}_\mathrm{P}^\mathsf{T} \odif{V}                                                                                                                                                 \\
             & \quad + \iiint_{V} \nabla\vect{\psi}_\mathrm{T}^\mathsf{T} \bab{\pab{c_\mathrm{w} \rho_\mathrm{w} K_\mathrm{wP} + c_\mathrm{v} \rho_\mathrm{w} K_\mathrm{vP}} \pab{\vect{\psi}_\mathrm{T}^\mathsf{T} \vect{T}}} \nabla\vect{\psi}_\mathrm{P} \odif{V} \\
             & \quad + \iiint_{V} \nabla\vect{\psi}_\mathrm{T}^\mathsf{T} \pab{\rho_\mathrm{w} L_\mathrm{v} K_\mathrm{vP}} \nabla\vect{\psi}_\mathrm{P} \odif{V}
          \end{split}
          \label{Eq:J_TP_def}
        \end{equation}

  \item[水-水ブロック:] $\mat{J}_\mathrm{HH}$ \\
        間隙水圧の変化に対する水分収支の応答を表す項である.
        \begin{align}
          \mat{J}_\mathrm{HH} = \pdv{\vect{R}_\mathrm{H}}{\vect{P}}
           & = \iiint_{V} \beta_0 C_\mathrm{HH} \vect{\psi}_\mathrm{H} \vect{\psi}_\mathrm{H}^\mathsf{T} \odif{V} \notag                   \\
           & \quad + \iiint_{V} \nabla\vect{\psi}_\mathrm{H}^\mathsf{T} \rho_\mathrm{w} K_\mathrm{P} \nabla\vect{\psi}_\mathrm{H} \odif{V}
        \end{align}

  \item[水-熱ブロック:] $\mat{J}_\mathrm{HT}$ \\
        温度変化が水分移動に与える影響を表す連成項である.
        \begin{align}
          \mat{J}_\mathrm{HT} = \pdv{\vect{R}_\mathrm{H}}{\vect{T}}
           & = \iiint_{V} \beta_0 C_\mathrm{HT} \vect{\psi}_\mathrm{H} \vect{\psi}_\mathrm{T}^\mathsf{T} \odif{V} \notag                          \\
           & \quad + \iiint_{V} \nabla\vect{\psi}_\mathrm{H}^\mathsf{T} \rho_\mathrm{w} K_\mathrm{T} \nabla\vect{\psi}_\mathrm{T} \odif{V} \notag \\
           & \quad - \iiint_{V} \nabla\vect{\psi}_\mathrm{H}^\mathsf{T} \vect{V}_\mathrm{HT} \vect{\psi}_\mathrm{T}^\mathsf{T} \odif{V}
        \end{align}
        ここで,$C_\mathrm{HT}$ は温度変化に伴う水分貯留量の変化率である.
        $K_\mathrm{wT}, K_\mathrm{vT}$ は温度勾配による水分輸送係数である.
        また,$\vect{V}_\mathrm{HT}$ は物性値の温度依存性に起因する補正項であり,$\vect{R}_\mathrm{H}$のフラックス項が負であるため,ヤコビアンへの寄与は負となる.
        \begin{equation}
          \vect{V}_\mathrm{HT} = \pdv{\vect{J}_\mathrm{m}}{T} \bigg|_{\nabla P, \nabla T \text{ const}}
        \end{equation}
\end{description}

\subsection{修正Picard反復法による線形化}
\label{Sec:PicardLinearization}

本解析では,温度 $\vect{T}$ と間隙水圧 $\vect{P}$ を同時に解く完全連成解析(Monolithic Approach)を採用する.
非線形方程式系の解法には,接線剛性行列(Jacobian)の更新を必要としない修正Picard反復法を用いる.
この手法では,第 $m$ 回目の反復において係数行列内の物性値を既知の解($\vect{T}^{m}, \vect{P}^{m}$)で評価して固定し,線形化された方程式系を解くことで更新解 $\vect{T}^{m+1}, \vect{P}^{m+1}$ を得る.

離散化された線形方程式系は,以下のブロック行列形式で記述される.
\begin{equation}
  \label{Eq:Monolithic-Picard-System}
  \begin{bmatrix}
    \mat{K}_\mathrm{TT} & \mat{K}_\mathrm{TH} \\
    \mat{K}_\mathrm{HT} & \mat{K}_\mathrm{HH}
  \end{bmatrix}^{m}
  \begin{Bmatrix}
    \vect{T} \\
    \vect{P}
  \end{Bmatrix}^{m+1}
  =
  \begin{Bmatrix}
    \vect{F}_\mathrm{T} \\
    \vect{F}_\mathrm{H}
  \end{Bmatrix}^{m}
\end{equation}
ここで,$\mat{K}$は一般化されたシステム行列,$\vect{F}_\mathrm{RHS}$は一般化された外力ベクトルを表す.

\subsubsection{熱移動ブロックの詳細}
熱移動支配方程式に対応する上段のブロック($\mat{K}_\mathrm{TT}, \mat{K}_\mathrm{TH}$)および右辺ベクトル $\vect{F}_\mathrm{T}$ は,以下の通り定義される.
ここで,非線形性の強い移流項(顕熱・潜熱移動)は右辺ベクトルに組み込むことで,左辺行列の安定化を図っている.

\begin{description}
  \item[熱-熱ブロック:] $\mat{K}_\mathrm{TT}$ \\
        温度変化に対する慣性項(熱容量)と熱伝導項,および温度に依存する境界条件項から構成される.
        \begin{align}
          \mat{K}_\mathrm{TT}
           & = \iiint_{V} \beta_0 C_\mathrm{TT} \vect{\psi}_\mathrm{T} \vect{\psi}_\mathrm{T}^\mathsf{T} \odif{V}
          + \iiint_{V} \nabla\vect{\psi}_\mathrm{T}^\mathsf{T} \tensor{R} \nabla\vect{\psi}_\mathrm{T} \odif{V} \notag      \\
           & \quad + \iint_{\partial V_\mathrm{HC}} \beta \vect{\psi}_\mathrm{T} \vect{\psi}_\mathrm{T}^\mathsf{T} \odif{S}
          + \iint_{\partial V_\mathrm{HH}} h \vect{\psi}_\mathrm{T} \vect{\psi}_\mathrm{T}^\mathsf{T} \odif{S}
          + \iint_{\partial V_\mathrm{HR}} \beta_\mathrm{HR} \vect{\psi}_\mathrm{T} \vect{\psi}_\mathrm{T}^\mathsf{T} \odif{S}
        \end{align}

  \item[熱-水ブロック:] $\mat{K}_\mathrm{TH}$ \\
        水圧変化が熱収支に与える影響を表す連成項である.ここでは相変化や密度変化に伴う熱容量の連成成分 $C_\mathrm{TH}$ を考慮する.
        \begin{equation}
          \mat{K}_\mathrm{TH} = \iiint_{V} \beta_0 C_\mathrm{TH} \vect{\psi}_\mathrm{T} \vect{\psi}_\mathrm{P}^\mathsf{T} \odif{V}
        \end{equation}

  \item [熱外力ベクトル:] $\vect{F}_\mathrm{T}$ \\
        境界条件からの既知の流入出,内部発熱に加え,前回の反復値で評価された移流フラックス,および時間積分の履歴項から構成される.
        \begin{align}
          \vect{F}_\mathrm{T}
           & = - \iint_{\partial V_\mathrm{HN}} q_\mathrm{N} \vect{\psi}_\mathrm{T} \odif{S}
          - \iint_{\partial V_\mathrm{HC}} \gamma \vect{\psi}_\mathrm{T} \odif{S} \notag                                                       \\
           & \quad + \iint_{\partial V_\mathrm{HH}} h T_\mathrm{env} \vect{\psi}_\mathrm{T} \odif{S}
          + \iint_{\partial V_\mathrm{HR}} \beta_\mathrm{HR} T_\mathrm{r} \vect{\psi}_\mathrm{T} \odif{S} \notag                               \\
           & \quad + \iiint_{V} \nabla\vect{\psi}_\mathrm{T}^\mathsf{T} \cdot \vect{j}_\mathrm{adv} \odif{V}
          + \iiint_{V} \vect{\psi}_\mathrm{T} S_\mathrm{T} \odif{V} \notag                                                                     \\
           & \quad - \iiint_{V} \vect{\psi}_\mathrm{T}^\mathsf{T} \pab{C_\mathrm{TT} T_\mathrm{hist} + C_\mathrm{TH} P_\mathrm{hist}} \odif{V}
        \end{align}
        ここで,$\vect{j}_\mathrm{adv}$ は前回の反復値を用いて算定された,流体移動に伴う顕熱および潜熱フラックスの総和である.
        \begin{equation}
          \vect{j}_\mathrm{adv} = \pab{c_\mathrm{w} \rho_\mathrm{w} \vect{j}_\mathrm{WL} T + c_\mathrm{v} \rho_\mathrm{w} \vect{j}_\mathrm{WV} T + \rho_\mathrm{w} L_\mathrm{v} \vect{j}_\mathrm{WV}}
        \end{equation}
        また,$T_\mathrm{hist}, P_\mathrm{hist}$ はBDF法における時間微分の履歴項 $\sum \beta_k X_{n+1-k}$ に対応する.
\end{description}

\subsubsection{水分移動ブロックの詳細}
水分移動支配方程式に対応する下段のブロックおよび右辺ベクトルについても,熱移動と同様に構築される.
水分移動における非線形項(透水係数など)は,Picard反復により前回の反復値を用いて線形化される.

\begin{description}
  \item[水-水ブロック:] $\mat{K}_\mathrm{HH}$ \\
        \begin{equation}
          \mat{K}_\mathrm{HH}
          = \iiint_{V} \beta_0 C_\mathrm{HH} \vect{\psi}_\mathrm{H} \vect{\psi}_\mathrm{H}^\mathsf{T} \odif{V}
          + \iiint_{V} \nabla\vect{\psi}_\mathrm{H}^\mathsf{T} \tensor{K}_\mathrm{H} \nabla\vect{\psi}_\mathrm{H} \odif{V}
        \end{equation}
        ここで,$\tensor{K}_\mathrm{H} = \rho_\mathrm{w}(K_\mathrm{wP} + K_\mathrm{vP})$ は物質移動係数の総和である.

  \item[水-熱ブロック:] $\mat{K}_\mathrm{HT}$ \\
        \begin{equation}
          \mat{K}_\mathrm{HT}
          = \iiint_{V} \beta_0 C_\mathrm{HT} \vect{\psi}_\mathrm{H} \vect{\psi}_\mathrm{T}^\mathsf{T} \odif{V}
          + \iiint_{V} \nabla\vect{\psi}_\mathrm{H}^\mathsf{T} \tensor{K}_\mathrm{T} \nabla\vect{\psi}_\mathrm{T} \odif{V}
        \end{equation}
        ここで,$\tensor{K}_\mathrm{T} = \rho_\mathrm{w}(K_\mathrm{wT} + K_\mathrm{vT})$ は温度勾配に起因する水分移動係数である.

  \item[水外力ベクトル:] $\vect{F}_\mathrm{H}$ \\
        重力項およびソース項を右辺に配置する.重力項は発散定理により正の符号で加算される.
        \begin{align}
          \vect{F}_\mathrm{H}
           & = - \iint_{\Gamma_\mathrm{HN}} q_{\mathrm{HN}} \vect{\psi}_\mathrm{H} \odif{S}
          + \iiint_{V} \nabla\vect{\psi}_\mathrm{H}^\mathsf{T} \cdot \pab{\rho_\mathrm{w} K_\mathrm{wP} \vect{g}} \odif{V} \notag                                                                        \\
           & \quad + \iiint_{V} \vect{\psi}_\mathrm{H} S_\mathrm{H} \odif{V} - \iiint_{V} \vect{\psi}_\mathrm{H}^\mathsf{T} \pab{C_\mathrm{HH} P_\mathrm{hist} + C_\mathrm{HT} T_\mathrm{hist}} \odif{V}
        \end{align}
\end{description}

\numberwithin{equation}{section}

\FloatBarrier
\section{有限要素法における実装}
\label{Sec:FEMImplementation}

\numberwithin{equation}{subsection}
\subsection{2次元アイソパラメトリック要素の統一的記述}
\label{Sec:2D_Isoparametric_Elements}

本解析では,2次元領域の空間離散化において,アイソパラメトリック要素(Isoparametric Element)を採用する.
これは,解析対象になる物理量の補間と,座標などの要素形状の定義に同一の基底関数を用いる手法である.
本節では,解析に用いる以下の4種類の要素を統一的に記述する.
\begin{itemize}
  \item 三角形一次要素 (Tri3) および 二次要素 (Tri6)
  \item 四角形一次要素 (Quad4) および 二次要素 (Quad8)
\end{itemize}
まず,要素内の任意の点における物理量 $T$ および 物理座標 $\vect{x} = (x, y)^\top$ は,要素内の節点値を用いて次式で近似される.
\begin{subequations}
  \label{Eq:Iso-Interp}
  \begin{align}
    T\pab{\xi, \eta}        & = \sum_{i=1}^{N_n} \psi_i\pab{\xi, \eta} T_i              \\
    \Pw\pab{\xi, \eta}      & = \sum_{i=1}^{N_n} \psi_i\pab{\xi, \eta} P_{\mathrm{w},i} \\
    \vect{x}\pab{\xi, \eta} & = \sum_{i=1}^{N_n} \psi_i\pab{\xi, \eta} \vect{x}_i
  \end{align}
\end{subequations}
ここで,$N_n$ は要素あたりの節点数,$\psi_i$ は正規化座標系 $\pab{\xi, \eta}$ で定義された形状関数である.
この定式化により,要素の幾何形状が歪んでいる場合でも,正規化座標系上での積分計算に帰着させることが可能となる.
座標変換に伴うJacobian行列 $\mat{J}$ は,要素タイプに関わらず次式で統一的に定義される.
\begin{equation}
  \label{Eq:General_Jacobian}
  \mat{J} =
  \begin{bmatrix}
    \displaystyle \sum \pdv{\psi_i}{\xi} x_i & \displaystyle \sum \pdv{\psi_i}{\eta} x_i \\[3mm]
    \displaystyle \sum \pdv{\psi_i}{\xi} y_i & \displaystyle \sum \pdv{\psi_i}{\eta} y_i
  \end{bmatrix}
\end{equation}

\FloatBarrier
\subsection{二次元要素と形状関数}
各要素の形状と節点配置を\cref{Fig:2D_Element_Library}に示す.
三角形要素には面積座標(Area Coordinates)$L_i$ を,四角形要素には自然座標(Natural Coordinates)$(\xi, \eta)$ を適用する.

\begin{figure}[H]
  \centering
  \begin{tikzpicture}[scale=1.5, every node/.style={scale=0.8}]
    \tikzstyle{node_style}=[circle, draw, fill=white, inner sep=1.5pt]
    \tikzstyle{elem_edge}=[thick]

    % --- (a) Tri3 ---
    \begin{scope}[shift={(0,3)}]
      \draw[elem_edge] (0,0) -- (2,0) -- (0,2) -- cycle;
      \draw[-latex] (0,0) -- (0.8,0) node[below]{$L_1$};
      \draw[-latex] (0,0) -- (0,0.8) node[left]{$L_2$};
      \node[node_style] at (0,0) {1};
      \node[node_style] at (2,0) {2};
      \node[node_style] at (0,2) {3};
      \node at (1, -0.5) {(a) Tri3 (Linear)};
    \end{scope}

    % --- (b) Tri6 ---
    \begin{scope}[shift={(4,3)}]
      \draw[elem_edge] (0,0) -- (2,0) -- (0,2) -- cycle;
      \node[node_style] at (0,0) {1};
      \node[node_style] at (2,0) {2};
      \node[node_style] at (0,2) {3};
      \node[node_style] at (1,0) {4};
      \node[node_style] at (1,1) {5};
      \node[node_style] at (0,1) {6};
      \node at (1, -0.5) {(b) Tri6 (Quadratic)};
    \end{scope}

    % --- (c) Quad4 ---
    \begin{scope}[shift={(0,0)}]
      \draw[elem_edge] (0,0) -- (2,0) -- (2,2) -- (0,2) -- cycle;
      \node[node_style] at (0,0) {1};
      \node[node_style] at (2,0) {2};
      \node[node_style] at (2,2) {3};
      \node[node_style] at (0,2) {4};
      \node at (1, -0.5) {(c) Quad4 (Linear)};
      % Axis
      \draw[-latex] (0.8,1) -- (1.4,1) node[right]{$\xi$};
      \draw[-latex] (1,0.8) -- (1,1.4) node[above]{$\eta$};
    \end{scope}

    % --- (d) Quad8 ---
    \begin{scope}[shift={(4,0)}]
      \draw[elem_edge] (0,0) -- (2,0) -- (2,2) -- (0,2) -- cycle;
      \node[node_style] at (0,0) {1};
      \node[node_style] at (2,0) {2};
      \node[node_style] at (2,2) {3};
      \node[node_style] at (0,2) {4};
      \node[node_style] at (1,0) {5};
      \node[node_style] at (2,1) {6};
      \node[node_style] at (1,2) {7};
      \node[node_style] at (0,1) {8};
      \node at (1, -0.5) {(d) Quad8 (Quadratic)};
    \end{scope}
  \end{tikzpicture}
  \caption{2次元アイソパラメトリック要素}\label{Fig:2D_Element_Library}
\end{figure}

\subsubsection{三角形要素群 (Triangular Elements)}
三角形要素では面積座標 $L_1, L_2, L_3$ を用いる.これらは $L_1+L_2+L_3=1$ を満たすため,独立変数は2つ(例: $L_1, L_2$)である.
一般の直交正規化座標 $(\xi, \eta)$ との対応は,通常 $\xi = L_1, \eta = L_2$ と定義される.

\begin{description}
  \item[一次三角形要素 (Tri3)]
        線形補間を行う最も基本的な要素である.
        \begin{equation}
          \psi_1 = L_1, \quad \psi_2 = L_2, \quad \psi_3 = L_3
        \end{equation}
  \item[二次三角形要素 (Tri6)]
        各辺の中点に節点を追加し,物理量の二次分布を表現可能とした要素である.
        \begin{align}
          \mbox{頂点節点:} & \quad \psi_1 = L_1(2L_1-1), \quad \psi_2 = L_2(2L_2-1), \quad \psi_3 = L_3(2L_3-1) \notag \\
          \mbox{中間節点:} & \quad \psi_4 = 4L_1L_2, \qquad~~ \psi_5 = 4L_2L_3, \qquad~~ \psi_6 = 4L_3L_1
        \end{align}
\end{description}

\subsubsection{四角形要素群 (Quadrilateral Elements)}
四角形要素では,正規化座標 $(\xi, \eta) \in [-1, 1] \times [-1, 1]$ を用いる.

\begin{description}
  \item[一次四角形要素 (Quad4)]
        双一次(Bilinear)補間を行う要素である.
        \begin{equation}
          \psi_i = \frac{1}{4}(1 + \xi_i \xi)(1 + \eta_i \eta) \quad (i=1,\dots,4)
        \end{equation}
        ここで $(\xi_i, \eta_i)$ は各節点の正規化座標である.
  \item[二次四角形要素 (Quad8)]
        各辺の中点に節点を追加したSerendipity族の二次要素である.中心節点を持たないため計算効率が良い.
        \begin{align}
          \mbox{隅節点 ($i=1\dots4$):} & \quad \psi_i = \frac{1}{4}(1 + \xi_i \xi)(1 + \eta_i \eta)(\xi_i \xi + \eta_i \eta - 1) \notag \\
          \mbox{中間節点 ($\xi_i=0$):}  & \quad \psi_i = \frac{1}{2}(1 - \xi^2)(1 + \eta_i \eta) \notag                                  \\
          \mbox{中間節点 ($\eta_i=0$):} & \quad \psi_i = \frac{1}{2}(1 + \xi_i \xi)(1 - \eta^2)
        \end{align}
\end{description}

\subsection{数値積分の統一的扱い}
剛性行列や質量行列の算出に必要な要素領域積分は,Gauss-Legendre積分公式を用いて次のように記述される.
\begin{equation}
  \label{Eq:Gauss-Legendre-2D}
  \int_{-1}^{+1}\int_{-1}^{+1} f\pab{\xi, \eta}\odif{\xi}\odif{\eta} \approx \sum_{i=1}^{N_\mathrm{sample}}\sum_{j=1}^{N_\mathrm{sample}}w_i w_j f\pab{\xi_i, \eta_j}
\end{equation}
ただし,$N_\mathrm{sample}$は各軸方向のサンプル点数,$w_i$は重み,$\xi_i$,$\eta_i$はガウス積分点である.

\eqref{Eq:Gauss-Legendre-2D}を用いたときの積分の概念図を示す.このとき,サンプル点は2点とする.

\begin{figure}[H]
  \centering
  \begin{tikzpicture}[scale=3.5]
    \coordinate (p1) at (-1,-1);
    \coordinate (p2) at (1,-1);
    \coordinate (p3) at (1,1);
    \coordinate (p4) at (-1,1);
    \coordinate (S1) at (-0.57735,-0.57735);
    \coordinate (S2) at (-0.57735, 0.57735);
    \coordinate (S3) at ( 0.57735,-0.57735);
    \coordinate (S4) at ( 0.57735, 0.57735);

    % 背景の領域(元々薄いグレーなのでそのまま)
    \draw[fill=black!10!white] (p1)--node[midway, below right]{$-1$}(p2)--node[midway, below right]{$+1$}(p3)--node[midway, above right]{$+1$}(p4)--cycle node[midway, below left]{$-1$};
    %axis
    \draw[-{latex}, line width=0.4mm] (-1.2,0) -- (1.2,0) node[above]{$\xi$};
    \draw[-{latex}, line width=0.4mm] (0,-1.2) -- (0,1.2) node[left]{$\eta$};

    \draw[-{latex}, line width=0.4mm] (-0.6,-1.5) --node[midway, below]{$i$} (0.6,-1.5);
    \draw[-{latex}, line width=0.4mm] (-1.5,-0.6) --node[midway, left]{$j$} (-1.5,0.6);

    \draw[-, line width=0.3mm] (-1.2, 0.57735) -- (1.2, 0.57735) node[right]{$\eta= \sqrt{1/3}$};
    \draw[-, line width=0.3mm] (-1.2,-0.57735) -- (1.2,-0.57735) node[right]{$\eta=-\sqrt{1/3}$};
    \draw[-, line width=0.3mm] (-0.57735,-1.2) node[below]{$\xi=-\sqrt{1/3}$} -- (-0.57735,1.2);
    \draw[-, line width=0.3mm] ( 0.57735,-1.2) node[below]{$\xi= \sqrt{1/3}$} -- ( 0.57735,1.2);

    % サンプル点の円(ピンクからグレーに変更)
    \foreach \i in {1,2,3,4} {
        \draw[fill=gray!60] (S\i) node {\i} circle[radius=0.08];
      }
  \end{tikzpicture}
  \caption{Gauss-Legendre積分のサンプル点を2点とした場合の概念図}
\end{figure}

サンプル点数に応じた重みとガウス積分点を\cref{Table:Gauss-Legendre-PointsAndWeights}に示す.なお,重み $w_i$ は1次元積分における値であり,2次元積分ではその積 $w_i w_j$ が用いられる.
\begin{longtable}{ccc}
  \caption{Gauss-Legendre積分のサンプル点数とガウス積分点,および重み}\label{Table:Gauss-Legendre-PointsAndWeights}                        \\
  \hline
  点数:$N_\mathrm{sample}$ & 座標:$\xi_i$,$\eta_i$                                      & 重み:$w_i$                       \\
  \hline
  $1$                    & $0$                                                      & $2$                            \\
  \hline
                         &                                                          &                                \\[-6mm]
  $2$                    & $\pm\sqrt{\dfrac{1}{3}}$                                 & $1$                            \\[7mm]
  \hline
                         &                                                          &                                \\[-6mm]
  $3$                    & $0$                                                      & $\dfrac{8}{9}$                 \\[7mm]
                         & $\pm\sqrt{\dfrac{3}{5}}$                                 & $\dfrac{5}{9}$                 \\[7mm]
  \hline
                         &                                                          &                                \\[-6mm]
  $4$                    & $\pm\sqrt{\dfrac{3}{7}-\dfrac{2}{7}\sqrt{\dfrac{6}{5}}}$ & $\dfrac{18+\sqrt{30}}{36}$     \\[7mm]
                         & $\pm\sqrt{\dfrac{3}{7}+\dfrac{2}{7}\sqrt{\dfrac{6}{5}}}$ & $\dfrac{18-\sqrt{30}}{36}$     \\[7mm]
  \hline
                         &                                                          &                                \\[-6mm]
  $5$                    & $0$                                                      & $\dfrac{128}{225}$             \\[7mm]
                         & $\pm\dfrac{1}{3}\sqrt{5-2\sqrt{\dfrac{10}{7}}}$          & $\dfrac{322+13\sqrt{70}}{900}$ \\[7mm]
                         & $\pm\dfrac{1}{3}\sqrt{5+2\sqrt{\dfrac{10}{7}}}$          & $\dfrac{322-13\sqrt{70}}{900}$ \\[7mm]
  \hline
\end{longtable}

有限要素法における数値積分では,被積分関数(形状関数の導関数やヤコビアンの積)の次数に応じて適切な積分点数を選択する必要がある.$N_\mathrm{sample}$ 点のGauss-Legendre積分は $2N_\mathrm{sample}-1$ 次以下の多項式を厳密に積分可能である.剛性行列の評価には通常,フル積分(Full Integration)が推奨されるが,計算効率の向上やロッキング現象の回避を目的として,意図的に積分点数を減らす低減積分(Reduced Integration)が用いられる場合もある.
ここで,要素タイプごとの標準的な積分規則を\cref{Table:Integration_Rules}に示す.
\begin{table}[H]
  \centering
  \caption{各要素タイプにおける標準的な数値積分則}
  \label{Table:Integration_Rules}
  \begin{tabular}{lcccc} \toprule
    要素    & 座標系        & 積分点数              & 積分次数     & 備考     \\ \midrule
    Tri3  & 面積座標       & 1点                & $O(h)$   & 重心積分   \\
    Tri6  & 面積座標       & 3点                & $O(h^2)$ & 辺の中点など \\
    Quad4 & $\xi-\eta$ & $2 \times 2 = 4$点 & $O(h^2)$ & フル積分   \\
    Quad8 & $\xi-\eta$ & $3 \times 3 = 9$点 & $O(h^3)$ & フル積分   \\ \bottomrule
  \end{tabular}
\end{table}

\FloatBarrier

\subsection{数値積分と物性値の評価方法}

本解析では,土壌の熱・水分物性値(容量係数,拡散テンソル,輸送ベクトル等)が温度 $T$ および間隙水圧 $\Pw$ に強く依存する非線形性を有する.
そのため,係数自体を節点値から形状関数で補間する手法は用いず,各ガウス積分点において補間された状態量に基づき,構成則を用いて直接評価する手法を採用する.

まず,要素内のガウス積分点 $p$(局所座標 $\xi_p, \eta_p$)における未知変数の値($T_p, P_{\mathrm{w},p}$)およびその勾配は,節点値 $\vect{T}^e, \vect{P}_\mathrm{w}^e$ と形状関数 $\vect{\psi}$,勾配行列 $\mat{B}$ を用いて以下のように計算される.
\begin{equation}
  \label{Eq:GaussPoint_Interp}
  \begin{aligned}
    T_p & = \vect{\psi}\pab{\xi_p, \eta_p} \vect{T}^e,            & \nabla T_p & = \mat{B}\pab{\xi_p, \eta_p}^\top \vect{T}^e   \\
    P_p & = \vect{\psi}\pab{\xi_p, \eta_p} \vect{P}_\mathrm{w}^e, & \nabla P_p & = \mat{B}\pab{\xi_p, \eta_p}^\top \vect{\Pw}^e
  \end{aligned}
\end{equation}
ここで,$\mat{B} = \nabla \vect{\psi}$ は形状関数の空間微分行列である。
これらを用いて,積分点 $p$ における物理係数(スカラー $A$,テンソル $\mat{M}$,ベクトル $\vect{V}$)は,構成則関数 $\mathcal{F}$ を用いて以下のように決定される.
\begin{equation}
  A_p = \mathcal{F}_A\pab{T_p, P_p}, \quad
  \mat{M}_p = \mathcal{F}_M\pab{T_p, P_p}, \quad
  \vect{V}_p = \mathcal{F}_V\pab{T_p, P_p, \nabla T_p, \dots}
\end{equation}

これを踏まえ,各要素行列の数値積分計算式を以下のように記述する.

\subsubsection{容量型行列(分布スカラー係数 $A$)}
質量行列や熱容量行列などがこれに該当する.積分点ごとの係数値 $A_p$ を用いて計算される.
\begin{align}
  \mat{K}_1^e
   & = \iint_{\Omega^e} \vect{\psi}^\top A\pab{T, \Pw} \vect{\psi} \det\mat{J} \odif{\xi}\odif{\eta} \notag                \\
   & \approx \sum_{p=1}^{N_\mathrm{int}} w_p \vect{\psi}\pab{\xi_p}^\top A_p \vect{\psi}\pab{\xi_p} \det\mat{J}\pab{\xi_p}
\end{align}

\subsubsection{拡散型行列(分布テンソル係数 $\mat{M}$)}
熱伝導行列や透水行列などがこれに該当する.積分点ごとのテンソル値 $\mat{M}_p$ を用いて計算される.
\begin{align}
  \mat{K}_2^e
   & = \iint_{\Omega^e} \mat{B}^\top \mat{M}\pab{T, \Pw} \mat{B} \det\mat{J} \odif{\xi}\odif{\eta} \notag                \\
   & \approx \sum_{p=1}^{N_\mathrm{int}} w_p \mat{B}\pab{\xi_p}^\top \mat{M}_p \mat{B}\pab{\xi_p} \det\mat{J}\pab{\xi_p}
\end{align}

\subsubsection{混合型行列(分布ベクトル係数 $\vect{V}$)}
移流項などがこれに該当する.積分点ごとのベクトル値 $\vect{V}_p$ を用いて計算される.
(ここでは $\nabla \psi \cdot \vect{V} \psi$ の型を例示する)
\begin{align}
  \mat{K}_3^e
   & = \iint_{\Omega^e} \pab{\mat{B}^\top \vect{V}\pab{T, \Pw}} \vect{\psi} \det\mat{J} \odif{\xi}\odif{\eta} \notag                \\
   & \approx \sum_{p=1}^{N_\mathrm{int}} w_p \pab{\mat{B}\pab{\xi_p}^\top \vect{V}_p} \vect{\psi}\pab{\xi_p} \det\mat{J}\pab{\xi_p}
\end{align}

\FloatBarrier
\section{線形方程式のためのKrylov部分空間法}
\label{Sec:KSP}

\subsection{BiCGSTAB}

BiCGSTAB (Biconjugate Gradient Stabilized) 法は,非対称な係数行列を持つ線形方程式系を解くための反復法である\parencite{Vorst-1992}
\footnote{JICFuS WIKI,BiCGSTAB法:\url{https://www.jicfus.jp:443/wiki/index.php?Bi-CGSTAB\%20\%E6\%B3\%95}}.

% ------------------------------------------------------------------
% Algorithm 1: Theoretical BiCGSTAB
% Label becomes: Alg:PreBiCGSTAB
% ------------------------------------------------------------------
\begin{AlgorithmBox}{Preconditioned Bi-CGSTAB method (Theoretical)}{PreBiCGSTAB}
  \setlength{\baselineskip}{17pt}
  \begin{algorithmic}[1]
    \State Set an initial value $\vect{x}_0$
    \State Compute the initial residual $\vect{r}_0=\vect{b}-\mat{A}\vect{x}_0$
    \State Choose an arbitrary vector $\hat{\vect{r}}_0$ such that $\inp{\hat{\vect{r}}_0}{\vect{r}_0}\neq 0$, e.g., $\hat{\vect{r}}_0=\vect{r}_0$
    \State Set $\vect{p}_0=\vect{r}_0$
    \For{$k=0,1,2,\cdots$}
    \State $\vect{v}_k = \mat{A} \mat{M}^{-1} \vect{p}_k$
    \State $\alpha_k = \dfrac{\inp{\hat{\vect{r}}_0}{\vect{r}_k}}{\inp{\hat{\vect{r}}_0}{\vect{v}_k}}$
    \State $\vect{s}_k = \vect{r}_k - \alpha_k \vect{v}_k$
    \If{$\norm{\vect{s}_k}_2 < tol$}
    \State $\vect{x}_{k+1} = \vect{x}_k + \alpha_k \mat{M}^{-1} \vect{p}_k$
    \State \textbf{break} (Converged)
    \EndIf
    \State $\vect{t}_k = \mat{A} \mat{M}^{-1} \vect{s}_k$
    \State $\omega_k = \dfrac{\inp{\vect{t}_k}{\vect{s}_k}}{\inp{\vect{t}_k}{\vect{t}_k}}$
    \State $\vect{x}_{k+1} = \vect{x}_k + \alpha_k \mat{M}^{-1} \vect{p}_k + \omega_k \mat{M}^{-1} \vect{s}_k$
    \State $\vect{r}_{k+1} = \vect{s}_k - \omega_k \vect{t}_k$
    \If{$\norm{\vect{r}_{k+1}}_2 < tol$}
    \State \textbf{break} (Converged)
    \EndIf
    \State $\beta_k = \dfrac{\alpha_k}{\omega_k} \times \dfrac{\inp{\hat{\vect{r}}_0}{\vect{r}_{k+1}}}{\inp{\hat{\vect{r}}_0}{\vect{r}_k}}$
    \State $\vect{p}_{k+1} = \vect{r}_{k+1} + \beta_k (\vect{p}_k - \omega_k \vect{v}_k)$
    \EndFor
  \end{algorithmic}
\end{AlgorithmBox}

実際の実装では,行列ベクトル積の回数とベクトル格納に必要なメモリを削減するため,以下のように中間変数を再利用する.

% ------------------------------------------------------------------
% Algorithm 2: Implementation BiCGSTAB
% Label becomes: Alg:IMP_PreBiCGSTAB
% ------------------------------------------------------------------
\begin{AlgorithmBox}{Implementation of preconditioned Bi-CGSTAB method}{IMP_PreBiCGSTAB}
  \setlength{\baselineskip}{17pt}
  \begin{algorithmic}[1]
    \State Set an initial value $\vect{x}_0=0.0$
    \State Compute the initial residual $\vect{r}_0=\vect{b}-\mat{A}\vect{x}_0$
    \State Create preconditioned matrix $\mat{M}^{-1}$
    \State Set initial shadow residual $\hat{\vect{r}}_0=\vect{r}_0$
    \State $\rho = \inp{\hat{\vect{r}}_0}{\vect{r}_0}$, $\rho_\mathrm{old} = 1.0$, $\alpha_k = 1.0$, $\omega_k = 1.0$
    \State $\vect{p} \leftarrow 0$, $\vect{v} \leftarrow 0$
    \For{$k=0,1,2,\cdots$}
    \If{$k = 0$}
    \State $\vect{p} = \vect{r}_0$
    \Else
    \State $\beta_k=\dfrac{\alpha_k}{\omega_k}\times\dfrac{\rho}{\rho_\mathrm{old}}$
    \State $\vect{p} = \vect{r}_k+\beta_k(\vect{p}-\omega_k\vect{v})$
    \EndIf

    \State Apply preconditioner: $\hat{\vect{p}}=\mat{M}^{-1}\vect{p}$
    \State $\vect{v}=\mat{A}\hat{\vect{p}}$

    \State $\alpha_k=\dfrac{\rho}{\inp{\hat{\vect{r}}_0}{\vect{v}}}$
    \State $\vect{s}_k=\vect{r}_k-\alpha_k\vect{v}$

    \State $resid_{s} = \norm{\vect{s}_k}_2$
    \If{$resid_{s} < tol$}
    \State $\vect{x}_{k+1}=\vect{x}_k+\alpha_k\hat{\vect{p}}$
    \State \textbf{break} (Converged)
    \EndIf

    \State Apply preconditioner: $\hat{\vect{s}}=\mat{M}^{-1}\vect{s}_k$
    \State $\vect{t}=\mat{A}\hat{\vect{s}}$

    \State $\omega_k=\dfrac{\inp{\vect{t}}{\vect{s}_k}}{\inp{\vect{t}}{\vect{t}}}$
    \If{$\omega_k = 0.0$}
    \State \textbf{fail} (div zero breakdown); \textbf{break}
    \EndIf

    \State $\vect{x}_{k+1}=\vect{x}_k+\alpha_k\hat{\vect{p}}+\omega_k\hat{\vect{s}}$
    \State $\vect{r}_{k+1}=\vect{s}_k-\omega_k\vect{t}$

    \State $resid = \norm{\vect{r}_{k+1}}_2$
    \If{$resid < tol$}
    \State \textbf{break} (Converged)
    \EndIf

    \State $\rho_\mathrm{old}=\rho$
    \State $\rho=\inp{\hat{\vect{r}}_0}{\vect{r}_{k+1}}$
    \If{$\rho = 0.0$}
    \State \textbf{fail} (breakdown); \textbf{break}
    \EndIf
    \EndFor
  \end{algorithmic}
\end{AlgorithmBox}

\subsection{GMRES}

GMRES (Generalized Minimal Residual) 法は,非対称な係数行列を持つ線形方程式系を解くための反復法である\parencite{Saad-1986}
\footnote{JICFuS WIKI,GMRES法:\url{https://www.jicfus.jp:443/wiki/index.php?GMRES\%28m\%29\%20\%E6\%B3\%95}}.
右前処理 (Right Preconditioning) を用いたリスタート付き GMRES(m) 法を示す.

% ------------------------------------------------------------------
% Algorithm 3: Theoretical GMRES
% Label becomes: Alg:PreFullGMRES_Corrected
% ------------------------------------------------------------------
\begin{AlgorithmBox}{Preconditioned GMRES(m) method (Right Preconditioning)}{PreFullGMRES_Corrected}
  \setlength{\baselineskip}{17pt}
  \begin{algorithmic}[1]
    \State Set an initial value $\vect{x}_0$
    \State Compute $\vect{r} = \vect{b} - \mat{A} \vect{x}_0$
    \While{$\norm{\vect{r}}_2 / \norm{\vect{b}}_2 > \varepsilon$}
    \State $\beta = \norm{\vect{r}}_2$
    \State $\vect{v}_1 = \vect{r} / \beta$
    \State Set $\vect{g} = \beta \vect{e}_1 \in \mathbb{R}^{m+1}$

    \For{$j=1, 2, \dots, m$}
    \State $\vect{w} = \mat{A} \mat{M}^{-1} \vect{v}_j$
    \For{$i=1, 2, \dots, j$}
    \State $h_{i,j} = \inp{\vect{w}}{\vect{v}_i}$
    \State $\vect{w} = \vect{w} - h_{i,j} \vect{v}_i$
    \EndFor
    \State $h_{j+1,j} = \norm{\vect{w}}_2$
    \State $\vect{v}_{j+1} = \vect{w} / h_{j+1,j}$

    \For{$i=1, 2, \dots, j-1$}
    \State $\begin{bmatrix} h_{i,j} \\ h_{i+1,j} \end{bmatrix} \leftarrow \begin{bmatrix} c_i  & s_i \\ -s_i & c_i \end{bmatrix} \begin{bmatrix} h_{i,j} \\ h_{i+1,j} \end{bmatrix} $
    \EndFor

    \State $c_j = \dfrac{h_{j,j}}{\sqrt{h_{j,j}^2 + h_{j+1,j}^2}}$
    \State $s_j = \dfrac{h_{j+1,j}}{\sqrt{h_{j,j}^2 + h_{j+1,j}^2}}$

    \State $h_{j,j} = c_j h_{j,j} + s_j h_{j+1,j}$
    \State $h_{j+1,j} = 0.0$

    \State $g_{j+1} = -s_j g_j$
    \State $g_j = c_j g_j$

    \If{$\abs{g_{j+1}} / \norm{\vect{b}}_2 < \varepsilon$}
    \State $m=j$; \textbf{break}
    \EndIf
    \EndFor

    \State Solve $\bar{\mat{H}}_m \vect{y}_m = \bar{\vect{g}}_m$ for $\vect{y}_m$ (via back-substitution)
    \State $\tilde{\vect{x}} = \vect{x}_0 + \mat{M}^{-1} \mat{V}_m \vect{y}_m$
    \State $\vect{r} = \vect{b} - \mat{A} \tilde{\vect{x}}$
    \State $\vect{x}_0 = \tilde{\vect{x}}$
    \EndWhile
  \end{algorithmic}
\end{AlgorithmBox}

実際の実装では,基底ベクトル $V=[\vect{v}_i]$ と,前処理を適用した基底 $\hat{V}=[\mat{M}^{-1}\vect{v}_i]$ を分けて保持する必要がある.

% ------------------------------------------------------------------
% Algorithm 4: Implementation GMRES
% Label becomes: Alg:IMP_LIS_FGMRES
% ------------------------------------------------------------------  
\begin{AlgorithmBox}{Implementation of right-preconditioned GMRES(m) method}{IMP_LIS_FGMRES}
  \setlength{\baselineskip}{17pt}
  \begin{algorithmic}[1]
    \State Set an initial value $\vect{x}_0$
    \State Compute $\vect{r} = \vect{b} - \mat{A} \vect{x}_0$
    \While{$\norm{\vect{r}}_2 / \norm{\vect{b}}_2 > \varepsilon$}
    \State $\beta = \norm{\vect{r}}_2$
    \State $\vect{v}_1 = \vect{r} / \beta$
    \State Set $\vect{g} = \beta \vect{e}_1 \in \mathbb{R}^{m+1}$
    \State $c \leftarrow 0$, $s \leftarrow 0$

    \For{$j=1, 2, \dots, m$}
    \State $\vect{z}_j = \mat{M}^{-1} \vect{v}_j$
    \State $\vect{w} = \mat{A} \vect{z}_j$
    \For{$i=1, 2, \dots, j$}
    \State $h_{i,j} = \inp{\vect{w}}{\vect{v}_i}$
    \State $\vect{w} = \vect{w} - h_{i,j} \vect{v}_i$
    \EndFor
    \State $h_{j+1,j} = \norm{\vect{w}}_2$
    \State $\vect{v}_{j+1} = \vect{w} / h_{j+1,j}$

    \For{$i=1, 2, \dots, j-1$}
    \State $h_{i,j}^\text{temp} = c_i h_{i,j} + s_i h_{i+1,j}$
    \State $h_{i+1,j} = -s_i h_{i,j} + c_i h_{i+1,j}$
    \State $h_{i,j} = h_{i,j}^\text{temp}$
    \EndFor

    \State $c_j = \dfrac{h_{j,j}}{\sqrt{h_{j,j}^2 + h_{j+1,j}^2}}$
    \State $s_j = \dfrac{h_{j+1,j}}{\sqrt{h_{j,j}^2 + h_{j+1,j}^2}}$

    \State $h_{j,j} = c_j h_{j,j} + s_j h_{j+1,j}$
    \State $h_{j+1,j} = 0.0$

    \State $g_{j+1} = -s_j g_j$
    \State $g_j = c_j g_j$

    \If{$\abs{g_{j+1}} < tol$}
    \State $m=j$; \textbf{break}
    \EndIf
    \EndFor

    \State Let $\bar{\mat{H}}_m$ be the $m \times m$ upper triangular matrix $h_{i,j}$
    \State Let $\bar{\vect{g}}_m$ be the $m \times 1$ vector $g_i$
    \State Solve $\bar{\mat{H}}_m \vect{y}_m = \bar{\vect{g}}_m$ for $\vect{y}_m$ (via back-substitution)

    \State Let $\mat{Z}_m = [\vect{z}_1, \dots, \vect{z}_m]$
    \State $\vect{x} = \vect{x}_0 + \mat{Z}_m \vect{y}_m$

    \State $\vect{r} = \vect{b} - \mat{A} \vect{x}$
    \State $\vect{x}_0 = \vect{x}$
    \EndWhile
  \end{algorithmic}
\end{AlgorithmBox}

\FloatBarrier
\numberwithin{equation}{section}

\FloatBarrier

% Results
\section{実験解析}
\numberwithin{equation}{subsection}
\subsection{実験概要}
\label{Sec:ExperimentalSetup}
実験には,\cref{Fig:experimental_setup}に示す円筒形の凍結実験装置を用いた.装置は内径\qty{0.2}{\meter},外径\qty{0.23}{\meter},高さ\qty{0.6}{\meter}のアクリル円筒と,円筒下端から\qty{0.2}{\meter}の位置を貫通する外径\qty{0.006}{\meter}のU字型凍結管で構成される.
供試体は鳥取砂丘砂を乾燥密度\qty{1600}{\kilogram.\meter^{-3}}で水中充填法により作製し,底部給水・上部排水により飽和状態を維持した.凍結開始前にDarcy流速\qty{1.0}{\meter.\day^{-1}}で36時間以上給水した後,凍結管内にブライン(エチレングリコール水溶液,比重\qty{1100}{\kilogram.\meter^{-3}})を循環させて内部から冷却した.

\FloatBarrier
\numberwithin{equation}{section}

\section{地下水流れ存在下での人工地盤凍結工法におけるパラメータスタディ}
\label{Sec:ParametricStudy}
% 本節ではこれまでに検討した数値モデルに対して,実験装置を用いた実験を行い,その結果を比較・検証する.

\numberwithin{equation}{subsection}
\subsection{概要}
\label{Sec:Param_Outline}

前章までは,室内模型実験およびその再現解析を通じて,構築した数値解析モデルの妥当性を検証した.
本章では,検証済みの数値モデルを実規模を想定した広域な解析領域に適用し,地下水流が存在する地盤における凍結挙動についてのパラメータスタディを行う.

地下水流は,上流部より温度の高い液状水を運ぶとともに,凍結管周辺の冷却された間隙水を下流側に移流させ,凍土壁の形成を阻害する主要な要因となる.
特に,複数の凍結管によって形成された凍土が結合し,遮水壁として機能するまでの所要時間は,地下水流速や凍結管の本数,配置間隔といった設計因子に強く依存する.
そこで本章では,解析領域内の初期地下水流速および凍結管の配置条件をパラメータとして変化させ,これらが凍結壁の閉塞時間に及ぼす影響を定量的かつ網羅的に検討する.
また,得られた数値解析結果に基づき,既存の設計理論式に含まれる幾何学的パラメータである代表長さの再評価を試みる.

\FloatBarrier
\subsection{解析条件および計算ケース}
\label{Sec:Param_Conditions}

解析対象として,凍結管が配置されている空間内のある一定の深さにおける\qtyproduct{30 x 30}{\meter}の水平断面領域を設定した.
凍結管は表面温度一定(\qty{-30}{\degreeCelsius})とし,解析領域の左端(流入境界)から\qty{10}{\meter}の地点に,領域中心軸に対して上下対称となるように配置した.

計算ケースとして,凍結管の本数を2,4,8,12本の4水準とし,それぞれの配置において凍結管間隔を\qtylist{0.6;0.8;1.0}{\meter}の一定間隔とした組み合わせについて検討を行った.
また,地下水流の影響を詳細に評価するため,初期地下水流速については,\qty{0.0}{\meter.\day^{-1}}から\qty{1.2}{\meter.\day^{-1}}の範囲で計19通りの流速条件を設定した.
これにより,流速の増大に伴う凍結閉塞時間の変化および閉塞限界近傍の非線形な挙動を捉えることを可能とした.
その他の解析条件(境界条件の設定等)および物性値については,前章の実験検証と同様とした.

\FloatBarrier
\subsection{結果}
\label{Sec:Param_Results}

\subsubsection{凍土壁閉塞時間の算出}
各計算ケースにおいて,得られた温度分布の経時変化に基づき,凍土壁が閉塞するまでの所要時間を算出した.
ここで「閉塞」とは,隣接する凍結管から成長した凍結領域が連結し,凍結管列の間隙が全て\qty{0}{\degreeCelsius}以下の領域で満たされた状態と定義した.

\subsubsection{双曲線フィッティングによる限界流速の同定}
それぞれの凍結管間隔・本数に対し,初期流速と凍土が閉塞するまでの時間について双曲線フィッティングを行った.
凍結管本数が多い条件では,初期流速が閉塞成立の限界(限界流速)に近づくにつれて閉塞時間が急増する傾向が見られたが,双曲線フィッティングはこうした全体的な挙動を概ね良好に捉えることができた.

特に凍結管が2本の場合,フィッティングされた双曲線の漸近値(閉塞時間が無限大となる流速)は,既存の理論式から算出される理論的な限界流速とよく一致した.
これより,本数値解析結果のフィッティングより得られる漸近値は,理論上の閉塞限界と等価であるとみなせる.
したがって,以下の考察においては,この数値的な漸近値をその計算ケースにおける「実効的な限界流速」として取り扱う.

\FloatBarrier
\subsection{考察}
\label{Sec:Param_Discussion}

\subsubsection{代表長さの逆算と評価}
前節の結果に基づき,限界流速式で算出される限界流速を数値的な漸近値に置き換えることで,式中の経験的パラメータである「代表長さ」を数値計算によって逆算することを試みた.
従来,この代表長さとしては凍土壁全長(凍結管列の全幅)が慣例的に用いられてきたが,管間隔や密度効果は考慮されていなかった.
このようにして数値的に求めた代表長さと,慣例的に代表長さとして用いられてきた凍土壁全長との比較を行ったところ,凍結管配置によっては両者に有意な乖離が生じることが確認された.

\subsubsection{新たな代表長さ算定式の構築}
上記の知見を踏まえ,より合理的な設計パラメータを導出するため,凍結管間隔,凍結管密度およびそれらの相互作用項を説明変数とする重回帰分析を行い,新たな代表長さの算定式を構築した.
分析の結果,提案式を用いることで,凍結管配置を考慮した代表長さを良好に整理できることが確認され,本手法の有効性が示された.

本研究で構築した代表長さの算定式は,凍結管間隔や凍結管密度など,あらかじめ決定可能な幾何学的条件のみを説明変数としている.
このため,従来設計者の経験的判断に委ねられていた代表長さを,客観的かつ定量的に決定することが可能となる.
本研究の成果は,解析的設計手法の信頼性および実用性を向上させ,地下水流動を伴う人工地盤凍結工事の設計において,過度に保守的な設計や不要な地盤改良の解消に寄与することが期待される.

\FloatBarrier
\numberwithin{equation}{section}

\FloatBarrier

% Discussion
\part{最終考察}
\label{Part:Discussion}
\clearpage

\section{数値解析モデルの妥当性と解析アプローチの合理性}
\label{Sec:Param_Discussion_Validity}
現象の再現性:
構築した熱・水理カップリングソルバーは,ゼロカーテン効果や定常的な温度場形成を正確に再現しており,潜熱放出と対流・伝導のエネルギーバランスを計算する上で十分な精度を有しています .

逆解析の有効性:
実験値に含まれる環境外乱をフィルタリングし,物理的整合性の取れた境界条件を同定する手法として,逆解析アプローチは極めて合理的です.これにより,次章以降の広範なパラメータスタディの信頼性が担保されています.

\section{代表長さ $l$ を支配する物理的メカニズムの解明}
\label{Sec:Param_Discussion_PhysicalMeaning}
水理的堰上げ(dam-up)効果: $l$ は単なる幾何形状の指標ではなく,地下水流に対する水理的抵抗の大きさを定量化するパラメータです .凍結管の本数が増えるほど,あるいは管間隔が広くなるほど,上流側の圧力が上昇し,未凍結部を通過する局所流速が加速されるため,限界流速 $V_{crit}$ は低下します .

幾何学的相互作用: 回帰分析の結果,代表長さ $l$ は凍土壁の総延長 $l_{total}$ だけでなく,配管密度($2a/L$)およびそれらの相互作用項に強く依存することが明らかになりました .

\section{提案モデルの工学的有用性と実務への適用}
曖昧さの排除: 従来,設計者の経験や仮定に委ねられていた $l$ を,具体的な配管レイアウトから算出可能な予測変数へと変換しました .これにより,高志の式を用いた簡便な限界流速判定の信頼性が大幅に向上します .

設計の最適化: 配管密度を高めることによる水理的堅牢性の向上は,特に長大な凍土壁において顕著です .この相乗効果を定量化したことは,大規模プロジェクトにおけるコストとリスクの最適化に直結します .

\section{適用範囲と今後の課題}
土質条件: 本モデルは砂質土などの高透水性地盤を対象としており,地下水流が凍結を阻害する最もクリティカルな条件下での設計を支援します .

幾何学的制約: 現時点では直線状の配置を前提としていますが,大規模な円形立坑などでは直線条件への近似が可能です .今後は,曲率を持つ非線形な形状への拡張が期待されます .

安全側の設計: 実務においては,還り温度などの保守的な値を入力することで,熱効率に対する安全率を考慮した設計運用が可能です .

%============================================================================================================
% BACK MATTER
%============================================================================================================
% Appendix
\makeatletter
\renewcommand{\theequation}{\Alph{section}.\arabic{equation}}
\makeatother

\appendix
\addcontentsline{toc}{part}{付録}

\section{ベクトル解析の基礎について}
\label{sec:VectorAnalysis}
本付録では,本修論を読むために助けになるベクトル解析の知識について解説する.本修論では,すべて右手系の直交座標系を用いている.そのときの正規直交基底$\vect{e}$は次のように定義される.
\begin{equation}
  \vect{e}_x=\begin{bmatrix}
    1 \\0\\0
  \end{bmatrix}, \quad
  \vect{e}_y=\begin{bmatrix}
    0 \\1\\0
  \end{bmatrix}, \quad
  \vect{e}_z=\begin{bmatrix}
    0 \\0\\1
  \end{bmatrix}
\end{equation}
\begin{FormulaBox}{Kronecker delta}{Kronecker-delta}
  Kronecker deltaは,$i$と$j$が等しいときに$1$,それ以外のときは$0$となる関数である.
  \begin{equation}
    \label{eq:Kronecker-delta}
    \delta_{ij}=\begin{cases}
      1 & \pab{i=j}     \\
      0 & \pab{i\neq j}
    \end{cases}
  \end{equation}
  Kronecker deltaは,$\vect{e}$の内積を用いて次のように表すことができる.
  \begin{equation}
    \label{eq:Kronecker-delta-inner-product}
    \delta_{ij}=\vect{e}_i\cdot\vect{e}_j
  \end{equation}
  Kronecker deltaは,行列の対角成分を表すのに便利である.
\end{FormulaBox}
\begin{FormulaBox}{Levi-Civita symbol}{Levi-Civita-symbol}
  Levi-Civita symbolは,添字$i,j,k$の並び方に応じて次のような値を取るものとして定義される.
  \begin{equation}
    \label{eq:Levi-Civita-symbol}
    \epsilon_{ijk}=\begin{cases}
      +1 & \pab{\pab{i,j,k}\in\Bab{\pab{1,2,3}, \pab{2,3,1}, \pab{3,1,2}}} \\
      -1 & \pab{\pab{i,j,k}\in\Bab{\pab{1,3,2}, \pab{3,2,1}, \pab{2,1,3}}} \\
      0  & \pab{\text{otherwise}}
    \end{cases}
  \end{equation}
  添字$ijk$ が, $\pab{1,2,3}$ の偶置換である場合は $+1$ ,奇置換である場合は $-1$ ,それ以外は$0$となる.正規直交基底のスカラー三重積で表示することもできる.
  \begin{equation}
    \label{eq:Levi-Civita-ScalarTripleProduct}
    \epsilon_{ijk}=\pab{\vect{e}_i\times\vect{e}_j}\cdot\vect{e}_k
  \end{equation}
  また,Levi-Civita symbolは循環性がある.$i\rightarrow j$,$j\rightarrow k$,$k\rightarrow i$のとき,$\epsilon_{ijk}=\epsilon_{jki}=\epsilon_{kij}$が成り立つ.
\end{FormulaBox}
ここで適当なベクトル$\vect{a}$,$\vect{b}$,$\vect{c}$を次のように定義する.
\begin{equation}
  \vect{a}=\begin{bmatrix}
    a_1 \\a_2\\a_3
  \end{bmatrix}, \quad
  \vect{b}=\begin{bmatrix}
    b_1 \\b_2\\b_3
  \end{bmatrix}, \quad
  \vect{c}=\begin{bmatrix}
    c_1 \\c_2\\c_3
  \end{bmatrix}
\end{equation}

\begin{FormulaBox}{ベクトルの内積}{Vector-InnerProduct}
  ベクトル$\vect{a}$と$\vect{b}$の内積は次のように定義される.
  \begin{equation}
    \label{eq:Vector-InnerProduct}
    \vect{a}\cdot\vect{b}=\sum_{i=1}^{3}a_ib_i=\pab{\vect{a}\cdot\vect{b}}_i
  \end{equation}
  また,内積はKronecker deltaを用いて次のように表すことができる.
  \begin{equation}
    \label{eq:Vector-InnerProduct-KroneckerDelta}
    \vect{a}\cdot\vect{b}=\delta_{ij}a_ib_j
  \end{equation}
\end{FormulaBox}
\begin{FormulaBox}{ベクトルの外積}{Vector-OuterProduct}
  ベクトル$\vect{a}$と$\vect{b}$の外積
  \begin{equation}
    \vect{a}\times\vect{b}=\begin{vmatrix}
      \vect{e}_x & \vect{e}_y & \vect{e}_z \\
      a_1        & a_2        & a_3        \\
      b_1        & b_2        & b_3
    \end{vmatrix}=\begin{bmatrix}
      a_2b_3-a_3b_2 \\
      a_3b_1-a_1b_3 \\
      a_1b_2-a_2b_1
    \end{bmatrix}
  \end{equation}
  の各成分はLevi-Civita symbolを用いて次のように表すことができる.
  \begin{equation}
    \label{eq:Levi-Civita-OuterProduct}
    \pab{\vect{a}\times\vect{b}}_i=\sum_{j=1}^{3}\sum_{k=1}^{3}\epsilon_{ijk}a_jb_k
  \end{equation}
\end{FormulaBox}
\begin{proof}
  \eqref{eq:Levi-Civita-symbol}を用いると
  \begin{align*}
      & \sum_{j=1}^{3}\sum_{k=1}^{3}\epsilon_{1jk}a_jb_k\notag                \\
    = & \epsilon_{111}a_1b_1+\epsilon_{112}a_1b_2+\epsilon_{113}a_1b_3\notag  \\
      & +\epsilon_{121}a_2b_1+\epsilon_{122}a_2b_2+\epsilon_{123}a_2b_3\notag \\
      & +\epsilon_{131}a_3b_1+\epsilon_{132}a_3b_2+\epsilon_{133}a_3b_3\notag \\
    = & a_2b_3-a_3b_2\notag                                                   \\
    = & \pab{\vect{a}\times\vect{b}}_1
  \end{align*}
  同様に
  \begin{align*}
      & \sum_{j=1}^{3}\sum_{k=1}^{3}\epsilon_{2jk}a_jb_k\notag                \\
    = & \epsilon_{211}a_1b_1+\epsilon_{212}a_1b_2+\epsilon_{213}a_1b_3\notag  \\
      & +\epsilon_{221}a_2b_1+\epsilon_{222}a_2b_2+\epsilon_{223}a_2b_3\notag \\
      & +\epsilon_{231}a_3b_1+\epsilon_{232}a_3b_2+\epsilon_{233}a_3b_3\notag \\
    = & a_3b_1-a_1b_3\notag                                                   \\
    = & \pab{\vect{a}\times\vect{b}}_2
  \end{align*}
  \begin{align*}
      & \sum_{j=1}^{3}\sum_{k=1}^{3}\epsilon_{3jk}a_jb_k\notag                \\
    = & \epsilon_{311}a_1b_1+\epsilon_{312}a_1b_2+\epsilon_{313}a_1b_3\notag  \\
      & +\epsilon_{321}a_2b_1+\epsilon_{322}a_2b_2+\epsilon_{323}a_2b_3\notag \\
      & +\epsilon_{331}a_3b_1+\epsilon_{332}a_3b_2+\epsilon_{333}a_3b_3\notag \\
    = & a_1b_2-a_2b_1\notag                                                   \\
    = & \pab{\vect{a}\times\vect{b}}_3
  \end{align*}
\end{proof}
\begin{FormulaBox}{Levi-Civita symbolの恒等式}{Levi-Civita-symbol-Identity}
  $n=3$のLevi-Civita symbolには次の恒等式が成り立つ.
  \begin{equation}
    \label{eq:Levi-Civita-Identity}
    \sum_{i=1}^{3}\epsilon_{ijk}\epsilon_{ilm}=\delta_{jl}\delta_{km}-\delta_{jm}\delta_{kl}
  \end{equation}
\end{FormulaBox}
\begin{proof}
  \eqref{eq:Levi-Civita-ScalarTripleProduct}を用いると,\eqref{eq:Levi-Civita-Identity}の左辺は次のように表される.
  \begin{align}
    \label{eq:Levi-Civita-Identity-proof-1}
    \sum_{i=1}^{3}\epsilon_{ijk}\epsilon_{ilm} & =\sum_{i=1}^{3}\bab[big]{\pab[big]{\vect{e}_i\times\vect{e}_j}\cdot\vect{e}_k}\bab[big]{\pab[big]{\vect{e}_i\times\vect{e}_l}\cdot\vect{e}_m}
  \end{align}
  一般にスカラー三重積は,それを構成する三つのベクトルを列ベクトルとする行列の行列式に等しい.よって\eqref{eq:Levi-Civita-Identity-proof-1}の右辺に現れたスカラー三重積を行列式に書き換えると,
  \begin{align}
    \label{eq:Levi-Civita-Identity-proof-2}
    \sum_{i=1}^{3}\epsilon_{ijk}\epsilon_{ilm} & =\sum_{i=1}^{3}\bab[big]{\pab[big]{\vect{e}_i\times\vect{e}_j}\cdot\vect{e}_k}\bab[big]{\pab[big]{\vect{e}_i\times\vect{e}_l}\cdot\vect{e}_m}\notag \\
                                               & = \sum_{i=1}^{3}
    \det\begin{bmatrix}
          \vect{e}_i & \vect{e}_j & \vect{e}_k
        \end{bmatrix}
    \det\begin{bmatrix}
          \vect{e}_i & \vect{e}_l & \vect{e}_m
        \end{bmatrix}
  \end{align}
  一般に転置行列の行列式はもとの行列の行列式と等しく,行列の積の行列式はそれぞれの行列の行列式の積に等しいので,\eqref{eq:Levi-Civita-Identity-proof-2}は次のように書き換えることができる.
  \begin{align}
    \label{eq:Levi-Civita-Identity-proof-3}
    \sum_{i=1}^{3}\epsilon_{ijk}\epsilon_{ilm} & =\sum_{i=1}^{3}
    \det\begin{bmatrix}
          \vect{e}_i^\top \\ \vect{e}_j^\top \\ \vect{e}_k^\top
        \end{bmatrix}
    \det\begin{bmatrix}
          \vect{e}_i & \vect{e}_l & \vect{e}_m
        \end{bmatrix}\notag                                                                                                               \\
                                               & =\sum_{i=1}^{3}\det~\pab{\begin{bmatrix}
                                                                              \vect{e}_i^\top \\ \vect{e}_j^\top \\ \vect{e}_k^\top\\
                                                                            \end{bmatrix}
    \begin{bmatrix}
        \vect{e}_i & \vect{e}_l & \vect{e}_m
      \end{bmatrix}}\notag                                                                                                                   \\
                                               & =\sum_{i=1}^{3}\det\begin{bmatrix}
                                                                      \vect{e}_i^\top \vect{e}_i & \vect{e}_i^\top \vect{e}_l & \vect{e}_i^\top \vect{e}_m \\
                                                                      \vect{e}_j^\top \vect{e}_i & \vect{e}_j^\top \vect{e}_l & \vect{e}_j^\top \vect{e}_m \\
                                                                      \vect{e}_k^\top \vect{e}_i & \vect{e}_k^\top \vect{e}_l & \vect{e}_k^\top \vect{e}_m \\
                                                                    \end{bmatrix}
  \end{align}
  ここで,ベクトルの内積が$\vect{a}\cdot\vect{b}=\vect{a}^\top\vect{b}$であり,$\vect{e}$が正規直交基底をなすことに注意すれば,
  \begin{align}
    \label{eq:Levi-Civita-Identity-proof-4}
    \sum_{i=1}^{3}\epsilon_{ijk}\epsilon_{ilm} & =\sum_{i=1}^{3}\det\begin{bmatrix}
                                                                      \vect{e}_i\cdot\vect{e}_i & \vect{e}_i\cdot\vect{e}_l & \vect{e}_i\cdot\vect{e}_m \\
                                                                      \vect{e}_j\cdot\vect{e}_i & \vect{e}_j\cdot\vect{e}_l & \vect{e}_j\cdot\vect{e}_m \\
                                                                      \vect{e}_k\cdot\vect{e}_i & \vect{e}_k\cdot\vect{e}_l & \vect{e}_k\cdot\vect{e}_m \\
                                                                    \end{bmatrix}\notag \\
                                               & =\sum_{i=1}^{3}\det\begin{bmatrix}
                                                                      \delta_{ii} & \delta_{il} & \delta_{im} \\
                                                                      \delta_{ji} & \delta_{jl} & \delta_{jm} \\
                                                                      \delta_{ki} & \delta_{kl} & \delta_{km} \\
                                                                    \end{bmatrix}\notag                                           \\
                                               & =\sum_{i=1}^{3}\det\begin{bmatrix}
                                                                      1           & \delta_{il} & \delta_{im} \\
                                                                      \delta_{ji} & \delta_{jl} & \delta_{jm} \\
                                                                      \delta_{ki} & \delta_{kl} & \delta_{km} \\
                                                                    \end{bmatrix}
  \end{align}
  \eqref{eq:Levi-Civita-Identity-proof-4}の行列式を余因子展開すれば,
  \begin{align}
    \label{eq:Levi-Civita-Identity-proof-5}
    \sum_{i=1}^{3}\epsilon_{ijk}\epsilon_{ilm} & =\sum_{i=1}^{3}\det\begin{bmatrix}
                                                                      1           & \delta_{il} & \delta_{im} \\
                                                                      \delta_{ji} & \delta_{jl} & \delta_{jm} \\
                                                                      \delta_{ki} & \delta_{kl} & \delta_{km} \\
                                                                    \end{bmatrix}\notag                                                                                                                                          \\
                                               & =\sum_{i=1}^{3}\Bab{\begin{vmatrix}
                                                                         \delta_{jl} & \delta_{jm} \\
                                                                         \delta_{kl} & \delta_{km} \\
                                                                       \end{vmatrix}+\delta_{il}\begin{vmatrix}
                                                                                                  \delta_{jm} & \delta_{ji} \\
                                                                                                  \delta_{km} & \delta_{ki} \\
                                                                                                \end{vmatrix}+\delta_{im}\begin{vmatrix}
                                                                                                                           \delta_{ji} & \delta_{jl} \\
                                                                                                                           \delta_{ki} & \delta_{kl} \\
                                                                                                                         \end{vmatrix}}\notag                                                                                                     \\
                                               & =\sum_{i=1}^{3}\bab{\pab{\delta_{jl}\delta_{km}-\delta_{jm}\delta_{kl}}+\delta_{il}\pab{\delta_{jm}\delta_{ki}-\delta_{ji}\delta_{km}}+\delta_{im}\pab{\delta_{ji}\delta_{kl}-\delta_{jl}\delta_{ki}}}
  \end{align}
  Kronecker-deltaの定義に注意し売れば,\eqref{eq:Levi-Civita-Identity-proof-5}の右辺は次のように書き換えることができる.
  \begin{subequations}
    \begin{equation}
      \label{eq:Levi-Civita-Identity-proof-6}
      \sum_{i=1}^{3}\delta_{jl}\delta_{km}-\delta_{jm}\delta_{kl}=3\pab{\delta_{jl}\delta_{km}-\delta_{jm}\delta_{kl}}
    \end{equation}
    \begin{equation}
      \label{eq:Levi-Civita-Identity-proof-7}
      \sum_{i=1}^{3}\delta_{il}\pab{\delta_{jm}\delta_{ki}-\delta_{ji}\delta_{km}}=\delta_{jm}\delta_{kl}-\delta_{jl}\delta_{km}
    \end{equation}
    \begin{equation}
      \label{eq:Levi-Civita-Identity-proof-8}
      \sum_{i=1}^{3}\delta_{im}\pab{\delta_{ji}\delta_{kl}-\delta_{jl}\delta_{ki}}=\delta_{jm}\delta_{kl}-\delta_{jl}\delta_{km}
    \end{equation}
  \end{subequations}
  よって,
  \eqref{eq:Levi-Civita-Identity-proof-5}は次のように書き換えることができる.
  \begin{align}
    \label{eq:Levi-Civita-Identity-proof-9}
    \sum_{i=1}^{3}\epsilon_{ijk}\epsilon_{ilm} & =\sum_{i=1}^{3}\bab{\pab{\delta_{jl}\delta_{km}-\delta_{jm}\delta_{kl}}+\delta_{il}\pab{\delta_{jm}\delta_{ki}-\delta_{ji}\delta_{km}}+\delta_{im}\pab{\delta_{ji}\delta_{kl}-\delta_{jl}\delta_{ki}}}\notag \\
                                               & =3\pab{\delta_{jl}\delta_{km}-\delta_{jm}\delta_{kl}}+\delta_{jm}\delta_{kl}-\delta_{jl}\delta_{km}+\delta_{jm}\delta_{kl}-\delta_{jl}\delta_{km}\notag                                                      \\
                                               & =\delta_{jl}\delta_{km}-\delta_{jm}\delta_{kl}
  \end{align}
\end{proof}

\begin{FormulaBox}{発散(Divergence)}{Divergence-Definition}
  ベクトル場$\vect{a}\pab{x, y, z}$の各成分の空間微分によって定義されるスカラー場を,$\vect{a}$の発散(Divergence)と呼び,$\nabla\cdot\vect{a}$あるいは$\div\vect{a}$と表記する.
  \begin{equation}
    \label{eq:Divergence-Def}
    \nabla\cdot\vect{a} = \pdv{a_1}{x} + \pdv{a_2}{y} + \pdv{a_3}{z} = \sum_{i=1}^{3}\pdv{a_i}{x_i}
  \end{equation}
  ここで,$\nabla$はナブラ演算子
  \begin{equation*}
    \nabla = \vect{e}_x \pdv{}{x} + \vect{e}_y \pdv{}{y} + \vect{e}_z \pdv{}{z}
  \end{equation*}
  を表し,形式的にナブラ演算子とベクトルの内積として扱うことができる.
\end{FormulaBox}

\begin{FormulaBox}{ガウスの発散定理(3次元)}{Gauss-Divergence-Theorem-3D}
  空間内の閉領域$V$とその境界である閉曲面$\partial V$を考える.ベクトル場$\vect{a}$が$V$を含む領域で連続な偏導関数を持つとき,以下の等式が成り立つ.
  \begin{equation}
    \label{eq:Gauss-Divergence-Theorem-3D}
    \int_{V} \nabla\cdot\vect{a} \odif{V} = \int_{\partial V} \vect{a}\cdot\vect{n} \odif{S}
  \end{equation}
  ここで,$\vect{n}$は閉曲面$\partial V$上の外向き単位法線ベクトル,$dV$は体積要素,$dS$は面積要素である.
\end{FormulaBox}

\begin{proof}
  領域$V$が$x, y, z$の各軸方向に関して単純な領域(任意の軸に平行な直線が,境界と高々2点で交わる領域)であると仮定する.
  まず,$\nabla\cdot\vect{a}$の第3項($z$微分成分)の体積積分を考える.$V$の$xy$平面への正射影を$D$とし,$V$の境界$\partial V$のうち,上側の面を$z=z_{top}\pab{x,y}$,下側の面を$z=z_{btm}\pab{x,y}$とする.
  \begin{align}
    \label{eq:Gauss-Proof-3D-1}
    \int_{V} \frac{\partial a_3}{\partial z} \, dV 
     & = \iint_{D} \bab{\int_{z_{btm}\pab{x,y}}^{z_{top}\pab{x,y}} \frac{\partial a_3}{\partial z} \, dz} \, dxdy \notag \\
     & = \iint_{D} \bab{a_3\pab{x,y,z_{top}} - a_3\pab{x,y,z_{btm}}} \, dxdy
  \end{align}
  ここで,上側の面における外向き法線ベクトル$\vect{n}_{top}$と$\vect{e}_z$のなす角を$\gamma_{top}$とすると,面素の関係$dxdy = \cos\gamma_{top} dS = \pab{\vect{n}_{top}\cdot\vect{e}_z}dS$が成り立つ.同様に下側の面では,外向き法線$\vect{n}_{btm}$が下を向くため,$\vect{n}_{btm}\cdot\vect{e}_z$は負となり,$dxdy = -\cos\gamma_{btm} dS = -\pab{\vect{n}_{btm}\cdot\vect{e}_z}dS$となる.
  これらを\eqref{eq:Gauss-Proof-3D-1}に代入し,閉曲面$\partial V$全体での積分に書き換えると,
  \begin{align}
    \label{eq:Gauss-Proof-3D-2}
    \iint_{D} a_3\pab{x,y,z_{top}} \, dxdy + \iint_{D} -a_3\pab{x,y,z_{btm}} \, dxdy 
     & = \int_{\partial V_{top}} a_3 \pab{\vect{n}\cdot\vect{e}_z} \, dS + \int_{\partial V_{btm}} a_3 \pab{\vect{n}\cdot\vect{e}_z} \, dS \notag \\
     & = \int_{\partial V} a_3 n_3 \, dS
  \end{align}
  となる.$x, y$成分についても同様の議論を行うことで,
  \begin{equation}
    \int_{V} \frac{\partial a_1}{\partial x} \, dV = \int_{\partial V} a_1 n_1 \, dS, \quad
    \int_{V} \frac{\partial a_2}{\partial y} \, dV = \int_{\partial V} a_2 n_2 \, dS
  \end{equation}
  が得られる.これら3つの式の和をとることで,\eqref{eq:Gauss-Divergence-Theorem-3D}が得られる.
  なお,複雑な形状の領域であっても,単純な領域の和に分割することで,本定理は同様に成立する.
\end{proof}

\begin{FormulaBox}{ガウスの発散定理(2次元)}{Gauss-Divergence-Theorem-2D}
  平面内の閉領域$S$とその境界である閉曲線$\partial S$を考える.2次元ベクトル場$\vect{a}=\begin{bmatrix}a_1 & a_2\end{bmatrix}^\top$について,以下の等式が成り立つ.
  \begin{equation}
    \label{eq:Gauss-Divergence-Theorem-2D}
    \int_{S} \nabla\cdot\vect{a} \, dS = \oint_{\partial S} \vect{a}\cdot\vect{n} \, dl
  \end{equation}
  ここで,$\nabla\cdot\vect{a} = \dfrac{\partial a_1}{\partial x} + \frac{\partial a_2}{\partial y}$であり,$\vect{n}$は境界$\partial S$上の外向き単位法線ベクトル,$l$は弧長パラメータである.
\end{FormulaBox}

\begin{proof}
  3次元の場合と同様に,$S$を$y$軸に平行な線分で切ったときに境界と高々2点で交わる単純な領域と仮定する.$\nabla\cdot\vect{a}$の第2項($y$微分成分)の面積分を考える.
  境界$\partial S$を,上側の曲線$y=y_{top}\pab{x}$と下側の曲線$y=y_{btm}\pab{x}$(定義域は$x_1 \le x \le x_2$)に分割する.
  \begin{align}
    \label{eq:Gauss-Proof-2D-1}
    \int_{S} \frac{\partial a_2}{\partial y} \, dS 
     & = \int_{x_1}^{x_2} \bab{\int_{y_{btm}\pab{x}}^{y_{top}\pab{x}} \frac{\partial a_2}{\partial y} \, dy} \, dx \notag \\
     & = \int_{x_1}^{x_2} \bab{a_2\pab{x,y_{top}} - a_2\pab{x,y_{btm}}} \, dx
  \end{align}
  ここで,境界上の線素ベクトル$d\vect{l}$を反時計回りに取ると,外向き法線ベクトル$\vect{n}$を用いて$\vect{n}\,dl = \pab{dy, -dx}^\top$の関係がある(あるいは幾何学的に,$dx = n_y dl / \sin\theta$等の関係を用いる).
  上側の曲線では$x$が増加するにつれて積分路は右へ進むが,$\vect{n}$の$y$成分は正であるため,$dx = \frac{dx}{dl}dl$に対し$n_2 dl \approx dx$の関係(厳密には$n_2 dl = dx$)が成り立つ.下側の曲線では積分路の向きと$x$の増分が逆になること等を考慮し,線積分へ書き換えると以下のようになる.
  \begin{equation}
    \int_{x_1}^{x_2} a_2\pab{x,y_{top}} \, dx - \int_{x_1}^{x_2} a_2\pab{x,y_{btm}} \, dx 
    = \int_{\partial S_{top}} a_2 n_2 \, dl + \int_{\partial S_{btm}} a_2 n_2 \, dl 
    = \oint_{\partial S} a_2 n_2 \, dl
  \end{equation}
  $x$成分についても同様に$\int_{S} \frac{\partial a_1}{\partial x} \, dS = \oint_{\partial S} a_1 n_1 \, dl$が示せるため,両者の和をとることで\eqref{eq:Gauss-Divergence-Theorem-2D}が得られる.
\end{proof}

\begin{FormulaBox}{発散の積の微分}{Identity-Div-Product}
  スカラー場$f$とベクトル場$\vect{A}$の積の発散について,以下の恒等式が成り立つ.
  \begin{equation}
    \label{eq:Identity-Div-Product}
    \nabla \cdot \pab{f \vect{A}} = \nabla f \cdot \vect{A} + f \nabla \cdot \vect{A}
  \end{equation}
  これは,1変数の積の微分公式 $\pab{uv}' = u'v + uv'$ の多次元(ベクトル解析)版に相当する.
\end{FormulaBox}

\begin{proof}
  直交座標系における成分計算によって示す.$\nabla$演算子の$i$成分を$\partial_i = \frac{\partial}{\partial x_i}$,$\vect{A}$の第$i$成分を$A_i$と表記し,アインシュタインの縮約記法を用いる.
  左辺を展開し,積の微分公式を適用すると以下のようになる.
  \begin{align}
    \nabla \cdot \pab{f \vect{A}} & = \partial_i \pab{f A_i} \notag                   \\
                                  & = \pab{\partial_i f} A_i + f \pab{\partial_i A_i}
  \end{align}
  ここで,第1項は勾配ベクトル$\nabla f$と$\vect{A}$の内積 $\pab{\nabla f} \cdot \vect{A}$ であり,第2項は$f$と発散 $\nabla \cdot \vect{A}$ の積であるため,
  \begin{equation}
    = \nabla f \cdot \vect{A} + f \nabla \cdot \vect{A}
  \end{equation}
  となり,等式が示された.
\end{proof}

% References
\addcontentsline{toc}{part}{引用文献}
\pagestyle{fancy}
\fancyhf{}
\fancyhead[R]{\thepage}
\renewcommand{\headrulewidth}{0pt}
\fancypagestyle{plain}{
  \fancyhf{}
  \fancyhead[R]{\thepage}
  \renewcommand{\headrulewidth}{0pt}
}
\printbibliography[heading=bibliography, title={引用文献}]
\clearpage
\section*{謝辞}
\label{Part:acknowledgements}
\addcontentsline{toc}{part}{謝辞}

本論文の執筆にあたり,多くの方々から多大なるご指導とご支援を賜りました.ここに記して深く感謝申し上げます.

まず,研究の遂行に際し終始ご指導・ご助言を賜りました指導教員の〇〇〇〇先生に心より御礼申し上げます.研究課題の設定から実験・解析,論文構成に至るまで,多角的な観点からご示唆を頂き,本研究を大きく前進させることができました.

また,研究室の〇〇先生ならびに〇〇先生には,日頃の議論や発表に対する貴重なご意見を頂きました.加えて,研究室の先輩・同輩の皆様には,日々の議論や作業面での助力を通じて多くの支えを頂きました.

本研究の一部は〇〇(共同研究,学内プロジェクト,助成金等)の支援を受けて実施されました.関係各位に感謝申し上げます.

最後に,学生生活を通じて温かく見守り支えてくれた家族に深く感謝いたします.
%============================================================================================================
\end{document}